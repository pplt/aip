\documentclass[11pt]{amsart}
%%%%%%%%%%%%%%%%%% Colors %%%%%%%%%%%%%%%%%%
\usepackage{xcolor}
\definecolor{nicered}{rgb}{0.6, 0, 0.1}
\definecolor{niceblue}{rgb}{0.06, 0.3, 0.57}
\definecolor{nicegreen}{rgb}{0.0, 0.51, 0.5}

%%%%%%%%%%%%%%%%%% Assorted Packages %%%%%%%%%%%%%%%%%%
\usepackage[colorlinks=true,pagebackref,hyperindex,citecolor=nicegreen,linkcolor=niceblue,urlcolor=nicered]{hyperref}
\usepackage{amsmath,amsthm,amsfonts,amssymb}
%\usepackage{color}
\usepackage{mathrsfs,stmaryrd,bm}
\usepackage[mathcal]{euscript}
\usepackage{mathtools,soul}
\usepackage{microtype}
\usepackage[shortlabels]{enumitem}
\usepackage{booktabs}
\usepackage{xspace}
\usepackage{caption,subcaption}
\captionsetup[subfigure]{subrefformat=simple,labelformat=simple}
\renewcommand\thesubfigure{(\sc \alph{subfigure})}
\usepackage[
ruled,
%linesnumbered,
vlined]{algorithm2e}

%shortfall and deficit
\newcommand{\short}{\operatorname{short}}
\newcommand{\ushort}{\operatorname{ushort}}
\newcommand{\deficit}{\operatorname{deficit}}
\newcommand{\udeficit}{\operatorname{udeficit}}

\newcommand{\denom}{d} 
\newcommand{\fsr}{\mathcal{R}}

\renewcommand{\S}{\mathcal{S}}
\newcommand{\pp}{\mathsf{p}}
\renewcommand{\tt}{\mathsf{t}}


\newcommand{\hooklongrightarrow}{\lhook\joinrel\longrightarrow}

\newcommand{\mspec}{\operatorname{mSpec}}
\newcommand{\spec}{\operatorname{Spec}}

%clever ref package
%must come before following 3 sections
\usepackage{cleveref}  %must be consistent with names in following 3 sections
\crefname{equation}{Eq.}{Eqs.}
\crefname{theorem}{Theorem}{Theorems}
\crefname{lemma}{Lemma}{Lemmas}
\crefname{corollary}{Corollary}{Corollaries}
\crefname{proposition}{Proposition}{Propositions}
\crefname{definition}{Definition}{Definitions}
\crefname{remark}{Remark}{Remarks}
\crefname{example}{Example}{Examples}
\crefname{notation}{Notation}{Notations}
\crefname{setup}{Setup}{Setups}
\crefname{question}{Question}{Questions}
\crefname{convention}{Convention}{Conventions}
\crefname{algorithm}{Algorithm}{Algorithms}
\newcommand{\creflastconjunction}{, and\nobreakspace}

 %theorem style environments
\newtheorem{theorem}{Theorem}[section]
\newtheorem{lemma}[theorem]{Lemma}
\newtheorem{corollary}[theorem]{Corollary}
\newtheorem{proposition}[theorem]{Proposition}
\newtheorem{thmintro}{Theorem}
\renewcommand{\thethmintro}{\Alph{thmintro}}

%definition style environments
\theoremstyle{definition}
\newtheorem{definition}[theorem]{Definition}
\newtheorem{setup}[theorem]{Setup}
\newtheorem{example}[theorem]{Example}

%remark style environments
\theoremstyle{remark}
\newtheorem{remark}[theorem]{Remark}
\newtheorem{notation}[theorem]{Notation}
\newtheorem{convention}[theorem]{Convention}
\newtheorem{problem}[theorem]{Problem}
\newtheorem*{claim}{Claim}

%numbering
\numberwithin{equation}{subsection} %Can replace {subsection} with {theorem} if you want

%spacing
%\usepackage{setspace}
%\singlespacing
%\onehalfspacing
%\doublespacing
%\setstretch{1.1}

%\setlength{\parskip}{0.4em}


%various thresholds
\DeclareMathOperator{\lct}{lct}
\DeclareMathOperator{\fpt}{fpt}
\newcommand{\ft}[2]{\operatorname{ft}(#1, #2)}

%ideals
\newcommand{\ideal}[1]{\langle #1 \rangle}
\newcommand{\ideala}{\mathfrak{a}}
\newcommand{\idealb}{\mathfrak{b}}
\newcommand{\ideald}{\mathfrak{d}}
\newcommand{\idealm}{\mathfrak{m}}
\newcommand{\idealp}{\mathfrak{p}}
\newcommand{\mon}{\operatorname{mon}}
\newcommand{\idealc}{\mathfrak{c}}
\newcommand{\J}{\mathcal{J}} % for multiplier ideals

%linear & integer programs
\newcommand{\LP}{\mathrm{P}}
\newcommand{\IP}{\Pi}
\newcommand{\ip}{\Theta}
\DeclareMathOperator{\im}{im}
\DeclareMathOperator{\opt}{opt}
\DeclareMathOperator{\val}{val}
\DeclareMathOperator{\feas}{feas}

%convexity
\DeclareMathOperator{\conv}{conv}
\DeclareMathOperator{\cone}{cone}
\DeclareMathOperator{\rb}{rb}
\DeclareMathOperator{\rs}{rs}
\DeclareMathOperator{\ri}{ri}

%euclidean space
\newcommand{\vv}[1]{\mathbf{#1}} %vectors
\newcommand{\iprod}[2]{\langle #1, #2 \rangle} %dot product
\newcommand{\norm}[1]{ \| #1 \| } % norm
\newcommand{\canvec}{\vv{e}}
\newcommand{\defpt}{\vv{c}}
%representation of rational numbers
\newcommand{\tail}[1]{\left[ #1 \right]}
\newcommand{\lpr}[2]{ [ \hspace{.01in} #1 \, \% \, #2 \hspace{.01in} ]} %least positive residue
\newcommand{\up}[1]{\left\lceil #1 \right\rceil} %ceiling
\newcommand{\down}[1]{\left\lfloor #1 \right\rfloor} %floor

%random
\DeclareMathOperator{\col}{col}
\DeclareMathOperator{\mf}{mf}
\renewcommand{\sp}{\operatorname{sp}}
%\DeclareMathOperator{\rep}{rep}
%\DeclareMathOperator{\lis}{list}
\newcommand{\Q}{\mathcal{Q}}
%\newcommand{\N}{\mathrm{N}}
\newcommand{\N}{\mathcal{N}}
\newcommand{\M}{\mathcal{M}}
\renewcommand{\O}{\mathcal{O}}
\newcommand{\Z}{\mathcal{Z}}

% newly-defined commands
\DeclareMathOperator{\diag}{diag}
\DeclareMathOperator{\crit}{crit}
\newcommand{\orep}{\mathbb{O}}
\newcommand{\witt}{\mathfrak{W}}
\newcommand{\graph}{\mathfrak{S}}
\newcommand{\sierp}{\mathscr{S}}
\newcommand{\fip}{\Sigma}
\DeclareMathOperator{\sprout}{sprout}
\newcommand{\sproutsfrom}[2]{#1 \leftarrow #2}
%\newcommand{\sproutsfrom}[2]{#1 \in \operatorname{sp} #2}
\newcommand{\collapse}{\widebar}

%sets
\newcommand{\kk}{\Bbbk}
\newcommand{\LL}{\mathbb{L}}
\newcommand{\FF}{\mathbb{F}}
\newcommand{\RR}{\mathbb{R}}
\newcommand{\RRnn}{\mathbb{R}_{\ge 0}}
\newcommand{\CC}{\mathbb{C}}
\newcommand{\ZZ}{\mathbb{Z}}
\newcommand{\QQ}{\mathbb{Q}}
\newcommand{\NN}{\mathbb{N}}
\renewcommand{\emptyset}{\varnothing}

\newcommand{\numvars}{m}

%inequalities
\renewcommand{\geq}{\geqslant}
\renewcommand{\leq}{\leqslant}
\renewcommand{\ge}{\geqslant}
\renewcommand{\le}{\leqslant}

%abbreviations
\newcommand{\cf}{\emph{cf}.\ }
\newcommand{\eg}{e.g., }
\newcommand{\ie}{i.e., }
\newcommand{\loccit}{\emph{loc.~cit.}}
\newcommand{\vs}{vs.\ }

\newcommand{\muCool}{$\mu$-uniform\xspace}
\newcommand{\nuCool}{$\nu$-uniform\xspace}
\newcommand{\mustata}{Musta{\c{t}}\u{a}\xspace}


%notes
\usepackage[textwidth=3.3 cm,textsize=small,shadow
%disable
%%option disable removes the notes
]{todonotes}
\newcommand{\comment}[2][]{\todo[linecolor=orange,backgroundcolor=orange!30!,caption={}, #1]{#2}} % color-name! intensity !
\newcommand{\alert}[2][]{\todo[linecolor=red,backgroundcolor=red!50!,caption={}, #1]{#2}} % color-name! intensity !
\newcommand{\details}[2][]{\todo[linecolor=cyan,backgroundcolor=cyan!40, caption={},#1]{#2}}

\newcommand{\emily}[2][]{\todo[linecolor=green,backgroundcolor=green!30!,caption={}, #1]{#2}}
\newcommand{\daniel}[2][]{\todo[linecolor=blue,backgroundcolor=blue!30!,caption={}, #1]{#2}}
\newcommand{\pedro}[2][]{\todo[linecolor=nicegreen,backgroundcolor=nicegreen!70!,caption={}, #1]{#2}}

%editing
%\renewcommand{\!}[1]{{\color{red}\text{$\star$\,}#1\,$\star$}}
\newcommand{\ol}[1]{\overline{#1}}

% Decent looking bars (by Hendrik Vogt)
\makeatletter
\let\save@mathaccent\mathaccent
\newcommand*\if@single[3]{%
  \setbox0\hbox{${\mathaccent"0362{#1}}^H$}%
  \setbox2\hbox{${\mathaccent"0362{\kern0pt#1}}^H$}%
  \ifdim\ht0=\ht2 #3\else #2\fi
  }
%The bar will be moved to the right by a half of \macc@kerna, which is computed by amsmath:
\newcommand*\rel@kern[1]{\kern#1\dimexpr\macc@kerna}
%If there's a superscript following the bar, then no negative kern may follow the bar;
%an additional {} makes sure that the superscript is high enough in this case:
\newcommand*\widebar[1]{\@ifnextchar^{{\wide@bar{#1}{0}}}{\wide@bar{#1}{1}}}
%Use a separate algorithm for single symbols:
\newcommand*\wide@bar[2]{\if@single{#1}{\wide@bar@{#1}{#2}{1}}{\wide@bar@{#1}{#2}{2}}}
\newcommand*\wide@bar@[3]{%
  \begingroup
  \def\mathaccent##1##2{%
%Enable nesting of accents:
    \let\mathaccent\save@mathaccent
%If there's more than a single symbol, use the first character instead (see below):
    \if#32 \let\macc@nucleus\first@char \fi
%Determine the italic correction:
    \setbox\z@\hbox{$\macc@style{\macc@nucleus}_{}$}%
    \setbox\tw@\hbox{$\macc@style{\macc@nucleus}{}_{}$}%
    \dimen@\wd\tw@
    \advance\dimen@-\wd\z@
%Now \dimen@ is the italic correction of the symbol.
    \divide\dimen@ 3
    \@tempdima\wd\tw@
    \advance\@tempdima-\scriptspace
%Now \@tempdima is the width of the symbol.
    \divide\@tempdima 10
    \advance\dimen@-\@tempdima
%Now \dimen@ = (italic correction / 3) - (Breite / 10)
    \ifdim\dimen@>\z@ \dimen@0pt\fi
%The bar will be shortened in the case \dimen@<0 !
    \rel@kern{0.6}\kern-\dimen@
    \if#31
      \overline{\rel@kern{-0.6}\kern\dimen@\macc@nucleus\rel@kern{0.4}\kern\dimen@}%
      \advance\dimen@0.4\dimexpr\macc@kerna
%Place the combined final kern (-\dimen@) if it is >0 or if a superscript follows:
      \let\final@kern#2%
      \ifdim\dimen@<\z@ \let\final@kern1\fi
      \if\final@kern1 \kern-\dimen@\fi
    \else
      \overline{\rel@kern{-0.6}\kern\dimen@#1}%
    \fi
  }%
  \macc@depth\@ne
  \let\math@bgroup\@empty \let\math@egroup\macc@set@skewchar
  \mathsurround\z@ \frozen@everymath{\mathgroup\macc@group\relax}%
  \macc@set@skewchar\relax
  \let\mathaccentV\macc@nested@a
%The following initialises \macc@kerna and calls \mathaccent:
  \if#31
    \macc@nested@a\relax111{#1}%
  \else
%If the argument consists of more than one symbol, and if the first token is
%a letter, use that letter for the computations:
    \def\gobble@till@marker##1\endmarker{}%
    \futurelet\first@char\gobble@till@marker#1\endmarker
    \ifcat\noexpand\first@char A\else
      \def\first@char{}%
    \fi
    \macc@nested@a\relax111{\first@char}%
  \fi
  \endgroup
}
\makeatother

\usepackage{subcaption}
\usepackage{bm}

\begin{document}

\title[Fractal and arithmetic programs]{Fractal and arithmetic programs, and the Frobenius powers of monomial ideals}
\author{Daniel J.\ Hern\'andez}
\author{Pedro Teixeira}
\author{Emily E. Witt}
\maketitle

\newcommand{\denom}{\ell} %We should stick to the same notation for the denominator of a special point.  Sometimes D is used, but I would prefer a lowercase letter.  Lowercase "d" makes sense, but we use that for the dimension of the ambient polynomial ring.

\newcommand{\CheckedBox}{\text{\rlap{$\checkmark$}}\Box}

\emily[inline]{
TO DO LIST:
\begin{enumerate}
 \item[$\CheckedBox$] Write essential statements/proofs.
 \item[$\CheckedBox$] Decide on name of paper.
%  \emily[inline]{How about \emph{Fractal and arithmetic programs, and the Frobenius powers of monomial ideals}?}
% \pedro[inline]{I like that!}
 \item[$\CheckedBox$] Decide on name for minimal coordinate.
 \comment[inline]{Going with ``special point''.}
% \pedro[inline]{
%     Do we have a consensus for ``special points''? 
% } 
% \daniel[inline]{Emily likes it.  I thought for a while, and couldn't come up with something better.  If anyone gets inspired, we can always change it later.  So, let's make this change.  Should we say ``Let $\vv{s}$ be a special point \emph{for} $(A, \vv{u})$ and use the notation $\operatorname{sp}(A, \vv{u})$ or $\operatorname{special}(A, \vv{u})$?  The more time passes, the more I appreciate computer programming notation (where we just write out entire words, or longer abbreviations), but I don't insist that we do that here.
% }
% \pedro[inline]{
%    I've made the changes.
%    ``Special point for $(A,\vv{u})$'' sounds good to me.
%    For the notation, $\operatorname{sp}(A,\vv{u})$ aligns better with other notation we are using ($\ri$, $\rb$, $\opt$, \ldots), so that's what I've used; however, if anyone doesn't like it, please feel free to change it.
% }
 
\item[$\CheckedBox$] Make sprout notation.
\comment[inline]{Going with $\sprout(A,\vv{u},p)$}
\item[$\Box$] Put in examples.
 \item[$\Box$] Decide on how to define $\widehat{\witt}$.  (Sprouting graph?)

 % \daniel[inline]{Can someone explain why the graphs now start at the $0$-th level? }
 % \pedro[inline]{
 %    I did that, because I think that's how we were thinking of them in our conversations in Lawrence.
 %    For example, we were saying that $\bigcup_{e\ge 1}\graph^e(A,p)$ is finite, because we were thinking of the initial level consisting of all (possibly infinitely many) small pairs as level 0.
 %    I have a slight preference for starting at 0, but wouldn't mind changing it back, if you prefer. 
 % }
 \item[$\CheckedBox$] Fill in/rewrite preliminaries.
 \item[$\Box$] Reorganize and motivate the $\IP$ and $\ip$, and their connection, in Sections 5 and 6.
 \item[$\CheckedBox$] ``Fractal linear program;'' solve $P$ in ``Sierpinski gasket''
 \item[$\Box$] Direct proof that $\delta$/$\Delta$ are independent of $\vv{s}$.
 \item[$\Box$] Minimize and/or ``algebrafy'' statements.
 \item[$\CheckedBox$] Background section on Frobenius powers, $\mu$, $\nu$, etc.
 \item[$\Box$] Derive some easy corollaries for very general hypersurfaces.
 \item[$\CheckedBox$] Generalize the definition of $\IP$ so that $\IP(A, \vv{u}q)$ becomes $\IP(A, \vv{u}, q)$?
 \comment[inline]{Changed to $\IP(A, \vv{u}, q)$}
 % \daniel[inline]{Any thoughts on this?  The main reason for doing so to me is that it starts to get ugly when we add subscripts to the terms in $(A, \vv{u}, q)$ like we do when dealing with paths and iterated lifting.  It is also more consistent with the notation for $\ip$.  But, I could live with the $\vv{u}q$ version if someone feels strongly about this.}
 % \emily[inline]{I like $(A, \vv{u}, q)$ best, but don't mind so much either way.}
 % \pedro[inline]{I started changing to $\IP(A,\vv{u},q)$. Are we going to abbreviate $\IP(A,\vv{u},1)$ by $\IP(A,\vv{u})$? Or insist on always having a $q$?
 %    % , so, for example, the statement of Lemma~7.3 would become ``[\ldots] the quotient when dividing any optimal point of $\IP_p(A, \vv{u}, qp^e)$ by $p^e$ must be optimal for $\IP_p(A, \vv{u}, q)$''?
 %    \emily[inline]{Nice observation.  Both Daniel and I think insisting on the ``$q$'' is preferable.}
 %    \pedro[inline]{Almost done! Do you guys feel that way too about \Cref{follow-leftovers: P}? Somehow it felt unnatural to stick a ``$q$'' in that statement\ldots}
 %    \emily[inline]{Hmm, I like it as-is.}
 %}
 \item[$\Box$] Replace ``Lemma'' with ``Proposition'', if there's no immediate application in sight.
\end{enumerate}
}

\pedro[inline]{
   Should we say ``associated to'' or ``associated with''?
   We're using the former, but Google Ngram shows that the latter is a lot more popular.
\emily[inline]{Either is OK with me.}
}

\newpage

\section{Preliminaries}

\subsection{Euclidean spaces, convexity, and polyhedra}
\label{ss: euclidean spaces and convexity}

We review in this subsection some of the terminology, notation, and constructions concerning Euclidean spaces and convex geometry used throughout the paper.
We use bold-face lower-case letters to denote points of the Euclidean space $\RR^n$, and the same letter, in regular font, to represent their coordinates (\eg $\vv{v}=(v_1,\ldots,v_n)$).
The points $(0,\ldots,0)$ and $(1,\ldots,1)$ are denoted $\vv{0}$ and $\vv{1}$, and the standard basis vectors of $\RR^n$ are denoted $\canvec_1,\ldots,\canvec_n$.

Given a point $\vv{u}\in \RR^n$, $\norm{\vv{u}}$ denotes its coordinate sum, $u_1+\cdots+u_n$.
\pedro{Maybe we should emphasize that this is not a norm, despite the notation?}
The standard inner product in $\RR^n$ is denoted by the usual angle brackets: $\iprod{\vv{u}}{\vv{v}} = u_1v_1 + \cdots + u_nv_n$.
An inequality between points of $\RR^n$ is a shorthand for a system of $n$ coordinatewise inequalities; for instance, $\vv{u}\le \vv{v}$ means that $u_i \le v_i$ for each $i=1,\ldots,n$.
In the same vein, operations on numbers are extended to points in $\RR^n$ in a coordinatewise fashion; for instance, $\up{\vv{u}}=(\up{u_1},\ldots,\up{u_n})$.

We say that a point $\vv{u}\in \RR^n$ is positive (respectively, nonnegative) if $\vv{u} > \vv{0}$ (respectively, $\vv{u}\ge \vv{0}$).
More generally, given a point $\vv{u}$ in a coordinate subspace $\mathcal{S}$ of $\RR^n$, we say that $\vv{u}$ is \emph{positive in $\mathcal{S}$} (respectively, \emph{nonnegative in $\mathcal{S}$}) if $\vv{u}$ is a positive (respectively, nonegative) linear combination of the standard basis vectors that span $\mathcal{S}$.

Turning to concepts and constructions of convex geometry, a (convex) \emph{polyhedron} in $\RR^n$ is a subset of $\RR^n$ obtained by intersecting finitely many closed halfspaces or, equivalently, a set consisting of all points $\vv{x}\in \RR^n$ satisfying an inequality of the form $A\vv{x}\le \vv{b}$, where $A$ is a matrix with $n$ columns.

The (convex) \emph{cone generated by $\vv{u}_1,\ldots,\vv{u}_k \in \RR^n$}, denoted $\cone(\vv{u}_1,\ldots,\vv{u}_k)$, is the set consisting of all \emph{conical combinations} of $\vv{u}_1, \ldots, \vv{u}_k$, that is, points of the form $\sum_{i=1}^k \lambda_i \vv{u}_i$, where the $\lambda_i$ are nonnegative real numbers.
Likewise, the \emph{convex hull of $\vv{u}_1,\ldots,\vv{u}_k$}, denoted $\conv(\vv{u}_1,\ldots,\vv{u}_k)$, is the set of all \emph{convex combinations} of $\vv{u}_1, \ldots, \vv{u}_k$, that is, points of the form $\sum_{i=1}^k \lambda_i \vv{u}_i$, where the $\lambda_i$ are nonnegative and $\sum_{i=1}^k \lambda_i = 1$.
The convex hull of a finite set of points is called a \emph{polytope}.

If $\mathcal{U}$ and $\mathcal{V}$ are subsets of $\RR^n$, their \emph{Minkowski sum} is the set
\[\mathcal{U}+\mathcal{V} \coloneqq \{\vv{u}+\vv{v}: \vv{u}\in \mathcal{U}\text{ and }\vv{v}\in \mathcal{V}\}.\]
The \emph{Minkowski--Weyl Theorem} asserts that a subset $\mathcal{P}$ of $\RR^n$ is a polyhedron if and only if $\mathcal{P}$ is the Minkowski sum of a polytope and a finitely generated cone.
The cone in this decomposition is the set of all directions $\vv{d} \in \RR^n$ in which $\mathcal{P}$ recedes, that is, $\vv{c} + \lambda \vv{d} \in \mathcal{P}$ for every $\vv{c} \in \mathcal{P}$ and $\lambda > 0$; it is uniquely determined by $\mathcal{P}$, and called the \emph{recession cone of $\mathcal{P}$}.

The Minkowski--Weyl Theorem gives us a couple of useful characterizations of polytopes: a polyhedron $\mathcal{P}$ is a polytope if and only if it is a bounded polyhedron or, equivalently, a polyhedron with a trivial recession cone. 

\pedro[inline]{
   Maybe we should gather what we need about faces and vertices of polyhedra right here.
}
\daniel[inline]{I'm not so sure about this.  At the moment, I feel like it is less distracting to just remind the reader of something (beyond the absolute basic definitions already covered) at the time they are used.  But, I could be convinced otherwise}

The \emph{relative interior} of a subset $\mathcal{U}$ of $\RR^n$, denoted $\ri \mathcal{U}$, is its interior relative to the smallest affine subset of $\RR^n$ containing $\mathcal{U}$.
\pedro{
   Is there a more concrete characterization for polyhedra/polytopes?
   E.g., points not in any proper face? Positive convex combinations of vertices?
}
\daniel{Yep!  Points not in any proper face.  If $S$ is a finite set with $\mathcal{P} = \conv(S)$, then $\ri \mathcal{P}$  consists of all points of the form $\sum_{\vv{s} \in S} \lambda_{\vv{s}} \vv{s}$ where the coefficients $\lambda_{\vv{s}}$ are positive, and sum to $1$.  So in particular, you could take $S$ to be the vertex set of $\mathcal{P}$.  Similarly, if $S$ is finite and $\mathcal{P} = \cone(S)$, then $\ri \mathcal{P}$ has a similar description, but we don't require that the coefficients sum to $1$. But, as I mentioned above, I'm not sure whether it is better to gather things here, or just mention them as we go along.}
For later use, we observe that, when restricted to convex sets, the relative interior operator commutes with Minkowski sums: if $\mathcal{U}$ and $\mathcal{V}$ are convex subsets of $\RR^n$, then $\ri(\mathcal{U}+\mathcal{V})=\ri \mathcal{U}+\ri \mathcal{V}$.

\begin{proposition}  
   \label{bounded polytope: P}
   Let $\vv{a}$ and $\vv{c}$ be points in $\RR^n$, and suppose that $\vv{a}$ has positive coordinates.
   If $\alpha$ is any real number, then the polyhedron consisting of all points $\vv{b} \in \RR^n$ such that  $\vv{b} \le \vv{c} \text{ and } \iprod{\vv{a}}{\vv{b}} \geq \alpha$ is bounded.
\end{proposition}

\begin{proof}
   It suffices to show that the given set is bounded from below.
   For each $\vv{b}$ in that set and each $i$ we have $\vv{b}\le \vv{c} + (b_i - c_i)\canvec_i$.
   As $\vv{a}$ has positive coordinates, $\alpha\le \iprod{\vv{a}}{\vv{b}}\le \iprod{\vv{a}}{\vv{c} + (b_i -c_i)\canvec_i} = 
  \iprod{\vv{a}}{\vv{c}} + a_i(b_i - c_i)$, so $b_i\ge (\alpha + a_ic_i - \iprod{\vv{a}}{\vv{c}})/a_i$.
\end{proof}

We conclude this subsection with a useful technical result.
Though variations of this proposition are well known, we include a simple proof, for lack of an appropriate reference.

\begin{proposition}  
\label{vertex: P}
Let $M$ be an $m \times n$ matrix and let $\vv{b} \in \RR^m$ be a point contained in the cone generated by the columns of $M$.  If $\Q$ is the polyhedron in $\RR^n$  consisting of all points $\vv{t}$ with $\vv{t} \geq \vv{0}$ and $M \vv{t} = \vv{b}$, then a point $\vv{t}^{\ast} \in \Q$ is a vertex of $\Q$ if and only if the columns of $M$ corresponding to the nonzero coordinates of $\vv{t}^{\ast}$ are linearly independent.  %In particular, $\Q$ contains a vertex.
\end{proposition}

\begin{proof}
   The fact that $\vv{b}$ lies in the cone generated by the columns of $M$ implies that $\Q$ is nonempty.
   Fix a point $\vv{t}^{\ast} \in \Q$.
   Before proceeding, recall that $\vv{t}^{\ast}$ is a vertex of $\Q$ if and only if an expression of $\vv{t}^{\ast}$ as a convex combination of points $\vv{r}$ and $\vv{s}$ in $\Q$ is only possible when $\vv{r}=\vv{s}=\vv{t}^{\ast}$.

   First, assume that the columns of $M$ corresponding to the nonzero coordinates of $\vv{t}^{\ast}$ are linearly independent, and suppose that $\vv{t}^{\ast} = \lambda \vv{r} + \mu \vv{s}$ is a convex combination of points $\vv{r}, \vv{s} \in \Q$.
   Since $\vv{r},\vv{s}\ge \vv{0}$, the $i$-th coordinate of $\vv{r}$ and of $\vv{s}$ are zero whenever the $i$-th coordinate of $\vv{t}^{\ast}$ is zero.
   On the other hand, the fact that $\vv{r}$ and $\vv{s}$ lie in $\Q$ also implies that
   \[ M \vv{t}^{\ast} = \vv{b} = M \vv{r} = M \vv{s}, \]
   and the assumption that the columns of $M$ corresponding to the nonzero coordinates of $\vv{t}^{\ast}$ are linearly independent then implies that $\vv{r}=\vv{s}=\vv{t}^{\ast}$.

Next, suppose that the columns of $M$ corresponding to the nonzero coordinates of $\vv{t}^{\ast}$ are linearly dependent.   In this case, we may fix a nonzero point $\vv{k} \in \RR^n$ with the property that $M \vv{k} = \vv{0}$, and such that the $i$-th coordinate of $\vv{k}$ is zero whenever the $i$-th coordinate of $\vv{t}^{\ast}$ is zero.  We claim that if $\varepsilon > 0$ is sufficiently small, then the points $\vv{t}^{\ast} \pm \varepsilon \vv{k}$ must lie in $\Q$.   As $\vv{t}^{\ast}$ is a convex combination of these points, it will then follow that $\vv{t}^{\ast}$ is not a vertex of $\Q$.  Towards the claim, note that $M(\vv{t}^{\ast} \pm \varepsilon \vv{k}) = M \vv{t}^{\ast} = \vv{b}$ for every $\varepsilon > 0$.  On the other hand, the condition relating the coordinates of $\vv{t}^{\ast}$ and $\vv{k}$ guarantees that $\vv{t}^{\ast} \pm \varepsilon \vv{k}$ is nonnegative for all $0 < \varepsilon \ll 1$.  
\end{proof}

\pedro[inline]{
   An alternative to the previous proposition is the following result, which we could just mention and give a reference (it appears in several books):

   \begin{proposition}
      Let $\mathcal{P}$ be the polyhedron defined by a system of inequalities $A \vv{x} \le \vv{b}$, where $A\in \RR^{m\times n}$, and $\vv{v}$ a vertex of $\mathcal{P}$.
      Then there exists $I \subseteq \{1,\ldots,m\}$ such that $\vv{v}$ is the unique solution to the system $A_I \vv{x} = \vv{b}_I$, where $A_I$ and $\vv{b}_I$ are obtained by selecting the $i$-th rows of $A$ and $\vv{b}$, for each $i\in I$.
   \end{proposition}

   Then \Cref{uniform denominators for vertices:  T} can be approached as follows:
   By \Cref{opt set: C}, $\opt \LP(A,\vv{u})$ is defined by $A\vv{s} \le \vv{u}$ and $\vv{s}\ge \vv{0}$, with equality in some specific coordinates, and thus defined by a system of inequalities $B \vv{x} \le \vv{b}$, where $\vv{b}$ is an integral vector and $B$ is a submatrix of the matrix $M$ obtained by stacking $A$, $-A$, the identity matrix $I_n$, and $-I_n$.
   Let $\denom$ be the least common multiple of the nonzero minors of $M$; then by the above result, every vertex of $\opt \LP$ is rational, with denominator $\denom$.

   \bigskip

   Hope this makes sense; if so, then I think this argument is slightly simpler, avoiding the linear bijection business.
}

\subsection{Monomial matrices}  A \emph{monomial matrix} is a matrix over $\ZZ$ with nonnegative, nonzero rows and columns.   If $A$ is a $d \times n$ monomial matrix, then we call $\ZZ^n$ the \emph{domain lattice}, and $\ZZ^d$ the \emph{range lattice}, of $A$.

%A  monomial matrix $B$ is a \emph{successor} of $A$ if $B$ can be obtained from $A$ by omitting some (possibly empty) subset of its rows.   The most relevant example of a successor is the collapse construction considered later in \Cref{newton-polyhedra: S}.  A successor $B$ of $A$ is  \emph{proper} if $B \neq A$.

\pedro[inline]{
   Maybe we could eliminate this subsection, and move the definition of monomial matrix to the section on Newton polyhedra.

   And maybe we could move Newton polyhedra here to this Preliminaries section, as a subsection. 

   Let me know your thoughts.
}
\daniel[inline]{I removed the stuff on \emph{successors}.  I agree the monomial matrix stuff should appear immediately before the definition of Newton polyhedra.  About moving Newton stuff to the preliminaries:  This seems reasonable, but I would at least want to emphasize (in the prose) that collapsing seems to be a new idea (unlike the rest of the preliminaries, which is much more established).  Collapsing seems to first appear in my paper with Emily;  I searched for a while, and couldn't find that construction anywhere else.}

\emily[inline]{Maybe we should introduce collapsing right before it is used, to emphasize that it is a new idea?}
\pedro[inline]{
   I like this idea.
   We could just put monomial matrices and the basics of Newton polyhedra here, and postpone collapsing.
   We can gather the results on collapsing in a subsection at the end of the current Section~4 or, better, have a whole section on collapsing.
}


\subsection{Linear programming}  

Let $\mathbb{D}$ be either $\RR$ or $\ZZ$.
A \emph{linear program} $\Pi$ in $\mathbb{D}^n$ is an optimization problem in which one seeks to maximize a fixed linear \emph{objective function} $\RR^n \to \RR$ on a subset of $\mathbb{D}^n$ defined by a fixed system of linear inequalities.
We refer to this subset as the \emph{feasible set} of $\Pi$, and denote it $\feas \Pi$, and we refer to the inequalities defining it as the \emph{constraints} of $\Pi$.
We say that the points of $\feas \Pi$ are \emph{feasible for $\Pi$}.
When $\mathbb{D} = \ZZ$, we refer to $\Pi$ as an \emph{integer linear program}, or simply \emph{integer program}, for short.  

If $\mathbb{D} = \RR$, then the feasible set is a polyhedron in $\RR^n$, and if $\mathbb{D} = \ZZ$, the feasible set is the set of lattice points in a polyhedron in $\RR^n$.  

In this article, we will only consider linear programs in which the objective function restricted to the feasible set attains a maximum (\eg this occurs whenever the constraints define a polytope).
In this case, a feasible point is \emph{optimal} if it maximizes the objective function, and the optimal value obtained by this function is called the \emph{value} of the program.
We use $\opt \Pi$ to denote optimal set of the linear program $\Pi$, and $\val \Pi$ to denote the value of $\Pi$. 

There are clearly multiple reasonable notions of equality for integer programs.
In this article,  we say that two integer programs are \emph{equal} if their objective functions and defining constraints are identical, and \emph{equivalent} if their objective functions are identical and their feasible sets agree. 

\subsection{Multinomial coefficients}

If $k$ is a nonnegative integer and $\vv{u} = (u_1,\ldots,u_n) \in \NN^n$ is a point with $\norm{\vv{u}} = k$, then the \emph{multinomial coefficient} $\binom{k}{u_1,\ldots,u_n}$, or $\binom{k}{\vv{u}}$ for short, is defined as follows: 
\[
   \binom{k}{\vv{u}} = \binom{k}{u_1,\ldots,u_n} \coloneqq \frac{k!}{u_1!\cdots u_n!}.
\]
If $\norm{\vv{u}} \ne k$, on the other hand, we set $\binom{k}{\vv{u}}=0$.
	
\begin{theorem}[\cite{dickson.multinomial}]
   \label{thm: dickson}
   Let $p$ be a prime integer, $k\in \NN$, and $\vv{u} \in \NN^n$.
   Write the terminating base $p$ expansions of $k$ and $\vv{u}$ as follows\textup:
   \begin{equation*}
      k = k_0+k_1p+k_2p^2+\cdots+k_rp^r\quad \text{and} \quad \vv{u}=\vv{u}_0+\vv{u}_1p+\vv{u}_2p^2+\cdots+\vv{u}_rp^r,
   \end{equation*}
   where $0\le k_i < p$ and $\vv{0}\le\vv{u}_i < p \cdot \vv{1}$ for each $i$.
   \textup{(}Note that it is possible that $k_r = 0$ or $\vv{u}_r = \vv{0}$.\textup{)}
   Then
   \[
      \binom{k}{\vv{u}}\equiv \binom{k_0}{\vv{u}_0}\binom{k_1}{\vv{u}_1}\cdots \binom{k_r}{\vv{u}_r} \mod{p}.
   \]
   In particular, $\binom{k}{\vv{u}}\not\equiv 0\bmod{p}$ if and only if $\norm{\vv{u}_i}=k_i$ for each $i$, that is, the components of $\vv{u}$ add up to $k$ without carrying \textup(base $p$\textup).
\qed
\end{theorem}

\begin{corollary}
   \label{cor: multinomial congruence}
   Let $k,l,e\in \NN$, with $l<p^e$, and $\vv{u},\vv{v}\in \NN^n$, with $\vv{v}<p^e\cdot \vv{1}$.
   Then
   \[
      \pushQED{\qed}
      \binom{kp^e+l}{\vv{u}p^e+\vv{v}}\equiv \binom{k}{\vv{u}}\binom{l}{\vv{v}} \mod{p}.\qedhere
      \popQED
   \]
\end{corollary}

\newpage

\section{Frobenius powers and critical exponents}

In this section, we recall the definition and basic properties of (generalized) Frobenius powers and critical exponents, as introduced in \cite{hernandez+etal.frobenius_powers}.
Let $R$ be a regular domain of characteristic $p > 0$, and let $\ideala$ be an ideal of $R$.
If $q$ is a power of $p$, then $\ideala^{[q]}$ denotes the standard $q$-th Frobenius power of $\ideala$, that is, the ideal generated by the $q$-th powers of the elements of $\ideala$.
Given a nonnegative integer $k$, with base $p$ expansion $k = k_0 + k_1 p + \cdots + k_r p^r$, the $k$-th Frobenius power of $\ideala$ is the ideal
\[\ideala^{[k]} \coloneqq \ideala^{k_0}\big(\ideala^{k_1}\big)^{[p]}\cdots \big(\ideala^{k_r}\big)^{[p^r]}.\]
More relevant to this article, though, is the description of $\ideala^{[k]}$ in terms of generators of $\ideala$: if $\ideala = \ideal{f_1,\ldots,f_n}$, then $\ideala^{[k]}$ is the ideal generated by the products $f_1^{a_1}\cdots f_n^{a_n}$, where $(a_1,\ldots,a_n)$ ranges over all $n$-tuples of nonnegative integers with $\binom{k}{a_1,\ldots,a_n}\not\equiv 0\bmod{p}$ \cite[Proposition~3.5]{hernandez+etal.frobenius_powers}.

Frobenius powers are extended to allow nonnegative real exponents, through the use of the Frobenius ``roots'' introduced in \cite{blickle+mustata+smith.discr_rat_FPTs}.
Explicitly, for a nonnegative rational exponent of the form $k/p^e$, we define
\[\ideala^{[k/p^e]} \coloneqq \big(\ideala^{[k]}\big)^{[1/p^e]},\]
and for an arbitrary positive real number $t$, we define $\ideala^{[t]}$ by taking approximations of $t$ from above by such rational numbers, in a way analogous to the definition of test ideals in \loccit
\pedro{Make this more explicit?}

Like test ideals and multiplier ideals, as $t$ varies, the Frobenius powers $\ideala^{[t]}$ form a nonincreasing chain, and are right-constant for positive $t$, \ie $\ideala^{[t+\epsilon]} = \ideala^{[t]}$, for $0<\epsilon \ll 1$.
The positive exponents where $\ideala^{[t]}$ ``jumps'' (that is, $\ideala^{[t-\epsilon]}\ne \ideala^{[t]}$, for all $0<\epsilon \le t$) are called the \emph{critical exponents} of $\ideala$.
These are the analogues of the jumping numbers of multiplier ideals, and of the $F$-jumping exponents of test ideals, and like their counterparts, they form a discrete set of rational numbers \cite[Corollary~5.8]{hernandez+etal.frobenius_powers}.

If $\ideala$ and $\idealb$ are nonzero proper ideals of $R$, with $\ideala \subseteq \sqrt\idealb$, the \emph{critical exponent of $\ideala$ with respect to $\idealb$} is the number
\begin{equation}\label{eq: defn of crit(a,b)}
   \crit(\ideala,\idealb) \coloneqq \min\big\{t\in \RRnn: \ideala^{[t]} \subseteq \idealb\big\}
      = \sup\big\{t\in \RRnn: \ideala^{[t]} \not\subseteq \idealb\big\}.
\end{equation}
This is indeed a critical exponent of $\ideala$, and moreover, every critical exponent $\lambda$ of $\ideala$ is of this form, for some $\idealb$ (take, for instance, $\idealb = \ideala^{[\lambda]}$).

We now describe a more explicit realization of the critical exponents of an ideal, which is central to this paper.
With $\ideala$ and $\idealb$ as above, given a nonnegative integer $e$, we set
\[\mu(\ideala,\idealb,p^e) \coloneqq \max\big\{k\in \NN : \ideala^{[k]} \not\subseteq \idealb^{[p^e]}\big\}.\]
Then $\big(\mu(\ideala,\idealb,p^e)/p^e\big)_e$ is a nondecreasing bounded sequence, and 
\begin{equation}\label{eq: crit as a limit of mus}
   \crit(\ideala,\idealb) = \lim_{e\to \infty} \frac{\mu(\ideala,\idealb,p^e)}{p^e} = \sup_{e\in \NN} \frac{\mu(\ideala,\idealb,p^e)}{p^e}.
\end{equation}
The $\mu(\ideala,\idealb,p^e)$ not only determine $\crit(\ideala,\idealb)$, but can also be recovered from $\crit(\ideala,\idealb)$, via truncations:
\begin{equation}\label{eq: recovering mus from crit}
   \mu(\ideala,\idealb,p^e) = \up{p^e\crit(\ideala,\idealb)} - 1.
\end{equation}

Before moving forward, we observe that the notions introduced in the last two paragraphs run parallel to the theory of $F$-thresholds.
With $\ideala$ and $\idealb$ as above, the \emph{$F$-threshold of $\ideala$ with respect to $\idealb$}, denoted $\ft{\ideala}{\idealb}$, is defined as in \eqref{eq: defn of crit(a,b)}, replacing the Frobenius power $\ideala^{[t]}$ with the test ideal $\tau(\ideala^t)$.
There is an explicit description for $\ft{\ideala}{\idealb}$ analogous to \eqref{eq: crit as a limit of mus}, where $\mu(\ideala,\idealb,p^e)$ is replaced with
\[\nu(\ideala,\idealb,p^e) \coloneqq \max\big\{k\in \NN : \ideala^{k} \not\subseteq \idealb^{[p^e]}\big\}.\]
However, there is no analogue to \eqref{eq: recovering mus from crit}, unless $\ideala$ is a principal ideal.

\subsection{Frobenius powers and critical exponents of monomial ideals}

We now introduce some notation and gather some basic results concerning Frobenius powers and critical exponents of monomial ideals.
We work in a polynomial ring over a field of characteristic $p>0$, in the variables $x=x_1,\ldots,x_d$.

\begin{notation}
   We adopt standard multi-index notation: if $\vv{u} = (u_1,\ldots,u_d)\in \NN^d$, then $x^\vv{u} = x_1^{u_1}\cdots x_d^{u_d}$.   If $\vv{u}\in \NN^d$, then $\diag(\vv{u})$ denotes the \emph{diagonal ideal} 
   \[ \diag(\vv{u}) = \ideal{x_1^{u_1},\ldots,x_d^{u_d}} = \ideal{x^{\vv{v}} : \vv{v} \in \NN^d \text{ and } \vv{v} \not < \vv{u}}.\]
   %\daniel{I added another easy description that I think we use often}
   When dealing with notation involving diagonal ideals, we shall typically replace $\diag(\vv{u})$ in the notation with $\vv{u}$.
   For instance, $\crit(\ideala,\diag(\vv{u}))$ will be simply denoted $\crit(\ideala,\vv{u})$.   
   Likewise, we shall often replace a monomial ideal with its exponent matrix in our notation.
%   For instance, if $\ideala$ is a monomial ideal of $R$, and $A$ is its exponent matrix, then $\crit(\ideala,\vv{u})$ will also be denoted $\crit(A,\vv{u})$.
\end{notation}

\begin{remark}\label{rmk: Frobenius powers of monomial ideals are monomial ideals}
   Since integral Frobenius powers and Frobenius roots of monomial ideals are themselves monomial ideals, the same is true for arbitrary Frobenius powers of monomial ideals. 
\end{remark}

\begin{proposition}\label{prop: description of frobenius powers in terms of crits}
   If $\ideala$ is a monomial ideal, then
   \[ x^{\vv{v}} \in \ideala^{[t]} \iff \ideala^{[t]} \not \subseteq \diag(\vv{v}+\vv{1}) \iff \crit(\ideala, \vv{v}+\vv{1}) > t.\]
   Consequently, $\ideala^{[t]} = \ideal{x^{\vv{v}} : \crit(\ideala, \vv{v}+\vv{1}) > t}$.
\end{proposition}

\begin{proof}
   The second equivalence follows immediately from \eqref{eq: defn of crit(a,b)}.
   As for the first, the forward implication is trivial, since $x^\vv{v} \notin\diag(\vv{v}+\vv{1})$, and conversely, if $\ideala^{[t]} \not \subseteq \diag(\vv{v}+\vv{1})$, then there exists $x^\vv{u} \in \ideala^{[t]}$ with $\vv{u} \le \vv{v}$, so $x^\vv{v}\in \ideal{x^\vv{u}} \subseteq \ideala^{[t]}$.
   The final conclusion holds because $\ideala^{[t]}$ is a monomial ideal, as noted in \Cref{rmk: Frobenius powers of monomial ideals are monomial ideals}.
\end{proof}

\begin{corollary}
   If $\ideala$ is a monomial ideal, then every critical exponent of~$\ideala$ is of the form $\crit(\ideala,\vv{u})$, for some $\vv{u} > \vv{0}$ in $\NN^d$.
\end{corollary}

\begin{proof}
   Let $\lambda$ be a critical exponent of $\ideala$, so that $\ideala^{[\lambda]}$ is properly contained in $\ideala^{[t]}$ for every $0 \le t <\lambda$.
   Since the critical exponents of $\ideala$ form a discrete set \cite[Corollary~5.8]{hernandez+etal.frobenius_powers}, the intersection of all such $\ideala^{[t]}$ properly contains $\ideala^{[\lambda]}$.
   This intersection---a monomial ideal by \Cref{rmk: Frobenius powers of monomial ideals are monomial ideals}---thus contains a monomial $x^\vv{v}$ not in $\ideala^{[\lambda]}$.
   \cref{prop: description of frobenius powers in terms of crits} then shows that $\ideala^{[t]}\not\subseteq \diag(\vv{v}+\vv{1})$, whenever $0\le t < \lambda$, but $\ideala^{[\lambda]} \subseteq \diag(\vv{v}+\vv{1})$, hence $\lambda = \crit(\ideala,\vv{v}+\vv{1})$.
\end{proof}

\daniel[inline]{
% Suppose $\ideala$ is a monomial ideal.  Then all of its Frobenius powers are also monomial ideals, and \[ x^{\vv{v}} \in \ideala^{[\lambda]} \iff \ideala^{[\lambda]} \not \subseteq \diag(\vv{v}+\vv{1}) \iff \crit(\ideala, \vv{v}+\vv{1}) > \lambda.\]  The first $\iff$ above is Proposition 2.5 from our examples paper.  Therefore, 
% \[ \ideala^{[\lambda]} = \langle x^{\vv{v}} : \crit(\ideala, \vv{v}+\vv{1}) > \lambda \rangle.\]
   % This tells us that every critical value of $\ideala$ is of the form $\crit(\ideala, \vv{u})$ with $\vv{u} > \vv{0}$.
   % Whatever we include in this section should be enough to at least justify this observation.
   When we look for corollaries, our ability to compute $\crit(\ideala, \vv{u})$ with $\vv{u} > \vv{1}$ will show that the ideals $\ideala^{[\lambda]}$ also vary ``uniformly'' with respect to the class of $p$ modulo some denominator $\denom$, in a way that we can make precise.
}

\daniel[inline]{What about adding a connection between $\ideala^{[t]}$ and $\tau(\ideala^{t'})$ where $t'$ is the greatest $F$-threshold less than $t$?  Like in the diagonal section of our examples paper?  Maybe we could wait to do this until we compare Frobenius powers and test ideals.}

\section{Newton polyhedra and collapsing}
\label{newton-polyhedra: S}

\comment[inline]{This section needs to be paraphrased, to avoid overlap with other paper}
\daniel[inline]{One thing we can do is change the presentation so that it is more informal.  We can include some general ideas, to make it intuitive, but can refer to the other paper for technical details that are obvious, but perhaps annoying to check.}

\subsection{Faces of Newton polyhedra}

The \emph{Newton polyhedron} associated to a monomial matrix $A$ with $d$ rows is the polyhedron in $\RR^d$ given by 
\[ \N = \conv( \col(A) ) + \cone( \vv{e}_1, \ldots, \vv{e}_d), \]
where $\col(A)$ is the set of columns of $A$.  

Recall that a proper subset $\O$ of $\N$ is a \emph{face} of $\N$ if there exists $\vv{a} \in \RR^d$ and $\alpha \in \RR$ are such that $\iprod{\vv{a}}{\vv{c}} \geq \alpha$ for all $\vv{c} \in \N$, with equality if and only if $\vv{c} \in \O$.
We say that such a point $\vv{a}$ \emph{defines} $\O$ in $\N$.
In this article, we will be concerned with faces $\O$ that do not lie in any coordinate subspace of $\RR^d$, which we call \emph{standard}.
\textbf{When considering a point that defines a standard face, we will always assume that we have scaled the point so that $\bm{\alpha = 1}$}.
\pedro{
   Trying to emphasize this, as it's essential to understand some computations that will follow.
   Maybe we should even use a ``Convention'' environment, and refer to it a couple of times, to remind the reader.
}

\begin{remark}\label{rmk: inner product with columns of A}
   In view of the above convention, if $\vv{a}$ defines a standard face $\O$ of the Newton polyhedron of $A$ and $\vv{s} \in \RR^n$ then $\iprod{\vv{a}}{A\vv{s}} \ge \norm{\vv{s}}$, and if $\vv{s}\ge \vv{0}$, then equality holds if and only if $s_i = 0$ whenever the $i$-th column of $A$ is not in $\O$.
\end{remark}

The following proposition is well known to experts, but we include the short proof to keep the article self-contained.

\begin{proposition} 
   \label{face: P}
   If $\vv{a} \in \RR^d$ defines a face $\O$ of a Newton polyhedron $\N$, then $\vv{a}$ is nonnegative, and the $i$-th coordinate of $\vv{a}$ is zero if and only if $\vv{u} + \lambda \vv{e}_i \in \O$  for every $\vv{u} \in \O$ and $\lambda > 0$.
   In particular, the supporting indices of $\vv{a}$ depend only on $\O$, and $\O$ is bounded if and only if $\vv{a}$ is positive. 
\end{proposition}

\daniel{I changed $\vv{c} \in \O$ to $\vv{u} \in \O$.  I think that the convention of these notes (at least, this is how it is in my head) is that letters lower in the alphabet (e.g., $\vv{a}, \vv{b}, \vv{c}$) represent points in the dual space to $\RR^d$ (that is, determine the linear function $\iprod{\vv{a}}{\cdot}$) and points higher in the alphabet $(e.g., \vv{u}, \vv{v}, \vv{w})$ represent points in $\RR^d$ itself}
\pedro{That's a useful convention. I'll keep it in mind.}
\begin{proof}
   If $\vv{u} \in \O$, then adding to $\vv{u}$ any nonnegative point in $\RR^d$ produces a point in $\N$.
   In particular, if $\iprod{\vv{a}}{\vv{u}} = \alpha$, then $\iprod{\vv{a}}{\vv{u} + \lambda \vv{e}_i} \geq \alpha$ for every standard basis vector $\vv{e}_i$ in $\RR^d$ and $\lambda > 0$.
   This observation implies that $a_i = \iprod{\vv{a}}{\canvec_i} \ge 0$ for each $i$, so $\vv{a} \geq \vv{0}$, and that $\vv{u} + \lambda \vv{e}_i \in \O$ for every $\lambda > 0$ if and only if $\iprod{\vv{a}}{\vv{e}_i} = 0$.  

Similar logic will show that if $\rb(\O) \coloneqq  \{ \vv{e}_i \in \RR^d : \iprod{\vv{a}}{\vv{e}_i} = 0\}$, then 
\begin{equation}
\label{face: e}
\O =  \conv( \col(A) \cap \O ) + \cone(\rb(\O))
\end{equation}
where we agree that the $\cone(\emptyset) = \{\vv{0}\}$.  We see from this that $\O$ is bounded if and only if $\rb(\O)$ is empty, which is equivalent to the third assertion.  
\end{proof}

\begin{definition}
   If $\vv{a} \in \RR^d$ defines $\O$, then the \emph{recession basis} of $\O$ is the set $\rb(\O)$ of all standard basis vectors $\vv{e}_i$ in $\RR^d$ such that the $i$-th coordinate of $\vv{a}$ is zero, and the \emph{recession subspace} of $\O$ is the subspace $\rs(\O)$ of $\RR^d$ spanned by $\rb(\O)$.  
\end{definition}

As noted above, these definitions depend only on $\O$, but not on the choice of $\vv{a}$.
In view of the Minkowski--Weyl Theorem (see \Cref{ss: euclidean spaces and convexity}), equation \eqref{face: e} implies that the cone generated by $\rb(\O)$ is the recession cone of $\O$, motivating our choice of terminology.

% Our choice of terminology is motivated by the following observation.

% \begin{remark}  Recall that the \emph{recession cone} of a polyhedron $\mathcal{Q}$ in $\RR^d$ is the set of all directions $\vv{d} \in \RR^d$ in which $\mathcal{Q}$ recedes, that is, $\vv{c} + \lambda \vv{d} \in \mathcal{Q}$ for every $\vv{c} \in \mathcal{Q}$ and $\lambda > 0$.  It is a well-known fact from convex geometry that  \eqref{face: e} implies that the cone generated by $\rb(\O)$ is the recession cone of $\O$.
% \end{remark}

\subsection{Collapses} 

\begin{definition}  
\label{collapse: D}
 Suppose that $\O$ is a proper face of the Newton polyhedron $\N$ in $\RR^d$ associated to a monomial matrix $A$ with $d$ rows.  

\begin{enumerate}
\item The set $\rb(\O)^{\perp}$ is the complement of $\rb(\O)$ in $\{ \vv{e}_1, \ldots, \vv{e}_d \}$, and $\rs(\O)^\perp$ is the subspace of $\RR^d$ spanned by $\rb(\O)^\perp$, that is, the orthogonal complement of $\rs(\O)$ in $\RR^d$.
\item The \emph{collapse} of a subset $X$ of $\RR^d$ along $\O$ is the image of $X$ under the canonical linear projection $\RR^d \longrightarrow \rs(\O)^{\perp}$.
\item The \emph{collapse} of $A$ along $\O$ is the matrix obtained from $A$ by collapsing each of its columns along $\O$.  That is, the collapse of $A$ along $\O$ is the matrix corresponding to the linear transformation  
%
\[ \RR^n \stackrel{A}{\xrightarrow{\hspace*{6mm}}} \RR^d \longrightarrow \rs(\O)^{\perp}.\]  
%
\end{enumerate}
\end{definition}     

\pedro[inline]{
   The way things are defined above, the collapse of $A$ is still a $d\times n$ matrix, where certain rows are filled with zeros (as opposed to having rows deleted).
   That makes perfect sense, but we'll have to do a couple of things to make the remarks and results that follow precise:
   \begin{itemize}
      \item Define ``monomial matrix in a coordinate subspace'' (rows corresponding to standard basis vectors in the coordinate subspace are nonzero; other rows are zero).
      The collapse of $A$ is, thus, a monomial matrix in $\rs(\O)^\perp$.
      \item Define ``Newton polyhedron in a coordinate subspace'' (the recession cone is not $\cone(\canvec_1,\ldots,\canvec_d)$; it's the cone generated by the canonical basis vectors in that coordinate subspace).
   \end{itemize}
}

\daniel[inline]{I am a bit confused by this comment.  The way I see things, the matrix associated to the linear transformation \[ \RR^n \stackrel{A}{\xrightarrow{\hspace*{6mm}}} \RR^d \longrightarrow \rs(\O)^{\perp}\] must be of size $\# \rb(\O) \times n$, as specified by the dimensions of the source and target;  as I am implicitly using the canonical bases of $\RR^n$ and $\RR \rb(\O)$ to turn this transformation into a matrix, we should definitely get the matrix obtained by \emph{omitting} some rows of $A$.  On the other hand, if we instead looked at
   \[ \RR^n \stackrel{A}{\xrightarrow{\hspace*{6mm}}} \RR^d \longrightarrow \rs(\O)^{\perp}  \hookrightarrow \RR^d \]
then we would get a matrix obtained by replacing some rows of $A$ with zero.  
} 
\pedro[inline]{
   Since $\rs(\O)^\perp$ is contained in $\RR^d$ and bases were not mentioned, I interpreted ``matrix of the linear transformation'' as ``matrix with respect to the canonical bases of $\RR^n$ and $\RR^d$.''
   But now I understand your interpretation: that $\collapse{A}$ is the matrix of $\RR^n \to \rs(\O)^\perp$ \emph{with respect to the standard basis of $\RR^n$ and the basis $\rb(\O)^\perp$ of $\rs(\O)^\perp$.}

   \medskip

   Under your interpretation, multiplication by $\collapse{A}$ returns ``coordinate vectors''; under mine, it returns ``the
   real thing''; for this reason, I think I prefer mine.
   For instance, under your interpretation, $\collapse{A} \vv{k}$ (an element of $\RR^{\# \rb(\O)^\perp}$) is not the same as $\collapse{A\vv{k}}$ (an element of $\RR^d$); instead, $\collapse{A} \vv{k}$ is the coordinate vector of $\collapse{A\vv{k}}$ with respect to the basis $\rb(\O)^\perp$. 
}
\daniel[inline]{I've always thought of $\rs(\O)^{\perp}$ as $\operatorname{span}_{\RR} \rb(\O)^{\perp}$, and not an abstract $\RR^{\# \rb(\O)^{\perp}}$, and given that $\collapse{A} = [ \collapse{\vv{a}_1} \cdots \collapse{\vv{a}_n} ]$, I've interpreted the right-hand side of $\collapse{A \vv{k}} = \collapse{A} \vv{k}$ as $\sum \collapse{\vv{a}_i} k_i$.  But, I see your point about the RHS being a coordinate vector.  Maybe we should think about polyhedra defined by linear transformations instead.  Regardless, keeping a row of zeros in the collapse is odd to me geometrically;  even when we are drawing collapsed Newton polyhedra, we aren't drawing them in the larger Euclidean space.  Similarly, when we collapse a monomial ideal, we are thinking about it as a monomial ideal in a polynomial ring in fewer variables, all of which appear as generators for the collapsed ideal.  So, my strong preference is to omit rows, even if it means abusing some  notation, or implicitly identifying a coordinate vector with its representation in terms of a given basis, especially since every vector space in sight has a canonical basis.}
\pedro[inline]{
   My preference would be to keep zeros, and avoid abuse of notation.
   I don't really see a conflict between keeping zeros versus drawing things in a smaller space.
   (If we have an object that lives in, say, $\RR^3$, but happens to be contained in some coordinate plane, it's natural to draw that object in the plane, instead of 3-space; we'd just say that we are only showing that subset of $\RR^3$\ldots)
   Nor do I see a problem with thinking of a monomial ideal in $x$ and $y$ as an ideal in $\Bbbk[x,y,z]$.

   \medskip
   
   If we go the other way, though, we should say right away that we will use coordinates with respect to the basis\footnote{Ordering the basis vectors, say, the natural way: by increasing indices.} $\rb(\O)^\perp$  to identify $\rs(\O)^\perp$ with $\RR^{\#\rb(\O)^\perp}$.
   OR, perhaps better, leave $\rs(\O)^\perp$ alone, and just define collapse as the composition
   \[\RR^d \stackrel{\text{proj}}{\longrightarrow} \rs(\O)^\perp \stackrel{\approx}{\longrightarrow} \RR^{\#\rb(\O)^\perp}.\]
}
  
\daniel[inline]{I should also add that there are some advantages to thinking about $\collapse{A}$ is obtained from $A$ by omitting columns, as opposed to replacing them with zero.  For instance, this interpretation is key in \Cref{aux program: D}, when defining the constraints of $\ip$.  If we instead replaced rows with zeros, then the modified definition of $\ip$ would be more involved.}
\pedro[inline]{
   In a way, we are already getting around these issues, by saying things like ``$\vv{a}$ is positive in $\rs(\O)$'' (meaning $a_i$ is positive whenever $\canvec_i \in \rb(\O)$).
   I think from here, it's just a small step to saying things like ``$\vv{a} < \vv{b}$ in $\rs(\O)$'' (meaning $\vv{b} - \vv{a}$ is positive in $\rs(\O)$).
}


\begin{remark}  The assumption that $\O \neq \N$ implies that $\rb(\O)^{\perp} \neq \emptyset$.
\end{remark}

Below, we adopt the notation established in \Cref{collapse: D}.
Furthermore,  $\collapse{A}$ (respectively, $\collapse{X}$) denotes the collapse of $A$ (respectively, a set $X$) along $\O$.


\begin{remark}
\label{collapse of a defining vector: R}
If $\vv{a} \in \RR^d$ defines $\O$ in $\N$, then the standard basis vectors in $\rb(\O)^{\perp}$ correspond to the nonzero coordinates of $\vv{a}$.  Consequently, $\iprod{\vv{a}}{\vv{c}} = \iprod{\collapse{\vv{a}}}{\collapse{\vv{c}}}$ for every $\vv{c} \in \RR^d$.
\pedro{May add that $\collapse{\vv{a}} = \vv{a}$}
\daniel{I guess this depends on what we decide above.}
\pedro{I think this is true regardless. Since $\vv{a} \in \rs(\O)^\perp$, it is fixed by the projection $\RR^d \to \rs(\O)^\perp$.}
\end{remark}

\begin{remark}
\label{collapse of monomial is monomial: R}
The collapsed matrix $\collapse{A}$ is monomial.  Indeed, each row of $\collapse{A}$ is a row of $A$, and hence is nonzero.   On the other hand, if $\vv{a} \in \RR^d$ defines $\O$, then \Cref{collapse of a defining vector: R} implies that the inner product of $\collapse{\vv{a}}$ with every column of $\collapse{A}$ is at least one.  In particular, the columns of $\collapse{A}$ are nonzero.
\end{remark}

\begin{proposition}\label{collapse of Newton polyhedron: P}
   If $\M$ is the Newton polyhedron in the coordinate subspace $\rs(\O)^{\perp}$ associated to the monomial matrix $\collapse{A}$, then $\collapse{\O}$ is a bounded face of $\M = \collapse{\N}$.
   In addition, if $\vv{a} \in \RR^d$ defines $\O$ in $\N$, then $ \collapse{\vv{a}}$ defines $\collapse{\O}$ in $\M$. 
\end{proposition}

\begin{proof}
By definition, the Newton polyhedron $\M$ equals
%
\[  \conv( \col(\collapse{A}) ) + \cone(\rb(\O)^{\perp}) =  \ol{\conv( \col(A))} + \ol{\cone(\vv{e}_1, \ldots, \vv{e}_d)} =  \collapse{\N}.\]

Given \Cref{collapse of a defining vector: R}, it is not difficult to verify that $\collapse{\vv{a}}$ defines $\collapse{\O}$ in $\M$ whenever $\vv{a} \in \RR^d$ defines $\O$ in $\N$.  The positivity of $\collapse{\vv{a}}$ in $\rs(\O)^{\perp}$ then implies that $\collapse{\O}$ is bounded.  Alternatively, one may project \eqref{face: e} to $\rs(\O)^{\perp}$ to see that the collapsed face $\collapse{\O}$ is the polytope $\conv( \collapse{ \col(A) \cap \O}) = \conv( \col(\collapse{A}) \cap \collapse{\O})$.  
\end{proof}

\newpage

\section{Linear programs associated to monomial pairs}

\begin{definition}
A \emph{monomial pair} $(A, \vv{u})$ consists of a monomial matrix $A$ and a positive point $\vv{u}$ in the range lattice of $A$.
\end{definition}

\begin{definition}
Given a monomial pair $(A,\vv{u})$, where $A$ is a matrix with $n$ columns, $\LP(A, \vv{u})$ is the linear program in $\RR^n$ defined as follows:
\begin{enumerate}
\item The constraints are $\vv{k} \geq \vv{0}$ and $A \vv{k} \leq \vv{u}$.
\item The objective function is $\vv{k} \mapsto \norm{\vv{k}}$.
\end{enumerate}
\end{definition}

\comment[inline]{The objective of this section is to connect this program to $F$-thresholds, and to study the optimal set of this linear program.  }

It is easy to see that the feasible set of the linear program $\LP(A,\vv{u})$ is bounded, and therefore a polytope in $\RR^n$.
Consequently, $\LP(A,\vv{u})$ has a well-defined value.

\subsection{Relations with $F$-thresholds}
\label{opt sets: SS}

\begin{definition}
   The \emph{$F$-threshold} of a monomial pair $(A, \vv{u})$, denoted $\ft{A}{\vv{u}}$, is the unique positive real number $\lambda$ with the property that $(1/\lambda)  \cdot \vv{u}$ lies in the boundary of the Newton polyhedron of $A$.  
\end{definition}

\begin{definition}
   The \emph{minimal face} of a monomial pair $(A, \vv{u})$ is the face $\mf(A, \vv{u})$ of the Newton polyhedron of $A$ that is minimal, with respect to inclusion, among the faces of $\N$ containing the scaled point $(1/\lambda) \cdot \vv{u}$, where $\lambda = \ft{A}{\vv{u}}$.\footnote{Recall that the intersection of two faces of $\N$ is also a face of $\N$. Thus, as minimality here is with respect to inclusion, it follows that there is a unique such minimal face.}
\end{definition}

The positivity of $\vv{u}$ implies that $\ft{A}{\vv{u}}$ is well-defined, and that $\mf(A, \vv{u})$ is a standard face of the Newton polyhedron of $A$.
Our choice of notation and terminology is justified by the observation that if $\ideala$ is a monomial ideal with exponent matrix $A$, then the positive real number $\lambda$ described above is the $F$-threshold of $\ideala$ with respect to the diagonal ideal $\diag(\vv{u})$; see \cite[Proposition~4.12]{budur+mustata+saito.roots_bs_polys_for_mon_ideals} and the comments preceding it.
\pedro{We can also justify this later, with our description of mus}

\pedro[inline]{
This (or something similar) could be a running example, to illustrate the various notions and results, like optimal set, collapses, etc.
\begin{example}\label{ex: ft}
   Let $A=\left(\begin{smallmatrix}5&3&4\\ 5&4&3\\ 2&8&5\end{smallmatrix}\right)$ and $\vv{u} = (1,1,1)$.
   The Newton polyhedron $\N$ of $A$ is shown in \Cref{fig: newton polyhedron}.
   The point $(17/4)\cdot \vv{u}$, shown in white, lies in the relative interior of the face
   \[\O = \conv(\col(A)) + \cone(\canvec_2),\]
   shown in dark blue.
   Thus, $\ft{A}{\vv{u}} = 4/17$ and $\mf(A,\vv{u}) = \O$.
\end{example}
}
\begin{figure}
   \centering
   \includegraphics[width=.45\textwidth]{Pictures/newton_polyhedron.pdf}
   \caption{The Newton polyhedron of the matrix $A$ described in \Cref{ex: ft}}
\label{fig: newton polyhedron}
\end{figure}


\begin{proposition}
   \label{FT descriptions: P}
   Let $(A,\vv{u})$ be a monomial pair, where $A$ is a matrix with $d$ rows.
   If $\N$ is the Newton polyhedron associated to $A$, then
   \[ \ft{A}{\vv{u}} = \min_{\vv{d}} \, \iprod{\vv{d}}{\vv{u}} = \iprod{\vv{c}}{\vv{u}} = \val \LP(A, \vv{u}), \]
   where the minimum is over all points $\vv{d} \in \RR^d$ that define a standard face of $\N$, and $\vv{c} \in \RR^d$ is any point that defines a face of $\N$ containing $\mf(A, \vv{u})$. 
\end{proposition}

\begin{proof}
   Let $\O$ be a face of $\N$ containing $\mf(A,\vv{u})$, and fix $\vv{c} \in \RR^d$ defining $\O$ in $\N$.
   Set $\lambda = \ft{A}{\vv{u}}$.
   By definition, $\vv{u} \in \lambda \O$, and so $\iprod{\vv{c}}{\vv{u}} = \lambda$.
   \pedro{Maybe a good point to remind the reader of our convention about rescaling points defining standard faces.}
   Similarly, if $\vv{d}$ defines a standard face of $\N$, then $\vv{u} \in \lambda \N$ implies that $\iprod{\vv{d}}{\vv{u}} \geq \lambda$.

   It remains to show that $\val \LP(A, \vv{u}) = \lambda$.
   Towards this, note that if $\vv{s}$ is feasible for $\LP = \LP(A, \vv{u})$, then $\vv{s}\ge \vv{0}$ and $A \vv{s} \leq \vv{u}$, and consequently
   \[\norm{\vv{s}} \le \iprod{\vv{c}}{A \vv{s}} \leq \iprod{\vv{c}}{\vv{u}} = \lambda,\]
   where the first inequality follows from \Cref{rmk: inner product with columns of A}, and the second from the nonnegativity of $\vv{c}$, established in \Cref{face: P}.
   We conclude that $\val \LP \leq \lambda$.

   On the other hand, \eqref{face: e} and our choice of $\O$ imply that
   \begin{equation}\label{cone containment: e}
      (1/\lambda) \cdot \vv{u} \in \O = \conv(\col(A) \cap \O) + \cone(\rb(\O)).
   \end{equation}
   Multiplying by $\lambda$, we obtain an expression $\vv{u} = A \vv{s} + \vv{w}$ with $\norm{\vv{s}} = \lambda$ and $\vv{w} \geq \vv{0}$.
   Evidently, the point $\vv{s}$ is feasible for $\LP$, and so $\val \LP \geq \lambda$.
\end{proof}

\pedro[inline]{
   \begin{example}\label{ex: illustrating FT descriptions}
      If $A$ and $\vv{u}$ are as in \Cref{ex: ft}, then the minimal face of $(A,\vv{u})$, shown in blue in \Cref{fig: newton polyhedron}, is defined by the point $\vv{a} = (3/17, 0, 1/17)$.
      \Cref{FT descriptions: P} then tells us that
      \[\ft{A}{\vv{u}} = \iprod{\vv{a}}{\vv{u}} = 4/17.\]

      From another perspective, \Cref{fig: splitting polytope} shows the feasible region of the linear program $\LP(A,\vv{u})$, with its optimal set,
      \[\opt \LP(A,\vv{u}) = \conv((1/17, 0, 3/17),(2/17, 1/17, 1/17)),\]
      highlighted in green.
      Thus, \Cref{FT descriptions: P} again tells us that
      \[\ft{A}{\vv{u}} = \val \LP(A,\vv{u}) = \norm{(1/17, 0, 3/17)} = 4/17.\]
   \end{example}
}
\begin{figure}
   \centering
   \includegraphics[width=.45\textwidth]{Pictures/splitting_polytope.pdf}
   \caption{The feasible region of the linear program $\LP(A,\vv{u})$ in \Cref{ex: illustrating FT descriptions}}
\label{fig: splitting polytope}
\end{figure}

\pedro[inline]{
   The stuff below seems out of place here.
   Maybe move it to the end of this section?
}

\comment[inline]{
The following is a consequence of the discreteness of the $F$-jumping exponents associated to an ideal in a regular ring \cite[Theorem~3.1]{blickle+mustata+smith.discr_rat_FPTs}.
However, to keep our discussion self-contained, we include an elementary proof in our specialized setting.

\begin{proposition}
\label{discreteness: L}
Given a monomial matrix $A$ and a real number $\beta > 0 $, there are only finitely many numbers of the form $\ft{A}{\vv{u}}$ less than $\beta$.    
%In particular, once $A$ is fixed, there are only finitely many numbers of the form $\ft{A}{\vv{u}}$ with $(A, \vv{u})$ very small.
\end{proposition}

\pedro[inline]{
   We can probably simplify this proof, as it is very believable that if $\vv{a}$ is nonnegative, then as $\vv{u}$ ranges over positive integral points, $\iprod{\vv{a}}{\vv{u}}$ takes on only finitely many values less than $\beta$ (when $a_i \ne 0$, this condition allows only finitely many choices for $u_i$; when $a_i = 0$, the value of $u_i$ does not affect the inner product).
}

\begin{proof}
   It suffices to show that there are only finitely many numbers $\ft{A}{\vv{u}}$ less than $\beta$ with $\mf(A, \vv{u}) = \O$ being fixed.
   Consider such a pair, and let $\collapse{\vv{u}}$ denote the collapse of the point $\vv{u}$ along $\O$.

   Note that if $\vv{a}$ defines $\O$, then \Cref{FT descriptions: P} tells us that
   %
   \begin{equation}\label{ft inner product identity: e}
      \ft{A}{\vv{u}} = \iprod{\vv{a}}{\vv{u}} = \iprod{\vv{a}}{\collapse{\vv{u}}}.
   \end{equation}
   %
   The inequality $\ft{A}{\vv{u}} < \beta$ then implies that the collapsed point $\collapse{\vv{u}}$ must lie in the set of all $\vv{w} \in \rs(\O)^{\perp}$ with $\vv{w} \ge \vv{0}$ and $\iprod{\vv{a}}{\vv{w}} < \beta$.
   However, the positivity of $\vv{a}$ in $\rs(\O)^{\perp}$ implies that this set is bounded (\eg one may argue as in the proof of \Cref{bounded polytope: P}), and therefore contains only finitely many lattice points.

   In summary, as $\vv{u}$ varies through all points with $\mf(A, \vv{u}) = \O$ and $\ft{A}{\vv{u}}$ less than $\beta$, the collapsed point $\collapse{\vv{u}}$ takes on only finitely values, and so the same must be true for the right-hand side of \eqref{ft inner product identity: e} above.
\end{proof}
}

The identity \eqref{cone containment: e} above implies that 
 %
%\[ \vv{u} \in \cone (\O) = \cone \left( (\col(A) \cap \O) \cup \rb(\O)  \right). \]
$\vv{u}$ is a conical combination of the columns of $A$ lying in $\O$ and the points in the recession basis of $\O$.
Typically, there are many ways to express $\vv{u}$ as such a conical combination, and as we see below, each such expression determines an optimal point of $\LP(A, \vv{u})$.

\begin{corollary}\label{opt set: C}
   Let $(A,\vv{u})$ be a monomial pair, where $A$ is a $d\times n$ matrix.
   A point $\vv{s} \in \RR^n$ is optimal for $\LP(A, \vv{u})$ if and only if it satisfies the following conditions.
\begin{enumerate}
\item  \label{mc coords: e} The coordinates of $\vv{s}$ are nonnegative, and the $i$-th coordinate of $\vv{s}$ is zero whenever the $i$-th column of $A$ is not contained in $\O = \mf(A, \vv{u})$.
\item  \label{mc decomposition: e} $\vv{u} = A \vv{s} + \vv{w}$ for some $\vv{w} \in  \cone(\rb(\O))$.   
%\item  \label{mc sum: e}$\norm{\vv{s}} = \ft{A}{\vv{u}}$.
\end{enumerate}
\end{corollary}

% \begin{proof}
%    Set $\LP = \LP(A, \vv{u})$ and $\lambda = \val \LP$, and fix $\vv{a} \in \RR^d$ that defines the face $\O = \mf(A, \vv{u})$ in the Newton polyhedron associated to $A$.

%    First, note that any point $\vv{s}$ satisfying the two conditions above must be feasible for $\LP$, and so it suffices to show that $\norm{\vv{s}} = \lambda$.
%    Towards this, the assumption on $\vv{w}$ in the expression $\vv{u} = A \vv{s} + \vv{w}$ implies that $\iprod{\vv{a}}{\vv{w}} = 0$, which allows us to compute that $\lambda = \iprod{\vv{a}}{\vv{u}} = \iprod{\vv{a}}{A\vv{s}} = \norm{\vv{s}}$, where the first equality follows from 
% \Cref{FT descriptions: P}, and the last from the assumption on the coordinates of $\vv{s}$ and the fact that the inner product of $\vv{a}$ with every column of $A$ contained in $\O$ is one.

% Next, suppose that $\vv{s}$ is optimal for $\LP$, and let $\vv{w}$ be the unique point in $\RR^d$ with $\vv{u} = A \vv{s} + \vv{w}$.  The optimality of $\vv{s}$ implies that $\norm{\vv{s}} = \lambda$, while the constraints of $\LP$ imply that $\vv{w} \geq \vv{0}$.  A direct computation shows that
% %
% \[ \lambda = \iprod{\vv{a}}{\vv{u}} = \iprod{\vv{a}}{A \vv{s}} + \iprod{\vv{a}}{\vv{w}} \geq \norm{\vv{s}} + \iprod{\vv{a}}{\vv{w}} = \lambda + \iprod{\vv{a}}{\vv{w}}, \]
% %
% which allows us to conclude that $\iprod{\vv{a}}{A \vv{s}} = \norm{\vv{s}}$ and $\iprod{\vv{a}}{\vv{w}} = 0$.
% It follows from these identities, and the fact that the standard basis vectors in $\rb(\O)^{\perp}$ correspond to the positive coordinates of $\vv{a}$, that the point $\vv{s}$ must satisfy the two asserted conditions.
% \end{proof}

\pedro[inline]{
   Trying to simplify the proof a bit; the original is commented out, in case someone wishes to restore it.
}

\begin{proof}
   Set $\LP = \LP(A, \vv{u})$ and $\lambda = \val \LP $, and fix $\vv{a} \in \RR^d$ that defines the face $\O = \mf(A, \vv{u})$ in the Newton polyhedron associated to $A$.
   Let $\vv{s} \in \RR^n$ and set $\vv{w} = \vv{u} - A\vv{s}$.
   % If $\vv{s}$ is optimal for $\LP$, then $\vv{s}\ge \vv{0}$, $\vv{w} \coloneqq \vv{u} - A\vv{s} \ge \vv{0}$, and $\norm{\vv{s}} = \lambda$.
   By \Cref{FT descriptions: P,rmk: inner product with columns of A},
   %
   \begin{equation}\label{eq 1}
      \lambda = \iprod{\vv{a}}{\vv{u}} = \iprod{\vv{a}}{A \vv{s}} + \iprod{\vv{a}}{\vv{w}} \geq \norm{\vv{s}} + \iprod{\vv{a}}{\vv{w}}.
   \end{equation}
   %
   If $\vv{s}$ is optimal for $\LP$, then $\vv{s} \ge \vv{0}$, $\vv{w} \ge \vv{0}$, and $\norm{\vv{s}} = \lambda$, and \eqref{eq 1} shows that $\iprod{\vv{a}}{A \vv{s}} = \norm{\vv{s}}$ and $\iprod{\vv{a}}{\vv{w}} = 0$.
   The first identity and \Cref{rmk: inner product with columns of A} show that $\vv{s}$ satisfies~(1); the second identity shows that $\vv{w}\in \cone(\rb(\O))$, so $\vv{s}$ satisfies~(2).
   Conversely, if $\vv{s}$ satisfies (1) and (2), then $\vv{s}$ is feasible for $\LP$, $\iprod{\vv{a}}{A \vv{s}} = \norm{\vv{s}}$, and $\iprod{\vv{a}}{\vv{w}} = 0$.
   By \eqref{eq 1}, $\norm{\vv{s}} = \lambda$, so $\vv{s}$ is optimal for $\LP$.
\end{proof}

\begin{theorem}  
\label{uniform denominators for vertices:  T}
Given a monomial matrix $A$, there exists a positive integer $\denom = \denom(A)$ such that for every monomial pair $(A, \vv{u})$, every vertex of $\opt \LP(A, \vv{u})$ is rational with denominator $\denom$.
\end{theorem}

\begin{proof}
Fix a monomial pair $(A, \vv{u})$. Set $\LP = \LP(A, \vv{u})$ and $\O = \mf(A, \vv{u})$.  Let $M$ be the matrix obtained from $A$ by omitting any columns not in $\O$, and inserting as a column each standard basis vector in $\rb(\O)$.  Finally, let $\denom = \denom(\O)$ be the least common multiple of all the nonzero minors of $M$.

If $\Q$ is the polyhedron consisting of all $\vv{t}$ in the domain of $M$ with $\vv{t} \geq \vv{0}$ and $M \vv{t} = \vv{u}$, then \Cref{opt set: C} implies that there exists a linear bijection $\opt \LP 
\leftrightarrow \Q$.  Furthermore, if $\vv{t}^{\ast}$ is a vertex of $\Q$, then \Cref{vertex: P} allows us to solve for the nonzero coordinates of $\vv{t}^{\ast}$ in the equation $M \vv{t}^{\ast} = \vv{u}$.  In particular, the fact that $\vv{u}$ has integer coordinates implies that the nonzero coordinates of $\vv{t}^{\ast}$ are rational with denominator $\denom = \denom(\O)$.  The linear bijection $\opt \LP \leftrightarrow \Q$ implies the same must be true for every vertex of $\opt \LP$.
\pedro{I think here we need to emphasize that this bijection is given by matrices with integral coordinates}

Our assertion then follows from the observation that since $A$ is fixed, as $(A, \vv{u})$ varies, there are only finitely many possibilities for $\O = \mf(A, \vv{u})$.
\end{proof}


\subsection{Special points and denominators}

\ \comment[inline]{Special points are how we construct canonical solutions to the integer program $\IP(A, \vv{u}, q)$;  see \Cref{canonical-feasible: T} for more details.}

Technicalities that arise in future sections whenever the minimal face of a monomial pair $(A, \vv{u})$ is unbounded force us to consider a certain distinguished subset of optimal points, in which we require a strengthening of condition \eqref{mc decomposition: e} in \Cref{opt set: C}.

\begin{definition}
\label{mc: D} 
Let $(A,\vv{u})$ be a monomial pair, and $\O = \mf(A, \vv{u})$.  A point $\vv{s}$ is a \emph{special point} for $(A, \vv{u})$ if it satisfies the following conditions.
\begin{enumerate}
\item $\vv{s} \in \opt \LP(A, \vv{u})$.
\item $\vv{u} = A \vv{s} + \vv{w}$ for some $\vv{w}$ in the relative interior of $\cone ( \rb(\O))$.  
\end{enumerate}
The set of all such points is denoted $\sp(A, \vv{u})$, and the set of all such points with rational coordinates is denoted $\sp_{\QQ}(A, \vv{u})$.  
\end{definition}

%We see below that $\sp(A, \vv{u})$ and $\opt \LP(A, \vv{u})$ are equal, or close to equal.

\begin{proposition}  
   \label{opt versus mc: P}
   Let $(A,\vv{u})$ be a monomial pair.
   If $\O = \mf(A, \vv{u})$ is bounded, then $\sp(A, \vv{u}) = \opt \LP(A, \vv{u})$.  Otherwise,  $\sp(A, \vv{u})$ is a nonempty convex subset of $\opt \LP(A, \vv{u})$ that contains the relative interior of this optimal set. 
\end{proposition}

\begin{proof}    
If $\O$ is bounded, then $\rb(\O) = \emptyset$, and so $\cone( \rb(\O)) = \{\vv{0} \}$ is equal to its relative interior.  Next, set $\lambda = \ft{A}{\vv{u}}$ and assume that $\O$ is unbounded.

 The minimality of $\O$ implies that $(1/\lambda)  \cdot \vv{u}$ cannot lie in any proper face of $\O$, and therefore, must lie in its relative interior.  Further, as the relative interior operation on convex sets commutes with Minkowski sums---see, \eg \cite[Theorem 4.10(b)]{vantiel.convex_analysis}---the decomposition in \eqref{face: e}  implies that $\vv{u} = \vv{v} + \vv{w}$ with $\vv{v} \in \lambda \conv(\col(A) \cap \O)$ and $\vv{w} \in \ri \cone(\rb(\O))$.  Any realization of $\vv{v}$ as $\lambda$ times a convex combination of the points in $\col(A) \cap \O$ then determines a special point.

 We have just shown that $\sp(A, \vv{u})$ is nonempty, and it is clear that this set is convex.  Next,  suppose that $\vv{e}_i \in \rb(\O)$.  If every vertex $\vv{s}$ of the optimal set of $\LP = \LP(A, \vv{u})$ was such that $A \vv{s}$ agreed with $\vv{u}$ in the $i$-th coordinate, then the same would be true for every point in the optimal set.   However, the special point constructed above shows that this is impossible.  Therefore, for every $\vv{e}_i \in \rb(\O)$, there exists a vertex $\vv{s}_i$ of $\opt \LP$ such that $A \vv{s}_i$ is less than $\vv{u}$ in the $i$-th coordinate.  Consequently, if $\vv{s}^{\ast}$ is any convex combination of these vertices of $\opt \LP$
 \pedro{\emph{These} vertices, or \emph{all} vertices? I'm a bit confused with the conclusion of this proof.}
 with positive coefficients, it follows that $A \vv{s}^{\ast}$ is less than $\vv{u}$ in the coordinate subspace $\rs(\O)$.
\daniel[inline]{It should be \emph{all} vertices.  I think it would help to address an earlier comment of yours.  The relative interior of a polytope (in this case, I have in mind $\opt \LP$) can be characterized as the set of all points that can be realized as a linear combination with \emph{positive coefficients} of the vertices of that polytope.   So, if $\vv{s}_1, \ldots, \vv{s}_{\ell}$ are the vertices of $\opt \LP$, then any element in $\ri \opt \LP$ looks like $\vv{s} = \lambda_1 \vv{s}_1 + \cdots \lambda_{\ell} \vv{s}_{\ell}$, where the $\lambda_i$ are positive.  There may be many such representations.  The feasibility of each vertex for $\LP$ implies that $A \vv{s}_i \leq \vv{u}$ for each $i$, and we showed that for each $\vv{e}_i \in \rb(\O)$, there exists an index $\phi(i)$ such that $A \vv{s}_{\phi(i)}$ is less than $\vv{u}$ in the $i$-th spot.  So, doesn't this mean that $A \vv{s} \leq \vv{u}$, and this inequality is strict for every $i$ with $\vv{e}_i \in \rb(\O)$? I hope so!}
\end{proof}

\comment[inline]{An example in which these sets (the set of special points, the optimal set, and the relative interior of the optimal set) all differ is commented out}
\pedro[inline]{
   Maybe we should include it.
   Or bring back \Cref{ex: ft} again, where $\opt \LP(A,\vv{u})$, highlighted in green in \Cref{fig: splitting polytope}, is the line segment connecting $(1/17, 0, 3/17)$ and $(2/17, 1/17, 1/17)$, and only the second endpoint is special for $(A,\vv{u})$.

}
% Consider the monomial matrix \[ A = \begin{bmatrix} a & 0 & c \\ 0 & b & c \\ 0 & 0 & d \end{bmatrix} \] 
% where $a,b,c$ are positive integers with $1/a + 1/b = 1/c$ and $d$ is any integer with $d>c$.  The maximal face of the splitting polytope is the edge connecting the points \[ \left( \frac{d-c}{da}, \frac{d-c}{db}, \frac{1}{d} \right) \text{ and } \left( \frac{1}{a}, \frac{1}{b}, 0 \right).\]  On the other hand, it is easy to check that the special points for $(A, \vv{1})$ consist of the points on this edge except for the first of these two  points.

\begin{definition}  Suppose that $A$ is a monomial matrix. A \emph{special denominator} for $A$ is a positive integer $\denom = \denom(A)$ such that for every monomial pair $(A, \vv{u})$, there exists a point $\vv{s} \in \sp(A, \vv{u})$ so that $\denom \cdot \vv{s}$ has integer coordinates.
\end{definition}

\begin{theorem}  
\label{special-denominators-exist:  T}
Special denominators exist.
\end{theorem}


\begin{proof}
   Let $A$ be a monomial matrix.
   Let $\ell_{\circ}$ be an integer satisfying the property described in \Cref{uniform denominators for vertices:  T} relative to $A$, and fix a monomial pair $(A, \vv{u})$.
   If $\O = \mf(A, \vv{u})$ is bounded, then $\sp(A, \vv{u}) = \opt \LP(A, \vv{u})$ by \Cref{opt versus mc: P}, and so every vertex in this set has denominator~$\ell_{\circ}$.

   Next, suppose $\O$ is unbounded, so that $A$ has $d \geq 2$ many rows.
   Without loss of generality, suppose that $\rb(\O) = \{ \vv{e}_1, \ldots, \vv{e}_c \}$ for some $1 \leq c \leq d-1$, and fix \emph{positive} integers $d_1, \ldots, d_c$ that sum to $d-1$.
   As demonstrated in the  proof of \Cref{opt versus mc: P}, for every index  $1 \leq i \leq c$, there exists a vertex $\vv{s}_i$ of $\opt \LP(A, \vv{u})$ for which $A \vv{s}_i$ is less than $\vv{u}$ in the $i$-th coordinate.
   It then follows from the definition of special point that the point
   \[ \frac{ d_1 \cdot \vv{s}_1 + \cdots + d_c \cdot  \vv{s}_c}{d-1}  \]
   lies in $\sp_{\QQ}(A, \vv{u})$ and has denominator $(d-1)\ell_{\circ}$.  
\end{proof}

\subsection{Collapsing}

Below, we describe the relationship between collapses and the other notions introduced in this section.

\begin{proposition}
   \label{collapse of mf and mc: P}
Consider a monomial pair $(A, \vv{u})$.  If $\collapse{A}$ \textup(respectively, $\collapse{X}$\textup) is the collapse of $A$ \textup(respectively, a set $X$\textup) along $\O = \mf(A, \vv{u})$, then the following hold.
\begin{enumerate}
\item $\mf(\collapse{A}, \collapse{\vv{u}}) = \collapse{\O}$ and $\ft{A}{\vv{u}} = \ft{\collapse{A}}{\collapse{\vv{u}}}$.
\item Each optimal point for $\LP(A, \vv{u})$ is also optimal for $\LP(\collapse{A}, \collapse{\vv{u}})$.  
\item Each special point for $(A, \vv{u})$ is a special point for $(\collapse{A}, \collapse{\vv{u}})$.
\end{enumerate}
\end{proposition}

\begin{proof}
Set $\lambda = \ft{A}{\vv{u}}$, so that $(1 / \lambda) \cdot \vv{u}$ lies in the relative interior of $\O$.  It is clear that projection preserves relative interiors, and so $(1/\lambda) \cdot \collapse{\vv{u}}$ must lie in the relative interior of $\collapse{\O}$, which is a bounded face of $\collapse{\N}$ by \Cref{collapse of Newton polyhedron: P}.  This observation demonstrates both that $\collapse{\O}$ is the minimal face of $\collapse{\N}$ containing $(1/\lambda) \cdot \collapse{\vv{u}}$, and that $\lambda = \ft{\collapse{A}}{\collapse{\vv{u}}}$.  

Next, note that \Cref{FT descriptions: P} and the above tells us that \[ \val \LP(A, \vv{u}) = \ft{A}{\vv{u}} = \ft{\collapse{A}}{\collapse{\vv{u}}} = \val \LP(\collapse{A}, \collapse{\vv{u}}). \] 
%
By construction, each nonzero row of $\collapse{A}$ is a row of $A$, and so the constraints of $\LP(\collapse{A}, \collapse{\vv{u}})$ are a subset of those of $\LP(A, \vv{u})$.  It follows that any optimal point for $\LP(A, \vv{u})$ must be optimal for $\LP(\collapse{A}, \collapse{\vv{u}})$.  The boundedness of $\collapse{\O}$, \Cref{opt versus mc: P}, and the preceding observation then allows us to conclude that
\begin{equation*}
   \sp(A, \vv{u}) \subseteq \opt \LP(A, \vv{u}) \subseteq \opt \LP(\collapse{A}, \collapse{\vv{u}}) = \sp(\collapse{A}, \collapse{\vv{u}}).
   \qedhere
\end{equation*}
\end{proof}

% \daniel[inline]{What do you guys think about including an example (picture) that describes the differences between the feasible set for $\LP(A, \vv{u})$ and $\LP(\collapse{A}, \collapse{\vv{u}})$?  In the few examples I've looked at, it seems like the optimal set of these programs is the same.  This probably isn't always true, but does someone know an example?  If so, it might be a good candidate for a picture.}
% \pedro[inline]{
%    I like that idea.
%    I've searched for random examples of $3 \times 3$ matrices (so we can picture not only the feasible sets, but also the Newton polyhedra) for which the optimal sets are different, and found that they are not super rare.
%    Below are pictures of a couple of examples.
%    In the first, $A=\left(\begin{smallmatrix}5&3&4\\ 5&4&3\\ 2&8&5\end{smallmatrix}\right)$; in the second, $A=\left(\begin{smallmatrix}8&4&4\\ 6&4&9\\ 7&9&4\end{smallmatrix}\right)$; $\vv{u} = (1,1,1)$ for both.
%    The feasible set for $\LP(A, \vv{u})$ is shown in blue, and is properly contained in the feasible set for $\LP(\collapse{A}, \collapse{\vv{u}})$, which also includes the yellow region.
%    The optimal set for $\LP(A, \vv{u})$ is shown in green, and is properly contained in the optimal set for $\LP(\collapse{A}, \collapse{\vv{u}})$, which also includes the red line segment.
% }
% \emily[inline]{These look awesome!  Maybe we should include a simple 2D example as well? }
% \pedro[inline]{I'll try to find a good 2D example.}

\pedro[inline]{
   \begin{example}
      Let $A$ and $\vv{u}$ be as in \Cref{ex: ft}.
      \Cref{fig: collapse}(\textsc{a}) shows the Newton polyhedron associated to the collapse $\collapse{A}$ of $A$ along $\O = \mf(A,\vv{u})$ (compare with \Cref{fig: newton polyhedron}).
      The point $(17/4)\cdot\collapse{\vv{u}}$, shown in white, lies in the relative interior of $\collapse{\O}$, shown in blue; thus, $\collapse{\O} = \mf(\collapse{A},\collapse{\vv{u}})$ and $\ft{\collapse{A}}{\collapse{\vv{u}}} = 4/17 = \ft{A}{\vv{u}}$.
      
      The feasible region of $\LP(A,\vv{u})$, shown in \Cref{fig: splitting polytope}, is properly contained in the feasible region for $\LP(\collapse{A},\collapse{\vv{u}})$, shown in \Cref{fig: collapse}(\textsc{b}), which highlights the difference between those sets in yellow.
      \Cref{fig: collapse}(\textsc{b}) also shows that the optimal set of $\LP(A,\vv{u})$ is properly contained in the optimal set of $\LP(\collapse{A},\collapse{\vv{u}})$, highlighting the difference between these sets in red.
      Thus, the containment established in \Cref{collapse of mf and mc: P}(2) may be proper.
   \end{example}
}
\begin{figure}
   \centering
   \begin{subfigure}{.49\textwidth}
      \centering
      \includegraphics[width=.8\textwidth]{Pictures/newton_polyhedron_of_collapse.pdf}
      \caption{}
   \end{subfigure}
   \begin{subfigure}{.49\textwidth}
      \centering
      \includegraphics[width=.8\textwidth]{Pictures/opt_for_collapse_may_change.pdf}
      \caption{}
   \end{subfigure}
   \caption{}
   \label{fig: collapse}
\end{figure}

\newpage
\section{An auxiliary integer program}

\pedro[inline]{Added some algebraic context.}
Let $\ideala$ be a monomial ideal in a polynomial ring in the variables $x=x_1,\ldots,x_d$ over a field of positive characteristic $p$.
Let $\vv{u}$ be a positive point in $\NN^d$, and $\ideald = \diag(\vv{u}) = \ideal{x_1^{u_1},\ldots,x_d^{u_d}}$.
Recall that for each $q$ a power of $p$ we define the number
\[\nu(\ideala,\ideald,q) \coloneqq \max\big\{k\in \NN : \ideala^{k} \not\subseteq \ideald^{[q]}\big\},\]
used in the computation of the $F$-threshold of $\ideala$ with respect to $\ideald$.
If $A \in \NN^{d\times n}$ is the exponent matrix of $\ideala$, then $\ideala^k$ is generated by monomials $x^{A\vv{k}}$, with $\vv{k}\in \NN^n$ and $\norm{\vv{k}} = k$.
Because $\ideald^{[q]} = \diag(\vv{u}q)$ is generated by monomials $x^\vv{v}$ with $\vv{v} \not< \vv{u}q$, the condition $\ideala^{k} \not\subseteq \ideald^{[q]}$ is equivalent to the existence of $\vv{k}$ as above satisfying $A\vv{k} < \vv{u}q$.
Thus, finding the maximum defining $\nu(\ideala,\ideald,q)$ is equivalent to maximizing $\norm{\vv{k}}$, with $\vv{k} \in \NN^n$ subject to the constraint $A\vv{k} < \vv{u}q$, which suggests the following definition.

\begin{definition}
   Given a monomial pair $(A,\vv{u})$ and a positive integer $q$, $\IP(A, \vv{u}, q)$ is the integer program in $\ZZ^n$ defined as follows:
\begin{enumerate}
\item The constraints are $\vv{k} \geq \vv{0}$ and $A \vv{k} < \vv{u}q$. 
\item The objective function is $\vv{k} \mapsto \norm{\vv{k}}$.
\end{enumerate}
\end{definition}

\begin{definition}
The \emph{image} of $\IP(A, \vv{u}, q)$ is the set \[ \im \IP(A, \vv{u}, q) = A ( \opt \IP(A, \vv{u}, q) ). \] 
\end{definition}

In the algebraic context established above, $\nu(\ideala,\ideald,q)$, which we will often denote $\nu(A,\vv{u},q)$, is the value of the integer program $\IP(A, \vv{u}, q)$.
Moreover, if $\nu = \nu(\ideala,\ideald,q) = \val \IP(A,\vv{u},q)$, then 
\begin{align*}
  \ideala^{\nu} &= \ideal{x^{A\vv{k}}: \norm{\vv{k}}=\nu}\\
  &\equiv \ideal{x^{A\vv{k}}: \norm{\vv{k}}=\nu\text{ and } A\vv{k} <\vv{u}q} \bmod \ideald^{[q]}\\
  &\equiv \ideal{x^{A\vv{k}}: \vv{k}\in \opt\IP(A,\vv{u},q)} \bmod \ideald^{[q]}\\
  &\equiv \ideal{x^{\vv{v}}: \vv{v} \in \im \IP(A,\vv{u},q)} \bmod \ideald^{[q]},
\end{align*}
so the image of $\IP(A,\vv{u},q)$ characterizes the ``leftovers'' of $\ideala^\nu$ modulo $\ideald^{[q]}$.

\subsection{Canonical feasible points}

We highlight a simple construction that associates to any point in $\sp_{\QQ}(A, \vv{u})$ a feasible point for $\IP(A, \vv{u}, q)$.
As a part of this, we call upon some basic notions from modular arithmetic.   

\begin{definition} If $m,n \in \ZZ$ are positive, then $\lpr{m}{n}$ is the \emph{least positive residue} of $m$ modulo $n$, \ie $m \equiv \lpr{m}{n} \bmod n$ and $1 \leq \lpr{m}{n} \leq n$.
\end{definition}

\begin{definition}
   \label{tail: D}
   Let $q$ be a positive integer.
   If $\lambda = a/b$ for some \emph{positive} integers $a$ and $b$, then we define
   \[ \tail{\lambda}_q = \frac{ \lpr{aq}{b}}{b}. \]
   Clearly, this expression depends only on $\lambda$, but not on the integers $a$ and $b$.
   Moreover, we set $[0]_q = 0$, and if $\vv{s} \in \QQ^n$ is nonnegative, then we define $\tail{\vv{s}}_q$ to be the point in $\QQ^n$ obtained by applying this operation to each coordinate of $\vv{s}$.
\end{definition}

\begin{remark}
\label{tail-basics: R}
Suppose that $\lambda \in \QQ$ and $q \in \ZZ$ are both positive, and write $\lambda = \frac{a}{b}$ with $a$ and $b$ positive integers.  Then $\tail{\lambda}_q$ is positive and rational, at most $1$,  and depends on  $q$ modulo $b$, but not on $q$ itself.  Furthermore, 
%
\[ \lambda q - \tail{\lambda}_q = \frac{aq-\lpr{aq}{b}}{b} \] is an integer, and in fact, is the \emph{greatest integer less than $\lambda q$}.
\end{remark}


\begin{lemma}
   \label{less than u: L}  Suppose that $\vv{s}$ is a special point for a monomial pair $(A, \vv{u})$.
   If $\vv{t}$ is a point in the domain of $A$ with $\vv{0} \leq \vv{t} \leq \vv{s}$, with the latter bound strict in every coordinate in which $\vv{s}$ is positive, then $A \vv{t} < \vv{u}$.
\end{lemma}

\begin{proof}  Set $\O = \mf(A, \vv{u})$.  The fact that $\vv{s} \in \sp(A, \vv{u})$  implies that \[ \vv{u} = A \vv{s} + \vv{w}\] for some positive point $\vv{w}$ in $\rs(\O)$.     The inequality $\vv{t} \leq \vv{s}$ induces the bound $A \vv{t} \leq A \vv{s} = \vv{u} - \vv{w}$, which shows that $A\vv{t}$ is less than $\vv{u}$ after projecting to the coordinate subspace $\rs(\O)$.  To conclude the proof, it suffices to show that the same is true in the complementary subspace $\rs(\O)^{\perp}$.  

Towards this, let $(\collapse{A},\collapse{\vv{u}})$ be the collapse of $(A,\vv{u})$ along $\O$.  Our choice of $\vv{t}$ implies that $\collapse{A}( \vv{s} - \vv{t})$ and $\collapse{A} \vv{s} = \collapse{\vv{u}}$ are both linear combinations with positive coefficients of the same set of columns of $\collapse{A}$.  Therefore, since $\collapse{\vv{u}} = \collapse{A} \vv{s}$ is positive in $\rs(\O)^{\perp}$, then the same must be true for $\collapse{A}(\vv{s} - \vv{t})$.  In other words, $\collapse{A} \vv{t} < \collapse{A} \vv{s} = \collapse{\vv{u}}$, which shows that $A \vv{t} < \vv{u}$ in $\rs(\O) \oplus \rs(\O)^{\perp}$.
\end{proof}


%A crucial notion in our upcoming arguments is the following basic construction from elementary number theory.
%
%\begin{definition}  If $\lambda \in \QQ$ is nonzero and $q$  is a positive integer, we set \[ \tail{\lambda}_q = \frac{ \lpr{aq}{b}}{b} \] where $\lambda = a/b$ is some representation for $\lambda$ and $\lpr{c}{b}$ is the least \emph{positive} residue of the integer $c$ modulo $b$.   In the case that $\lambda = 0$, we set $\tail{\lambda}_q = 0$.  We extend this definition to points of $\QQ^n$ in a coordinate-wise manner.
%\end{definition}
%
%\begin{remark} This definition is clearly independent of the representation of $\lambda \neq 0$ as a fraction.  In fact, in this case, the positive rational number $\tail{\lambda}_q$ depends only on the residue of $q$ modulo the least denominator of $\lambda$.
%\end{remark}
%
%\begin{remark}  If $\vv{s} \in \QQ^n$ has nonnegative coordinates, then \[ \vv{s}q - \tail{\vv{s}}_q \] has nonnegative integer coordinates for every positive integer $q$.  We will use this fact often in the sequel, typically without explicitly mentioning it.
%\end{remark}



%One again, suppose $(A, \vv{u})$ is a monomial pair and that $q$ is a positive integer.  


\begin{theorem}
\label{canonical-feasible: T}
If $\vv{s} \in \sp_{\QQ}(A, \vv{u})$ and $q \in \ZZ$ is positive, then  \[ \vv{s}q - \tail{\vv{s}}_q \in \feas \IP(A, \vv{u}, q).\] 
\end{theorem}

\begin{proof}  \Cref{tail-basics: R} tells us that if $\vv{t} = \vv{s} - (1/q) {\tail{\vv{s}}_q}$, then $\vv{t}q = \vv{s}q -\tail{\vv{s}}_q$ has nonnegative integer coordinates.  In fact, \Cref{tail-basics: R} also allows us to apply \Cref{less  than u: L} with $\vv{t}$ as defined here to see that $A (\vv{s} q - \tail{\vv{s}}_q ) =  A\vv{t}q <  \vv{u}q$.  
\end{proof}

%To clarify the statements of the results that follow, we fix the following notation.

%\begin{setup}
%\label{collapse: S}
%Let $(A, \vv{u})$ be a $d \times n$ monomial pair with $\O = \mf(A, \vv{u})$.  Let $\collapse{X}$ denote the collapse of a subset $X$ of $\RR^d$ along the face $\O$, and let $\collapse{A}$ be the collapse of $A$ along $\O$.  With these conventions, $\collapse{A \vv{t}} = \collapse{A} \vv{t}$ for every $\vv{t} \in \RR^n$.    
%\end{setup}
%
%\begin{remark}  The main upshot to working with the collapsed matrix $\collapse{A}$ is the following observation:  If $\vv{s} \in \sp(A, \vv{u})$, then though it is not necessarily true that $A \vv{s}$ equals $\vv{u}$, it will always be the case that 
%\[ \collapse{A} \vv{s} = \collapse{A \vv{s}} = \collapse{\vv{u}}.\] 
%\end{remark}



%\subsection{A critical secondary linear program}  We continue to adopt the notation in \Cref{collapse: S}.  Below, we define and study a secondary integer program that will play a key role in our solution of $\IP(A, \vv{u}, q)$ for $q \gg 0$.
%
%
%\begin{definition} If $\vv{s} \in \sp_{\QQ}(A, \vv{u})$ and $q>0$ is an integer, then
%\[ \ip_q(A, \vv{u}, \vv{s}) \] 
%is  the integer program  in $\ZZ^n$ is defined as follows:
%\begin{enumerate}
%\item The objective function is $\vv{k} \mapsto \norm{\vv{k}}$.
%\item The constraints are that the $i$-th coordinate of $\vv{k}$ is nonnegative whenever the $i$-th coordinate of $\vv{s}$ is zero, and that $\collapse{A} \vv{k}  < \collapse{A} \tail{\vv{s}}_q$.
%\end{enumerate}
%
%The \emph{image} of $\ip_q(A, \vv{u}, \vv{s})$ is the subset of $\ZZ \rb(\O)^{\perp}$ given by \[ \im \ip_q(A, \vv{u}, \vv{s})  = \collapse{A} (\opt \ip_q(A, \vv{u}, \vv{s})). \] 
%\end{definition}
%
%Before establishing the basic properties of these programs, we establish their relevance to the problem at hand.

%\emily[inline]{The following could be discussion in the prose that leads to the definition of $\Theta$. }
%\daniel[inline]{I'm giving this a shot below.}

\begin{remark}[Comparisons with canonical feasible points] 
\label{comparison: R}
Adopt the context of \Cref{canonical-feasible: T}, and fix a point $\vv{k}$ that is feasible for $\IP = \IP(A, \vv{u}, q)$.    

Our goal is to describe some natural constraints on the difference between $\vv{k}$ and the feasible point  described in \Cref{canonical-feasible: T}.  Toward this, set 
%
\[ \vv{h} =  \vv{k} - \vv{s}q + \tail{\vv{s}}_q \]   
%
and, let $\collapse{\mathcal{X}}$ denote the collapse of $\mathcal{X}$ along the face $\O = \mf(A, \vv{u})$.  


Notice that if $s_i = 0$, then $h_i  = k_i \geq 0$, where the last bound follows from the nonnegativity constraint of $\IP$.  The definition of $\vv{h}$ and constraints of $\IP$ also tell us that $A ( \vv{s}q-\tail{\vv{s}}_q + \vv{h}) = A \vv{k} < \vv{u}q = A \vv{s}q + \vv{w}q$, where $\vv{w} \in \rs(\O)$ is as in \Cref{mc: D}.  Collapsing this inequality, keeping in mind that $\collapse{\vv{w}} = \vv{0}$, and rearranging terms, shows that $\collapse{A \vv{h}} < \collapse{A \tail{\vv{s}}_q}$.
\end{remark}




\subsection{Another integer program}


\begin{definition}
A \emph{monomial list} $(A, \vv{u}, \vv{s}, q)$ consists of the following data.
\begin{enumerate}
\item A monomial pair $(A, \vv{u})$.
\item A rational special point $\vv{s} \in \sp_{\QQ}(A, \vv{u})$.
\item A positive integer $q$.
\end{enumerate}
\end{definition}

We call a monomial list whose first term is the matrix $A$ an $A$-list.

\begin{definition}  
\label{aux program: D}
If $(A, \vv{u}, \vv{s}, q)$ is a monomial list, then
\[ \ip(A, \vv{u}, \vv{s}, q) \] 
is the integer program in domain lattice of $A$ defined as follows:
\begin{enumerate}
\item The objective function is $\vv{h} \mapsto \norm{\vv{h}}$.
\item The constraints are that the $i$-th coordinate of $\vv{h}$ is nonnegative whenever the $i$-th coordinate of $\vv{s}$ is zero, and that \[ \collapse{A \vv{h}}  < \collapse{A \tail{\vv{s}}_q}\]
\daniel{We should remind the reader that the collapsed inequalities are a subset of the ones determined by $A$} where $\collapse{\mathcal{X}}$ is the collapse of $\mathcal{X}$ along the face $\O = \mf(A, \vv{u})$.\end{enumerate}
\end{definition}


\begin{definition}
The \emph{image} of $\ip(A, \vv{u}, \vv{s}, q)$ is the set \[ \im \ip(A, \vv{u}, \vv{s}, q)  =  \collapse{A (\opt \ip(A, \vv{u}, \vv{s}, q))}\] 
where $\collapse{\mathcal{X}}$ is the collapse of $\mathcal{X}$ along the face $\O = \mf(A, \vv{u})$.
\end{definition}

\begin{proposition}  
\label{comparison: P}
If $(A, \vv{u}, \vv{s}, q)$ is a monomial list, then $\feas \IP(A, \vv{u}, q)$ lies in 
\[ \vv{s}q - \tail{\vv{s}}_q + \feas \ip (A, \vv{u}, \vv{s}, q).\]
\end{proposition}

\begin{lemma}
\label{tail projection: L}
If $(A, \vv{u}, \vv{s}, q)$ is a monomial list and $\O = \mf(A, \vv{u})$, then $\collapse{A}\tail{\vv{s}}_q$ is a positive lattice point in $\rs(\O)^{\perp}$, where $\collapse{A}$ is the collapse of $A$ along $\O$.
\end{lemma}

\begin{proof}  By construction, $\vv{s}q - \tail{\vv{s}}_q $ has nonnegative integer coordinates, and the identity 
$\collapse{\vv{u}} q =\collapse{A} \vv{s} q = \collapse{A} ( \vv{s}q - \tail{\vv{s}}_q ) +\collapse{A} \tail{\vv{s}}_q$ then shows that $\collapse{A} \tail{\vv{s}}_q$ must also have integer coordinates.   To see that this vector is positive in $\rs(\O)^{\perp}$, note that $\collapse{\vv{u}} = \collapse{A} \vv{s}$ and $\collapse{A} \tail{\vv{s}}_q$ are both linear combinations with positive coefficients of the same set of columns of $\collapse{A}$.  Given this, it is easy to see that since $\collapse{\vv{u}} = \collapse{A} \vv{s}$ is positive in $\rs(\O)^{\perp}$, the same must be true for $\collapse{A} \tail{\vv{s}}_q$.
\end{proof}


\begin{remark}
\label{collapsed aux program: R}
Suppose $(A, \vv{u}, \vv{s}, q)$ is a monomial list.  If $(\collapse{A}, \collapse{\vv{u}})$ is the collapse of $(A ,\vv{u})$ along $\O = \mf(A, \vv{u})$, then \Cref{collapse of mf and mc: P} implies that 
\[ (\collapse{A}, \collapse{\vv{u}}, \vv{s}, q) \] is also a monomial list.  It is then clear from \Cref{aux program: D} that 
\[ \ip(A, \vv{u}, \vv{s}, q) = \ip(\collapse{A}, \collapse{\vv{u}}, \vv{s}, q). \] 
\end{remark}



The following may be regarded as a partial converse to \Cref{comparison: R}

\emily[inline]{Restate as a lemma first.  Do this point-by-point, and appeal to the finiteness of the $\mathbb{O}$.
We could say ``if $\vv{h}$ is optimal for $\Theta$, then $\ldots$ is optimal for $\ldots$''
}


\daniel[inline]{This has been changed to a ``point-by-point" statement.  This was necessary, since in its current placement, we haven't established any finiteness properties of $\ip$, and the old proof used $\mathbb{O}$}

\newpage
\begin{proposition}
\label{uniform value: P}
Consider a monomial list $(A, \vv{u}, \vv{s}, q)$, and a denominator $\ell$ for the special point $\vv{s}$.  If $\vv{h} \in \opt \ip(A, \vv{u}, \vv{s}, q)$, and $q/\ell$ is greater than every coordinate of $\vv{1} - \vv{h}$, and every coordinate of $A \vv{h}$, then the point 
$\vv{s}q - \tail{\vv{s}}_q + \vv{h}$ is optimal for $\IP(A, \vv{u}, q)$. 
\end{proposition}

\begin{proof} We start by describing what it means $q$ to be large.  Fix a positive denominator  $\denom \in \ZZ$ for $\vv{s} \in \sp_{\QQ}(A, \vv{u})$, and choose $q \gg 0$ so that $q/\denom$ is greater than every coordinate of $\vv{1} - \vv{h}$, and every coordinate of $A \vv{h}$.

Let $\collapse{\mathcal{X}}$ denote the collapse of $\mathcal{X}$ along $\O = \mf(A, \vv{u})$, and write  \[ \vv{u} = A \vv{s} + \vv{w} \] for some $\vv{w}$ that is positive in $\rs(\O)$, as in \Cref{mc: D}.  As $\vv{u}$ has integer coordinates, it follows that $\denom$ is also a denominator for $\vv{w}$.  

\Cref{tail-basics: R} tells us $\vv{k} := \vv{s}q - \tail{\vv{s}}_q + \vv{h}$ has integer coordinates, and we claim that $\vv{k} \geq \vv{0}$, that is, $\vv{s}q \geq \tail{\vv{s}}_q - \vv{h}$, whenever $q \gg 0$.  Indeed, if the $i$-th coordinate of $\vv{s}$ is zero, then so is the $i$-th coordinate of $\tail{\vv{s}}_q$, while the feasibility of  $\vv{h}$ for $\ip = \ip(A, \vv{u}, \vv{s}, q)$ implies that the $i$-th coordinate of $\vv{h}$ is nonnegative.  On the other hand, if the $i$-th coordinates of $\vv{s}$ is positive, then it must be at least $q/\denom$, and so the $i$-th coordinate of $\vv{s}q$ is at least $q/\denom$, which is greater than $1 - h_i$ by our choice of $q \gg 0$.  However, \Cref{tail-basics: R} also tells us that the $i$-th coordinate of $\tail{\vv{s}}_q - \vv{h}$ is at most $1-h_i$.  In summary, we have just shown that $\vv{k}$ is a nonnegative lattice point whenever $q \gg 0$.

Thus, $\vv{k}$ is feasible for $\IP = \IP(A, \vv{u}, q)$ if and only if
\[ A\vv{k} = A (\vv{s}q - \tail{\vv{s}}_q + \vv{h})  < \vv{u}q = A {\vv{s}}q + \vv{w}q.\] 
which we rewrite as 
\begin{equation} 
\label{equivalent ineq: e}
A \vv{h} < A \tail{\vv{s}}_q + \vv{w}q.
\end{equation}

After projecting to $\rs(\O)^{\perp}$, the bound \eqref{equivalent ineq: e} becomes $\collapse{A \vv{h}} < \collapse{A \tail{\vv{s}}}_q$, which holds by the feasibility of $\vv{h}$ for $\ip$.  If $\O$ is unbounded, then the projection of the right-hand side of \eqref{equivalent ineq: e} to $\rs(\O)$ is at least $\vv{w}q$.  However, as $\denom$ is a denominator for $\vv{w}$, every coordinate of $\vv{w}q$ is at least $q/\denom$,  and our choice of $q \gg 0$ then guarantees that \eqref{equivalent ineq: e} holds after projecting to $\rs(\O)$.  We conclude that \eqref{equivalent ineq: e} holds throughout $\RR^d = \rs(\O) \oplus \rs(\O)^{\perp}$.

In summary, we have just shown that $\vv{k}$ is feasible for $\IP$, and so 
\[ \val \IP \geq \norm{\vv{k}} = \ft{A}{\vv{u}} \cdot q - \norm{\tail{\vv{s}}_q} + \val \ip \] 
where above we have used that $\vv{s} \in \sp_{\QQ}(A, \vv{u})$ and $\vv{h} \in \opt \ip$.  To establish the optimality of $\vv{k}$, it remains to show that the $\val \IP$ equals this lower bound.  However, this is a consequence of \Cref{comparison: P}.
\end{proof}




\subsection{Some finiteness properties}  
We now explore some finiteness properties, and our results are of two types:  \Cref{bounded value: L} and \Cref{finite image: C} concern the integer program $\ip$ associated to some fixed monomial list,  while \Cref{finitely many secondary programs: L} and \Cref{finitely many coord sums: C} concern the nature of these programs as the list varies.

\begin{lemma}
\label{bounded value: L} 
If $(A, \vv{u}, \vv{s}, q)$ is a monomial list, then $0 \leq  \val  \ip(A, \vv{u}, \vv{s}, q) < \norm{\tail{\vv{s}}_q}$.  
\end{lemma}

\begin{proof}   
Fix a point $\vv{a} \in \RR^d$ that defines $\O  = \mf(A, \vv{u})$, and let $\collapse{X}$ denote the collapse of a subset $X$ along $\O$.  Thus, if $\collapse{A}$ is the collapse of $A$ along $\O$, then
\[ \iprod{{\vv{a}}}{A \vv{t}} = \iprod{\collapse{\vv{a}}}{\collapse{A \vv{t}}} = \iprod{\collapse{\vv{a}}}{\collapse{A} \vv{t}} \] for every $\vv{t}$.  With this notation in hand, we begin the proof below.

The product $\vv{a}^{\mathrm{T}} A $ is a row vector whose $i$-th coordinate is the inner product of $\vv{a}$ with the $i$-th column of $A$ (and so is at least one).   In fact, if the $i$-th coordinate of a point $\vv{k}$ feasible for $\ip = \ip(A, \vv{u}, \vv{s}, q)$ were negative, then the $i$-th coordinate of $\vv{s}$ must be positive;  thus, the $i$-th column of $A$ must lie in $\O$, and so the $i$-th coordinate of $\vv{a}^{\mathrm{T}} A$ must equal one.  In particular, 
%
\begin{equation} 
\label{bound in inner product: e}
\norm{\vv{k}} \leq (\vv{a}^{\mathrm{T}} A) \vv{k} =  \vv{a}^{\mathrm{T}} (A \vv{k}) = \iprod{\vv{a}}{A \vv{k}} = \iprod{\collapse{\vv{a}}}{\collapse{A} \vv{k}} 
\end{equation}
whenever $\vv{k}$ is feasible for $\ip$, and a similar argument will show that 
\begin{equation}  
\label{norm of tail: e}
\norm{\tail{\vv{s}}_q} =  \iprod{\vv{a}}{A \tail{\vv{s}}_q} = \iprod{\collapse{\vv{a}}}{\collapse{A} \tail{\vv{s}}_q}.
\end{equation}

Consequently, if $\vv{k}$ is feasible for $\ip$, then the constraint $\collapse{A}\vv{k} <\collapse{A} \tail{\vv{s}_q}$ and the above observations combine to tell us that \[ \norm{\vv{k}} \leq \iprod{\collapse{\vv{a}}}{\collapse{A} \vv{k}} < \iprod{\collapse{\vv{a}}}{\collapse{A} \tail{\vv{s}_q}} = \norm{\tail{\vv{s}_q}}\] 
which demonstrates that $\val \ip < \norm{\tail{\vv{s}}_q}$.  Finally, the positivity of $\collapse{A}\tail{\vv{s}}_q$ described in \Cref{tail projection: L} implies that $\vv{0}$ is feasible for $\ip$.
\end{proof}

\emily[inline]{Let's try to construct an example in which the optimal set of $\ip$ is infinite.}

\begin{corollary}
\label{finite image: C}
If $(A, \vv{u}, \vv{s}, q)$ is a monomial list, then $\im \ip(A, \vv{u}, \vv{s}, q)$ is finite.
\end{corollary}

\begin{proof}  Adopt the notation from the proof of \Cref{bounded value: L}.

If $\vv{k}$ is optimal for $\ip$, then \eqref{bound in inner product: e} implies that \[ \val \ip = \norm{\vv{k}} \leq \iprod{\collapse{\vv{a}}}{\collapse{A} \vv{k}}\] and so $\collapse{A} \vv{k}$ is a lattice point in the polyhedron of all points $\vv{b}$  in $\rs(\O)^{\perp}$ with $\vv{b} < \collapse{A} \tail{\vv{s}_q}$  and $\iprod{\collapse{\vv{a}}}{\vv{b}} \geq \val \ip$.  The positivity of $\collapse{\vv{a}}$ in $\rs(\O)^{\perp}$ and \Cref{bounded polytope: P} then tell us  that this polyhedron is bounded.  
\end{proof}

\emily[inline]{Maybe we should make this a Theorem, and explain that this is a very important finiteness property. 
Potentially move it up before \Cref{bounded value: L}.}

\begin{lemma} 
\label{finitely many secondary programs: L} 
If $A$ is fixed, then there are only finitely many integer programs of the form $\ip(A, \vv{u}, \vv{s}, q)$ as we vary over all $A$-lists $(A, \vv{u}, \vv{s}, q)$.
\end{lemma}

\begin{proof}  Consider a monomial list $(A, \vv{u}, \vv{s}, q)$.  As $A$ is fixed, there are only finitely many possibilities for $\O = \mf(A, \vv{u})$, and only finitely many possibilities for the set of supporting indices of any point $\vv{s} \in \sp_{\QQ}(A ,\vv{u})$.  

Next, let $\collapse{A}$ be the collapse of $A$ along the face $\O$.  If $\vv{s} \in \sp_{\QQ}(A, \vv{u})$, then $\vv{0} \leq \tail{\vv{s}}_q \leq \vv{1}$ for every integer $q > 0$, where $\vv{1}$ is the vector in the domain lattice of $A$ consisting of all ones.  Consequently, $\vv{0} \leq \collapse{A} \tail{\vv{s}}_q \leq \collapse{A}\, \vv{1}$, and as \Cref{tail projection: L} tells us that $\collapse{A} \tail{\vv{s}}_q$ has integer coordinates, it follows that there are only finitely many possibilities for this point.
\end{proof}

\begin{corollary} 
\label{finitely many coord sums: C}
 If $A$ is fixed, then there are only finitely many rational numbers of the form $ \norm{\tail{\vv{s}}_q}$ as we vary over all $A$-lists $(A, \vv{u}, \vv{s}, q)$.  
\end{corollary}

\begin{proof}  This follows from \eqref{norm of tail: e} and the proof of \Cref{finitely many secondary programs: L}.
\end{proof}


These finiteness properties above facilitate the following result.

\newcommand{\fsr}{\mathcal{R}}

\begin{theorem}[Existence of finite sets of representatives]  
\label{fsr-exist: T}
Given a monomial matrix $A$, there exists a finite subset $\fsr(A)$ of the domain lattice of $A$ with the following property\textup:  For every monomial list $(A, \vv{u}, \vv{s}, q)$, and for every point $\vv{v} \in \im \ip(A, \vv{u}, \vv{s}, q)$, there exists a point $\vv{h} \in \fsr(A)$ with $\vv{h} \in \opt \ip(A, \vv{s}, \vv{u}, q)$ and $\collapse{ A \vv{h}} = \collapse{A} \vv{h} =  \vv{v}$, where $\collapse{\mathcal{X}}$ denotes the collapse of $\mathcal{X}$ along $\O = \mf(A, \vv{u})$.
\end{theorem}

\begin{proof}  \Cref{finite image: C} implies that for every monomial list $(A, \vv{u}, \vv{s}, q)$,  there exists a \emph{finite} subset $\fsr(A, \vv{u}, \vv{s}, q)$ of $\opt (A, \vv{u}, \vv{s}, q)$ such that 
\[ \collapse{A}(\fsr(A, \vv{u}, \vv{s}, q))  = \ol{ A(\fsr(A, \vv{u}, \vv{s}, q)) } = \im \ip (A, \vv{u}, \vv{s}, q) \] 
and \Cref{finitely many secondary programs: L} then implies that these sets may be chosen in such a way so that $\fsr(A) = \cup \, \fsr(A, \vv{u}, \vv{s}, q)$ is finite, where the union is over all $A$-lists.
\end{proof}


\newpage
\section{Toward solving $\IP$}
\label{solving: S}

Suppose that $(A, \vv{u})$ is a monomial pair and that $q$ is positive integer.
The goal in this subsection is to demonstrate that the value and image of $\IP(A, \vv{u}, q)$ vary with $q$ in a uniform way as $q \to \infty$.

\subsection{Relating the two integer programs}
\label{relating-programs: ss}

\ \daniel[inline]{This needs updating.  Will give it a shot soon}

\begin{corollary}  
\label{uniform value and image: C}
Given a monomial matrix $A$, there exists an integer $\beta = \beta(A)$ satisfying the following condition\textup:
If $(A, \vv{u})$ is a monomial pair with $\O = \mf(A, \vv{u})$, $\vv{s} \in \sp_{\QQ}(A, \vv{u})$ is a point with denominator $D$, and $q>\beta D$, then 
%
\[ \val \IP(A, \vv{u}, q) = \ft{A}{\vv{u}} \cdot q - \norm{\tail{\vv{s}}_q} + \val \ip(A, \vv{u}, \vv{s}, q) \] 
%
and 
\[ \ol{\im \IP(A, \vv{u}, q)} = \collapse{\vv{u}} q - \collapse{A} \tail{\vv{s}}_q + \im \ip(A, \vv{u}, \vv{s}, q) \] 
where $\collapse{A}$ \textup(respectively, $\collapse{X}$\textup) is the collapse of $A$  \textup(respectively, $X$\textup) along $\O$.
\end{corollary}

\begin{proof}
Let $\beta$ be as in \Cref{uniform value: P}.  Fix a monomial pair $(A, \vv{u})$ with $\O = \mf(A, \vv{u})$, a point $\vv{s} \in \sp_{\QQ}(A, \vv{u})$ with denominator $D$, and an integer $q > \beta D$.  Let $\collapse{A}$ and $\collapse{X}$ be as above.

The asserted value of $\IP(A, \vv{u}, q)$ follows from \Cref{uniform value: P}.  Next, fix a point $\vv{k} \in \opt \IP(A, \vv{u}, q)$, and let $\vv{h}$ be the unique lattice point such that $\vv{k} = \vv{s}q - \tail{\vv{s}}_q + \vv{h}$.  \Cref{comparison: R} implies that $\vv{h}$ is feasible for $\ip = \ip(A, \vv{u}, \vv{s}, q)$, while the optimality of $\vv{k}$ tells us that $\norm{\vv{k}} = \val \IP(A, \vv{u}, q)$.  Keeping in mind our formula for $\val \IP(A, \vv{u}, q)$, this equality tells us $\norm{\vv{h}} = \val \ip$.    Therefore, $\vv{h}$ must be optimal for $\ip$,  and so $\collapse{A} \vv{h} \in \im \ip$.  Furthermore, as $\collapse{A} \vv{s} = \collapse{\vv{u}}$, 

\[ \collapse{A} \vv{k} = \collapse{\vv{u}} q - \collapse{A} \tail{\vv{s}}_q + \collapse{A} \vv{h}\]  
which shows that $\collapse{A} ( \opt \IP(A, \vv{u}, q))$ = $\ol{\im \IP(A, \vv{u}, q)}$ is contained in 
\[ \collapse{\vv{u}} q - \collapse{A} \tail{\vv{s}}_q + \im \ip.\]

We now establish the opposite containment:  \Cref{uniform value: P} tells us that \[  \vv{s}q - \tail{\vv{s}}_q + \orep(A, \vv{u}, \vv{s}, q)\] is optimal for $\IP(A, \vv{u}, q)$,  while \[ \collapse{A}( \orep(A, \vv{u}, \vv{s}, q)) = \im \ip \] by \Cref{orep: D}.   It follows that $\collapse{A}(\opt \IP(A, \vv{u}, q)) = \ol{\im \IP(A, \vv{u}, q)}$ contains the set $\collapse{\vv{u}} q - \collapse{A} \tail{\vv{s}}_q + \im \ip$.
\end{proof}

\subsection{Some useful invariants}
\label{useful-invariants: ss}

In this subsection, we study the quantities appearing in \Cref{uniform value and image: C} above.  We begin with a fundamental observation.

\emily[inline]{Can we give a direct proof that $\delta$ does not depend on $\vv{s}$?}

\begin{corollary}  
\label{independence: C} Fix a monomial pair $(A, \vv{u})$ and an integer $q>0$.  If $\collapse{A}$ is the collapse of $A$ along $\O = \mf(A, \vv{u})$, then the quantities
\[   \delta(A, \vv{u}, \vv{s}, q)  = \norm{\tail{\vv{s}}_q}  - \val \ip(A, \vv{u}, \vv{s}, q)\] and 
\[ \Delta(A, \vv{u}, \vv{s}, q)  = \collapse{A} \tail{\vv{s}}_q - \im  \ip( A, \vv{u}, \vv{s}, q)  \] 
do not depend on  $\vv{s} \in \sp_{\QQ}(A, \vv{u})$.  
\end{corollary}

\begin{proof}
Fix $\vv{s}$ and $\vv{s}'$ in $\sp_{\QQ}(A, \vv{u})$, as well as a common denominator $\denom$ for these points.  As these quantities clearly depend only on $q \bmod \denom$, it suffices to show that $\delta(A, \vv{u}, \vv{s}, q) = \delta(A, \vv{u}, \vv{s}', q)$  and $ \Delta(A, \vv{u}, \vv{s}, q) = \Delta(A, \vv{u}, \vv{s}', q)$ whenever $q \gg 0$.  However, this is follows from \Cref{uniform value and image: C}.
\end{proof}

\begin{definition}  
\label{independence: D}  

Given a monomial pair $(A, \vv{u})$ and positive integer $q$, we set 
 \[ \delta(A, \vv{u}, q) = \norm{\tail{\vv{s}}_q}  - \val \ip(A, \vv{u}, \vv{s}, q)\] and 
\[\Delta(A, \vv{u}, q) = \collapse{A} \tail{\vv{s}}_q - \im  \ip( A, \vv{u}, \vv{s}, q)  \]
where  $\vv{s} \in \sp_{\QQ}(A, \vv{u})$, and $\collapse{A}$ is the collapse of $A$ along $\O = \mf(A, \vv{u})$. 
\end{definition}

%\begin{remark} Above, we referred to \Cref{uniform value and image: C} to deduce the independence of $\delta(A, \vv{u}, q)$ and $\Delta(A, \vv{u}, q)$ on the rational point $\vv{s} \in \sp_{\QQ}(A, \vv{u})$.  Though it seems likely that this can be established with a more direct argument, we have yet to identify one. 
%\end{remark}

\begin{lemma}  
\label{independence: L}  
If $\collapse{A}$ is the collapse of $A$ along $\O = \mf(A, \vv{u})$ and $q>0$ is an integer, then the following hold.

\begin{enumerate}
\item $\delta(A, \vv{u}, q)$  is a positive rational number.
\item $\Delta(A, \vv{u}, q)$ is a finite set of positive lattice points in $\rs(\O)^{\perp}$.
\item No column of $\collapse{A}$ is less than any point in $\Delta(A, \vv{u}, q)$.
\end{enumerate}
\end{lemma}

\begin{proof} 
Fix a point $\vv{s} \in \sp_{\QQ}(A, \vv{u})$ with which to compute $\delta = \delta(A, \vv{u}, q)$ and $\Delta = \Delta(A, \vv{u}, q)$.  \Cref{bounded value: L} implies that $\delta$ is positive, and \Cref{finite image: C} that $\Delta$ is a finite subset of $\ZZ \rb(\O)^{\perp}$.   The positivity of $\Delta$ in this lattice is a consequence of the constraints of $\ip = \ip(A, \vv{u}, \vv{s}, q)$.  These constraints also imply that no column of $\collapse{A}$ is less than any point in $\Delta$.  Indeed, if $\vv{k}$ is optimal for $\ip$, then optimality implies that  $\collapse{A}( \vv{k} + \vv{e}_i) \not < \collapse{A} \tail{\vv{s}}_q$ for each standard basis vector $\vv{e}_i$ in the domain of $\collapse{A}$, which we rewrite as  \[ \collapse{A} \vv{e}_i \not < \collapse{A} \tail{\vv{s}}_q - \collapse{A} \vv{k}.\] 
We conclude that no column of $\collapse{A}$ is less than $\collapse{A}\tail{\vv{s}}_q - \collapse{A}\vv{k}$.
\end{proof}

We conclude with some finiteness properties.


\emily[inline]{Maybe we should restate \Cref{finitely many deltas for a fixed A: P} more precisely. }


\begin{proposition}
\label{finitely many deltas for a fixed A: P}
 Given a monomial matrix $A$, there are only finitely many objects of the form $\delta(A, \vv{u}, q)$ and $\Delta(A, \vv{u}, q)$.
\end{proposition}

\begin{proof}
This follows immediately from \Cref{finitely many secondary programs: L} and \Cref{finitely many coord sums: C}.
\end{proof}

\begin{remark}  
\label{comparing deltas: R}
If $(\collapse{A}, \collapse{\vv{u}})$ is the collapse of $(A, \vv{u})$ along $\O = \mf(A, \vv{u})$, then  
\[ \delta(A, \vv{u}, q) = \delta(\collapse{A}, \collapse{\vv{u}}, q)  \text{ and }  \Delta(A, \vv{u},q) = \Delta(\collapse{A}, \collapse{\vv{u}}, q)\] for all integers $q>0$ (e.g., this follows from \Cref{collapsed aux program: R}).   Consequently, one may replace the point in $\sp_{\QQ}(A, \vv{u})$ in \Cref{independence: D}   with one in $\sp_{\QQ}(\collapse{A}, \collapse{\vv{u}})$ without affecting the value of $\delta(A, \vv{u}, q)$ and $\Delta(A, \vv{u}, q)$.
\end{remark}

\begin{remark}
\label{pair periodicity: R}
If $(A, \vv{u})$ is fixed, then $\delta(A, \vv{u}, q)$ and $\Delta(A, \vv{u}, q)$ are periodic in $q$.  Indeed, if $\denom$ is the denominator of some point in $\sp_{\QQ}(A, \vv{u})$, then 
\begin{equation}
\label{periodicity: e}
 \delta(A, \vv{u}, p) = \delta(A, \vv{u}, q)  \text{ and } \Delta(A, \vv{u}, p) = \Delta(A, \vv{u}, q)
\end{equation} whenever $p \equiv q \bmod \denom$.    In fact, \Cref{comparing deltas: R} tells us that the same is true if instead $\denom$ is the denominator of a point in $\sp_{\QQ}(\collapse{A}, \collapse{\vv{u}})$.
\end{remark}

\begin{remark}
\label{uniform periodicity: R}
 If only $A$ is specified, then there exists a uniform integer $\denom$ such that \eqref{periodicity: e} holds for every monomial pair $(A, \vv{u})$ whenever $p \equiv q \bmod \denom$.  
 
 Indeed,  this follows from the observation that if $\denom$ is as in \Cref{uniform denominators for mc:  T}, then we may compute  $\delta(A, \vv{u}, q)$ and $\Delta(A, \vv{u}, q)$ for all monomial pairs $(A, \vv{u})$ and integers $q>0$ in terms of a point in $\sp_{\QQ}(A, \vv{u})$ with denominator $\denom$.
\end{remark}

We record another application of \Cref{uniform denominators for mc:  T} below.

\begin{theorem}
\label{uniform uniform value and image: T}
Given a monomial matrix $A$, there exists an integer $\beta = \beta(A)$ with the following property\textup:
If $q > \beta$ and $(A, \vv{u})$ is a monomial pair, then
\[ \val \IP(A, \vv{u}, q) = \ft{A}{\vv{u}} \cdot q - \delta(A, \vv{u}, q) \] and
\[ \ol{ \im \IP(A, \vv{u}, q)} = \collapse{\vv{u}}q - \Delta(A, \vv{u},q) \] where $\collapse{X}$ denotes the collapse of a subset $X$ along $\O = \mf(A, \vv{u})$.
\end{theorem}

\begin{proof}  
\Cref{uniform denominators for mc:  T}  tells us that once $A$ has been fixed, there exists a positive integer $D$ such that for every monomial pair $(A, \vv{u})$, there exists a point in $\sp_{\QQ}(A, \vv{u})$ with denominator $D$.  Therefore, if $\beta_{\circ}$  is any integer satisfying the condition stated in \Cref{uniform value and image: C}, then we may take $\beta = D \beta_{\circ}$.  
\end{proof}

The following is a consequence of \Cref{comparing deltas: R} and \Cref{uniform uniform value and image: T}.

\begin{corollary}
Given a monomial matrix $A$, there exists an integer $\beta$ with the following property\textup:  If $q > \beta$ and $(A, \vv{u})$ is a monomial pair with $\O = \mf(A, \vv{u})$, then $\val \IP(A, \vv{u}, q) = \val \IP(\collapse{A}, \collapse{\vv{u}}, q)$ and $\ol{ \im \IP(A, \vv{u}, q)} = \im \IP(\collapse{A}, \collapse{\vv{u}}, q)$ where $\collapse{A}$ \textup(respectively, $\collapse{X}$\textup) is the collapse of $A$  \textup(respectively, $X$\textup) along $\O$.
\end{corollary}

\emily[inline]{The following could just replace the above statement?  Replace integer programming language with algebraic language? For instance, as follows?}

\begin{corollary}
For $p \gg 0$, 
 $\nu(A, \vv{u}, q) = \nu(\collapse{A}, \collapse{\vv{u}}, q)$. 
\end{corollary}


\newpage




\newpage


\section{Arithmetic integer programming}


\emily[inline]{Motivate this via $\mu$s.}

In this section, we consider a variant of an integer program in which we impose an additional, and {highly} nonlinear, constraint.  As this new constraint is arithmetic in nature, we call such an optimization problem an \emph{arithmetic integer program}, and we will focus exclusively on one such family of optimization problems.  We define the terms \emph{feasible, optimal}, and \emph{value} relative to an arithmetic program in the analogous way.   

In what follows, $(A, \vv{u})$ is a $d \times n$ monomial pair and $p$ is a prime integer.
\pedro{Maybe we should also say that $q$, from here on, will always denote a power of $p$.}

\begin{definition} 
\label{aip: D}
If $q$ is a power of $p$, then $\IP_p(A, \vv{u}, q)$ is the arithmetic integer program in $\ZZ^n$ defined as follows:
\begin{enumerate}
\item The linear constraints are $\vv{k} \geq \vv{0}$ and $A \vv{k} < \vv{u}q$.  
\item The nonlinear (arithmetic) constraint is that $\binom{\norm{\vv{k}}}{\vv{k}} \not \equiv 0 \bmod p$.
By \Cref{thm: dickson}, this is equivalent to the condition that  if \[ \vv{k} = \vv{k}_0 + \cdots + \vv{k}_l \cdot  p^r\] is the unique terminating base $p$ expansion of $\vv{k}$, then $\norm{\vv{k}_e} < p$ for all $0 \leq e \leq r$.
\item The objective function is $\vv{k} \mapsto \norm{\vv{k}}$.
\end{enumerate}

\end{definition}


\daniel[inline]{It is possible that we don't use the image of this program anywhere.  Maybe only the image of $\ip$.}
\pedro[inline]{
   This image appears in multiple places: \Cref{follow-leftovers: P}, \Cref{arithmetic uniform value and image: T}, \Cref{cor: upper bound for higher mus}.
}
\begin{definition}
The \emph{image} of $\IP_p(A, \vv{u}, q)$ is the set $\im \IP_p(A, \vv{u}, q)$ of all points  $A \vv{k}$ with $\vv{k} \in \opt \IP_p(A, \vv{u}, q)$. 
\end{definition}

We seek to understand the behavior of the arithmetic program $\IP_p(A, \vv{u}, p^e)$ for all $p \gg 0$ and $e \geq 1$.
As will soon be apparent, these programs are more subtle than their non-arithmetic analogs.
We gather some basic general results pertaining to these programs below;  more specialized arguments will appear in the next section.

\begin{lemma} 
   \label{optimal division: L}
   If $(A, \vv{u})$ is a monomial pair and $q$ is a power of $p$, then the quotient when dividing any optimal point of $\IP_p(A, \vv{u}, qp^e)$ by $p^e$ must be optimal for $\IP_p(A, \vv{u}, q)$.
\end{lemma}

\begin{proof}  Suppose $\vv{g} \in \opt \IP_p(A, \vv{u}, qp^e)$ and write 
\[ \vv{g} = \vv{h} p^e + \vv{k} \]
with $\vv{h}$ and $\vv{k}$ in $\NN^d$ such that every coordinate of $\vv{k}$ is less than $p^e$.
By \Cref{cor: multinomial congruence}, the arithmetic constraint satisfied by $\vv{g}$ implies that both $\binom{\norm{\vv{h}}}{\vv{h}}$ and $\binom{\norm{\vv{k}}}{\vv{k}}$ are nonzero mod $p$.
By construction,  the base $p$ expansion of $\vv{k}$ is of the form $\vv{k} = \vv{k}_0 + \cdots + \vv{k}_{e-1} \cdot p^{e-1}$, and so the arithmetic constraint satisfied by $\vv{k}$ implies that $\norm{\vv{k}} < p^e$.
Consequently, if $\vv{h}$ were not optimal for $\IP_p(A, \vv{u}, q)$, then there would exist $\vv{m}$ feasible for $\IP_p(A, \vv{u}, q)$ with $\norm{\vv{m}} \geq \norm{\vv{h}} + 1$, which would lead to a point $\vv{m}p^e$ feasible for $\IP_p(A, \vv{u}, qp^e)$ whose norm is \[ \norm{\vv{m}}p^e \geq \norm{\vv{h}} \cdot p^e + p^e >  \norm{\vv{h}} \cdot p^e + \norm{\vv{k}} = \norm{\vv{g}}\] which contradicts the optimality of $\vv{g}$. % We conclude that $\vv{h} \in \opt \IP_p(A, \vv{u}, q)$.
\end{proof}

We record some corollaries of \Cref{optimal division: L} below.

\daniel[inline]{Should we just think about \Cref{natural bounds: C} algebraically?}

\begin{corollary}
   \label{natural bounds: C}
   If $(A, \vv{u})$ is a monomial pair, $q$ is a power of $p$, and $e$ is a positive integer, then
   \[ \val \IP_p(A, \vv{u}, q) \cdot p^e \leq \val \IP_p(A, \vv{u}, qp^e) < (\val \IP_p(A, \vv{u}, q) +1) \cdot p^e. \]
\end{corollary}
\begin{proof}
These bounds follow from a direct computation of the norm of the optimal point $\vv{g}$ in the proof of \Cref{optimal division: L}
\end{proof}

\begin{corollary}
   \label{cor: mu comparison}
   If $(A, \vv{u})$ and $(B, \vv{v})$ are monomial pairs such that
   \[ \val \IP_p(A, \vv{u}, q) > \val \IP_p(B, \vv{v}, q)\]
   for some $q$, then $\val \IP_p(A, \vv{u}, qp^e) > \val \IP_p(B, \vv{v}, qp^e)$ for all $e \geq 0$.
\end{corollary}

\begin{proof}
   If $\val \IP_p(A, \vv{u}, q) \geq \val \IP_p(B, \vv{v}, q) + 1$, then \Cref{natural bounds: C} tells us that
   \begin{align*}
     \val \IP_p(A, \vv{u}, qp^e)  &\geq \val \IP_p(A, \vv{u}, q) \cdot p^e \\
                                  &\geq (\val \IP_p(B, \vv{v}, q)+1)\cdot p^e \\
                                  & > \val \IP_p(B, \vv{u}, qp^e). \qedhere
   \end{align*}
\end{proof}

\subsection{Small pairs}

\ \pedro[inline]{Postpone introduction of medium small points until immediately before definition of $\widehat{\graph}$ graph?}

\begin{definition}
A monomial pair $(A, \vv{u})$ is \emph{small} $\vv{u}$ is not greater than any column of $A$, and is \emph{very small} if $\ft{A}{\vv{u}}$ is at most one.
\end{definition}

\begin{remark}
\label{finitely many small but not very small: R}
Geometrically, $(A, \vv{u})$ is small if and only if $\vv{u}$ does not lie in the interior of the upper staircase associated to the columns of $A$, and very small if $\vv{u}$ does not lie in the interior of the Newton polyhedron of $A$.  

It is clear from this geometric interpretation that ``very small'' implies ``small.''  Furthermore, once $A$ is fixed, there are only finitely monomial pairs $(A, \vv{u})$ that are small, but not very small. \daniel{Do we need a proof?}
\end{remark}

\begin{lemma}
\label{refined-discreteness: L}
Given a monomial matrix $A$, there exists $\delta = \delta(A)$ such that $\ft{A}{\vv{u}} < \delta$ whenever $(A, \vv{u})$ is small.
\end{lemma}

\begin{proof}   Fix a small pair $(A, \vv{u})$ with $\O = \mf(A, \vv{u})$.  If $\epsilon$ is the number of columns of $A$ lying on $\O$, then it suffices to prove that $\ft{A}{\vv{u}} \leq \epsilon$.

By means of contradiction, suppose that $\ft{A}{\vv{u}} > \epsilon$.  If $\vv{s} \in \sp(A,\vv{u})$, then $\norm{\vv{s}} = \ft{A}{\vv{u}} > \epsilon$, and as $\vv{s}$ has at most $\epsilon$ nonzero entries, some entry of $\vv{s}$ must be greater than $1$.  Thus, $A \vv{s}$ is greater than some column of $A$.  However, our choice of $\vv{s}$ also implies that $A \vv{s} \leq \vv{u}$, which then implies that $\vv{u}$ is greater than some column of $A$, contradicting the smallness of $(A, \vv{u})$.
\end{proof}

\begin{lemma}
   \label{trivial value bound: L}
   If $(A, \vv{u})$ is small, then
   \[ \val \IP_p(A, \vv{u}, p^e) \leq p^{e} -1 \]
   for every $e \geq 0$.
\end{lemma}


\begin{proof}
   Note that $(A, \vv{u})$ is small if and only if $\vv{0}$ is the only feasible point for $\IP(A, \vv{u}, 1)$.
   Thus, $\val \IP_p(A, \vv{u}, 1) = \val \IP(A, \vv{u}, 1) = 0$, and the assertion then follows from \Cref{natural bounds: C}.
\end{proof}

%\begin{proof} If $\vv{k}$ is feasible for $\IP_p(A, \vv{u}, p^e)$ and $ p^e \vv{e}_i  \leq \vv{k}$ for some standard basis vector $\vv{e}_i$ of $\ZZ^n$, then $p^e A \vv{e}_i  \leq A \vv{k} < \vv{u}p^e$, and therefore $A \vv{e}_i < \vv{u}$, which contradicts the smallness of $(A, \vv{u})$.  Thus, every coordinate of $\vv{k}$ is less than $p^e$, and so the base $p$ expansion of $\vv{k}$ is of the form $\vv{k} = \vv{k}_0 + \cdots + \vv{k}_{e-1} p^{e-1}$.  The arithmetic constraint of the program then implies that $\norm{\vv{k}} \leq p^e-1$.
%\end{proof}

% \emily[inline]{If 
% \[
%  \ideala^{\left[ \mu_\ideala^{\vv{u}}(q) \right] } \equiv \langle x^{\vv{u}q - \vv{z}} \mid \vv{z} \in \Z \subseteq \NN_+^d \rangle \bmod \operatorname{diag}(\vv{u}q)\]
%  then 
%  \[
% \operatorname{crit}(\ideala, \vv{u}) = \frac{1}{q}\left( \mu^{\vv{u}}_\ideala(q) + \max \{ \operatorname{crit}(\ideala, \vv{z}) \mid \vv{z} \in \Z \}  \} \right)
% \]
% }


\begin{proposition}
   \label{follow-leftovers: P}
   Suppose $(A, \vv{u})$ is a monomial pair.
   If
   \[ \im \IP(A, \vv{u}, 1) = \vv{u} - \Z\]
   then every monomial pair $(A, \vv{z})$ with $\vv{z} \in \Z$ is small, and if $p \gg 0$ and $e \geq 0$, then
   \[ \val \IP_p(A, \vv{u}, p^e) = \val \IP(A, \vv{u}, 1) \cdot p^e + \max \val \IP_p(A, \vv{z}, p^e) \]
   where the maximum is over all points $\vv{z} \in \Z$.
\end{proposition}

\daniel[inline]{This proof seems way too long.  Might be shortened if we think about things algebraically}

\begin{proof}
   The constraints of $\IP(A, \vv{u}, 1)$ imply that $\Z$ is a finite set of lattice points with positive coordinates.   These constraints and optimality also imply that if $\vv{e}$ is a standard basis vector in the domain of $A$, then no point in the Minkowski sum $\vv{e} + \opt \IP(A, \vv{u}, 1)$ can be feasible for $\IP(A, \vv{u}, 1)$.  Applying $A$ to this shows that no point in 
\[ A \vv{e} + \im \IP(A, \vv{u}, 1) = A \vv{e} + \vv{u} - \Z \] 
is less than $\vv{u}$.  Thus, $A \vv{e}$ is not less than any point in $\Z$, and as $\vv{e}$ was arbitrary, it follows that $(A, \vv{z})$ is small for every $\vv{z} \in \Z$.

The finiteness of $\Z$ allows us to choose $p$ large enough so that \[ \val \IP(A, \vv{v}, 1) \leq p -1 \] for every point $\vv{v} \in \Z \cup \{ \vv{u} \}$.  In this case, every feasible point for $\IP(A, \vv{v}, 1)$  automatically satisfies the arithmetic constraint of $\IP_p(A, \vv{v}, 1)$, which allows us to conclude that $\IP(A, \vv{v}, 1) = \IP_p(A, \vv{v}, 1)$.  In particular,
\[ \im \IP_p(A, \vv{u}, 1) =\vv{u} - \Z. \] 

Next, fix $\vv{g}$ optimal for $\IP_p(A, \vv{u}, p^e)$.  If $\vv{h}$ is the quotient, and $\vv{k}$ the remainder, when dividing $\vv{g}$ by $p^e$, then \Cref{optimal division: L} tells us that $\vv{h}$ is optimal for $\IP_p(A, \vv{u}, 1)$, so that $A \vv{h} = \vv{u}-\vv{z}$ for some $\vv{z} \in \Z$.  The feasibility of $\vv{g}=\vv{h}p^e + \vv{k}$ for $\IP_p(A, \vv{u}, p^e)$ then implies the feasibility of $\vv{k}$ for $\IP_p(A, \vv{z}, p^e)$.  This establishes that $\norm{\vv{g}} = \val \IP_p(A, \vv{u}, p^e)$ is at most the asserted value.

To establish the opposite inequality, suppose $\vv{z}^{\ast}$ is a point in $\Z$ with $\val \IP_p(A, \vv{z}^{\ast}, p^e)$ maximal.  By virtue of being in $\Z$, we may write $\vv{z}^{\ast} = \vv{u} - A \vv{g}^{\ast}$ for some $\vv{g}^{\ast} \in \opt \IP(A, \vv{u}, 1)$.  If $\vv{k}^{\ast}$ is optimal for $\IP_p(A, \vv{z}^{\ast}, p^e)$, then a direct computation will show that 
$\vv{h}^{\ast} = \vv{g}^{\ast} p^e + \vv{k}^{\ast}$ satisfies the linear constraint of  $\IP_p(A, \vv{u}, p^e)$.  Furthermore, the feasibility of $\vv{k}^{\ast}$ implies that $\binom{\norm{\vv{k}^{\ast}}}{\vv{k}^{\ast}} \not \equiv 0 \bmod p$, and the smallness of $(A, \vv{z}^{\ast})$ and \Cref{trivial value bound: L} tell us that $\norm{\vv{k}^{\ast}} \leq p^e-1$.  On the other hand, our choice of $p \gg 0$ tells us that $\norm{\vv{g}^{\ast}} = \val \IP_p(A, \vv{u}, 1) = \val \IP(A, \vv{u}, 1) \leq p-1$, and it follows that $\vv{h}^{\ast}$ also satisfies that the arithmetic constraint of $\IP_p(A, \vv{u}, p^e)$.
\end{proof}


We have just shown that to compute the value of $\IP_p(A, \vv{u}, p^e)$ for all $p \gg 0$ and $e \geq 1$, we may assume that $(A, \vv{u})$ is small.

\pedro[inline]{Commented out a theorem about mus for medium small pairs, that is now a corollary to ILL.}
% Below, consider an important special case of this simplified situation.

% \daniel[inline]{\Cref{trivial max value: T} looks like it could follow from ILL and then just modifying the $\vv{k}'s$ immediately.  But, the following is more direct.}

% \begin{theorem}
% \label{trivial max value: T}  Given a monomial matrix $A$, there exists an integer $\beta$ with the following property\textup:   
% If $(A, \vv{u})$ is small, but not very small, then  \[ \val \IP_p(A, \vv{u}, p^e) = p^e-1\] for every $p > \beta$ and $e \geq 1$.
% \comment{Compare Theorem~6.4 of \emph{Frobenius Powers}}
% \end{theorem}

% \begin{proof} Suppose that $(A, \vv{u})$ is small, but not very small, and fix $\vv{s} \in \sp_{\QQ}(A, \vv{u})$, so that $\norm{\vv{s}} = \ft{A}{\vv{u}} > 1$.   
% Set  $\vv{t} = \vv{s} / \norm{\vv{s}}$ and note that $\vv{0} \leq \vv{t} \leq \vv{s}$, with the latter inequality strict in every coordinate in which $\vv{s}$ is positive.  

% Next, fix an index $i$ such that the $i$-th coordinate of $\vv{s}$ is positive.  As $\tail{\vv{t}}_p$ obtains only finitely many values as $p$ varies, our choice of $i$ guarantees that 
% \[ 0 \leq \vv{t} - \frac{\tail{\vv{t}}_p}{p} + \frac{\norm{\tail{\vv{t}}_p}}{p} \cdot \vv{e}_i  \leq \vv{s} \]
% for all $p \gg 0$, with the latter inequality strict in every coordinate in which $\vv{s}$ is positive.  For such $p \gg 0$,  \Cref{less than u: L} tells us that 
% %
%  \[ A \left(  \vv{t}p - \tail{\vv{t}}_p + \norm{\tail{\vv{t}}_p} \cdot \vv{e}_i  \right) < \vv{u}p. \]
% %
 
%  By construction, $\norm{\vv{t}} = 1$, and as $\vv{t}p - \tail{\vv{t}}_p$ has nonnegative integer coordinates, we have that $\norm{\tail{\vv{t}}_p}$ is also a positive integer.  In summary, 
%   \[ \vv{k}_p  =   \vv{t}p - \tail{\vv{t}}_p+ (\norm{\vv{t}}_p - 1) \cdot \vv{e}_i   \] has nonnegative integer coordinates and satisfies $A \vv{k}_p < \vv{u}p - A \vv{e}_i$.  A direct calculation will also show that $\norm{\vv{k}_p} = p-1$. 
  
%  Given this, it is straightforward to verify that the point
%  \[ \vv{k}_p \cdot p^{e-1} + (p^{e-1} - 1) \cdot \vv{e}_i \]
%  has norm $p^e-1$ and is feasible for $\IP_p(A, \vv{u}, p^e)$ for all $e \geq 1$.  \Cref{trivial value bound: L} then allows us to conclude that $\val \IP_p(A, \vv{u}, p^e)  = p^e-1$ for all $e \geq 1$.
 
% To conclude the proof, it suffices to recall that, as noted in \Cref{finitely many small but not very small: R},  there are only finitely many $(A, \vv{u})$ that are small, but not very small.
% \end{proof}


\comment[inline]{At this point, it suffices to deal with the case that $(A, \vv{u})$ is very small}

\daniel[inline]{Should we state this more algebraically?}

\begin{theorem}
\label{arithmetic uniform value and image: T}   Given a monomial matrix $A$, there exists an integer $\beta$ with the following property\textup:  
If $(A, \vv{u})$ is a very small and $p > \beta$, then  \[ \val \IP_p(A, \vv{u}, p) = \ft{A}{\vv{u}} \cdot p - \delta(A, \vv{u}, p). \] 
and 
\[ \ol{ \im \IP_p(A, \vv{u}, p)} = \collapse{\vv{u}}p - \Delta(A, \vv{u}, p) \] where $\collapse{X}$ denotes the collapse of a subset $X$ of $\RR^d$ along $\O = \mf(A, \vv{u})$.
\end{theorem}

\begin{proof}  If $\beta$ is as in \Cref{uniform uniform value and image: T}, then \[ \val \IP(A, \vv{u}, p) = \ft{A}{\vv{u}} \cdot p - \delta(A, \vv{u}, p) \] for every monomial pair $(A, \vv{u})$ and $p > \beta$.  If this pair is very small, so that $\ft{A}{\vv{u}} \leq 1$, the positivity of $\delta(A, \vv{u},p)$ will then imply that this quantity is less than $p$.  Consequently, every $\vv{k}$ feasible for $\IP(A, \vv{u}, p)$ satisfies $\norm{\vv{k}} \leq p-1$, and therefore satisfies the arithmetic constraint of $\IP_p(A, \vv{u}, p)$.  We conclude that $\IP(A, \vv{u}, p) = \IP_p(A, \vv{u}, p)$ whenever $(A, \vv{u})$ is very small and $p > \beta$.
\end{proof}

\emily[inline]{If $(A, \vv{u})$ is small but not very small, then $\mu_\ideala^{\vv{u}}(p) = p-1$, so $\mu_\ideala^{\vv{u}}(p) \neq \nu_\ideala^{\vv{u}}(p)$.
In this case, although our description of $\ideala^{\nu_\ideala^{\vv{u}}(p)}$ does not depend on $p$, we \emph{can} have that the generators of $\ideala^{\mu_\ideala^{\vv{u}}(p)}
= \ideala^{p-1}$ depend on $p$:  For instance, if $\ideala = \langle x, y, \rangle$, then $\nu_\ideala^{\vv{u}}(p) = 2p-2$  and $\mu_\ideala^{\vv{u}}(p) = p-1$.  Moreover, $x^{p-(p+1)/2}y^{p-(p+1)/2} \in \ideala^{p-1}$.
(Actually, all minimal generators of $\ideala^{p-1}$ depend on $p$.)
}




\newpage


\section{A graph}

\daniel[inline]{Perhaps we should motivate why we would want to look at $p$-sprouts.  The point is that they determine which $\mu$'s we should compute next, at least in terms of the collapse.}


\subsection{Sprouting}

\begin{definition}
\label{p-sprout: D}
We say that $(B, \vv{v})$ is a \emph{$p$-sprout} of a monomial pair $(A, \vv{u})$ whenever the following conditions are satisfied.
\begin{enumerate}
\item $B$ is the collapse of $A$ along the minimal face $\O = \mf(A, \vv{u})$.
\item $\vv{v}$ is any point in $\Delta(A, \vv{u}, p)$.
\end{enumerate}
\end{definition}



\begin{remark}
\label{p-sprout: R} 
As noted in \Cref{collapse of monomial is monomial: R}, the collapse of a monomial matrix along a face of its Newton polyhedron is monomial, and so a $p$-sprout of a monomial pair is also a monomial pair.  Furthermore,   \Cref{independence: L} tells us that there are only finitely many $p$-sprouts of a fixed monomial pair, and that each such sprouted pair is small. 
 \end{remark}

The following statement may be regarded as a refinement of the upper bound given in \Cref{natural bounds: C}, at least when $p$ is large enough.

\emily[inline]{It seems like we prove that $\mu(A, \vv{u}, q) = \mu(\collapse{A}, \collapse{\vv{u}}, q)$.  Let's make sure to state that later.
}

\begin{corollary}\label{cor: upper bound for higher mus}
Given a monomial matrix $A$, there exists an integer $\beta$ with the following property\textup:  If $(A, \vv{u})$ is very small, then
%
\[ \val \IP_p(A, \vv{u}, p^{e+1})  \leq  \val \IP_p(A, \vv{u}, p) \cdot p^e +  \max \val \IP_p(B, \vv{v}, p^e) \] 
%
for all $p > \beta$ and $e \geq 1$, where the maximum is over all $p$-sprouts $(B, \vv{v})$ of $(A, \vv{u})$.  
\end{corollary}

\begin{proof}  Let $\beta$ be as in \Cref{arithmetic uniform value and image: T}, and fix a monomial pair $(A, \vv{u})$ that is very small.
Suppose $\vv{g}$ is optimal for $\IP_p(A, \vv{u}, p^{e+1})$, and let $\vv{h}$ and $\vv{k}$ be the quotient and remainder, respectively, when dividing $\vv{g}$ by $p^e$.

Let $\collapse{X}$ be the collapse of a subset $X$ of $\RR^d$ along $\O = \mf(A, \vv{u})$.  \Cref{optimal division: L} tells us that $\vv{h}$ must be optimal for $\IP_p(A, \vv{u}, p)$, and \Cref{arithmetic uniform value and image: T} then implies that $B \vv{h} = \collapse{A \vv{h}} \in \ol{\im \IP_p(A, \vv{u}, p)} = \collapse{\vv{u}}p - \Delta(A, \vv{u}, p)$ for all $p > \beta$.   
Therefore, for $p > \beta$, we may write \[ B \vv{h} = \collapse{\vv{u}}p - \vv{v}\] for some $\vv{v} \in \Delta(A, \vv{u}, p)$.  On the other hand, our choice of $\vv{g}$ guarantees that $A \vv{g} < \vv{u}p^{e+1}$, which leads to the inequality $B \vv{h} p^e + B \vv{k} = B \vv{g} <  \collapse{\vv{u}}p^{e+1}$  in $\rs(\O)^{\perp}$.  Comparing this with the above description of $B \vv{h}$ shows that \[ B \vv{k} < \vv{v} p^e \] which allows us to conclude that $\vv{k} \in \IP_p(B, \vv{v}, p^e)$.  %The corollary then follows from the fact that $\norm{\vv{g}} = \norm{\vv{h}} \cdot p^e + \norm{\vv{k}}$.
\end{proof}


\subsection{The Sprouting graph}


\begin{definition} \daniel{By the way, I think Emily and I have shown that the collapse of a collapse of $A$ is a collapse of $A$.  This will mean that the only matrices that can appear in $\S_p(A)$ are the collapses of $A$.  We don't gain any stronger theoretical finiteness properties, but this might simplify any implementations}
Given a monomial matrix $A$ and a prime $p$, define
\begin{enumerate}
   \item $\graph^0(A,p) = \{(A,\vv{u}) : (A,\vv{u})\text{ is a small monomial pair} \}$;
   \item $\displaystyle\graph^{e+1}(A,p) = \bigcup_{(B,\vv{v})\in \graph^e(A,p)}\sprout(B,\vv{v},p)$ for $e \geq 0$. 
\end{enumerate}
\end{definition}


\emily[inline]{We think that $\{ \graph^e : e \geq 1 \}$ is finite, but only care that $\bigcup_{e=1}^\infty \graph^e(A)$ is.}

\emily[inline]{verify that $\graph_e(A, \vv{u})$ and $\graph_e(A)$ are eventually periodic}
\daniel[inline]{In the remark (or wherever) when we gather some basic finiteness properties, at least state that there are only finitely many matrices appearing in any vertex of $\graph_p(A)$ as $p$ varies}

%\subsection*{Finiteness properties}
%
%Once $A$ is fixed,
%\begin{itemize}
% \item $\bigcup_{e=1}^\infty \graph^e(A,p)$ is finite.
% \item There exist $D$ such that for all $e \geq 1$ and $(B, \vv{v}) \in \graph^e(A,p)$, there exists $\vv{s} \in \sp(B, \vv{v})$ with denominator $D$. 
% \item $\mathbb{O}(A)$ is finite, and $\bigcup_{(B, \vv{v}) \in \graph^e(A), \text{ some } e} \mathbb{O}(B, \vv{v}, \vv{s}, p)$ is finite.
% \item Add the last point
%\end{itemize}

\newpage

\begin{theorem}[Iterated lifting]
\label{ILL: T}
   For each monomial matrix $A$, there exists an integer $\beta = \beta(A)$ with the following property\textup:
   If $p>\beta$ and $(A_1, \vv{u}_1) \to (A_2, \vv{u}_2) \to \cdots \to (A_e, \vv{u}_e)$ is a path in $\graph_p(A)$, then for every $1 \leq i \leq e$, there exists a point $\vv{k}_i \in \opt \IP(A_i, \vv{u}_i,p)$  for which 
 \[
  \vv{k}_1 p^{e-1} + \vv{k}_2 p^{e-2} + \cdots + \vv{k}_{e-1} p + \vv{k}_e \in \feas \IP(A_1, \vv{u}_1, p^e).
 \]
\end{theorem}

\begin{proof}\daniel{The ``finiteness properties" part of this proof is slightly different than what we sketched in Lawrence, but the rest of the argument follows what we talked about then.  I think it is correct, but it would be nice if someone could verify this.}  We start by describing what it means $p$ to be large.  Toward this, let $M_1, \cdots, M_l$\daniel{Update these $M_i$ if we include a proof that the collapse of a collapse of $A$ is a collapse of $A$.} be the finitely many monomial matrices obtained from $A$ by omitting some of its columns.  \Cref{special-denominators-exist:  T} allows us to fix a positive integer $\denom$ that is a special denominator for each such matrix.  We may also fix a finite set of representatives $\fsr(M_i)$ for each such matrix, as described in \Cref{fsr-exist: T}.  Set $\fsr = \fsr(M_1) \cup \cdots \cup \fsr(M_l)$, and let $\Omega$ be the set consisting of all coordinates of all points in $A(\fsr)$.  We stress that $\fsr$ and $\Omega$ are finite sets determined by $A$, and do not depend in any way on $p$.  

Consider the conditions \eqref{p-big-1} and \eqref{p-big-2} below.  
%
\begin{align}
\tag{$\heartsuit$} \label{p-big-1}
\text{$p/\denom$ is greater than any coordinate of any point in $\vv{1} - \fsr$.} \\ 
 \label{p-big-2}
\tag{$\diamondsuit$}\text{$p^e / \denom > \sum_{i=1}^e \omega_i \cdot p^{e-i}$ for every $e \geq 1$ and $\omega_1, \cdots, \omega_e \in \Omega$,}
\end{align}

The finiteness of $\fsr$,  and \Cref{positive-polynomial: L} below, imply that there exists an integer $\beta = \beta(\denom, \fsr)$ for which \eqref{p-big-1} and \eqref{p-big-2} hold whenever $p > \beta$.  In what follows, we assume that $p$ is chosen so that these conditions are satisfied.

Now, consider a finite path \[ (A_1, \vv{u}_1) \to (A_2, \vv{u}_2) \to \cdots \to (A_e, \vv{u}_e) \] in $\graph_p(A)$.  For every $1 \leq i \leq e$, set $\O_i = \mf(A, \vv{u}_i)$, and fix a special point $\vv{s}_i \in \sp(A_i, \vv{u}_i)$ with denominator $\denom$.  If $1 \leq i < e$, then the sprouting $(A_i, \vv{u}_i) \to (A_{i+1}, \vv{u}_{i+1})$ tells us that $A_{i+1}$ is the collapse of $A_i$ along $\O_i$, and that $\vv{u}_{i+1} \in \Delta(A_i, \vv{u}_i, q) = A_{i+1} \tail{\vv{s}_i}_p - \im  \ip( A_i, \vv{u}_i, \vv{s}_i, p)$.  Theorem \Cref{fsr-exist: T}, and our choice of $\fsr$, then allow us to fix a point $\vv{h}_i$ in $\fsr \cap \opt \ip ( A_i, \vv{u}_i, \vv{s}_i, p)$ such that 
$\vv{u}_{i+1} = A_{i+1} \tail{\vv{s}_i}_p - A_{i+1} \vv{h}_i$.  Finally, we take $\vv{h}_e$ to be an arbitrary point in the nonempty set $\fsr \cap \opt \ip ( A_e, \vv{u}_e, \vv{s}_e, p)$.


Next, for every $1 \leq i \leq e$,  we define
  \[
\vv{k}_i = \vv{s}_i \cdot p - [\vv{s}_i]_p + \vv{h}_i.
\]
Observe that \eqref{p-big-1} and \eqref{p-big-2} imply that for every $1 \leq i \leq e$, the quantity $p/\ell$ is greater than every coordinate of $\vv{1}-\vv{h}_i$, and every coordinate of $A_i \vv{h}_i$.  It then follows from \Cref{uniform value: P} (or rather, its proof) that 
\begin{equation}
\label{optimality-for-each-component: e}
\vv{k}_i \in \opt \IP(A_i, \vv{u}_i,p)
\end{equation}
for every $1 \leq i \leq e$.

We will now induce on $e$ to prove that $\sum_{i=1}^e \vv{k}_i \cdot p^{e-i}$ is feasible for $\IP(A_1, \vv{u}_1, p^e)$.  When $e = 1$, this follows from \eqref{optimality-for-each-component: e}.  Next, suppose that $e \geq 2$.  Our induction hypothesis applied to the truncated path  
\[ (A_2, \vv{u}_2) \to \cdots \to (A_e, \vv{u}_e) \]
%
tells us that $\vv{k}^{\ast} = \sum_{i=2}^e \vv{k}_i \cdot p^{e-i} \in \feas \IP(A_2, \vv{u}_2, p^{e-1})$.  To complete the induction step, we must show that $\vv{k}_1 p^{e-1} + \vv{k}^{\ast}$ is feasible for $\IP(A_1, \vv{u}_1, p^e)$.  However,  \eqref{optimality-for-each-component: e} implies that this point has nonnegative integer coordinates, and hence, we must only show that $A_1 ( \vv{k}_1 p^{e-1} + \vv{k}^{\ast} ) < \vv{u}_1 p^e$.

To do so,  recall that our choice of $\vv{s}_1 \in \sp(A_1, \vv{u}_1)$ allows us to express $\vv{u}_1$ as 
$\vv{u}_1 = A_1 \vv{s}_1 + \vv{w}$, where $\vv{w}$ is some point in $\rs(\O_1)$ that is positive in this Euclidean space.  This expression implies that the special denominator $\denom$ is also a denominator for $\vv{w}$.  It then follows from the definition of $\vv{k}_1$ and this expression for $\vv{u}_1$ that the inequality $A_1 ( \vv{k}_1 p^{e-1} + \vv{k}^{\ast} ) < \vv{u}_1 p^e$ is equivalent to 
%
\begin{equation}
\label{target-inequality: e}
  A_1( - \tail{\vv{s}_1}_p + \vv{h}_1 ) \cdot p^{e-1} + A_1\vv{k}^{\ast} < \vv{w} p^e.
\end{equation}

Given that the target of $A_1$ is $\rs(\O_1) \oplus \rs(\O_1)^{\perp}$, it suffices to verify that \eqref{target-inequality: e} holds after projection to each of these Euclidean spaces.  We first consider the projection to $\rb(\O_1)^{\perp}$.  As $A_2$ is the collapse of $A_1$ along $\O_1$, the projection of $A_1 \vv{m}$ to $\rb(\O_1)^{\perp}$ equals $A_2 \vv{m}$ for every $\vv{m}$ in the domain of $A_1$.  In particular, the projection of $A_1( - \tail{\vv{s}_1}_p + \vv{h}_1 )$ to this subspace is $A_2 ( - \tail{\vv{s}_1}_p + \vv{h}_1 )$, which equals $-\vv{u}_2$, by our choice of $\vv{h}_1$.  Thus, projecting \eqref{target-inequality: e} to $\rs(\O)^{\perp}$ yields $-\vv{u}_2 \cdot p^{e-1} + A_2 \vv{k}^{\ast} < \vv{0}$, which holds as $\vv{k}^{\ast} \in \feas \IP(A_2, \vv{u}_2, p^{e-1})$.

We now consider the projection of \eqref{target-inequality: e} to $\rs(\O_1)$, and given that $\tail{\vv{s}_1}_p \geq \vv{0}$, it suffices to verify that the projection of the stronger inequality \[ \sum_{i=1}^{e} A_1 \vv{h}_i \cdot p^{e-i} = A_1 \vv{h}_1 \cdot p^{e-1} + A_1 \vv{k}^{\ast} < \vv{w}p^e \] to $\rs(\O)^{\perp}$ holds.  However, keeping in mind that $\ell$ is a denominator of $\vv{w}$, which is positive in $\rs(\O_1)$, every coordinate of the projection of $\vv{w} p^e$ to $\rs(\O_1)$ is at least $p^e / \ell$, while every coordinate of $\sum_{i=1}^{e} A_1 \vv{h}_i \cdot p^{e-i}$ is of the form $\sum_{i=1}^{e} \omega_i \cdot p^{e-i}$ for some $\omega_1, \cdots, \omega_e \in \Omega$.  Thus, the condition \eqref{p-big-2} tells us that this stronger inequality holds after projecting to $\rs(\O_1)$.

We have just verified that \eqref{target-inequality: e} holds throughout $\rs(\O_1) \oplus \rs(\O_1)^{\perp}$, which allows us to conclude the induction step, and hence, our proof.
\end{proof}

\newpage

\begin{lemma}
   \label{positive-polynomial: L}
   Given a real number $w > 0$, and a set $\Omega$ of real numbers that is bounded from above, there exists an integer $\beta = \beta(w, \Omega)$ satisfying the following condition\textup:
   If $p > \beta$, then for every integer $e \geq 1$, and for every $\omega_1, \ldots, \omega_e \in \Omega$, we have that $wp^{e} >  \omega_1 \cdot p^{e-1} + \cdots + \omega_{e-1} \cdot p + \omega_e$.
\end{lemma}


\begin{proof}
Let $\lambda$ be any positive upper bound for $\Omega$.  Suppose that $p > (w+\lambda)/w$, which after rearranging terms, is equivalent to the condition $w(p-1) - \lambda > 0$.  Multiplying this by $p^e$ and adding the positive number $\lambda$ then shows that
%
\[ wp^e ( p-1 ) - \lambda (p^e-1) > 0 \] for every integer $e \geq 1$.   If, in addition, we also suppose that $p -1 > 0$, then we may divide the above by this quantity to conclude that \[ w p^e - \lambda \cdot \frac{ p^e - 1}{p-1} = wp^e - \lambda p^{e-1} - \cdots - \lambda p - \lambda > 0 \] for every integer $e \geq 1$.   Moving every term but $wp^e$ to the right-hand side of this inequality, the resulting bound is enough to conclude our proof.
\end{proof}

%
\daniel[inline]{I ended up combining these.  We can split them up later if anyone (possibly, me) prefers this.  I also added the hypothesis that the pairs in the first path were small, which was missing.}
% 
\begin{corollary}\label{cor: iterated lifting}
Given a monomial matrix $A$, there exists an integer $\beta = \beta(A)$ such that the following hold for every $p > \beta$ and path \daniel{We haven't defined the graph $\graph_p(A)$ yet (i.e., the arrows and ``levels" of the vertices hasn't been discussed.  Once we do this, we should define ``$\in''$ to mean ``path in", as this might save us some writing, and it is intuitive.  What does everyone think?}\[ (A_1, \vv{u}_1) \to \cdots \to (A_e, \vv{u}_e)  \in \graph_p(A).\]  
\begin{enumerate}
\item If $(A_i, \vv{u}_i)$ is very small for every $1 \leq i \leq e$, then \[ \mu(A_1, \vv{u}_1, p^e) \geq \sum_{i=1}^e \mu(A_i, \vv{u}_i, p) \, p^{e-i}.\] 
\item If $(A_i, \vv{u}_i)$ is very small for $1 \leq i < e$, but the last pair $(A_e, \vv{u}_e)$ is not very small, then for every integer $s \geq 0$, 
 \[ \mu(A_1, \vv{u}_1, p^{e+s}) \geq \sum_{i=1}^{e-1} \mu(A_i, \vv{u}_i, p) \, p^{e+s-i} + p^{s+1}-1. \]
\end{enumerate}
\end{corollary}

%\begin{proposition}\label{prop: ineq for higher mus in terminal paths}  Given a monomial matrix $A$, there exists an integer $\beta = \beta(A)$ satisfying the following condition\textup:  If $p > \beta$ and
% \[
%  (A_1, \vv{u}_1) \to (A_2, \vv{u}_2) \to \cdots \to (A_e, \vv{u}_e)
% \]
% is a path in $\graph(A,p)$ such that  $(A_i, \vv{u}_i)$ is very small for $1 \leq i < e$, and such that the terminal vertex $(A_e, \vv{u}_e)$ is not very small, then  
%\[
% \mu(A_1, \vv{u}_1, p^{e+s}) \geq \mu(A_1, \vv{u}_1, p) p^{e+s-1} + \cdots + \mu(A_{e-1}, \vv{u}_{e-1}, p) p^{s+1} + p^{s+1} - 1.
%\] for every integer $s \geq 0$.
%\end{proposition}

\begin{proof} The proofs of each assertion are similar; we only prove the second, which is more involved.  Let $\beta = \beta(A)$ be as in \Cref{ILL: T}.  If $p > \beta$, then \Cref{ILL: T} tells us that there exists $\vv{k}_i \in \opt \IP(A_i, \vv{u}_i, p)$ for which \[ \vv{k}^{\ast} = \sum_{1 \leq i < e} \vv{k}_i \cdot p^{e-i} + \vv{k}_e \in \feas \IP(A_1, \vv{u}_1, p^e).\]

  The assumption on the points in the path implies that $\norm{\vv{k}_i} \leq p-1$ for all $1 \leq i < e$, while $\norm{\vv{k}_e} \geq p$.  Thus, there exists a point $\vv{g}$ in the domain lattice of $A$ such that $\norm{\vv{g}} = p-1$ and $\vv{0} \leq \vv{g} \leq \vv{k}_e$, with the last inequality strict in at least one coordinate, say, in the first coordinate.  Thus, $\vv{0} \leq \vv{g} + \vv{e}_1 \leq \vv{k}_e$.

Fix an integer $s \geq 0$, and set   
%
\[ \vv{h} = \sum_{1 \leq i < e} \vv{k}_i \cdot p^{e+s-i} + (\vv{g} + \vv{e}_1) \cdot p^{s} - \vv{e}_1 \]
%
The bound $\vv{0} \leq \vv{g} + \vv{e}_1 \leq \vv{k}_e$ implies that $\vv{h} \leq \vv{k}^{\ast}  p^s$, and the feasibility of $\vv{k}^{\ast}$ for $\IP(A_1, \vv{u}_1, p^e)$ then implies that  $\vv{h}$ is feasible for $\IP(A_1, \vv{u}_1, p^{e+s})$.  To see that $\vv{h}$ is also feasible for the arithmetic version of this program, simply observe that the base $p$ expansion of $\vv{h}$ is given by 
%
\[ \vv{h} = \sum_{1 \leq i < e} \vv{k}_i \cdot p^{e+s-i} + \vv{g} \cdot p^{s} + (p-1) \vv{e}_1 \cdot p^{s-1} + \cdots + (p-1) \vv{e}_1 \]
%
and recall that $\norm{\vv{k}_i} \leq p-1$ for every $1 \leq i < e$.  Therefore, 
%
\[ \mu(A_1, \vv{u}_1, p^{e+s}) \geq \norm{\vv{h}} = \sum_{1 \leq i < e} \mu(A_i, \vv{u}_i, p) \cdot p^{e+s-i}+ p^{s+1}-1. \qedhere\]
%
\end{proof}

\begin{corollary}
   Given a monomial matrix $A$, there exists an integer $\beta = \beta(A)$ with the following property\textup: If $p > \beta$ and $(A, \vv{u})$ is small, but not very small, then $\mu(A,u,p^e) = p^e-1$ for every $e \geq 1$.
   \qed
\end{corollary}



%\emily[inline]{This is a terminating path in $\widehat{\graph}(A)$ (Pedro's version! i.e., all pairs are small and the last is medium-small)} 
%\daniel[inline]{I restated this so that the graph always terminates at a medium-small vertex, even when $e = 1$.  This should then imply that the critical exponent of a medium small pair equals $1$, which would allow us to remove an earlier theorem}

\newpage
\subsection{The Sprouting graph for a very small pair}
\begin{definition}
   Suppose that $(A, \vv{u})$ is very small.
   For $e \geq 0$, we define the set $\widehat{\graph}^e(A,\vv{u},p)$ inductively as follows:
\begin{enumerate}
\item $\widehat{\graph}^0(A, \vv{u}, p)$ consists of the single monomial pair $(A, \vv{u})$.
\item Suppose that $\widehat{\graph}^e(A, \vv{u}, p)$ has been defined for some integer $e \geq 0$, and let $S$ be the set of all $p$-sprouts of all monomial pairs in $\widehat{\graph}^e(A, \vv{u}, p)$.
If  $S$ is empty (that is, $\widehat{\graph}^e(A, \vv{u}, p)$ itself is empty) or contains a pair that is not very small, then \[ \widehat{\graph}^{e+1}(A, \vv{u}, p) = \emptyset.\]  
\emily{or say $\emptyset$ whenever $\graph^e(A, \vv{u}, p)$ is empty, or contains a medium-small pair}
Otherwise, $\widehat{\graph}^{e+1}(A, \vv{u}, p)$ is the set of all sprouts $(B, \vv{v})$ in $S$ satisfying the following conditions:    

\begin{enumerate}
\item Among all pairs in $S$, the value of  $\ft{B}{\vv{v}}$ is maximal.
\item Among all pairs in $S$ that achieve this maximum, the value of $\delta(B, \vv{v}, p)$ is minimal.
\end{enumerate}
Consequently, the value $\mu(B, \vv{v},p)$ is maximized among all elements in $S$ when $p \gg 0$. 
\end{enumerate}
\end{definition}


\begin{proposition}
   Given a monomial matrix $A$, there exists an integer $D$ such that $\widehat{\graph}(A, \vv{u}, p) = \widehat{\graph}(A, \vv{u}, q)$ for every monomial pair $(A, \vv{u})$ whenever $p \equiv q \bmod D$.
\end{proposition}

\alert[inline]{Include the proof.  It has to do with some finiteness property of $\ip$.}


% \begin{corollary}
% If $(A, \vv{u})$ is a monomial pair and $p \gg 0$, then
% \[ \val \IP_p(B, \vv{v}, p) = \ft{B}{\vv{v}} \cdot p - \delta_p(B, \vv{v}) \] for any vertex $(B, \vv{v})$ of $\widehat{\graph}_p(A, \vv{u})$.    In addition, if $(B, \vv{v})$ and $(D, \vv{z})$ are any two such vertices, then $\val \IP_p(B, \vv{v}) < \val \IP_p(D, \vv{z})$ if and only if $\ft{B}{\vv{v}} < \ft{D}{\vv{z}}$, or these two quantities agree and $\delta_p(B, \vv{v}) > \delta_p(D, \vv{z})$.  
% \end{corollary}


\begin{lemma}\label{lem: upper bound for higher mu}
   Given a monomial matrix $A$, there exists an integer $\beta= \beta(A)$ for which the following holds\textup:
   For each $p>\beta$ and $e\ge 1$, if $(A, \vv{u})$ is a very small monomial pair and $(A_1, \vv{u}_1) \to \cdots \to (A_e, \vv{u}_e)$ is a path in $\widehat{\graph}(A, \vv{u}, p)$,  then 
   \[
      \mu(B, \vv{v}, p^e) \le \mu(A_1, \vv{u}_1, p)p^{e-1} + \mu(A_2, \vv{u}_2, p)p^{e-2} + \cdots + \mu(A_{e}, \vv{u}_{e}, p)
   \]
   for any vertex $(B, \vv{v})$ of $\widehat{\graph}(A, \vv{u}, p)$ on the same level as $(A_1, \vv{u}_1)$.
\end{lemma}

\begin{proof}
   Choose $\beta = \beta(A)$ so that the conclusions of \Cref{arithmetic uniform value and image: T,cor: upper bound for higher mus} \pedro{Maybe more?} hold for each one of the finitely many monomial pairs in $\widehat{\graph}(A, \vv{u},p)$ whenever $p > \beta$, and fix such $p$.
    By virtue of \Cref{arithmetic uniform value and image: T} and the construction of $\widehat{\graph}(A,\vv{u},p)$, the assumption that $(A_1, \vv{u}_1)$ and $(B, \vv{v})$ lie on the same level implies that $\mu(B,\vv{v},p) = \mu(A_1, \vv{u}_1, p)$, proving the result for $e = 1$.

    Suppose that $e \geq 2$ and the result holds for paths of length $e-2$.
    \Cref{cor: upper bound for higher mus} tells us that
    \begin{align*}
      \mu(B,\vv{v},p^e) &\le \mu(B,\vv{v},p) p^{e-1} + \max_{\sproutsfrom{(C,\vv{z})}{(B,\vv{v})}} \ \mu(C,\vv{z},p^{e-1}) \\
      &= \mu(A_1,\vv{u}_1,p) p^{e-1} + \max_{\sproutsfrom{(C,\vv{z})}{(B,\vv{v})}} \ \mu(C,\vv{z},p^{e-1}),
    \end{align*}
    and to complete our inductive step, it suffices to show that
    \begin{equation}\label{ineq}
        \mu(C,\vv{z},p^{e-1}) \le \mu(A_2,\vv{u}_1,p) p^{e-2} + \cdots + \mu(A_e,\vv{u}_e,p)
    \end{equation}
    for each $(C,\vv{z})$ sprouting from $(B,\vv{v})$.
    
    Towards this, first note that if $(C, \vv{z})$ does not lie in $\widehat{\graph}(A,\vv{u},p)$, then \Cref{cor: mu comparison} implies that $\mu(C, \vv{z},p^{e-1}) < \mu(A_2,\vv{u}_2, p^{e-1})$, and the induction hypothesis applied to $(A_2, \vv{u}_2) \to \cdots \to (A_e, \vv{u}_e)$ and $(A_2,\vv{u}_2)$ itself gives \eqref{ineq}.
    On the other hand, if $(C, \vv{z})$ lies in $\widehat{\graph}(A, \vv{u},p)$, then it lies on the same level as $(A_2, \vv{u}_2)$, and  our induction hypothesis applied to the path $(A_2, \vv{u}_2) \to \cdots \to (A_e, \vv{u}_e)$ and the point $(C, \vv{z})$ once again gives us \eqref{ineq}, completing the proof.
\end{proof}

%%% OLD PROOF
% Choose $p \gg 0$ so that the conclusions of Corollary ??\daniel{The preceding corollary?} hold for $(A, \vv{u})$.  In this case, the assumption that $(A_1, \vv{u}_1)$ and $(B, \vv{v})$ lie in the same level implies that $\ft{A_1}{\vv{u}_1} = \ft{B}{\vv{v}}$ and $\delta_p(A_1, \vv{u}_1) = \delta_p(B, \vv{u})$, and so 
% \[ \val \IP_p(B, \vv{v}, p) = \val \IP_p (A_1, \vv{u}_1, p) = \ft{A_1}{\vv{u}_1} \cdot p - \delta_p(A_1, \vv{u}_1). \] 

% We will induce on $e \geq 1$.  The above observation settles the base case $e=1$.  Next, suppose that $e \geq 2$, and consider a path  \[ (A_1, \vv{u}_1) \to (A_2, \vv{u}_2) \to \cdots \to (A_e, \vv{u}_e)\] in $\graph_p(A, \vv{u})$.  If $C$ is the collapse of $B$ along the face $\O = \mf(B, \vv{u})$, then Corollary ?? \daniel{Which Corollary is this referring to?} and the above expression for $\val \IP_p(B, \vv{v}, p)$ tell us that the value of the arithmetic program $\IP_p(B, \vv{v}, p^{e+1})$ is at most
% %
% \[  \left( \ft{A_1}{\vv{u}_1} \cdot p - \delta_p(A_1, \vv{u}_1) \right) \cdot p^e +  \max_{\vv{z}} \val \IP_p(C, \vv{z}, p^e) \]
% %
% where the maximum is over all $\vv{z} \in \Delta_p(B, \vv{v})$.   Therefore, to complete our inductive step, it suffices to show that this maximum is at most
% %
% \begin{equation}
% \label{target bound: e}
% \left( \sum_{s=2}^e \frac{ \ft{A_s}{\vv{u}_s}  \cdot p - \delta_p(A_s, \vv{u}_s)}{p^s} \right) \cdot p^{e-1}.  \end{equation}

% Towards this, first note that if $(C, \vv{z})$ is not among the vertices of $\graph_p$ of level equal to that of $(A_2, \vv{u}_2)$ for any $\vv{z} \in \Delta_p(B, \vv{v})$, then Corollary \!{} and our choice of $p \gg 0$ implies that $\val \IP_p(C, \vv{z}, p) < \val \IP_p(A_2, \vv{u}_2, p)$ for every $\vv{z} \in \Delta_p(B, \vv{z})$.  In light of this, Corollary \!{} then tells us that 
% \[  \max_{\vv{z}} \val \IP_p(C, \vv{z}, p^e) < \val \IP_p(A_2, \vv{u}_2, p^e)\]  
% and our induction hypothesis applied to the path 
% \[ (A_2, \vv{u}_2) \to \cdots \to (A_e, \vv{u}_e) \] and the initial point $(A_2, \vv{u}_2)$ itself then tells us that the value of the program $\IP_p(A_2, \vv{u}_2, p^e)$ is at most the quantity in \eqref{target bound: e}.  

% On the other hand, if $(C, \vv{z})$ is a vertex of $\graph_p(A, \vv{u})$ of level equal to that of $(A_2, \vv{u}_2)$ for some $\vv{z} \in \Delta_p(B, \vv{v})$, then our induction hypothesis applied to the path 
% $(A_2, \vv{u}_2) \to \cdots \to (A_e, \vv{u}_e)$ and the point $(C, \vv{z})$ once again tells us that the value of the program $\IP_p(A_2, \vv{u}_2, p^e)$ is at most the quantity in \eqref{target bound: e}.  With this, we conclude the induction step, and hence, our proof. 
% \end{proof}



\begin{theorem}\label{thm: formula for higher mu}
   Given a monomial matrix $A$, there exists an integer $\beta= \beta(A)$ for which the following holds\textup:
   For each $p>\beta$ and $e\ge 1$, if $(A, \vv{u})$ is a very small monomial pair and $(A_1, \vv{u}_1) \to \cdots \to (A_e, \vv{u}_e)$ is a path in $\widehat{\graph}(A, \vv{u}, p)$,  then 
   \[ 
      \mu(A_1, \vv{u}_1, p^e) = \mu(A_1,\vv{u}_1,p)p^{e-1} + \cdots + \mu(A_{e-1},\vv{u}_{e-1},p)p + \mu(A_e,\vv{u}_e,p).
   \]
   If, in addition, that path is terminal \textup(that is, $(A_e,\vv{u}_e)$ is not very small\textup), then
      \[ 
 \mu(A_1, \vv{u}_1, p^{e+s}) = \mu(A_1, \vv{u}_1, p) p^{e+s-1} + \cdots + \mu(A_{e-1}, \vv{u}_{e-1}, p) p^{s+1} + p^{s+1} - 1
\]
for every nonnegative integer $s$.
\end{theorem}

\begin{proof}
   The first identity follows from \Cref{cor: iterated lifting}(1) and \Cref{lem: upper bound for higher mu}, while the second follows from \Cref{cor: iterated lifting}(2) and the first, together with the fact that $\mu(A_1, \vv{u}_1, p^{e+s}) \le \mu(A_1, \vv{u}_1, p^{e}) p^s+p^s-1$, which follows from the second inequality in \Cref{natural bounds: C}.
\end{proof}

\alert[inline]{
\begin{corollary}\label{cor: constant mus}
   If $(A,\vv{u})$ is a small monomial pair and  $\beta = \beta(A)$ is as in \Cref{lem: upper bound for higher mu}, then $\crit$ and $\mu$ are constant on each level of $\widehat{\graph}(A,\vv{u},p)$ for every $p\ge \beta$. 
\end{corollary}
%\todo[inline]{Restate more precisely}
\pedro[inline]{
   I'm not a believer anymore.
   If $(A_1,\vv{u}_1)$ and $(B,\vv{v})$ are as in \Cref{lem: upper bound for higher mu}, we only get the inequality $\mu(B,\vv{v},p^e) \le \mu(A_1,\vv{u}_1,p^e)$, and to conclude that we have equality we'd need to know that there is a path of length $e-1$ starting at $(B,\vv{v})$.
   But we don't know if we have this kind of uniformity in the graph.
}

\daniel[inline]{I think this can be remedied, but we will need to define the graph in a different way, to allow for paths that terminate and infinite paths at the same time.  The point is that at some fixed level $e$, a vertex should sprout if and only if that vertex is very small.  This sounds more like what you were suggesting in Lawrence.}
}
\newpage


\comment[inline]{The point of this Lemma is to show positivity of the coefficients in the polynomials that define the $\mu$'s. A consequence is that if $\ideala$ is $\idealm$-primary and homogeneous, then all coefficients of every intermediate power of $p$ in this polynomials vanishes.}

\begin{lemma}  If  $p>0$ is prime and $(B, \vv{v})$ is a $p$-sprout of  $(A, \vv{u})$, then \[ \delta(A, \vv{u}, p) \geq \ft{B}{\vv{v}}\]
with equality if $A$ is the monomial matrix associated to a monomial ideal that is homogeneous with respect to some positive $\ZZ$-grading, and primary to the ambient homogeneous maximal ideal.
\end{lemma}

\begin{proof}
By definition of $p$-sprout,  $B$ is the collapse of $A$ along the face $\O = \mf(A, \vv{u})$ of the Newton polyhedron $\N$ associated to $A$.  Suppose that $A$ has $d$ rows, and let $\collapse{X}$ denote the collapse of a subset $X$ of $\RR^d$ along $\O$.  

If $\vv{a} \in \RR^d$ defines the face $\O$ in $\N$, then \Cref{collapse of Newton polyhedron: P}  states that $\collapse{\vv{a}}$ defines the standard face $\collapse{\O}$ of $\collapse{\N}$, the Newton polyhedron associated to $B$, and \Cref{FT descriptions: P} then implies that \[\ft{B}{\vv{v}} \leq \iprod{\collapse{\vv{a}}}{\vv{v}}.\]

Thus, it suffices to show that $\iprod{\collapse{\vv{a}}}{\vv{v}} \leq \delta(A, \vv{u}, p)$.  However, by definition of $p$-sprout, $\vv{v} \in \Delta(A, \vv{u}, p)$, and so fixing a point $\vv{s} \in \sp_{\QQ}(A, \vv{u})$, we may write $ \vv{v} = B \tail{\vv{s}}_p - B \vv{h}$ for some $\vv{h}$  optimal for $\ip = \ip(A, \vv{u}, \vv{s}, q)$.  Our choice of $\vv{a}$ guarantees that the inner product of $\collapse{\vv{a}}$ with each column of $B$ is at least one, with equality whenever that column lies on $\collapse{\O}$.  Arguing as in the proof of \Cref{bounded value: L}, one may show that $\iprod{\collapse{\vv{a}}}{B \tail{\vv{s}}_p} = \norm{\tail{\vv{s}}_p}$ and \[ \iprod{\collapse{\vv{a}}}{ B \vv{h}} \geq \norm{\vv{h}} = \val \ip \] 
which allows us to conclude that \[ \iprod{\collapse{\vv{a}}}{\vv{v}} \leq \norm{\tail{\vv{s}}_p} - \val \ip= \delta(A, \vv{u}, p).\]

We now address the last assertion:  If $A$ satisfies these additional conditions, then homogeneity implies that the convex hull of the columns of $A$ is a proper face $\O$ of $\N$.  The $\mathfrak{m}$-primary assumption further implies that $\O$ is a facet, and that $\mf(A, \vv{z}) = \O$ for every $\vv{z} \in \ZZ_+^d$.  Furthermore, the positivity of the grading implies that the point $\vv{a} \in \RR^d$ defining this face must have positive coordinates, and so $\O$ must be bounded.  In this case, collapsing along this face is simply the identity map on $\RR^d$, and so in particular, $B=A$.  Given this, one may retrace the steps above to see that every inequality involving the inner product of $\vv{a} = \collapse{\vv{a}}$ with another point must be, in fact, an equality.  The details are left to the reader.
\end{proof}


\todo[inline]{Point out that the levels of $\widehat{\graph}(A, \vv{u})$ are eventually periodic. This will give an independent proof the rationality of critical exponents.  The way we present a formula for critical exponents may need this observation.}




%%%%%%%%%%%%%%%%%%%%%%%%%%%%%%%%%%%%%%%%%%%%%%%%%%%%%

\section{Fractal linear programs}

%%%%%%%%%%%%%%%%%%%%%%%%%%%%%%%%%%%%%%%%%%%%%%%%%%%%%

\begin{definition}
   The \emph{Sierpi\'nski $p$-gasket} in $\RRnn^n$ is the set $\sierp_{p,n}$
   \pedro{Should we include $n$ in the notation? E.g., $\sierp_p^n$?}
   \emily{There might be some confusion that this is a set $\sierp_{p}$ to the $n^\text{th}$ power?  What about $\sierp_{p,n}$?}
   \pedro{I like that, and have incorporated it.}
   consisting of all points $\vv{t}\in \RRnn^n$ for which there exist a sequence of points $\{ \vv{t}_e \}_{e=1}^\infty$ in $\NN^n$ for which all $\norm{\vv{t}_e} < p$ and $q$, a power of $p$, such that
 \[
\vv{t} = q\cdot\Big(\frac{\vv{t}_1}{p} +\frac{\vv{t}_2}{p^2}+\cdots +\frac{\vv{t}_e}{p^e} + \cdots \Big).  
 \]
\end{definition}

From the definition, we immediately see that a point in $\RRnn^n$ is in $\sierp_{p,n}$ if the unique nonterminating base $p$ expansions of its coordinates \emph{add without carrying}.
However, this is not a complete description of $\sierp_{p,n}$.
For instance, when $p=2$, the sum of $\frac{1}{4} = \frac{1}{2^3} + \frac{1}{2^4} + \frac{1}{2^5} + \cdots$ with itself carries at infinitely many places, yet we see that $\left(\frac{1}{4}, \frac{1}{4}\right)$ is in the Sierpi\'nski $2$-gasket in $\RRnn^2$ after writing one summand simply as $\frac{1}{4} = \frac{1}{2^2}$, \ie in its \emph{terminating} expansion.

This description of the Sierpi\'nski $p$-gasket in terms of expansions is not hard to translate geometrically into its geometric description as a fractal. 
For instance, the points of $\sierp_{2,2}$ in $[0,1]^2$ form the familiar Sierpi\'nski triangle:   
The points $\vv{t}$ in the unit square that have \emph{no} expansion $\vv{t} = \frac{\vv{t}_1}{2} +\frac{\vv{t}_2}{2^2}+\cdots +\frac{\vv{t}_e}{2^e} + \cdots$ for which all $\vv{t}_e \in \NN^2$, and $\norm{\vv{t}_1} < 2$ are precisely the points $(x,y) \in [0,1]^2$ lying above the line $x+y=1$, so we remove the triangle given by $x+y>1$.
At the next stage, the points with no expansion satisfying $\norm{\vv{t}_1} < 2$ and  $\norm{\vv{t}_2} < 2$ are those in the triangle given by $x, y < \frac{1}{2}$ and $x+y > \frac{1}{2}$.  
The condition on expansions at the third place removes three triangles from the remaining set, and we can continue on analogously. 

\Cref{fig: sierpinski 3-gasket} illustrates the self-similarity of the Sierpi\'nski $3$-gasket in $\RR^2$.  In general, notice that since each can be realized by removing a union of open sets from $\RR^n$, all $\sierp_{p,n}$ are closed sets.  
\begin{figure}
\begin{subfigure}{.49\textwidth}
  \centering
  \includegraphics[width=.9\linewidth]{Pictures/sierpinski3_a.pdf}
  \caption{Sierpi\'nski 3-gasket in $[0,1]^2$}
\end{subfigure}
\begin{subfigure}{.49\textwidth}
  \centering
  \includegraphics[width=.9\linewidth]{Pictures/sierpinski3_b.pdf}  
  \caption{Sierpi\'nski 3-gasket in $[0,9]^2$}
\end{subfigure}
\caption{}
\label{fig: sierpinski 3-gasket}
\end{figure}

Remarkably, the critical exponent of a monomial pair $(A, \vv{u})$ can be computed in terms of the Sierpi\'nski $p$-gasket, providing a geometric realization of this value.
Toward making this relation precise, consider the following optimization problem, which adds the extra ``fractal'' constraint from the definition of $\sierp_{p,n}$ to 
the linear program $\LP(A, \vv{u})$, after replacing the condition $A \vv{t} \leq \vv{u}$ with $A \vv{t} < \vv{u}$. 

\begin{definition}
Given a monomial pair $(A, \vv{u})$, the optimization problem $\fip_p(A,\vv{u})$ consists of maximizing $\vv{t}\mapsto \norm{\vv{t}}$, subject to the constraints $\vv{t} \geq \vv{0}$, $A\vv{t} < \vv{u}$, and $\vv{t}\in \sierp_{p,n}$. 
We define the feasible set for $\fip_p(A,\vv{u})$, $\feas \fip_p(A,\vv{u})$, to be the \emph{closure} of the set of all $\vv{t}\in \sierp_{p,n}$ satisfying $A\vv{t} < \vv{u}$.
% The value of the problem, $\val \fip_p(A,\vv{u})$, is defined as the supremum of $\norm{\vv{t}}$ among all $\vv{t} \in \feas \fip_p(A, \vv{u})$.  
\end{definition}

\begin{remark}
The points in $\vv{t}\in \sierp_{p,n}$ satisfying $A\vv{t} < \vv{u}$ are precisely those in the intersection of 
the feasible set of the linear program $\LP(A, \vv{u})$ with the Sierpi\'nski $p$-gasket.  
Since this intersection is a compact set, $\feas\fip_p(A,\vv{u})$ is also compact, and $\val \fip_p(A, \vv{u}) = \max\{ \norm{\vv{t}} : \vv{t} \in \feas \fip_p(A, \vv{u}) \}$, the value of $\fip_p(A, \vv{u})$,  exists and is finite.  
%Moreover, there is exists an optimal point
%$\vv{t} \in \feas \fip_p(A, \vv{u})$ such that $\norm{\vv{t}} = \val \fip_p(A, \vv{u})$.
\end{remark}

\begin{example} \label{ex: feas fip}
 Consider the optimization problem $\fip_p(A, \vv{u})$, where 
\[ A = \begin{pmatrix}
 3&11\\ 11&2 \\ 5&10 \\ 2&0
 \end{pmatrix} 
\text{ and } \vv{u} = \begin{pmatrix} 1 \\ 1 \\ 1 \\ \end{pmatrix}.
\]
In \Cref{fig: feas fip}, the feasible set for $\fip_p(A,\vv{u})$ is shown in blue, the feasible set for $\LP(A,\vv{u})$ is shown in gray, and the optimal set is highlighted in green, for small values of $p$.  
\begin{figure}
  \centering
\begin{subfigure}{.49\textwidth}
\centering
  \includegraphics[width=.8\textwidth]{Pictures/ex4_char2.pdf}\hskip .04\textwidth
\caption{$p=2$}  
\end{subfigure} 
\begin{subfigure}{.49\textwidth}
\centering
\includegraphics[width=.8\textwidth]{Pictures/ex4_char3.pdf}
\caption{$p=3$}  
\end{subfigure}
\begin{subfigure}{.49\textwidth}
\centering
  \includegraphics[width=.8\textwidth]{Pictures/ex4_char5.pdf}\hskip .04\textwidth
\caption{$p=5$}  
\end{subfigure}  
\begin{subfigure}{.49\textwidth}
\centering
  \includegraphics[width=.8\textwidth]{Pictures/ex4_char7.pdf}
\caption{$p=7$}  
\end{subfigure}  
\caption{The feasible set of $\Sigma_p(A, \vv{u})$ for $A$ and $\vv{u}$ described in \Cref{ex: feas fip}, for various $p$ values.}
\label{fig: feas fip}
\end{figure}
\end{example}

\begin{proposition}  
Given a monomial pair $(A, \vv{u})$, we have that 
\[\crit(A,\vv{u}) = \val\fip_p(A,\vv{u}).\]
\end{proposition}

\begin{proof}
For each $q=p^e$,  $\frac{1}{q}\cdot\feas\IP_p(A,\vv{u}, q)$ is contained in $\feas \fip_p(A,\vv{u})$, so 
\[
\val\fip_p(A,\vv{u}) \ge \displaystyle \lim_{q\to \infty}\frac{\val\IP_p(A,\vv{u}, q)}{q} = \crit(A,\vv{u}).
 \]
On the other hand, fix $\vv{t} \in \feas \LP(A, \vv{u}) \cap \sierp_{p,n}$, fix $q \geq 1$ a power of $p$, and a sequence $\{ \vv{t}_e \}_{e=1}^\infty$ in $\NN^n$, for which all $\norm{\vv{t}_e} < p$ and 
$\vv{t} = q \cdot \left( \frac{\vv{t}_1}{p} +\frac{\vv{t}_2}{p^2}+\cdots \right)$.  For $e \geq 1$, let 
 $\vv{t}^\star_{p^e} = q \cdot \left( \frac{\vv{t}_1}{p} + \cdots + \frac{\vv{t}_e}{p^e}  \right)$, so that $p^e \cdot \vv{t}^\star_{p^e} \in \IP_p(A, \vv{u}, p^e)$ and $\val \IP_p(A, \vv{u}, p^e) \geq p^e \norm{\vv{t}^\star_{p^e}}$. 
 Dividing by $p^e$ and taking limits, we find that  
 \[
\crit(A,\vv{u}) = \lim_{q\to \infty} \frac{\val \IP_p(A, \vv{u}, q)}{q} \geq \lim_{q \to \infty}   \norm{\vv{t}^\star_q} = \norm{\vv{t}}.
 \]
Finally, given $\vv{s} \in \feas \fip_p(A, \vv{u}, p)$, fix a sequence of points $\{ \vv{s}_e \}_{e=1}^\infty$ in $\feas \LP(A, \vv{u}) \cap \sierp_{p,n}$ that limit to $\vv{s}$. Since $\vv{t} \mapsto \norm{\vv{t}}$ defines a continuous function $\RR^n \to \RR$, we have that $\crit(A, \vv{u}) \geq \lim \limits_{e \to \infty} \norm{ \vv{s}_e } =  \norm{ \vv{s} }$.
Now, since $\vv{s}$ is an arbitrary element of $\feas \fip_p(A, \vv{u}, p)$, we have that
$\crit(A, \vv{u}) \geq \val \fip_p(A, \vv{u})$, and equality holds. 
\end{proof}

\newpage

\emily[inline]{

\textbf{Important Questions}.

\begin{enumerate}
 \item Does a medium-small pair always have a medium-small sprout?
 We think the answer is NO:  
 Let $A = \begin{bmatrix} 3 & 0 \\ 0 & 3 \end{bmatrix}$ and $\vv{u} = (2,2)$, so that 
$(A, \vv{u})$ is small but not very small.  Then the unique special point is $\vv{s} = (2/3,2/3)$, so that if $p=2 \bmod 3$, then $[\vv{s}]_p = (2[p\%3]/3, 2[p\%3]/3) = (1/3,1/3)$. 

The value of $\Theta(A, \vv{u}, \vv{s}, p)$ is $0$
using the bounds in this paper, and from this, we can find that the only element of $\Delta(A, \vv{u}, \vv{s}, p)$ is $(1,1)$, which is very small. 

\item Is it true that if some digit of a critical exponent of the monomial ideal $\ideala$ equals $p-1$, then \emph{all subsequent digits} must also be $p-1$.  This seems to be true if we run into a \emph{medium small} point.  Are there points $\vv{v}$ with $\mu(A,\vv{v}, p) = p-1$ where $(A,\vv{v})$ is not medium small?  Sure, look at $A = \begin{bmatrix} 2 & 0 \\ 0 & 2 \end{bmatrix}$ and $\vv{v} = (1,1)$.  Then our Frobenius examples paper should tell us that $(A, \vv{v})$ is very small but $\mu(A, \vv{u}, p)$ should equal $1$ often.
\item We saw earlier that a medium small pair need not sprout a medium small pair.  But does a pair $(A, \vv{u})$ that is small and satisfies $\mu(A, \vv{u}, p) = p-1$ then must it sprout a pair $(B, \vv{v})$ with $\mu(B, \vv{v}, p) = p-1$?

\item Pedro pointed out the better question is that if $(A, \vv{u})$ is small and $\mu(A, \vv{u}, p) = p-1$, then is the whole critical exponent $\crit(A, \vv{u}) = 1$?
\item The answer to the last question is FALSE:  In our Frobenius examples paper, there is a critical point $1-(1/p^2) = p-1 : p-2 : \overline{p-1}$.
\end{enumerate}  
}


{\small
\bibliographystyle{amsalpha}
\bibliography{bibdatabase}
}	


\end{document}
