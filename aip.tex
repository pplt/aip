\documentclass[11pt]{amsart}
%%%%%%%%%%%%%%%%%% Colors %%%%%%%%%%%%%%%%%%
\usepackage{xcolor}
\definecolor{nicered}{rgb}{0.6, 0, 0.1}
\definecolor{niceblue}{rgb}{0.06, 0.3, 0.57}
\definecolor{nicegreen}{rgb}{0.0, 0.51, 0.5}

%%%%%%%%%%%%%%%%%% Assorted Packages %%%%%%%%%%%%%%%%%%
\usepackage[colorlinks=true,pagebackref,hyperindex,citecolor=nicegreen,linkcolor=niceblue,urlcolor=nicered]{hyperref}
\usepackage{amsmath,amsthm,amsfonts,amssymb}
%\usepackage{color}
\usepackage{mathrsfs,stmaryrd,bm}
\usepackage[mathcal]{euscript}
\usepackage{mathtools,soul}
\usepackage{microtype}
\usepackage[shortlabels]{enumitem}
\usepackage{booktabs}
\usepackage{xspace}
\usepackage{caption,subcaption}
\captionsetup[subfigure]{subrefformat=simple,labelformat=simple}
\renewcommand\thesubfigure{(\sc \alph{subfigure})}
\usepackage[
ruled,
%linesnumbered,
vlined]{algorithm2e}

%shortfall and deficit
\newcommand{\short}{\operatorname{short}}
\newcommand{\ushort}{\operatorname{ushort}}
\newcommand{\deficit}{\operatorname{deficit}}
\newcommand{\udeficit}{\operatorname{udeficit}}

\newcommand{\denom}{d} 
\newcommand{\fsr}{\mathcal{R}}

\renewcommand{\S}{\mathcal{S}}
\newcommand{\pp}{\mathsf{p}}
\renewcommand{\tt}{\mathsf{t}}


\newcommand{\hooklongrightarrow}{\lhook\joinrel\longrightarrow}

\newcommand{\mspec}{\operatorname{mSpec}}
\newcommand{\spec}{\operatorname{Spec}}

%clever ref package
%must come before following 3 sections
\usepackage{cleveref}  %must be consistent with names in following 3 sections
\crefname{equation}{Eq.}{Eqs.}
\crefname{theorem}{Theorem}{Theorems}
\crefname{lemma}{Lemma}{Lemmas}
\crefname{corollary}{Corollary}{Corollaries}
\crefname{proposition}{Proposition}{Propositions}
\crefname{definition}{Definition}{Definitions}
\crefname{remark}{Remark}{Remarks}
\crefname{example}{Example}{Examples}
\crefname{notation}{Notation}{Notations}
\crefname{setup}{Setup}{Setups}
\crefname{question}{Question}{Questions}
\crefname{convention}{Convention}{Conventions}
\crefname{algorithm}{Algorithm}{Algorithms}
\newcommand{\creflastconjunction}{, and\nobreakspace}

 %theorem style environments
\newtheorem{theorem}{Theorem}[section]
\newtheorem{lemma}[theorem]{Lemma}
\newtheorem{corollary}[theorem]{Corollary}
\newtheorem{proposition}[theorem]{Proposition}
\newtheorem{thmintro}{Theorem}
\renewcommand{\thethmintro}{\Alph{thmintro}}

%definition style environments
\theoremstyle{definition}
\newtheorem{definition}[theorem]{Definition}
\newtheorem{setup}[theorem]{Setup}
\newtheorem{example}[theorem]{Example}

%remark style environments
\theoremstyle{remark}
\newtheorem{remark}[theorem]{Remark}
\newtheorem{notation}[theorem]{Notation}
\newtheorem{convention}[theorem]{Convention}
\newtheorem{problem}[theorem]{Problem}
\newtheorem*{claim}{Claim}

%numbering
\numberwithin{equation}{subsection} %Can replace {subsection} with {theorem} if you want

%spacing
%\usepackage{setspace}
%\singlespacing
%\onehalfspacing
%\doublespacing
%\setstretch{1.1}

%\setlength{\parskip}{0.4em}


%various thresholds
\DeclareMathOperator{\lct}{lct}
\DeclareMathOperator{\fpt}{fpt}
\newcommand{\ft}[2]{\operatorname{ft}(#1, #2)}

%ideals
\newcommand{\ideal}[1]{\langle #1 \rangle}
\newcommand{\ideala}{\mathfrak{a}}
\newcommand{\idealb}{\mathfrak{b}}
\newcommand{\ideald}{\mathfrak{d}}
\newcommand{\idealm}{\mathfrak{m}}
\newcommand{\idealp}{\mathfrak{p}}
\newcommand{\mon}{\operatorname{mon}}
\newcommand{\idealc}{\mathfrak{c}}
\newcommand{\J}{\mathcal{J}} % for multiplier ideals

%linear & integer programs
\newcommand{\LP}{\mathrm{P}}
\newcommand{\IP}{\Pi}
\newcommand{\ip}{\Theta}
\DeclareMathOperator{\im}{im}
\DeclareMathOperator{\opt}{opt}
\DeclareMathOperator{\val}{val}
\DeclareMathOperator{\feas}{feas}

%convexity
\DeclareMathOperator{\conv}{conv}
\DeclareMathOperator{\cone}{cone}
\DeclareMathOperator{\rb}{rb}
\DeclareMathOperator{\rs}{rs}
\DeclareMathOperator{\ri}{ri}

%euclidean space
\newcommand{\vv}[1]{\mathbf{#1}} %vectors
\newcommand{\iprod}[2]{\langle #1, #2 \rangle} %dot product
\newcommand{\norm}[1]{ \| #1 \| } % norm
\newcommand{\canvec}{\vv{e}}
\newcommand{\defpt}{\vv{c}}
%representation of rational numbers
\newcommand{\tail}[1]{\left[ #1 \right]}
\newcommand{\lpr}[2]{ [ \hspace{.01in} #1 \, \% \, #2 \hspace{.01in} ]} %least positive residue
\newcommand{\up}[1]{\left\lceil #1 \right\rceil} %ceiling
\newcommand{\down}[1]{\left\lfloor #1 \right\rfloor} %floor

%random
\DeclareMathOperator{\col}{col}
\DeclareMathOperator{\mf}{mf}
\renewcommand{\sp}{\operatorname{sp}}
%\DeclareMathOperator{\rep}{rep}
%\DeclareMathOperator{\lis}{list}
\newcommand{\Q}{\mathcal{Q}}
%\newcommand{\N}{\mathrm{N}}
\newcommand{\N}{\mathcal{N}}
\newcommand{\M}{\mathcal{M}}
\renewcommand{\O}{\mathcal{O}}
\newcommand{\Z}{\mathcal{Z}}

% newly-defined commands
\DeclareMathOperator{\diag}{diag}
\DeclareMathOperator{\crit}{crit}
\newcommand{\orep}{\mathbb{O}}
\newcommand{\witt}{\mathfrak{W}}
\newcommand{\graph}{\mathfrak{S}}
\newcommand{\sierp}{\mathscr{S}}
\newcommand{\fip}{\Sigma}
\DeclareMathOperator{\sprout}{sprout}
\newcommand{\sproutsfrom}[2]{#1 \leftarrow #2}
%\newcommand{\sproutsfrom}[2]{#1 \in \operatorname{sp} #2}
\newcommand{\collapse}{\widebar}

%sets
\newcommand{\kk}{\Bbbk}
\newcommand{\LL}{\mathbb{L}}
\newcommand{\FF}{\mathbb{F}}
\newcommand{\RR}{\mathbb{R}}
\newcommand{\RRnn}{\mathbb{R}_{\ge 0}}
\newcommand{\CC}{\mathbb{C}}
\newcommand{\ZZ}{\mathbb{Z}}
\newcommand{\QQ}{\mathbb{Q}}
\newcommand{\NN}{\mathbb{N}}
\renewcommand{\emptyset}{\varnothing}

\newcommand{\numvars}{m}

%inequalities
\renewcommand{\geq}{\geqslant}
\renewcommand{\leq}{\leqslant}
\renewcommand{\ge}{\geqslant}
\renewcommand{\le}{\leqslant}

%abbreviations
\newcommand{\cf}{\emph{cf}.\ }
\newcommand{\eg}{e.g., }
\newcommand{\ie}{i.e., }
\newcommand{\loccit}{\emph{loc.~cit.}}
\newcommand{\vs}{vs.\ }

\newcommand{\muCool}{$\mu$-uniform\xspace}
\newcommand{\nuCool}{$\nu$-uniform\xspace}
\newcommand{\mustata}{Musta{\c{t}}\u{a}\xspace}


%notes
\usepackage[textwidth=3.3 cm,textsize=small,shadow
%disable
%%option disable removes the notes
]{todonotes}
\newcommand{\comment}[2][]{\todo[linecolor=orange,backgroundcolor=orange!30!,caption={}, #1]{#2}} % color-name! intensity !
\newcommand{\alert}[2][]{\todo[linecolor=red,backgroundcolor=red!50!,caption={}, #1]{#2}} % color-name! intensity !
\newcommand{\details}[2][]{\todo[linecolor=cyan,backgroundcolor=cyan!40, caption={},#1]{#2}}

\newcommand{\emily}[2][]{\todo[linecolor=green,backgroundcolor=green!30!,caption={}, #1]{#2}}
\newcommand{\daniel}[2][]{\todo[linecolor=blue,backgroundcolor=blue!30!,caption={}, #1]{#2}}
\newcommand{\pedro}[2][]{\todo[linecolor=nicegreen,backgroundcolor=nicegreen!70!,caption={}, #1]{#2}}

%editing
%\renewcommand{\!}[1]{{\color{red}\text{$\star$\,}#1\,$\star$}}
\newcommand{\ol}[1]{\overline{#1}}

% Decent looking bars (by Hendrik Vogt)
\makeatletter
\let\save@mathaccent\mathaccent
\newcommand*\if@single[3]{%
  \setbox0\hbox{${\mathaccent"0362{#1}}^H$}%
  \setbox2\hbox{${\mathaccent"0362{\kern0pt#1}}^H$}%
  \ifdim\ht0=\ht2 #3\else #2\fi
  }
%The bar will be moved to the right by a half of \macc@kerna, which is computed by amsmath:
\newcommand*\rel@kern[1]{\kern#1\dimexpr\macc@kerna}
%If there's a superscript following the bar, then no negative kern may follow the bar;
%an additional {} makes sure that the superscript is high enough in this case:
\newcommand*\widebar[1]{\@ifnextchar^{{\wide@bar{#1}{0}}}{\wide@bar{#1}{1}}}
%Use a separate algorithm for single symbols:
\newcommand*\wide@bar[2]{\if@single{#1}{\wide@bar@{#1}{#2}{1}}{\wide@bar@{#1}{#2}{2}}}
\newcommand*\wide@bar@[3]{%
  \begingroup
  \def\mathaccent##1##2{%
%Enable nesting of accents:
    \let\mathaccent\save@mathaccent
%If there's more than a single symbol, use the first character instead (see below):
    \if#32 \let\macc@nucleus\first@char \fi
%Determine the italic correction:
    \setbox\z@\hbox{$\macc@style{\macc@nucleus}_{}$}%
    \setbox\tw@\hbox{$\macc@style{\macc@nucleus}{}_{}$}%
    \dimen@\wd\tw@
    \advance\dimen@-\wd\z@
%Now \dimen@ is the italic correction of the symbol.
    \divide\dimen@ 3
    \@tempdima\wd\tw@
    \advance\@tempdima-\scriptspace
%Now \@tempdima is the width of the symbol.
    \divide\@tempdima 10
    \advance\dimen@-\@tempdima
%Now \dimen@ = (italic correction / 3) - (Breite / 10)
    \ifdim\dimen@>\z@ \dimen@0pt\fi
%The bar will be shortened in the case \dimen@<0 !
    \rel@kern{0.6}\kern-\dimen@
    \if#31
      \overline{\rel@kern{-0.6}\kern\dimen@\macc@nucleus\rel@kern{0.4}\kern\dimen@}%
      \advance\dimen@0.4\dimexpr\macc@kerna
%Place the combined final kern (-\dimen@) if it is >0 or if a superscript follows:
      \let\final@kern#2%
      \ifdim\dimen@<\z@ \let\final@kern1\fi
      \if\final@kern1 \kern-\dimen@\fi
    \else
      \overline{\rel@kern{-0.6}\kern\dimen@#1}%
    \fi
  }%
  \macc@depth\@ne
  \let\math@bgroup\@empty \let\math@egroup\macc@set@skewchar
  \mathsurround\z@ \frozen@everymath{\mathgroup\macc@group\relax}%
  \macc@set@skewchar\relax
  \let\mathaccentV\macc@nested@a
%The following initialises \macc@kerna and calls \mathaccent:
  \if#31
    \macc@nested@a\relax111{#1}%
  \else
%If the argument consists of more than one symbol, and if the first token is
%a letter, use that letter for the computations:
    \def\gobble@till@marker##1\endmarker{}%
    \futurelet\first@char\gobble@till@marker#1\endmarker
    \ifcat\noexpand\first@char A\else
      \def\first@char{}%
    \fi
    \macc@nested@a\relax111{\first@char}%
  \fi
  \endgroup
}
\makeatother

\usepackage{caption}
\usepackage{booktabs}
\usepackage{subcaption}
\captionsetup[subfigure]{subrefformat=simple,labelformat=simple}
\renewcommand\thesubfigure{(\sc \alph{subfigure})}
\usepackage{bm}

%shortfall and deficit
\newcommand{\short}{\operatorname{short}}
\newcommand{\deficit}{\operatorname{deficit}}


\begin{document}

\title[Fractal and arithmetic programs]{Fractal programs, arithmetic programs, and the Frobenius powers of monomial ideals}
\author{Daniel J.~Hern\'andez}
\author{Pedro Teixeira}
\author{Emily E.~Witt}
\maketitle

\newcommand{\denom}{\ell} %We should stick to the same notation for the denominator of a special point.  Sometimes D is used, but I would prefer a lowercase letter.  Lowercase "d" makes sense, but we use that for the dimension of the ambient polynomial ring.

\pedro[inline]{What if we use $m$ for the number of variables/rows, to free $d$ to be the preferred notation for denominators?
\emily[inline]{I replaced instances of $d$ with the command {\tt \text{numvars}}, though please change it if you notice have missed some--I set it to $m$ in the meantime. It does look like $m$ is used in several other ways, so I think now I'll vote to keep ``$d$.''
}}
\pedro[inline]{
   I like $m$, especially in ``$m\times n$ matrix'' (though we'll have to search and replace the article ``a'' with ``an'').
   I noticed $m$ being used (in other contexts) in the following places:
   \begin{itemize}
      \item In the discussion after the definition of Sierpinski gasket; that could be replaced with anything else ($s$, for example).
      \item In the definition of least positive residue; it would be OK to leave it alone, or replace it with anything else.
      \item $\vv{m}$ appears in the proofs of \Cref{optimal division: L,ILL: T}; again it would be OK to leave it alone, or replace it with anything else.
      \item In \Cref{colon-product-stabilization: L} and in the proof of \Cref{monomial-noetherian-decomposition: L}. Those would need to be replaced.
   \end{itemize}
   It also appeared in the general discussion of optimization, but I replaced it with $n$, as our optimization problems are in $\RR^n$ and $\ZZ^n$.   
}
\emily[inline]{I am commenting out the items on the list we've done that don't look like we need for reference.}

\vspace{.3cm}

\newcommand{\CheckedBox}{\text{\rlap{$\checkmark$}}\Box}
\details[inline]{
TO DO LIST:
\begin{enumerate}
%\item[$\CheckedBox$] Write essential statements/proofs.
%\item[$\CheckedBox$] Decide on name of paper.
\item[$\CheckedBox$] Decide on name for minimal coordinate.
 \comment[inline]{Going with ``special point''.}
\item[$\CheckedBox$] Make sprout notation.
\comment[inline]{Going with $\sprout(A,\vv{u},p)$}
\item[$\Box$] Put in examples.
\pedro[inline]{Started working on it.}
 \item[$\Box$] Decide on how to define $\widehat{\witt}$.  (Sprouting graph?)
 %\item[$\CheckedBox$] Fill in/rewrite preliminaries.
 %\item[$\CheckedBox$] Reorganize and motivate the $\IP$ and $\ip$, and their connection, in Sections 5 and 6.
 %\item[$\CheckedBox$] ``Fractal linear program;'' solve $P$ in ``Sierpinski gasket''
 \item[$\Box$] Direct proof that $\delta$/$\Delta$ are independent of $\vv{s}$.
 \item[$\Box$] Minimize and/or ``algebrafy'' statements.
 %\item[$\CheckedBox$] Background section on Frobenius powers, $\mu$, $\nu$, etc.
 \item[$\Box$] Derive some easy corollaries for very general hypersurfaces.
 \item[$\CheckedBox$] Generalize the definition of $\IP$ so that $\IP(A, \vv{u}q)$ becomes $\IP(A, \vv{u}, q)$?
 \comment[inline]{Changed to $\IP(A, \vv{u}, q)$}
 \item[$\Box$] Replace ``Lemma'' with ``Proposition'', if there's no immediate application in sight.
 \emily[inline]{I went carefully through the current doc (4/14/20) and found that \Cref{refined-discreteness: L} is not explicitly cited.  The lemma right after \Cref{cor: constant mus} not currently cited, but a note before it gives a consequence we should add.  \Cref{noncontainment mod p: L} appears only to be mentioned in \Cref{mon-operation-modulo-p: T}, and the proof of the latter recovers a special case of \Cref{noncontainment mod p: L}.  All other lemmas are used to prove subsequent statements.}
\end{enumerate}
}

\details[inline]{
   TO-DO LIST (Continued):
   \begin{itemize}
      \item[$\Box$] Add remark describing how to verify inequalities $A\vv{k} < \vv{v}$ using projections/collapse.
      %\item[$\CheckedBox$] Add remark that points out that $\collapse{A\vv{k}} = \collapse{A}\vv{k}$.
      %\item[$\CheckedBox$] In Section 4, we should talk about both $\nu$'s and $F$-thresholds. We could then introduce $\IP(A,\vv{u},q)$, immediately relate its value to the $\nu$'s, and relate $\val \LP(A,\vv{u})$ to the normalized limit of $\IP(A,\vv{u},q)$, just like we do later for the fractal programs $\fip_p$ and the arithmetic programs $\IP_p$.  Most of this has already been written, and it should just be moved from Section 5.
      %\item[$\CheckedBox$] Similarly, introduce $\fip_p$ and $\IP_p$ together, and describe their connection right away.
      %\item[$\CheckedBox$] Prior to Definition 4.1, state our goal for describing $F$-thresholds and $\nu$'s.  Use this to motivate definition of monomial pair
      %%%%% Notation %%%%%
      \item[$\Box$] Uniformize $\collapse{A(X)} = \collapse{A}(X)$.
      \item[$\CheckedBox$] Use $\vv{d}$ for defining points?
      %Pro: opens up $\vv{a}$ for columns of $A$.  Con:  maybe $\collapse{\vv{d}}$ looks ugly?  Another con is $d$ is used for number of variables and $\RR^\numvars$.
      Emily suggest using $\vv{c}$.  This is probably a good idea.
      \pedro[inline]{I macroed the defining points as \texttt{defpt}, and currently defined it as $\vv{c}$.}
      %\item[$\CheckedBox$] Uniformize the way we describe programs (e.g., as in Section 9).
      \item[$\CheckedBox$] Uniformize the way we introduce collapses in statements.
      Daniel now prefers saying ``$\collapse{X}$ denotes the collapse of $X$ along $\O$\ldots'' without specifying what $X$ is; Pedro proposed to simply write ``Let the superscript bar denote collapse along\ldots''  I think we agreed Pedro's way is better.
      \pedro[inline]{Gave it a shot. Apparently, ``overbar'', ``overline'', and ``overscore'' are all accepted words for the superscript bar. I used ``overbar'', but feel free to change it, if you prefer a different term.}
      \item[$\Box$] Right now, we are referring to ``the'' monomial matrix associated to a monomial idea.  But, we don't want to restrict to associating columns to minimal generators, since this may not be preserved by collapse.  So, we don't seem to have a canonical monomial matrix, just one associated to every set of generators.
      \emily[inline]{Would ``a monomial matrix'' be OK?}
      \pedro[inline]{That sounds good to me.}
      \item[$\CheckedBox$] Label $\canvec_1$, $\canvec_2$, $\canvec_3$ in Figure 1, and other figures.
      \item[$\CheckedBox$] Change ``image'' to ``optimal image'', at least in definitions. optim instead of im?
      \item[$\Box$] Remove image of arithmetic program, restate things algebraically.
      %\item[$\CheckedBox$] Draw entire line of $\norm{\vv{s}} = \val \fip_p$.
      \item[$\Box$] Add the analog to Corollary 2.4 for $F$-thresholds and test ideals?
      \item[$\Box$] Does 7.10 come from 7.3, if we replace 1 with $q$ and $\IP$ with $\IP_p$? If so, we should write the proof this way, and replace the comment preceding 7.11 with a remark (that makes clear the reduction to the small case)
      \item[$\Box$] Does Corollary 8.3 come from 7.10 too?
      \item[$\Box$] Identify where ``equal'' and ``equivalent'' linear/integer programs are pointed out, and define this at that point.
   \end{itemize}
}

\newpage

% \emily[inline]{I've been thinking about the readability of the paper (including for the referee), and since it is also going to be quite long, I propose to move (at least some of) the preliminaries to an appendix.   Then we could get to ``new math'' sooner. What do you think?}
% \pedro[inline]{
%    An appendix on convex geometry would make sense to me; it could include most of what we have here in 1.1, and perhaps more (\eg definition and characterizations of faces, vertices, etc.)
%    Other things, like the terminology associated to linear programs and multinomial coefficients, could be spread through the paper, ``on demand''.
%    Maybe the rest should stay here? (Notation and conventions we use all the time, Newton polyhedra, monomial matrices/ideals\ldots)
% }








% \subsection{Linear programming}
%
% \ \pedro[inline]{
%    Since we are trying to minimize these preliminaries and get to the point, I think we can ditch this section.
%    We can just quickly introduce the terminology and notation after defining the first linear/integer program, and say something to the effect that this terminology and notation will be carried out to other optimization problems we'll consider.
% }
%
% Let $\mathbb{D}$ be either $\RR$ or $\ZZ$.
% A \emph{linear program} $\Pi$ in $\mathbb{D}^n$ is an optimization problem in which one seeks to maximize a fixed linear \emph{objective function} $\RR^n \to \RR$ on a subset of $\mathbb{D}^n$ defined by a fixed system of linear inequalities.
% We refer to this subset as the \emph{feasible set} of $\Pi$, and denote it $\feas \Pi$, and we refer to the inequalities defining it as the \emph{constraints} of $\Pi$.
% We say that the points of $\feas \Pi$ are \emph{feasible for $\Pi$}.
% When $\mathbb{D} = \ZZ$, we refer to $\Pi$ as an \emph{integer linear program}, or simply \emph{integer program}, for short.
%
% If $\mathbb{D} = \RR$, then the feasible set is a polyhedron in $\RR^n$, and if $\mathbb{D} = \ZZ$, the feasible set is the set of lattice points in a polyhedron in $\RR^n$.
%
% In this article, we only consider linear programs in which the objective function restricted to the feasible set attains a maximum (\eg this occurs whenever the constraints define a polytope).
% In this case, a feasible point is \emph{optimal} if it maximizes the objective function, and the \emph{value} of the program is the
% value of the objective function at an optimal point.
% We use $\opt \Pi$ to denote set of optimal points of a linear program $\Pi$, and $\val \Pi$ to denote the value of $\Pi$.
%
% There are several reasonable notions of equality for integer programs.
% In this article,  we say that two integer programs are \emph{equivalent} if their objective functions are identical and their feasible sets agree,
% and are \emph{equal} if their objective functions and defining constraints are identical.


% \subsection{Euclidean spaces, convexity, and polyhedra}
% \label{ss: euclidean spaces and convexity}
%
% We review in this subsection some of the terminology, notation, and constructions concerning Euclidean spaces and convex geometry used throughout the paper.
% We use bold-face lower-case letters to denote points of the Euclidean space $\RR^n$, and the same letter, in regular font, to represent their coordinates (\eg $\vv{v}=(v_1,\ldots,v_n)$).
% The points $(0,\ldots,0)$ and $(1,\ldots,1)$ are denoted $\vv{0}$ and $\vv{1}$, and the standard basis vectors of $\RR^n$ are denoted $\canvec_1,\ldots,\canvec_n$.
%
% Given a point $\vv{u}\in \RR^n$, $\norm{\vv{u}}$ denotes its coordinate sum, $u_1+\cdots+u_n$.
% The standard inner product in $\RR^n$ is denoted by the usual angle brackets: $\iprod{\vv{u}}{\vv{v}} = u_1v_1 + \cdots + u_nv_n$.
% An inequality between points of $\RR^n$ is a shorthand for a system of $n$ coordinatewise inequalities; for instance, $\vv{u}\le \vv{v}$ means that $u_i \le v_i$ for each $i=1,\ldots,n$.
% In the same vein, operations on numbers are extended to points in $\RR^n$ in a coordinatewise fashion; for instance, $\up{\vv{u}}=(\up{u_1},\ldots,\up{u_n})$.
%
% We say that a point $\vv{u}\in \RR^n$ is positive (respectively, nonnegative) if $\vv{u} > \vv{0}$ (respectively, $\vv{u}\ge \vv{0}$).
% More generally, given a point $\vv{u}$ in a coordinate subspace $\mathcal{S}$ of $\RR^n$, we say that $\vv{u}$ is \emph{positive in $\mathcal{S}$} (respectively, \emph{nonnegative in $\mathcal{S}$}) if $\vv{u}$ is a positive (respectively, nonnegative) linear combination of the standard basis vectors that span $\mathcal{S}$.
%
% Turning to concepts and constructions of convex geometry, a (convex) \emph{polyhedron} in $\RR^n$ is a subset of $\RR^n$ obtained by intersecting finitely many closed halfspaces or, equivalently, a set consisting of all points $\vv{x}\in \RR^n$ satisfying an inequality of the form $A\vv{x}\le \vv{b}$, where $A$ is a matrix with $n$ columns.
%
% The (convex) \emph{cone generated by $\vv{u}_1,\ldots,\vv{u}_k \in \RR^n$}, denoted $\cone(\vv{u}_1,\ldots,\vv{u}_k)$, is the set consisting of all \emph{conical combinations} of $\vv{u}_1, \ldots, \vv{u}_k$, that is, points of the form $\sum_{i=1}^k \lambda_i \vv{u}_i$, where the $\lambda_i$ are nonnegative real numbers.
% Likewise, the \emph{convex hull of $\vv{u}_1,\ldots,\vv{u}_k$}, denoted $\conv(\vv{u}_1,\ldots,\vv{u}_k)$, is the set of all \emph{convex combinations} of $\vv{u}_1, \ldots, \vv{u}_k$, that is, points of the form $\sum_{i=1}^k \lambda_i \vv{u}_i$, where the $\lambda_i$ are nonnegative and $\sum_{i=1}^k \lambda_i = 1$.
% The convex hull of a finite set of points is called a \emph{polytope}.
%
% If $\mathcal{U}$ and $\mathcal{V}$ are subsets of $\RR^n$, their \emph{Minkowski sum} is the set
% \[\mathcal{U}+\mathcal{V} \coloneqq \{\vv{u}+\vv{v}: \vv{u}\in \mathcal{U}\text{ and }\vv{v}\in \mathcal{V}\}.\]
% The \emph{Minkowski--Weyl Theorem} asserts that a subset $\mathcal{P}$ of $\RR^n$ is a polyhedron if and only if $\mathcal{P}$ is the Minkowski sum of a polytope and a finitely generated cone.
% The cone in this decomposition is the set of all directions $\vv{d} \in \RR^n$ in which $\mathcal{P}$ recedes, that is, $\vv{c} + \lambda \vv{d} \in \mathcal{P}$ for every $\vv{c} \in \mathcal{P}$ and $\lambda > 0$; it is uniquely determined by $\mathcal{P}$, and called the \emph{recession cone of $\mathcal{P}$}.
%
% The Minkowski--Weyl Theorem gives us a couple of useful characterizations of polytopes: a polyhedron $\mathcal{P}$ is a polytope if and only if it is a bounded polyhedron or, equivalently, a polyhedron with a trivial recession cone.
%
% %\pedro[inline]{
% %   Maybe we should gather what we need about faces and vertices of polyhedra right here.
% %}
% %\daniel[inline]{I'm not so sure about this.  At the moment, I feel like it is less distracting to just remind the reader of something (beyond the absolute basic definitions already covered) at the time they are used.  But, I could be convinced otherwise}
% %\pedro[inline]{
% %   Yes, maybe it's more efficient to introduce what we need ``on demand'', so the reader does not need to be coming back to this section all the time.
% %   That said, I think \emph{some} definition (perhaps the most generic definition) of face and vertex should be given here, for completeness (but maybe not every fact or every characterization we need).
% %}
% The \emph{relative interior} of a subset $\mathcal{U}$ of $\RR^n$, denoted $\ri \mathcal{U}$, is its interior relative to the smallest affine subset of $\RR^n$ containing $\mathcal{U}$.
% % \pedro{
% %    Is there a more concrete characterization for polyhedra/polytopes?
% %    E.g., points not in any proper face? Positive convex combinations of vertices?
% % }
% % \daniel{Yep!  Points not in any proper face.  If $S$ is a finite set with $\mathcal{P} = \conv(S)$, then $\ri \mathcal{P}$  consists of all points of the form $\sum_{\vv{s} \in S} \lambda_{\vv{s}} \vv{s}$ where the coefficients $\lambda_{\vv{s}}$ are positive, and sum to $1$.  So in particular, you could take $S$ to be the vertex set of $\mathcal{P}$.  Similarly, if $S$ is finite and $\mathcal{P} = \cone(S)$, then $\ri \mathcal{P}$ has a similar description, but we don't require that the coefficients sum to $1$. But, as I mentioned above, I'm not sure whether it is better to gather things here, or just mention them as we go along.}
% For later use, we observe that, when restricted to convex sets, the relative interior operator commutes with Minkowski sums: if $\mathcal{U}$ and $\mathcal{V}$ are convex subsets of $\RR^n$, then $\ri(\mathcal{U}+\mathcal{V})=\ri \mathcal{U}+\ri \mathcal{V}$.
%
% \begin{proposition}
%    \label{bounded polytope: P}
%    Let $\vv{c}$ and $\vv{u}$ be points in $\RR^n$, and suppose that $\vv{c}$ has positive coordinates.
%    If $\alpha$ is any real number, then the polyhedron consisting of all points $\vv{v} \in \RR^n$ such that  $\vv{v} \le \vv{u}$ and $\iprod{\vv{c}}{\vv{v}} \geq \alpha$ is bounded.
% \end{proposition}
%
% \begin{proof}
%    It suffices to show that the given set is bounded from below.
%    For each $\vv{v}$ in that set and each $i$ we have $\vv{v}\le \vv{u} + (v_i - u_i)\canvec_i$.
%    As $\vv{c}$ has positive coordinates, $\alpha\le \iprod{\vv{c}}{\vv{v}}\le \iprod{\vv{c}}{\vv{u} + (v_i -u_i)\canvec_i} =
%   \iprod{\vv{c}}{\vv{u}} + c_i(v_i - u_i)$, so $v_i \ge (\alpha + c_iu_i - \iprod{\vv{c}}{\vv{u}})/c_i$.
% \end{proof}
%
% We conclude this subsection with a useful technical result.
% Though variations of this proposition are well known, we include a simple proof, for lack of an appropriate reference.
%
% \begin{proposition}
% \label{vertex: P}
% Let $M$ be an $m \times n$ matrix and let $\vv{b} \in \RR^m$ be a point contained in the cone generated by the columns of $M$.  If $\Q$ is the polyhedron in $\RR^n$  consisting of all points $\vv{t}$ with $\vv{t} \geq \vv{0}$ and $M \vv{t} = \vv{b}$, then a point $\vv{t}^{\ast} \in \Q$ is a vertex of $\Q$ if and only if the columns of $M$ corresponding to the nonzero coordinates of $\vv{t}^{\ast}$ are linearly independent.  %In particular, $\Q$ contains a vertex.
% \end{proposition}
%
% \begin{proof}
%    The fact that $\vv{b}$ lies in the cone generated by the columns of $M$ implies that $\Q$ is nonempty.
%    Fix a point $\vv{t}^{\ast} \in \Q$.
%    Before proceeding, recall that $\vv{t}^{\ast}$ is a vertex of $\Q$ if and only if an expression of $\vv{t}^{\ast}$ as a convex combination of points $\vv{r}$ and $\vv{s}$ in $\Q$ is only possible when $\vv{r}=\vv{s}=\vv{t}^{\ast}$.
%
%    First, assume that the columns of $M$ corresponding to the nonzero coordinates of $\vv{t}^{\ast}$ are linearly independent, and suppose that $\vv{t}^{\ast} = \lambda \vv{r} + \mu \vv{s}$ is a convex combination of points $\vv{r}, \vv{s} \in \Q$.
%    Since $\vv{r},\vv{s}\ge \vv{0}$, the $i$-th coordinate of $\vv{r}$ and of $\vv{s}$ are zero whenever the $i$-th coordinate of $\vv{t}^{\ast}$ is zero.
%    On the other hand, the fact that $\vv{r}$ and $\vv{s}$ lie in $\Q$ also implies that
%    \[ M \vv{t}^{\ast} = \vv{b} = M \vv{r} = M \vv{s}, \]
%    and the assumption that the columns of $M$ corresponding to the nonzero coordinates of $\vv{t}^{\ast}$ are linearly independent then implies that $\vv{r}=\vv{s}=\vv{t}^{\ast}$.
%
% Next, suppose that the columns of $M$ corresponding to the nonzero coordinates of $\vv{t}^{\ast}$ are linearly dependent.   In this case, we may fix a nonzero point $\vv{k} \in \RR^n$ with the property that $M \vv{k} = \vv{0}$, and such that the $i$-th coordinate of $\vv{k}$ is zero whenever the $i$-th coordinate of $\vv{t}^{\ast}$ is zero.  We claim that if $\varepsilon > 0$ is sufficiently small, then the points $\vv{t}^{\ast} \pm \varepsilon \vv{k}$ must lie in $\Q$.   As $\vv{t}^{\ast}$ is a convex combination of these points, it will then follow that $\vv{t}^{\ast}$ is not a vertex of $\Q$.  Towards the claim, note that $M(\vv{t}^{\ast} \pm \varepsilon \vv{k}) = M \vv{t}^{\ast} = \vv{b}$ for every $\varepsilon > 0$.  On the other hand, the condition relating the coordinates of $\vv{t}^{\ast}$ and $\vv{k}$ guarantees that $\vv{t}^{\ast} \pm \varepsilon \vv{k}$ is nonnegative for all $0 < \varepsilon \ll 1$.
% \end{proof}
%
% \pedro[inline]{
%    An alternative to the previous proposition is the following result, which we could just mention and give a reference (it appears in several books):
%
%    \begin{proposition}
%       Let $\mathcal{P}$ be the polyhedron defined by a system of inequalities $A \vv{x} \le \vv{b}$, where $A\in \RR^{m\times n}$, and $\vv{v}$ a vertex of $\mathcal{P}$.
%       Then there exists $I \subseteq \{1,\ldots,m\}$ such that $\vv{v}$ is the unique solution to the system $A_I \vv{x} = \vv{b}_I$, where $A_I$ and $\vv{b}_I$ are obtained by selecting the $i$-th rows of $A$ and $\vv{b}$, for each $i\in I$.
%    \end{proposition}
%
%    Then \Cref{uniform denominators for vertices:  T} can be approached as follows:
%    By \Cref{opt set: P}, $\opt \LP(A,\vv{u})$ is defined by $A\vv{s} \le \vv{u}$ and $\vv{s}\ge \vv{0}$, with equality in some specific coordinates, and thus defined by a system of inequalities $B \vv{x} \le \vv{b}$, where $\vv{b}$ is an integral vector and $B$ is a submatrix of the matrix $M$ obtained by stacking $A$, $-A$, the identity matrix $I_n$, and $-I_n$.
%    Let $\denom$ be the least common multiple of the nonzero minors of $M$; then by the above result, every vertex of $\opt \LP$ is rational, with denominator $\denom$.
%
%    \bigskip
%
%    Hope this makes sense; if so, then I think this argument is slightly simpler, avoiding the linear bijection business.
% }
%
% \subsection{Monomial ideals, monomial matrices, and Newton polyhedra}
% \label{monomial newton preliminaries: ss}
% We work in the polynomial ring $\kk[x_1, \ldots, x_\numvars]$ over a field $\kk$, and adopt standard notation for describing monomials in this ring:  If $\vv{u} \in \NN^\numvars$, then \[ x^{\vv{u}} = x_1^{u_1} \cdots x_\numvars^{u_\numvars}.\]
%  \daniel{Return to this and add whatever we need to to support our discussion in \Cref{sec: LPs}.}
%
% A \emph{monomial matrix} is a matrix over $\ZZ$ with nonnegative, nonzero rows and columns.
% If $A$ is a $d \times n$ monomial matrix, then we call $\ZZ^n$ the \emph{domain lattice}, and $\ZZ^\numvars$ the \emph{target lattice}, of $A$.
%
% The \emph{Newton polyhedron} of a monomial matrix $A$ with $d$ rows is the polyhedron in $\RR^\numvars$ given by
% \[ \N = \conv( \col(A) ) + \cone( \canvec_1, \ldots, \canvec_\numvars), \]
% where $\col(A)$ is the set of columns of $A$.
%
% Recall that a proper subset $\O$ of $\N$ is a \emph{face} of $\N$ if there exists $\defpt \in \RR^\numvars$ and $\alpha \in \RR$ are such that $\iprod{\defpt}{\vv{v}} \geq \alpha$ for all $\vv{v} \in \N$, with equality if and only if $\vv{v} \in \O$.
% We say that such a point $\defpt$ \emph{defines} $\O$ in $\N$.  In this article, we are largely concerned with faces $\O$ of $\N$ that do not lie in any coordinate subspace of $\RR^\numvars$, which we call \emph{standard}.
%
% \begin{convention}
% \label{alpha=1: convention}
% Let $\O$, $\defpt \in \RR^\numvars$, and $\alpha \in \RR$ be as above.  If $\O$ is standard, then $\alpha$ must be positive, which allows us to rescale $\defpt$ so as to assume that $\alpha = 1$.   Thus, throughout this article, we always assume that we have normalized in this way when considering defining points of standard faces.
% \end{convention}
%
% \Cref{alpha=1: convention} leads to the following useful observation.
%
% \begin{proposition}\label{prop: inner product with columns of A}
%    If $\defpt \in \RR^\numvars$ defines a standard face $\O$ of $\N$, and $\vv{s} \in \RR^n$ has nonnegative coordinates, then $\iprod{\defpt}{A\vv{s}} \geq \norm{\vv{s}}$,  and equality holds if and only if $s_i = 0$ whenever the $i$-th column of $A$ is not in $\O$.
% \end{proposition}
%
% \begin{proof}
% If $\vv{a}_i$ denotes the $i$-th column of $A$, then \Cref{alpha=1: convention} and the nonnegativity of $\vv{s}$ imply that $\iprod{\defpt}{\vv{a}_i} \cdot s_i \geq s_i$ for every $1 \leq i \leq n$, with equality if and only if $s_i = 0$ or $\vv{a}_i \in \O$.
% Thus,
% \[ \iprod{\defpt}{A\vv{s}} = \sum_{i=1}^n \iprod{\defpt}{\vv{a}_i} \cdot s_i \geq  \sum_{i=1}^n s_i  = \norm{\vv{s}},\]
% and equality holds if and only if $s_i = 0$ whenever $\vv{a}_i \notin \O$.
% \end{proof}
%
% The following proposition is well known to experts, but we include the short proof to keep the article self-contained.
%
% \begin{proposition}
%    \label{face: P}
%    If $\defpt \in \RR^\numvars$ defines a face $\O$ of a Newton polyhedron $\N$, then $\defpt$ is nonnegative, and the $i$-th coordinate of $\defpt$ is zero if and only if $\vv{u} + \lambda \canvec_i \in \O$  for every $\vv{u} \in \O$ and $\lambda > 0$.
%    In particular, the supporting indices of $\defpt$ depend only on $\O$, and $\O$ is bounded if and only if $\defpt$ is positive.
% \end{proposition}
%
% \begin{proof}
%    If $\vv{u} \in \O$, then adding to $\vv{u}$ any nonnegative point in $\RR^\numvars$ produces a point in $\N$.
%    In particular, if $\iprod{\defpt}{\vv{u}} = \alpha$, then $\iprod{\defpt}{\vv{u} + \lambda \canvec_i} \geq \alpha$ for every standard basis vector $\canvec_i$ in $\RR^\numvars$ and $\lambda > 0$.
%    This observation implies that $a_i = \iprod{\defpt}{\canvec_i} \ge 0$ for each $i$, so $\defpt \geq \vv{0}$, and that $\vv{u} + \lambda \canvec_i \in \O$ for every $\lambda > 0$ if and only if $\iprod{\defpt}{\canvec_i} = 0$.
%
% Similar logic will show that if $\rb(\O) \coloneqq  \{ \canvec_i \in \RR^\numvars : \iprod{\defpt}{\canvec_i} = 0\}$, then
% \begin{equation}
% \label{face: e}
% \O =  \conv( \col(A) \cap \O ) + \cone(\rb(\O))
% \end{equation}
% where we agree that the $\cone(\emptyset) = \{\vv{0}\}$.  We see from this that $\O$ is bounded if and only if $\rb(\O)$ is empty, which is equivalent to the third assertion.
% \end{proof}
%
% \begin{definition}
%    If $\defpt \in \RR^\numvars$ defines $\O$, then the \emph{recession basis} of $\O$ is the set $\rb(\O)$ of all standard basis vectors $\canvec_i$ in $\RR^\numvars$ such that the $i$-th coordinate of $\defpt$ is zero, and the \emph{recession subspace} of $\O$ is the subspace $\rs(\O)$ of $\RR^\numvars$ spanned by $\rb(\O)$.
% \end{definition}
%
% As noted above, these definitions depend only on $\O$, but not on the choice of $\defpt$.
% In view of the Minkowski--Weyl Theorem (see \Cref{ss: euclidean spaces and convexity}), equation \eqref{face: e} implies that the cone generated by $\rb(\O)$ is the recession cone of $\O$, motivating our choice of terminology.
%
% \subsection{Linear programming}
%
% Let $\mathbb{D}$ be either $\RR$ or $\ZZ$.
% A \emph{linear program} $\Pi$ in $\mathbb{D}^n$ is an optimization problem in which one seeks to maximize a fixed linear \emph{objective function} $\RR^n \to \RR$ on a subset of $\mathbb{D}^n$ defined by a fixed system of linear inequalities.
% We refer to this subset as the \emph{feasible set} of $\Pi$, and denote it $\feas \Pi$, and we refer to the inequalities defining it as the \emph{constraints} of $\Pi$.
% We say that the points of $\feas \Pi$ are \emph{feasible for $\Pi$}.
% When $\mathbb{D} = \ZZ$, we refer to $\Pi$ as an \emph{integer linear program}, or simply \emph{integer program}, for short.
%
% If $\mathbb{D} = \RR$, then the feasible set is a polyhedron in $\RR^n$, and if $\mathbb{D} = \ZZ$, the feasible set is the set of lattice points in a polyhedron in $\RR^n$.
%
% In this article, we will only consider linear programs in which the objective function restricted to the feasible set attains a maximum (\eg this occurs whenever the constraints define a polytope).
% In this case, a feasible point is \emph{optimal} if it maximizes the objective function, and the optimal value obtained by this function is called the \emph{value} of the program.
% We use $\opt \Pi$ to denote optimal set of the linear program $\Pi$, and $\val \Pi$ to denote the value of $\Pi$.
%
% There are clearly multiple reasonable notions of equality for integer programs.
% In this article,  we say that two integer programs are \emph{equal} if their objective functions and defining constraints are identical, and \emph{equivalent} if their objective functions are identical and their feasible sets agree.
%
% \subsection{Multinomial coefficients}
% \ \daniel[inline]{I don't think we consider $\binom{k}{\vv{u}}$ unless $\norm{\vv{u}} = k$.  If so, why don't we just define $\binom{\norm{\vv{u}}}{\vv{u}}$?}
% \pedro[inline]{
%    I think we do need the more general definition in the statement of Dickson's theorem, below.
%    The one advantage of the current definition is that there's one less condition to state when writing out generators for $\ideala^{[k]}$ ($\binom{k}{\vv{k}} \not\equiv 0 \bmod p$, as opposed to $\norm{\vv{k}} = k$ and $\binom{\norm{\vv{k}}}{\vv{k}} \not\equiv 0 \bmod p$.)
%    I took advantage of that in a couple of places.
% }
% \emily[inline]{I think this ``zero'' convention is standard; at least it is familiar for binomial coefficients.  Actually, do we really need to define a multinomial coefficient?  It seems sufficient to me to simply say that \hl{we use  $\binom{k}{\vv{u}}$ to denote the binomial coefficient $\binom{k}{u_1,\ldots,u_n}$, which equals zero if $\norm{\vv{u}} \neq k$}.}
% \pedro[inline]{
%    Sounds good to me. Let's just include the highlighted sentence immediately before our first use of multinomial coefficients.
% }
%
% If $k$ is a nonnegative integer and $\vv{u} = (u_1,\ldots,u_n) \in \NN^n$ is a point with $\norm{\vv{u}} = k$, then the \emph{multinomial coefficient} $\binom{k}{u_1,\ldots,u_n}$, or $\binom{k}{\vv{u}}$ for short, is defined as follows:
% \[
%    \binom{k}{\vv{u}} = \binom{k}{u_1,\ldots,u_n} \coloneqq \frac{k!}{u_1!\cdots u_n!}.
% \]
% If $\norm{\vv{u}} \ne k$, on the other hand, we set $\binom{k}{\vv{u}}=0$.
%
%
% \emily[inline]{I am under the impression that the following theorem and corollary are only used in Section 7.  What do you think about putting them together in a remark right before they are used?}
% \pedro[inline]{
%    That sounds good to me.
%    We'll just combine all of this with the sentence immediately after Definition~7.1, all in one remark.
% }
% \begin{theorem}[\cite{dickson.multinomial}]
%    \label{thm: dickson}
%    Let $p$ be a prime integer, $k\in \NN$, and $\vv{u} \in \NN^n$.
%    Write the terminating base $p$ expansions of $k$ and $\vv{u}$ as follows\textup:
%    \begin{equation*}
%       k = k_0+k_1p+k_2p^2+\cdots+k_rp^r\quad \text{and} \quad \vv{u}=\vv{u}_0+\vv{u}_1p+\vv{u}_2p^2+\cdots+\vv{u}_rp^r,
%    \end{equation*}
%    where $0\le k_i < p$ and $\vv{0}\le\vv{u}_i < p \cdot \vv{1}$ for each $i$.
%    \textup{(}Note that it is possible that $k_r = 0$ or $\vv{u}_r = \vv{0}$.\textup{)}
%    Then
%    \[
%       \binom{k}{\vv{u}}\equiv \binom{k_0}{\vv{u}_0}\binom{k_1}{\vv{u}_1}\cdots \binom{k_r}{\vv{u}_r} \mod{p}.
%    \]
%    In particular, $\binom{k}{\vv{u}}\not\equiv 0\bmod{p}$ if and only if $\norm{\vv{u}_i}=k_i$ for each $i$, that is, the components of $\vv{u}$ add up to $k$ without carrying \textup(base $p$\textup).
% \qed
% \end{theorem}
%
% \begin{corollary}
%    \label{cor: multinomial congruence}
%    Let $k,l,e\in \NN$, with $l<p^e$, and $\vv{u},\vv{v}\in \NN^n$, with $\vv{v}<p^e\cdot \vv{1}$.
%    Then
%    \[
%       \pushQED{\qed}
%       \binom{kp^e+l}{\vv{u}p^e+\vv{v}}\equiv \binom{k}{\vv{u}}\binom{l}{\vv{v}} \mod{p}.\qedhere
%       \popQED
%    \]
% \end{corollary}

\newpage
\section{Introduction}

Let $\kk$ be a field of prime characteristic $p>0$ with $[\kk:\kk^p] < \infty$, and consider the polynomial ring $R = \kk[x_1, \ldots, x_\numvars]$.  Given ideals $\ideala$ and $\idealb$ of $R$ with ${\ideala}$ contained in $\sqrt \idealb$, and a nonnegative integer $e$, we define
%
\[\nu(\ideala,\idealb,p^e) \coloneqq \max\big\{k\in \NN : \ideala^{k} \not\subseteq \idealb^{[p^e]}\big\}\]
%
where $\idealb^{[p^e]} = \langle g^{p^e} : g \in \idealb \rangle$ is the $p^e$-th Frobenius power of $\idealb$.
The condition that $\ideala \subseteq \sqrt \idealb$ guarantees that each $\nu(\ideala, \idealb, p^e)$ is a well-defined integer.

The growth rate of these integers as a function of the nonnegative integer $e$ is of independent interest.
It is not hard to see that the sequence $\big(\nu(\ideala,\idealb,p^e)/p^e\big)_{e=0}^{\infty}$ is nondecreasing and bounded,  and we define
\[ \ft{\ideala}{\idealb} = \lim_{e \to \infty} \frac{\nu(\ideala, \idealb, p^e)}{p^e}  = \sup_{e\in \NN} \frac{\nu(\ideala,\idealb,p^e)}{p^e}, \]
which we call the \emph{$F$-threshold of $\ideala$ with respect to $\idealb$}.

% \daniel{Motivate the $\nu$'s by tying them to roots of BS polynomials; motivate $F$-thresholds by relating them to behavior of test ideals.  Relate them to multiplier ideals}


\daniel[inline]{This is essentially copied verbatim from \cite[Problem 3.8]{mustata+takagi+watanabe.F-thresholds}.  It also appeared as \cite[Problem 2.3]{budur+mustata+saito.roots_bs_polys} .  We should probably restate it more precisely, and explain its significance.}

\begin{problem}
   Find conditions on the ideal $\ideala$ such that the following holds.
   Given an ideal $\idealb$ with $\ideala \subseteq \sqrt \idealb$ and a nonnegative integer $e$, there exists a positive integer $N$ and polynomials $P_i(t) \in \QQ[t]$ of degree $e$ for every $i$ that is a unit modulo $N$ such that $\nu(\ideala_p, \idealb_p, p^e) = P_i(p)$ whenever $p$ is sufficiently large and $p \equiv i \bmod N$.
   When could $N$ be chosen independently of $\idealb$ and $e$?
\end{problem}

\begin{problem}[{\cite[Problem 3.8]{mustata+takagi+watanabe.F-thresholds}, \cite[Problem 2.3]{budur+mustata+saito.roots_bs_polys}}]
Identify a class of ideals $\ideala$ for which, given $e \geq 1$, the following property holds:  There exists an integer $N$ for which $\nu_\ideala^\idealb(p^e)$ is a quasi-polynomial on the units modulo $???$
\end{problem}
%
\daniel[inline]{Emphasize that when $\ideala$ is monomial and $\idealb$ is arbitrary, then the $F$-threshold of $\ideala_p$ with respect to $\idealb_p$ is independent of $p$ for all $p \gg 0$.  This should be explicitly pointed out in the appendix, I guess.}
%
\begin{theorem}
   \label{general-nu-theorem: T}
   Given a monomial ideal $\ideala$ of $\QQ[x_1, \ldots, x_\numvars]$, and an arbitrary ideal $\idealb$ of this polynomial ring with $\ideala \subseteq \sqrt{\idealb}$, there exists positive integers $\ell = \ell(\ideala)$ and  $\beta = \beta(\ideala, \idealb)$, and for every integer $1 \leq r < \ell$ relatively prime to $\ell$ a rational number $\delta(\ideala, \idealb, r)$ satisfying the following conditions.
   If $p^e > \beta(\ideala, \idealb)$ and $p^e \equiv r \bmod \ell$, then $ \nu(\ideala_p, \idealb_p, p^e) = \ft{\ideala}{\idealb} \cdot p^e - \delta(\ideala, \idealb, r)$.
\end{theorem}

As above, let $\kk$ be a field of prime characteristic $p>0$ with $[\kk: \kk^p]$ finite.
If $\ideala$ is an ideal of $\kk[x_1, \ldots, x_\numvars]$, then the Frobenius powers of $\ideala$ are a family of ideals $\ideala^{[t]}$ indexed by a nonnegative real parameter $t$.
When the exponent is an integer, then the corresponding Frobenius power can be described concretely in terms of generators as follows:
If $k$ is a natural number, and $\ideala$ is generated by polynomials $f_1, \ldots, f_n$, then $\ideala^{[k]}$ is the ideal generated by the products $f^{\vv{u}} = f_1^{u_1}\cdots f_n^{u_n}$, ranging over all points $\vv{u} \in \NN^n$ for which the multinomial coefficient $\binom{k}{\vv{u}}$ is nonzero modulo $p$  \cite[Proposition~3.5]{hernandez+etal.frobenius_powers}.

Now, if $\idealb$ is an ideal of $\kk[x_1, \ldots, x_\numvars]$ with $\ideala \subseteq \sqrt{\idealb}$, then in analogy with the above definitions, given a nonnegative integer $e$, we define
\[\mu(\ideala,\idealb,p^e) \coloneqq \max\big\{k\in \NN : \ideala^{[k]} \not\subseteq \idealb^{[p^e]}\big\}.\]
Then $\big(\mu(\ideala,\idealb,p^e)/p^e\big)_{e=0}^{\infty}$ is a nondecreasing bounded sequence, and
\begin{equation}\label{eq: crit as a limit of mus}
   \crit(\ideala,\idealb) = \lim_{e\to \infty} \frac{\mu(\ideala,\idealb,p^e)}{p^e} = \sup_{e\in \NN} \frac{\mu(\ideala,\idealb,p^e)}{p^e}.
\end{equation}
is the critical exponent of $\ideala$ with respect to $\idealb$.

\begin{theorem}
   \label{general-mu-theorem: T}
   Given a monomial ideal $\ideala$ of $\QQ[x_1, \ldots, x_\numvars]$, there exists a positive integer $\ell = \ell(\ideala)$, and for every ideal $\idealb$ with $\ideala \subseteq {\idealb}$ and for every integer $1 \leq r < \ell$ relatively prime to $\ell$, there exist a positive integer $\beta = \beta(\ideala, \idealb)$ and a sequence $\big(\epsilon(\ideala, \idealb, r, e)\big)_{e=1}^{\infty}$ of nonnegative rational numbers satisfying the following condition.
   If $p > \beta$ and $p \equiv r \bmod \ell$, then
   \[ \mu(\ideala_p, \idealb_p, p^e) = \ft{\ideala}{\idealb} \cdot p^e - \sum_{s=1}^{e} \epsilon(\ideala, \idealb, r, s) \cdot p^{e-s}.\]
\end{theorem}

\newpage
\section{Notation and basic notions}


\subsection{Euclidean spaces}
\label{ss: euclidean spaces and convexity}
%We review in this subsection some of the terminology, notation, and constructions concerning Euclidean spaces used throughout the paper.
We use bold-face lower-case letters to denote points of the Euclidean space $\RR^n$, and the same letter, in regular font, to represent their coordinates (\eg $\vv{v}=(v_1,\ldots,v_n)$).
The points $(0,\ldots,0)$ and $(1,\ldots,1)$ are denoted $\vv{0}$ and $\vv{1}$, and we write the standard basis vectors of $\RR^n$ as $\canvec_1,\ldots,\canvec_n$.

Given a point $\vv{u}\in \RR^n$, $\norm{\vv{u}}$ denotes its coordinate sum, $u_1+\cdots+u_n$.
The standard inner product in $\RR^n$ is denoted by the usual angle brackets: $\iprod{\vv{u}}{\vv{v}} = u_1v_1 + \cdots + u_nv_n$.
An inequality between points of $\RR^n$ is a shorthand for a system of $n$ coordinatewise inequalities; e.g., $\vv{u}\le \vv{v}$ means that $u_i \le v_i$ for each $i=1,\ldots,n$.
In the same vein, operations on numbers are extended to points in $\RR^n$ in a coordinatewise fashion; for instance, $\up{\vv{u}}=(\up{u_1},\ldots,\up{u_n})$.

We say that a point $\vv{u}\in \RR^n$ is positive (respectively, nonnegative) if $\vv{u} > \vv{0}$ (respectively, $\vv{u}\ge \vv{0}$).
\daniel{Revisit this when deciding how to describe collapsing}More generally, given a point $\vv{u}$ in a coordinate subspace $\mathcal{S}$ of $\RR^n$, we say that $\vv{u}$ is \emph{positive in $\mathcal{S}$} (respectively, \emph{nonnegative in $\mathcal{S}$}) if $\vv{u}$ is a positive (respectively, nonnegative) linear combination of the standard basis vectors that span $\mathcal{S}$.

\subsection{Monomial matrices, monomial ideals, and pairs}
\label{monomial newton preliminaries: ss}
We work in the polynomial ring $\kk[x_1, \ldots, x_\numvars]$ over a field $\kk$, and adopt standard notation for describing monomials.  That is, if $\vv{u} \in \NN^\numvars$, then $x^{\vv{u}} = x_1^{u_1} \cdots x_\numvars^{u_\numvars}$.

\begin{definition}[Monomial matrices]
\label{monomial matrix: D}
A \emph{monomial matrix} is a matrix over $\ZZ$ with nonnegative, nonzero rows and columns.
If $A$ is a $\numvars \times n$ monomial matrix, then we call $\ZZ^n$ the \emph{domain lattice} and $\ZZ^\numvars$ the \emph{target lattice} of $A$. 
\end{definition}

\begin{remark}[Connections with monomial ideals]
\label{monomial matrix ideal: R}  Consider a proper monomial ideal $\ideala$ of $\kk[x_1, \ldots, x_{\numvars}]$.  Given a generating set $\ideala = \langle x^{\vv{a}_1}, \ldots, x^{\vv{a}_n} \rangle$, not necessarily minimal, with the property that each variable of the ambient polynomial ring appears in some generator, it is not difficult to see that \[A= \begin{bmatrix} \vv{a}_1 & \cdots & \vv{a}_n \end{bmatrix}\] is a  $\numvars \times n$ monomial matrix.  Clearly, as this process depends on the choice of a distinguished generating set for $\ideala$, there are many monomial matrices $A$ we can assign to $\ideala$, though each such matrix will necessarily have $m$ rows.
\end{remark}

\begin{remark}[Generators in terms of monomial matrices]
   \label{generators-via-exponent-matrix: R}  If $\ideala$ is a proper monomial ideal, and $A= \begin{bmatrix} \vv{a}_1 & \cdots & \vv{a}_n \end{bmatrix}$ is a monomial matrix associated to $\ideala$, then the generators of powers of $\ideala$ can be described compactly in terms of the matrix $A$.
   Indeed, this follows from the fact that the product $(x^{\vv{a}_1})^{k_1} \cdots (x^{\vv{a}_n})^{k_n}$ can be written as $x^{A \vv{k}} = x^{k_1 \vv{a}_1 + \cdots + k_n \vv{a}_n}$.
   That is, $\ideala^\ell$ is generated by all monomials of the form $x^{A \vv{k}}$ with $\vv{k} \in \NN^n$ satisfying $\norm{\vv{k}} = \ell$.
%
% \begin{equation}
% \label{generators-via-exponent-matrix: e}
% \ideala^n = \langle x^{A \vv{k}} : \vv{k} \in \NN^n \text{ and } \norm{\vv{k}}=n \rangle.
% \end{equation}
\end{remark}



\begin{definition}[Diagonal ideals]  A  \emph{diagonal ideal} of $\kk[x_1, \ldots, x_{\numvars}]$ is any \emph{proper} ideal of the form $\ideald = \langle x_1^{u_1}, \ldots, x_{\numvars}^{u_{\numvars}} \rangle$.  In this case,  we call $\ideald$ the diagonal ideal associated to the \emph{positive} point $\vv{u} = (u_1, \ldots, u_{\numvars}) \in \ZZ^{\numvars}$.
\end{definition}

\begin{definition}[Pairs]  
\label{pairs: D}
   A \emph{monomial pair} $(A, \vv{u})$ consists of an $\numvars \times n$ monomial matrix $A$, and a positive point $\vv{u} \in \ZZ^m$, the target lattice of $A$.  
   
   An \emph{ideal pair} $(\ideala, \ideald)$ consists of a proper monomial ideal $\ideala$ of $\kk[x_1, \ldots, x_{\numvars}]$, along with a diagonal ideal $\ideald$ of the same ambient polynomial ring.
\end{definition}

\begin{remark}[Association of pairs]
\label{associate of pairs: R}
 As in \Cref{monomial matrix ideal: R}, we may associate to each ideal pair $(\ideala, \ideald)$ in $\kk[x_1, \ldots, x_{\numvars}]$ a monomial pair $(A, \vv{u})$.  Observe that, though there need not be a canonical choice for the $m \times n$ matrix $A$, the positive point $\vv{u} \in \ZZ^m$ is uniquely determined by the diagonal ideal $\ideald$.
\end{remark}

\subsection{Optimization}  Optimization problems play a central role in this paper.  Here, we briefly recall the basic framework of standard \emph{programming}.

Suppose that $\mathcal{X}$ is either $\RR^n$ or $\ZZ^n$.  By a \emph{program $\Omega$ in the ambient space $\mathcal{X}$}, we mean an optimization problem seeking to maximize some given linear \emph{objective function} $\RR^n \to \RR$ on some subset of $\mathcal{X}$ defined by a system of linear inequalities, called the \emph{constraints} of the program $\Omega$. If $\mathcal{X} = \RR^n$, then the points in $\mathcal{X}$ satisfying these constraints define a polyhedron in $\RR^n$, while if $\mathcal{X} = \ZZ^n$, then the points in $\mathcal{X}$ satisfying these constraints instead consist of the lattice points in some polyhedron in $\RR^n$.  In either case, we call this set of points in $\mathcal{X}$ the \emph{feasible set} of the program.  We denote it by $\feas \Omega$, and say that the points in this set are \emph{feasible} for the program $\Omega$.

To distinguish between these cases, we refer to a program in $\mathcal{X} = \RR^n$ as a {real linear program}, or \emph{linear program} for short, and to a program in $\mathcal{X}=\ZZ^n$ as an {integer linear program}, or \emph{integer program} for short.

Every program $\Omega$ we consider in this article has the property that the values taken on by the objective function on $\feas \Omega$ are bounded from above, and in this case, the maximum value achieved by the objective function on this feasible set is called the \emph{value} of the program $\Omega$, and is denoted $\val \Omega$.  The subset of $\feas \Omega$ where the objective function achieves this maximum value is called the \emph{optimal set} of $\Omega$, and is denoted $\opt \Omega$.   When $\Omega$ is a linear program in $\RR^n$, its optimal set is a face of the polyhedron $\feas \Omega$ in $\RR^n$.


\newpage
\section{Connections with optimization}
\label{sec: LPs}
  
Throughout this section, we consider an ideal pair $(\ideala, \ideald)$ in $\kk[x_1, \ldots, x_{\numvars}]$, as described in \Cref{pairs: D}.   Our goal here, and for much of the rest of the article, is to study the integers $\nu(\ideala, \ideald, p^e)$ and $\mu(\ideala, \ideald, p^e)$ and rational numbers $\ft{\ideala}{\ideald}$ and $\crit(\ideala, \ideald)$ defined in the introduction.  As will soon be apparent, it is natural to do so in terms of various optimization problems.

\begin{remark}
This context might seem restrictive, given that our ultimate goal is to understand the nature of the above mention numerical invariants when the diagonal ideal $\ideald$ is replaced with an arbitrary ideal $\idealb$.  

However, as we explain in \Cref{monomial-reduction: A}, understanding the situation for all ideal pairs $(\ideala, \ideald)$ with $\ideala$ fixed, but $\ideald$ varying, leads to \Cref{general-nu-theorem: T,general-mu-theorem: T}.
\end{remark}

To facilitate the transition from algebra to optimization, fix a monomial pair $(A, \vv{u})$ associated to $(\ideala, \ideald)$, as described in \Cref{associate of pairs: R}.  In particular, $A$ is an $m \times n$ monomial matrix, $\vv{u} \in \ZZ^m$ is a positive lattice point, and 
\[ \ideald = \langle x^{u_1}, \ldots, x^{u_{\numvars}} \rangle \subseteq \kk[x_1, \ldots, x_{\numvars}] \] is the diagonal ideal associated to $\vv{u}$.

\subsection{Relations with integer and linear programs}  We start by studying the integers $\nu(\ideala, \ideald, p^e)$ and rational numbers $\ft{\ideala}{\ideald}$.  First, recall that for each $q$ a power of $p$, the integer $\nu(\ideala, \ideald, q)$ is defined as 
\[\nu(\ideala,\ideald,q) = \max\big\{\ell \in \NN : \ideala^{\ell} \not\subseteq \ideald^{[q]}\big\}.\]

As noted in \Cref{generators-via-exponent-matrix: R},  the ideal $\ideala^{\ell}$ is generated by all monomials of the form $x^{A\vv{k}}$, with $\vv{k}\in \NN^n$ and $\norm{\vv{k}} = \ell$.  Furthermore, as $\ideald^{[q]}$ is the diagonal ideal associated to $\vv{u}q$, we have that $\ideala^{\ell} \not\subseteq \ideald^{[q]}$ if and only if there is some generator $x^{A \vv{k}}$ of $\ideala^{\ell}$ with $A\vv{k} < \vv{u}q$.  Thus, computing $\nu(\ideala,\ideald,q)$ is equivalent to maximizing the value of $\norm{\vv{k}}$, with $\vv{k} \in \ZZ^n$, and subject to the constraints $\vv{k} \geq \vv{0}$ and $A\vv{k} < \vv{u}q$.
This observation motivates the following definition.

\begin{definition}
\label{IP: D}
   If $q$ is a positive integer, then $\IP(A, \vv{u}, q)$ is the integer program in the domain lattice of $A$ which consists of maximizing the objective function $\vv{k} \mapsto \norm{\vv{k}}$ subject to the constraints $\vv{k} \geq \vv{0}$ and $A \vv{k} \leq \vv{u}q - \vv{1}$.
\end{definition}

It is easy to deduce from the fact that the matrix $A$ is a monomial matrix (i.e., has nonnegative, nonzero rows and columns) that the feasible set of $\IP(A,\vv{u},q)$ is finite, and hence, this integer program has a well-defined value.  In fact, the discussion preceding \Cref{IP: D} demonstrates that 
%
\begin{equation}
\label{nu as program value: eq}
\nu(\ideala,\ideald,q) = \val \IP(A,\vv{u},q).
\end{equation}

\daniel[inline]{I propose to describe things in terms of shortfalls, as opposed to optimal images.}

The constraints of the program $\IP(A,\vv{u},q)$ imply that for every subset $\mathcal{F}$ of its feasible set,  the image of $\mathcal{F}$ under $A$, denoted $A(\mathcal{F})$, consists of points in the target lattice of $A$ less than $\vv{u}q$.  In particular, the Minkowski difference $\vv{u}q-A(\mathcal{F})$ consists of positive lattice points in the target lattice of $A$.

\begin{definition}[Shortfalls]  The \emph{shortfall} of the integer program $\IP(A, \vv{u}, q)$, denoted $\short \IP(A, \vv{u}, q)$, is the Minkowski difference
%
\[ \short \IP(A, \vv{u}, q) = \vv{u}q - A( \opt \IP(A, \vv{u}, q)).\]  
Equivalently, the shortfall of $\IP(A, \vv{u},q)$ is determined by the condition
\[ A(\opt \IP(A, \vv{u}, q)) = \vv{u}q - \short \IP(A, \vv{u}, q).\]
%

%
The preceding discussion illustrates the shortfall of $\IP(A, \vv{u}, q)$ consists of positive lattice points in the target lattice of $A$.\end{definition}

\begin{remark}[Algebraic interpretation of shortfalls]
\label{shortfall motivation: R}
If $q$ is a power of $p$, and $\nu = \nu(\ideala,\ideald,q) = \val \IP(A,\vv{u},q)$, then the shortfall of $\IP(A,\vv{u},q)$ can be interpreted in terms of the ideal pair $(\ideala, \ideald)$ as the ``leftovers'' of $\ideala^\nu$ mod $\ideald^{[q]}$.

More precisely, the discussion preceding \Cref{aip: D} tells us that the generators of $\ideala^{\nu}$ not contained in $\ideald^{[q]}$ are precisely the monomials of the form $x^{A \vv{k}}$, with $\vv{k}$ in the domain lattice of $A$, and satisfying $\norm{\vv{k}} = \nu$ and $A\vv{k}<\vv{u}q$.  

Keeping in mind that $\nu= \val \IP(A, \vv{u}, q)$, these conditions on $\vv{k}$ are equivalent to $\vv{k}$ being optimal for $\IP(A, \vv{u}, q)$.  Thus, the generators of $\ideala^{\nu}$ not in $\ideald^{[q]}$ are precisely the monomials of the form $x^{\vv{w}}$, with the exponent vector $\vv{w}$ lying in $A(\opt \IP(A, \vv{u}, q)) = \vv{u}q - \short \IP(A, \vv{u}, q)$.  In other words,
\begin{equation}
\label{algebraic shortfall PI general: e}
 \ideala^{\nu(\ideala, \ideald, q)} \equiv \ideal{x^{\vv{u}q-\vv{v}}: \vv{v} \in \short \IP(A,\vv{u},q)} \bmod \ideald^{[q]}.
 \end{equation}
%\[ \ideala^{\nu(\ideala, \ideald, q)} \equiv \ideal{x^{\vv{v}}: \vv{v} \in \im \IP(A,\vv{u},q)} \bmod \ideald^{[q]}.\]
%\begin{align*}
%  \ideala^{\nu} &= \ideal{x^{A\vv{k}}: \norm{\vv{k}}=\nu}\\
%  &\equiv \ideal{x^{A\vv{k}}: \norm{\vv{k}}=\nu\text{ and } A\vv{k} <\vv{u}q} \bmod \ideald^{[q]}\\
%  &\equiv \ideal{x^{A\vv{k}}: \vv{k}\in \opt\IP(A,\vv{u},q)} \bmod \ideald^{[q]}\\
%  &\equiv \ideal{x^{\vv{v}}: \vv{v} \in \im \IP(A,\vv{u},q)} \bmod \ideald^{[q]}.
%\end{align*}
\end{remark}



We now shift our attention from the integers $\nu(\ideala, \ideald, q)$ to the $F$-threshold $\ft{\ideala}{\ideald}$.  First, note that \eqref{nu as program value: eq} and the definition of $\ft{\ideala}{\ideald}$ imply that 
%
\begin{equation}
\label{ft as limit of normalized program values: eq}
\ft{\ideala}{\ideald} = \lim_{e\to\infty} \frac{\nu(\ideala,\ideald,p^e)}{p^e} = \lim_{e\to\infty} \frac{\val \IP(A,\vv{u},p^e)}{p^e}.
\end{equation}

Continuing with our theme, we seek to describe this number in terms of a certain optimization program, which we define below.

\begin{definition}
\label{LP: D}
 $\LP(A, \vv{u})$ is the linear program in the domain of $A$ which consists of maximizing $\vv{k} \mapsto \norm{\vv{k}}$ subject to the constraints $\vv{k} \geq \vv{0}$ and $A \vv{k} \leq \vv{u}$.
\end{definition}

Once again, the fact that $A$ is monomial implies that the feasible set of the linear program $\LP(A,\vv{u})$ is bounded, and therefore a polytope in the domain of $A$.  Consequently, $\LP(A,\vv{u})$ has a well-defined value.
%In order to relate this value to the Newton polyhedron of $A$ we need the following definition.


\begin{proposition}
\label{ft as val LP: P}
The value of $\LP(A, \vv{u})$ equals the $F$-threshold $\ft{\ideala}{\ideald}$.
\end{proposition}

\begin{proof}
  Recall that, as $A$ is a $\numvars \times n$ monomial matrix, the domain of $A$ is $\RR^n$, and the domain lattice of $A$ is $\ZZ^n$.  If $\vv{k} \in \ZZ^n$ is optimal for $\IP(A,\vv{u},q)$, then the scaled point $\vv{k}/q$ is feasible for $\LP = \LP(A, \vv{u})$, so $\val \LP \geq \val \IP(A, \vv{u}, q)/q$ for all $q$.
   Thus, $\val \LP$ is at least equal to the limits appearing in \eqref{ft as limit of normalized program values: eq}.

   Conversely, given an optimal point $\vv{t} \in \RR^n$ for $\LP$,
   define $\vv{t}_q \in \ZZ^n$ as the point whose $i$-th coordinate is $0$ if $t_i=0$, and otherwise equals $\lceil q t_i \rceil - 1$, the greatest integer less than $q t_i$.
   By design, and the fact that $A$ is a monomial matrix, the $i$-th entry of $A \vv{t}_q$ is either less than the $i$-th entry of $A (q \vv{t})$, which itself is less than or equal to $u_iq$, by the feasibility of $\vv{t}$ for $\LP$, or equals $0$,  in which case it is also less than $u_i q$ since $\vv{u}$ is positive. This observation demonstrates that $\vv{t}_q$ is feasible for $\IP(A, \vv{u}, q)$, and so
   \[\val \IP(A, \vv{u}, q) \geq \norm{\vv{t}_q} \geq \sum_{i=1}^n \big(\lceil q t_i \rceil - 1\big) \geq q \norm{\vv{t}} -n = q \cdot \val \LP - n.\]
   
   Dividing by $q$ and taking the limit as $q$ tends to infinity, we find that the limits in \eqref{ft as limit of normalized program values: eq} are at least $\val \LP$, which completes the proof.
\end{proof}


\subsection{Relations with arithmetic and fractal programs}

 We now turn our attention to the integers $\mu(\ideala, \ideald, p^e)$ and rational numbers $\crit(\ideala,\ideald)$.  First, recall that for each $q$ a power of $p$, the integer $\mu(\ideald, \ideald, q)$ is defined as 
\[\mu(\ideala,\ideald,q) \coloneqq \max\big\{ \ell \in \NN : \ideala^{[\ell]} \not\subseteq \ideald^{[q]}\big\},\]
where $\ideala^{[\ell]}$ is the $\ell$-th (generalized) Frobenius power of $\ideala$.

As noted in the introduction, if $\idealb$ is an ideal of a ring of characteristic $p$ with generators $g_1, \ldots, g_n$, then $\idealb^{[\ell]}$ is the ideal generated by the products $g_1^{k_1}\cdots g_n^{k_n}$, ranging over $\vv{k} \in \NN^n$ with $\norm{\vv{k}}=\ell$, and for which  $\binom{\ell}{\vv{k}} =
\frac{\ell!}{k_1 ! \cdots k_n !}$ is nonzero modulo $p$  \cite[Proposition~3.5]{hernandez+etal.frobenius_powers}.
In light of this, \Cref{generators-via-exponent-matrix: R} tells us that $\ideala^{[\ell]}$ is generated by the monomials of the form $x^{A \vv{k}}$ with $\vv{k} \in \NN^n$ satisfying $\norm{\vv{k}}=\ell$ and $\binom{\ell}{\vv{k}} \not\equiv 0 \bmod p$.  As in the previous subsection, we may conclude that $\mu(\ideala,\ideald,q)$ is the maximum value of $\norm{\vv{k}}$, with $\vv{k} \in \NN^n$ subject to the linear constraint $A\vv{k} < \vv{u}q$, but also subject to the highly nonlinear constraint $\binom{\norm{\vv{k}}}{\vv{k}} \not\equiv 0 \bmod{p}$.  Motivated by this, we introduce a variant of an integer program that we call an \emph{arithmetic integer program}.
%As this new constraint is arithmetic in nature, we call such an optimization problem an \emph{arithmetic integer program}.

\begin{definition}
\label{aip: D}
If $q$ is a power of $p$, then $\IP_p(A, \vv{u}, q)$ is the \emph{arithmetic integer program} in the domain lattice of $A$ which consists of maximizing the objective function $\vv{k} \mapsto \norm{\vv{k}}$ subject to the linear constraints $\vv{k} \geq \vv{0}$ and $A \vv{k} \leq \vv{u}q - \vv{1}$, and the arithmetic constraint $\binom{\norm{\vv{k}}}{\vv{k}} \not \equiv 0 \bmod p$.
\end{definition}

We define the terms \emph{feasible, optimal}, and \emph{value} relative to the arithmetic program $\IP_p(A, \vv{u}, q)$ in analogy with those for the integer program $\IP(A, \vv{u}, q)$.


\begin{remark}[A characterization of the arithmetic constraint]
 \label{dickson: R}
   Consider $n \in \NN$ and $\vv{n} \in \NN^n$.
   In base $p$, we may uniquely express these quantities as $n= \sum_{e=0}^s n_e \, p^e$ and $\vv{n}=\sum_{e=0}^s \vv{n}_e \, p^e$
%\begin{equation*}
%n= \sum_{e=0}^s n_e \, p^e \quad \text{and} \quad \vv{n}=\sum_{e=0}^s \vv{n}_e \, p^e
%\end{equation*}
where $0\le n_e < p$ and $\vv{0}\le\vv{n}_e < p  \vv{1}$ for each $0 \leq e \leq s$, and where $n_s$ and $\vv{n}_s$ are allowed to be zero.
In any case, \cite{dickson.multinomial} then tells us that
$\binom{n}{\vv{n}}\equiv \binom{n_0}{\vv{n}_0}\binom{n_1}{\vv{n}_1}\cdots \binom{n_s}{\vv{n}_s} \mod{p}$.
%\[
%    \binom{n}{\vv{n}}\equiv \binom{n_0}{\vv{n}_0}\binom{n_1}{\vv{n}_1}\cdots \binom{n_t}{\vv{n}_t} \mod{p}.
%\]

In particular, the multinomial coefficient $\binom{n}{\vv{n}}$ is nonzero mod $p$ if and only if $\norm{\vv{n}_e}=n_e$ for each $e$, a condition that is sometimes described by saying that the entries of the vector $\vv{n} \in \NN$ sum to $n$ \emph{without carrying} in base $p$.

In light of this, the arithmetic constraint in \Cref{aip: D} is equivalent to the following condition:  If
 $\vv{k} = \vv{k}_0 + \vv{k}_1 \, p\cdots + \vv{k}_t \, p^s$ is the base $p$ expansion of the nonnegative lattice point $\vv{k}$, then $\norm{\vv{k}_e} < p$ for all $0 \leq e \leq t$.
\end{remark}

Clearly, the feasible set of $\IP_p(A, \vv{u}, q)$ lies in the feasible set of $\IP(A, \vv{u},q)$, and hence, is finite.  
In particular, this arithmetic program has a well-defined value, and the discussion preceding \Cref{aip: D} tells us that
%
\begin{equation}
\label{mu as program value: eq}
\mu(\ideala,\ideald,q) = \val \IP_p(A,\vv{u},q)
\end{equation}
%

\begin{definition}[Shortfalls]  The \emph{shortfall} of the arithmetic integer program $\IP_p(A, \vv{u}, q)$, denoted $\short \IP_p(A, \vv{u}, q)$, is the Minkowski difference
%
\[ \short \IP_p(A, \vv{u}, q) = \vv{u}q - A( \opt \IP_p(A, \vv{u}, q)).\]  
Equivalently, the shortfall of $\IP_p(A, \vv{u},q)$ is determined by the condition
\[ A(\opt \IP_p(A, \vv{u}, q)) = \vv{u}q - \short \IP_p(A, \vv{u}, q).\]
%

%
\end{definition}

The shortfall of $\IP_p(A, \vv{u}, q)$ consists of positive lattice points in the target lattice of $A$, and arguing as in \Cref{shortfall motivation: R}, one can show that
%
\begin{equation}
\label{algebraic shortfall PI_p general: e}
 \ideala^{[\mu(\ideala, \ideald, q)]} \equiv \ideal{x^{\vv{u}q-\vv{v}}: \vv{v} \in \short \IP_p(A,\vv{u},q)} \bmod \ideald^{[q]}.
 \end{equation}



Turning to the critical exponents $\crit(\ideala, \ideald)$, the identity \eqref{mu as program value: eq} implies that
%
\begin{equation}
\label{crit as limit of normalized program values: eq}
\crit(\ideala,\ideald) = \lim_{e\to\infty} \frac{\mu(\ideala,\ideald,p^e)}{p^e} = \lim_{e\to\infty} \frac{\val \IP_p(A,\vv{u},p^e)}{p^e}.
\end{equation}
%

In what follows, in analogy with \Cref{ft as val LP: P}, we seek to relate these quantities to the value of a variant of a linear program that we call a \emph{fractal linear program}.  To do so, we require the following concept.

\begin{definition}
\label{sierpinski: D}
   The \emph{Sierpi\'nski $p$-gasket of dimension $n$} is the set $\sierp_{p,n}$ consisting of all points $\vv{t}\in \RR^n$ for which there exist $s \in \ZZ$ and a sequence of points $( \vv{t}_e )_{e=s}^\infty$ in $\NN^n$ such that $\norm{\vv{t}_e} < p$ for all $e \geq s$, and
 \[
\vv{t} = \sum_{e=s}^{\infty} \frac{\vv{t}_e}{p^e}.
 \]
\end{definition}

Note that the integer $s \in \ZZ$ in \Cref{sierpinski: D} can be negative.
It is also immediate that $\vv{t} \in \sierp_{p,n}$ if and only if $p^m  \vv{t} \in \sierp_{p,n}$ for some (equivalently, for every) integer $m \in \ZZ$; this reflects the self-symmetry that will be observed in the pictures that follow. \daniel{Include a pointed reference?  Who knows where these images will end up!}
In particular, when determining whether $\vv{t} \in \RR^n$ lies in $\sierp_{p,n}$, we may rescale by a power of $p$  and assume that $\vv{t} \in [0,1]^n$.

Recall that every point $t \in [0,1]$ either has a unique base $p$ expansion, which is necessarily non-terminating, or else $p^m t \in \NN$ for some $m \in \ZZ$, in which case $t$ has both a terminating and a non-terminating expansion.  Thus, if $\vv{t} \in [0,1]^n$ is such that no component of $p^m \vv{t}$ is an integer for each $m \in \NN$, then there is a unique sequence $\{ \vv{t}_e \}_{e=1}^{\infty}$ in $ \NN^n$ such that $\vv{t} = \sum_{e=1}^{\infty} \frac{\vv{t}_e}{p^e}$, and such point lies in ${\sierp}_{p,n}$ if and only if all $\norm{\vv{t}_e}$ are less than $p$.  However, things can be more subtle when some $p^m \vv{t}$ has an integer component.

\begin{example}
If $p=2$, then $(1/4, 1/4) \in \sierp_{2,2}$.  Indeed, for the first component, take the non-terminating binary expansion $\frac{1}{4} = \frac{1}{2^3} + \frac{1}{2^4} + \frac{1}{2^5} + \cdots$, and for the second component, simply take the expansion $\frac{1}{4} = \frac{1}{2^2}$.  Note that if one instead considered the non-terminating expansion for both components, or the terminating expansion for both components, then the resulting expansion of $(1/4, 1/4)$ would fail to satisfy the condition in \Cref{sierpinski: D}.
\end{example}

This description of the Sierpi\'nski $p$-gasket in terms of expansions is not hard to translate geometrically into its description as a fractal.

\begin{example}
\label{sierpinski triangle: E}
The set $\sierp_{2,2} \cap [0,1]^2$ is the familiar Sierpi\'nski triangle, together with the line segments connecting $(1,0)$ to $(1,1)$, and $(1,1)$ to $(0,1)$.
Indeed, the points $\vv{t}$ in the unit square $[0,1]^2$ that have \emph{no} binary expansion $\vv{t} = \sum_{e=0}^{\infty}\frac{\vv{t}_e}{2^e}$ as above with $\norm{\vv{t}_1} < 2$ are precisely the points in the open triangle  $T = \{ (a,b) \in \RR^2 : a,b < 1, a+b > 1 \}$.
At the next stage, the points $\vv{t} \in [0,1]^2$ with \emph{no} expansion $\vv{t} = \sum_{e=0}^{\infty}\frac{\vv{t}_e}{2^e}$ with both  $\norm{\vv{t}_1} < 2$ and $\norm{\vv{t}_2} < 2$ are those in the union of open triangles $T$ and $\frac{1}{2} \cdot T$.  The condition on expansions at the third place removes three additional open triangles from this set, and we can continue analogously.
\end{example}


As suggested by \Cref{sierpinski triangle: E}, each $\sierp_{p,n}$  can be realized by removing a union of open simplices from $\RRnn^n$, and hence, is a closed set.  
\Cref{fig: sierpinski 3-gasket} illustrates the self-similarity of the $2$-dimensional Sierpi\'nski $3$-gasket.

\begin{figure}
\begin{subfigure}{.49\textwidth}
  \centering
  \includegraphics[width=.9\linewidth]{Pictures/sierpinski3_a.pdf}
  \caption{Restriction to $[0,1]^2$}
\end{subfigure}
\begin{subfigure}{.49\textwidth}
  \centering
  \includegraphics[width=.9\linewidth]{Pictures/sierpinski3_b.pdf}
  \caption{Restriction to $[0,9]^2$}
\end{subfigure}
\caption{The $2$-dimensional Sierpi\'nski 3-gasket}
\label{fig: sierpinski 3-gasket}
\end{figure}


Remarkably, the critical exponent of a monomial pair $\crit(\ideala, \ideald)$ described in \eqref{crit as limit of normalized program values: eq} can be computed in terms of the Sierpi\'nski $p$-gasket, providing a geometric interpretation for this value.  To motivate our approach, we first note that the feasible set of the linear program $\LP(A, \vv{u})$, whose value equals the $F$-threshold $\ft{\ideala}{\ideald}$ by \Cref{ft as val LP: P}, is simply the closure with respect to the Euclidean topology of the set $\{ \vv{t} \in \RR^n : \vv{t} \geq \vv{0} \text{ and } A\vv{t} < \vv{u} \}$.  In what follows, we consider a similar optimization problem,  replacing the conditions that $\vv{t} \in \RR^n$ and $\vv{t} \geq \vv{0}$ with the ``fractal constraint" that $\vv{t} \in \sierp_{p,n}$.

\begin{definition}
\label{fractal program: D}
The \emph{fractal linear program} $\fip_p(A,\vv{u})$ consists of maximizing the objective function $\vv{t}\mapsto \norm{\vv{t}}$ on the Euclidean closure of the set \[ \{ \vv{t} \in \sierp_{p,n} : A \vv{t} < \vv{u} \}.\]  We call this closure the feasible set of $\fip_p(A, \vv{u})$, and denote it by $\feas \fip_p(A, \vv{u})$.
% The value of the problem, $\val \fip_p(A,\vv{u})$, is defined as the supremum of $\norm{\vv{t}}$ among all $\vv{t} \in \feas \fip_p(A, \vv{u})$.
\end{definition}

\begin{remark}
   Note that the closure in the above definition need not agree with the closed set $\{ \vv{t} \in \sierp_{p,n} : A \vv{t} \le \vv{u} \}$.
   For example, if $A$ is the $2\times 2$ identity matrix, $\vv{u} = (1,1)$, and $p=2$, then the closure of the set $\{ \vv{t} \in \sierp_{2,2} : A \vv{t} < \vv{u} \}$ is simply the classical Sierpi\'nski triangle---the set of points $\vv{t}$ in $\sierp_{2,2}$ with $\norm{\vv{t}} \le 1$. The set $\{ \vv{t} \in \sierp_{2,2} : A \vv{t} \le \vv{u} \}$, on the other hand, also includes the line segments from $(1,0)$ to $(1,1)$, and from $(1,1)$ to $(0,1)$.
\end{remark}

\begin{remark}[On the values and optimal sets of fractal programs]  The discussion preceding \Cref{fractal program: D} implies that $\feas \fip_p(A, \vv{u})$ is contained in the feasible set of $\LP(A, \vv{u})$, and hence, is bounded.  Thus, $\feas \fip_p(A, \vv{u})$ is compact, and so $\val \fip_p(A, \vv{u})$, the value of $\fip_p(A, \vv{u})$, can be defined as 
   \[ \max \{ \norm{\vv{t}}: \vv{t} \in \feas \fip_p(A, \vv{u}) \} = \sup \{ \norm{\vv{t}} : \vv{t} \in \sierp_{p,n}, A \vv{t} < \vv{u} \} \] which is a well-defined real number.
   As usual, we define the optimal set of $\fip_p(A, \vv{u})$ to be the set $\opt \fip_p(A, \vv{u})$ of feasible points where this maximum is attained.
%Moreover, there is exists an optimal point
%$\vv{t} \in \feas \fip_p(A, \vv{u})$ such that $\norm{\vv{t}} = \val \fip_p(A, \vv{u})$.
\end{remark}

\begin{example} \label{ex: feas fip}
 Consider the fractal linear program $\fip_p = \fip_p(A, \vv{u})$, where
\[ A = \begin{bmatrix}
 3&11\\ 11&2 \\ 5&10 \\ 2&0
 \end{bmatrix}
\quad \text{and} \quad \vv{u} = \begin{bmatrix} 1 \\ 1 \\ 1 \\ \end{bmatrix}.
\]
\Cref{fig: feas fip} illustrates the key features of the program $\fip_p$ for small values of $p$.  The feasible set for $ \fip_p$ is displayed in blue, the feasible set for $\LP = \LP(A,\vv{u})$ in gray, and the line of points with coordinate sum equal to $\val \fip_p$ in green.  Thus,  $\opt \fip_p$ is simply the intersection of the blue points and green points.

%%%%%%%%%%%%%%%%%%%%%%%%%%%%%%%%%%%%%%%%%%%%%%%%%%%%%%%%%%%%%%%%%%%%%%%%%
%START OF FIGURE DISPLAYING FRACTAL PROGRAM
%%%%%%%%%%%%%%%%%%%%%%%%%%%%%%%%%%%%%%%%%%%%%%%%%%%%%%%%%%%%%%%%%%%%%%%%%

\begin{figure}
  \centering
\begin{subfigure}{.49\textwidth}
\centering
  \includegraphics[width=.9\textwidth]{Pictures/ex4_char2.pdf}\hskip .04\textwidth
   \captionsetup{labelformat=empty}
   \caption{$p=2$}
  % \caption{
  %    $
  %    \begin{array}{l}
  %      \opt \fip_2 = \conv\big(\big(\frac1{20},\frac3{40}\big),\big(\frac1{12},\frac1{24}\big)\big)\\[2mm]
  %      \val \fip_2 = \frac18
  %    \end{array}
  %    $
  % }
\end{subfigure}
\begin{subfigure}{.49\textwidth}
\centering
\includegraphics[width=.9\textwidth]{Pictures/ex4_char3.pdf}
  \captionsetup{labelformat=empty}
  \caption{$p=3$}
% \caption{
%      $
%      \begin{array}{l}
%        \opt \fip_3 = \conv\big(\big(\frac1{36},\frac1{12}\big),\big(\frac7{81},\frac2{81}\big)\big)\\[2mm]
%        \val \fip_3 = \frac19
%      \end{array}
%      $
% }
\end{subfigure}

\bigskip

\begin{subfigure}{.49\textwidth}
\centering
  \includegraphics[width=.9\textwidth]{Pictures/ex4_char5.pdf}\hskip .04\textwidth
  \captionsetup{labelformat=empty}
  \caption{$p=5$}
  % \caption{
  %    $
  %    \begin{array}{l}
  %      \opt \fip_5 = \opt \LP = \big\{\big(\frac2{25}, \frac3{50}\big)\big\}\\[2mm]
  %      \val \fip_5 = \frac7{50}
  %    \end{array}
  %    $
  % }
\end{subfigure}
\begin{subfigure}{.49\textwidth}
\centering
  \includegraphics[width=.9\textwidth]{Pictures/ex4_char7.pdf}
  \captionsetup{labelformat=empty}
  \caption{$p=7$}
  % \caption{
  %    $
  %    \begin{array}{l}
  %      \opt \fip_7 = \big\{\big(\frac{19}{245}, \frac3{49}\big)\big\}\\[2mm]
  %      \val \fip_7 = \frac{34}{245}
  %    \end{array}
  %    $ }
\end{subfigure}
\caption{The feasible and optimal sets of $\fip_p(A, \vv{u})$ for $A$ and $\vv{u}$ described in \Cref{ex: feas fip}, for small values of $p$}
\label{fig: feas fip}
\end{figure}


%%%%%%%%%%%%%%%%%%%%%%%%%%%%%%%%%%%%%%%%%%%%%%%%%%%%%%%%%%%%%%%%%%%%%%%%%
%END OF FIGURE DISPLAYING FRACTAL PROGRAM
%%%%%%%%%%%%%%%%%%%%%%%%%%%%%%%%%%%%%%%%%%%%%%%%%%%%%%%%%%%%%%%%%%%%%%%%%

\Cref{table: feas fip details} provides a more precise description of these quantities.
It is worth mentioning that when $p=5$, the optimal sets of $\fip_p$ and $\LP$ agree, and consequently so do their values.
On the other hand, if $p=2$, $3$, or $7$, then $\val \fip_p < \val \LP$.

%%%%%%%%%%%%%%%%%%%%%%%%%%%%%%%%%%%%%%%%%%%%%%%%%%%%%%%%%%%%%%%%%%%%%%%%%
%START OF TABLE DISPLAYING FRACTAL PROGRAM DATA
%%%%%%%%%%%%%%%%%%%%%%%%%%%%%%%%%%%%%%%%%%%%%%%%%%%%%%%%%%%%%%%%%%%%%%%%%


\begin{table}
\begin{center}
\begingroup
\setlength{\tabcolsep}{8pt} % Default value: 6pt
\renewcommand{\arraystretch}{1.4} % Default value: 1
\begin{tabular}{ccc}
  \toprule
  $p$ & $\val \fip_p$ & $\opt \fip_p$  \\
  \midrule
  $2$ & $\frac18$ & $\conv\big(\big(\frac1{20}, \frac3{40}\big),\big(\frac1{12}, \frac1{24}\big)\big)$ \\
  $3$ & $\frac19$ & $\conv\big(\big(\frac{1}{36}, \frac1{12}\big),\big(\frac7{81}, \frac2{81}\big)\big)$ \\
  $5$ & $\frac7{50}$ & $\big\{\big(\frac2{25}, \frac3{50}\big) \big\}$  \\
  $7$ & $\frac{34}{245}$ & $\big\{\big(\frac{19}{245}, \frac3{49}\big) \}$ \\
  \bottomrule
\end{tabular}
\endgroup
% The \begingroup ... \endgroup pair ensures the separation
% parameters only affect this particular table, and not any
% sebsequent ones in the document.
\end{center}
\medskip
\caption{The values and optimal sets of $\fip_p(A, \vv{u})$ for $A$ and $\vv{u}$ described in \Cref{ex: feas fip}, for small values of $p$}
\label{table: feas fip details}
\end{table}

%%%%%%%%%%%%%%%%%%%%%%%%%%%%%%%%%%%%%%%%%%%%%%%%%%%%%%%%%%%%%%%%%%%%%%%%%
%END OF FIGURE DISPLAYING FRACTAL PROGRAM DATA
%%%%%%%%%%%%%%%%%%%%%%%%%%%%%%%%%%%%%%%%%%%%%%%%%%%%%%%%%%%%%%%%%%%%%%%%%

\end{example}


\begin{proposition}
The value of $\fip_p(A,\vv{u})$ equals $\crit(\ideala, \ideald)$.
\end{proposition}

\begin{proof}
   If $e \in \NN$, then the constraints of $\IP_p(A, \vv{u}, p^e)$ imply that $p^{-e}  \feas \IP_p(A, \vv{u}, p^e)$ lies in $\feas \fip_p(A,\vv{u})$.
   It then follows from this and \eqref{crit as limit of normalized program values: eq} that
   %
   \[
      \val\fip_p(A,\vv{u}) \ge \displaystyle \lim_{e \to \infty}\frac{\val\IP_p(A,\vv{u}, p^e)}{p^e} = \crit(\ideala, \ideald).
   \]

   Next, fix a point $\vv{t} \in \sierp_{p,n}$ with $A \vv{t} < \vv{u}$.
   By definition of $\sierp_{p,n}$, we may fix an integer $s$, and a sequence $\{ \vv{t}_l \}_{l=s}^\infty$ in $\NN^n$ with $\norm{\vv{t}_l} < p$ for all $l \geq s$, such that $\vv{t} = \sum_{l=s}^{\infty} \frac{\vv{t}_l}{p^l}$.
   For each integer $e \geq 1$, set $\vv{t}^\star_{p^e} = \sum_{l=s}^{e} \frac{\vv{t}_l}{p^l}$.

   We claim that $p^e  \vv{t}^\star_{p^e}$ is feasible for $\IP_p(A, \vv{u}, p^e)$.
   To see why, first note that \Cref{dickson: R} tells us that this point satisfies the arithmetic constraint of $\IP_p(A, \vv{u}, p^e)$.
   Furthermore, $\vv{t}^{\star}_{p^e} \leq \vv{t}$, and so $A \vv{t}^{\star}_{p^e} \leq A \vv{t} < \vv{u}$, which demonstrates that $p^e \vv{t}^\star_{p^e}$ satisfies the linear constraints of $\IP_p(A, \vv{u}, p^e)$.
   Dividing by $p^e$ and taking limits, we find that
   \[
      \crit(\ideala, \ideald) = \lim_{e \to \infty} \frac{\val \IP_p(A, \vv{u}, p^e)}{p^e} \geq \lim_{e \to \infty}   \norm{\vv{t}^\star_{p^e}} = \norm{\vv{t}}.
   \]

   We have just shown that the objective function $\vv{t} \mapsto \norm{\vv{t}}$ is at most $\crit(\ideala, \ideald)$ on the set $\{ \vv{t} \in \sierp_{p,n} : A \vv{t} < \vv{u} \}$, and so the same must be true on the closure of this set, which agrees with the feasible set of $\fip_p(A, \vv{u})$, by definition.
   Restated,  $ \val \fip_p(A, \vv{u})$ is at most $\crit(\ideala,  \ideald)$, and so equality holds.
\end{proof}

\subsection{An outline}

\ \daniel[inline]{I imagine placing an outline here that points out the steps we are going to take to solve all of these optimization problems.}

\newpage
\section{Connections with Newton polyhedra}

Here, as in \Cref{sec: LPs},  we use $A$ to denote a fixed $\numvars \times n$ monomial matrix.
In fact, throughout this section, we fix a monomial pair $(A, \vv{u})$, so that $\vv{u}$ is a point in $\NN^\numvars$ with positive entries.  Our goal is to study the linear program $\LP(A, \vv{u})$ in terms of the \emph{Newton polyhedron} associated to the matrix $A$.

\begin{definition}
The \emph{Newton polyhedron} associated to the monomial matrix $A$ is the polyhedron in $\RR^\numvars$ given by
\[ \N = \conv( \col(A) ) + \cone( \canvec_1, \ldots, \canvec_\numvars), \]
where $\col(A)$ is the set of columns of $A$.
\end{definition}

\begin{definition}
   \label{defn: face}
   A proper subset $\O$ of $\N$ is a \emph{face} of $\N$ if there exists $\defpt \in \RR^\numvars$ and a nonnegative real number $\alpha$ are such that $\iprod{\defpt}{\vv{v}} \geq \alpha$ for all $\vv{v} \in \N$, with equality if and only if $\vv{v} \in \O$.
   We say that such a point $\defpt$ \emph{defines} $\O$ in $\N$.
   If $\O$ does not lie in any proper coordinate subspace of $\RR^\numvars$, we call it a \emph{standard} face of $\N$.
\end{definition}

\begin{remark}
   \label{rmk: nonnegativity of defining point}
   The requirement that the number $\alpha$ in \Cref{defn: face} be nonnegative is an arbitrary convention, which will play a role in \Cref{alpha=1: convention} below.
   We observe that the defining point $\defpt$ is nonnegative as well.
   Indeed, if $\vv{u} \in \O$, then $\vv{u} + \canvec_i$ lies in $\N$, for each $i$.
   Consequently, $\alpha \le \iprod{\defpt}{\vv{u} + \canvec_i} = \alpha + c_i$, and hence $c_i \ge 0$ for each $i$.
   \pedro{Looks like the fact that $\alpha$ is nonnegative doesn't really matter here.}
\end{remark}

In this article, we are largely concerned with standard faces, and for those we have a helpful convention.

\begin{convention}
\label{alpha=1: convention}
Take $\O \subseteq \RR^\numvars$, $\defpt \in \RR^\numvars$, and $\alpha \in \RR$ as above.  If $\O$ is standard, then $\alpha$ must be positive, which allows us to rescale $\defpt$ so as to assume that $\alpha = 1$.   Thus, throughout this article, we always assume that we have normalized in this way when considering defining points of standard faces.
\end{convention}

This convention leads to the following useful observation.

%Convention \Cref{alpha=1: convention} leads to the following useful observation.

\begin{proposition}\label{prop: inner product with columns of A}
   If $\defpt \in \RR^\numvars$ defines a standard face $\O$ of $\N$, and $\vv{s}~\in~\RR^n$ has nonnegative entries, then the inner product $\iprod{\defpt}{A\vv{s}}$ is at least $\norm{\vv{s}}$, ~and~equality holds if and only if $s_i = 0$ whenever~the~$i$-th~column~of~$A$~is~not~in~$\O$.
\end{proposition}

\begin{proof}
If $\vv{a}_i$ denotes the $i$-th column of $A$, then \Cref{alpha=1: convention} and the nonnegativity of $\vv{s}$ imply that $\iprod{\defpt}{\vv{a}_i}  s_i \geq s_i$ for every $1 \leq i \leq n$, with equality if and only if $s_i = 0$ or $\vv{a}_i \in \O$.
Thus,
\[ \iprod{\defpt}{A\vv{s}} = \sum_{i=1}^n \iprod{\defpt}{\vv{a}_i} s_i \geq  \sum_{i=1}^n s_i  = \norm{\vv{s}},\]
and equality holds if and only if $s_i = 0$ whenever $\vv{a}_i \notin \O$.
\end{proof}

% The following proposition is well known to experts, but we include the short proof to keep the article self-contained.

% \begin{proposition}
%    \label{face: P}
%    If $\defpt \in \RR^\numvars$ defines a face $\O$ of a Newton polyhedron $\N$, then $\defpt$ is nonnegative, and the $i$-th coordinate of $\defpt$ is zero if and only if $\vv{u} + \lambda \canvec_i \in \O$  for every $\vv{u} \in \O$ and $\lambda > 0$.
%    In particular, the supporting indices of $\defpt$ depend only on $\O$, and $\O$ is bounded if and only if $\defpt$ is positive.
% \end{proposition}

% \begin{proof}
%    If $\vv{u} \in \O$, then adding to $\vv{u}$ any nonnegative point in $\RR^\numvars$ produces a point in $\N$.
%    In particular, if $\iprod{\defpt}{\vv{u}} = \alpha$, then $\iprod{\defpt}{\vv{u} + \lambda \canvec_i} \geq \alpha$ for every standard basis vector $\canvec_i$ in $\RR^\numvars$ and $\lambda > 0$.
%    This observation implies that $c_i = \iprod{\defpt}{\canvec_i} \ge 0$ for each $i$, so $\defpt \geq \vv{0}$, and that $\vv{u} + \lambda \canvec_i \in \O$ for every $\lambda > 0$ if and only if $c_i = \iprod{\defpt}{\canvec_i} = 0$.

% Similar logic will show that if $\rb(\O) \coloneqq  \{ \canvec_i : \iprod{\defpt}{\canvec_i} = 0\}$, then
% \begin{equation}
% \label{face: e}
% \O =  \conv( \col(A) \cap \O ) + \cone(\rb(\O)),
% \end{equation}
% where we agree that the $\cone(\emptyset) = \{\vv{0}\}$.  We see from this that $\O$ is bounded if and only if $\rb(\O)$ is empty, which is equivalent to the third assertion.
% \end{proof}

\begin{definition}
   The \emph{minimal face} of $(A, \vv{u})$ is the face $\mf(A, \vv{u})$ of $\N$ that is minimal, with respect to inclusion, among the faces containing the unique point where the cone generated by $\vv{u}$ intersects the boundary of $\N$.\footnote{Recall that the intersection of two faces of $\N$ is also a face of $\N$. Thus, as minimality here is with respect to inclusion, it follows that there is a unique such minimal face.}
\end{definition}

\begin{proposition}
   \label{FT descriptions: P}
   The following numbers coincide\textup:
   \begin{enumerate}
      \item\label{value} The value of the linear program $\LP(A, \vv{u})$.
      \item\label{limit} The limit of $q^{-1}\val \IP(A,\vv{u},q)$ as $q$ tends to infinity.
      \item\label{lambda} The unique positive real number $\lambda$ with the property that the scaled point $\lambda^{-1}\vv{u}$ lies in the boundary of $\N$.
      \item\label{new ip} The inner product $\iprod{\vv{c}}{\vv{u}}$, where $\vv{c} \in \RR^\numvars$ is a point defining the minimal face $\mf(A, \vv{u})$ in $\N$ \textup(adhering to \Cref{alpha=1: convention}\textup).
   \end{enumerate}
\end{proposition}

As noted in \Cref{ft as val LP: P}, the quantity appearing in \Cref{FT descriptions: P}(1) can be described algebraically as an $F$-threshold associated to a monomial pair, which motivates the following definition.

\begin{definition}
\label{FT: D}
   The \emph{$F$-threshold} of a monomial pair $(A, \vv{u})$ is the number described in \Cref{FT descriptions: P}.  We denote it by $\ft{A}{\vv{u}}$.
\end{definition}

\begin{proof}[Proof of \Cref{FT descriptions: P}]
 \Cref{ft as val LP: P} and \eqref{ft as limit of normalized program values: eq} tells us that the quantities in \eqref{value} and \eqref{limit} above agree.  In what follows, let $\lambda$ be as in \eqref{lambda} above.


    Let $\O = \mf(A,\vv{u})$, and fix a point $\vv{c} \in \RR^\numvars$ defining $\O$ in $\N$.  Our choice of $\lambda$ allows us to write
    $\vv{u} = \lambda \vv{w}$ for some $\vv{w} \in \O$,  and in view of \Cref{alpha=1: convention}, it follows that $\iprod{\vv{c}}{\vv{u}} = \lambda \iprod{\vv{c}}{\vv{w}} = \lambda$.
   This establishes the equality of the numbers described in \eqref{lambda} and \eqref{new ip}.

Therefore, to conclude our proof,  it remains to show that $\val \LP = \lambda$, where $\LP = \LP(A, \vv{u})$.  Towards this, note that if $\vv{s} \in \feas \LP$, then $\vv{s}\ge \vv{0}$ and $A \vv{s} \leq \vv{u}$, and consequently
  $\norm{\vv{s}} \le \iprod{\vv{c}}{A \vv{s}} \leq \iprod{\vv{c}}{\vv{u}} = \lambda$,
   where the first inequality follows from \Cref{prop: inner product with columns of A}, and the second from the nonnegativity of $\vv{c}$, shown in \Cref{rmk: nonnegativity of defining point}.
   This shows that the value of $\LP$ is at most $\lambda$. 

   % On the other hand, \eqref{face: e} and our choice of $\O$ imply that
   % \begin{equation}\label{cone containment: e}
   %    \lambda^{-1} \vv{u} \in \O = \conv(\col(A) \cap \O) + \cone(\rb(\O)).
   % \end{equation}
   % Multiplying by $\lambda$, we obtain an expression $\vv{u} = A \vv{s} + \vv{w}$ with $\vv{s}, \vv{w} \geq \vv{0}$ and $\norm{\vv{s}} = \lambda$.
   % Evidently, the point $\vv{s}$ is feasible for $\LP$, and so $\val \LP \geq \lambda$.
   On the other hand, since $\lambda^{-1}\vv{u} \in \N = \conv(\col(A)) + \cone(\vv{e}_1,\ldots,\vv{e}_\numvars)$, we can write $\lambda^{-1}\vv{u} = A\vv{s}+\vv{w}$ with $\vv{s}, \vv{w} \geq \vv{0}$ and $\norm{\vv{s}} = 1$.
   Multiplying by $\lambda$, we see that  $\lambda\vv{s}$ is feasible for $\LP$, and so the value of $\LP$ is at least $\lambda$.
\end{proof}

\begin{example}\label{ex: ft}
   We examine the different characterizations of $\ft{A}{\vv{u}}$ given in \Cref{FT descriptions: P}, for 
   \[A=\begin{bmatrix}5&3&4\\ 5&4&3\\ 2&8&5\end{bmatrix} \quad \text{and} \quad \vv{u} =
      \begin{bmatrix} 1 \\ 1\\ 1 \end{bmatrix}.\]
   The Newton polyhedron $\N$ of $A$ is shown in \Cref{fig: newton polyhedron}.
   \begin{figure}
   \centering
   \begin{subfigure}{.48\textwidth}
      \centering
      \includegraphics[width=.9\textwidth]{Pictures/newton_polyhedron.pdf}\\[1.4mm]
      \caption{The Newton polyhedron of $A$}
      \label{fig: newton polyhedron}
   \end{subfigure}
   \begin{subfigure}{.48\textwidth}
      \centering
      \includegraphics[width=.8\textwidth]{Pictures/splitting_polytope.pdf}
      \caption{The feasible region of $\LP(A,\vv{u})$}
      \label{fig: splitting polytope}
   \end{subfigure}
      \caption{Illustration for \Cref{ex: ft}}
   \label{fig: newton polyhedron and splitting polytope}
   \end{figure}
   The point $(17/4)\cdot \vv{u}$, shown in white, lies in the relative interior of the face
   \[\O = \conv(\col(A)) + \cone(\canvec_2),\]
   shown in blue. \daniel{Is this really blue?  The 3(B) figure looks blue; this looks purple}
   \pedro{It is blue, but looks purplish because of Mathematica's lighting.
      Maybe best would be to avoid mentions to colors, as these things may end up being printed in B\&W.
   }
   Thus, $\mf(A,\vv{u}) = \O$, and the description given in \Cref{FT descriptions: P}\eqref{lambda} tells us that $\ft{A}{\vv{u}} = 4/17$.

   The minimal face $\O$ is defined by the point $\defpt = (3/17, 0, 1/17)$, so \Cref{FT descriptions: P}\eqref{new ip} tells us that
   $\ft{A}{\vv{u}} = \iprod{\defpt}{\vv{u}} = 4/17$.

   From yet another perspective, \Cref{fig: splitting polytope} shows the feasible region of the linear program $\LP(A,\vv{u})$, with its optimal set,
   \[\opt \LP(A,\vv{u}) = \conv\bigg(\bigg(\frac1{17}, 0, \frac3{17}\bigg),\bigg(\frac2{17}, \frac1{17}, \frac1{17}\bigg)\bigg),\]
   highlighted in green.
   Thus, \Cref{FT descriptions: P}\eqref{value} again tells us that
   \[\ft{A}{\vv{u}} = \val \LP(A,\vv{u}) = \bigg\|\bigg(\frac1{17}, 0, \frac3{17}\bigg)\bigg\| = \frac4{17}.\]
\end{example}

\subsection{Characterizing optimal points of $\LP(A,\vv{u})$}

The identity \eqref{cone containment: e} above implies that
 %
%\[ \vv{u} \in \cone (\O) = \cone \left( (\col(A) \cap \O) \cup \rb(\O)  \right). \]
$\vv{u}$ is a conical combination of the columns of $A$ lying in $\O$ and the points in the recession basis of $\O$.
Typically, there are many ways to express $\vv{u}$ as such a conical combination, and as we see below, the set of all such expressions parameterizes  $\opt \LP(A, \vv{u})$.

\begin{proposition}\label{opt set: P}
   Let $(A,\vv{u})$ be a monomial pair, where $A$ is a $\numvars\times n$ matrix.
   A point $\vv{s} \in \RR^n$ is optimal for $\LP(A, \vv{u})$ if and only if it satisfies the following conditions.
\begin{enumerate}
\item  \label{mc coords: e} The coordinates of $\vv{s}$ are nonnegative, and the $i$-th coordinate of $\vv{s}$ is zero whenever the $i$-th column of $A$ is not contained in $\O = \mf(A, \vv{u})$.
\item  \label{mc decomposition: e} $\vv{u} = A \vv{s} + \vv{w}$ for some $\vv{w} \in  \cone(\rb(\O))$.
%\item  \label{mc sum: e}$\norm{\vv{s}} = \ft{A}{\vv{u}}$.
\end{enumerate}
\end{proposition}

\begin{proof}
   Set $\LP = \LP(A, \vv{u})$ and $\lambda = \val \LP $, and fix $\defpt \in \RR^\numvars$ that defines the face $\O = \mf(A, \vv{u})$ in the Newton polyhedron of $A$.
   Let $\vv{s} \in \RR^n$ and set $\vv{w} = \vv{u} - A\vv{s}$.
   By \Cref{prop: inner product with columns of A},
   %
   \begin{equation}\label{eq 1}
      \lambda = \iprod{\defpt}{\vv{u}} = \iprod{\defpt}{A \vv{s}} + \iprod{\defpt}{\vv{w}} \geq \norm{\vv{s}} + \iprod{\defpt}{\vv{w}}.
   \end{equation}
   %
   If $\vv{s}$ is optimal for $\LP$, then $\vv{s} \ge \vv{0}$, $\vv{w} \ge \vv{0}$, and $\norm{\vv{s}} = \lambda$, and \eqref{eq 1} shows that $\iprod{\defpt}{A \vv{s}} = \norm{\vv{s}}$ and $\iprod{\defpt}{\vv{w}} = 0$.
   The first identity and \Cref{prop: inner product with columns of A} show that $\vv{s}$ satisfies~(1); the second identity shows that $\vv{w}\in \cone(\rb(\O))$, so $\vv{s}$ satisfies~(2).
   Conversely, if $\vv{s}$ satisfies (1) and (2), then $\vv{s}$ is feasible for $\LP$, $\iprod{\defpt}{A \vv{s}} = \norm{\vv{s}}$, and $\iprod{\defpt}{\vv{w}} = 0$.
   By \eqref{eq 1}, $\norm{\vv{s}} = \lambda$, so $\vv{s}$ is optimal for $\LP$.
\end{proof}

\begin{theorem}
\label{uniform denominators for vertices:  T}
Given a monomial matrix $A$, there exists a positive integer $\denom = \denom(A)$ such that for every monomial pair $(A, \vv{u})$, every vertex of $\opt \LP(A, \vv{u})$ is rational with denominator $\denom$.
\end{theorem}

\begin{proof}
Fix a monomial pair $(A, \vv{u})$. Set $\LP = \LP(A, \vv{u})$ and $\O = \mf(A, \vv{u})$.  Let $M$ be the matrix obtained from $A$ by omitting any columns not in $\O$, and inserting as a column each standard basis vector in $\rb(\O)$.  Finally, let $\denom = \denom(\O)$ be the least common multiple of all the nonzero minors of $M$.

If $\Q$ is the polyhedron consisting of all $\vv{t}$ in the domain of $M$ with $\vv{t} \geq \vv{0}$ and $M \vv{t} = \vv{u}$, then \Cref{opt set: P} implies that there exists a linear bijection $\opt \LP
\leftrightarrow \Q$.  Furthermore, if $\vv{t}^{\ast}$ is a vertex of $\Q$, then \Cref{vertex: P} allows us to solve for the nonzero coordinates of $\vv{t}^{\ast}$ in the equation $M \vv{t}^{\ast} = \vv{u}$.  In particular, the fact that $\vv{u}$ has integer coordinates implies that the nonzero coordinates of $\vv{t}^{\ast}$ are rational with denominator $\denom = \denom(\O)$.  The linear bijection $\opt \LP \leftrightarrow \Q$ implies the same must be true for every vertex of $\opt \LP$.
\pedro{I think here we need to emphasize that this bijection is given by matrices with integral coordinates}

Our assertion then follows from the observation that since $A$ is fixed, as $(A, \vv{u})$ varies, there are only finitely many possibilities for $\O = \mf(A, \vv{u})$.
\end{proof}

\subsection{Special points and denominators}

Technicalities that arise in future sections whenever the minimal face of a monomial pair $(A, \vv{u})$ is unbounded force us to consider a certain distinguished subset of optimal points, in which we require a strengthening of condition \eqref{mc decomposition: e} in \Cref{opt set: P}.

\begin{definition}
\label{mc: D}
Let $(A,\vv{u})$ be a monomial pair, and $\O = \mf(A, \vv{u})$.  A point $\vv{s}$ is a \emph{special point} for $(A, \vv{u})$ if it satisfies the following conditions.
\begin{enumerate}
\item $\vv{s} \in \opt \LP(A, \vv{u})$.
\item $\vv{u} = A \vv{s} + \vv{w}$ for some $\vv{w}$ in the relative interior of $\cone ( \rb(\O))$.
\end{enumerate}
The set of all such points is denoted $\sp(A, \vv{u})$, and the set of all such points with rational coordinates is denoted $\sp_{\QQ}(A, \vv{u})$.
\end{definition}

\begin{proposition}
   \label{opt versus mc: P}
   Let $(A,\vv{u})$ be a monomial pair.
   If $\O = \mf(A, \vv{u})$ is bounded, then $\sp(A, \vv{u}) = \opt \LP(A, \vv{u})$.  Otherwise,  $\sp(A, \vv{u})$ is a nonempty convex subset lying between $\opt \LP(A, \vv{u})$ and $\ri \opt \LP(A, \vv{u})$.
\end{proposition}

\begin{proof}
   If $\O$ is bounded, then $\rb(\O) = \emptyset$, and so $\cone( \rb(\O)) = \{\vv{0} \}$ is equal to its relative interior.
   Thus, in view of \Cref{opt set: P}, condition (2) in \Cref{mc: D} is redundant, and $\sp(A, \vv{u}) = \opt \LP(A, \vv{u})$.
   
   Next, set $\lambda = \ft{A}{\vv{u}}$ and assume that $\O$ is unbounded.
   The minimality of $\O$ implies that $\lambda^{-1}\vv{u}$ is not in any proper face of $\O$, and therefore, must lie in its relative interior.  Further, as the relative interior operation on convex sets commutes with Minkowski sums---see, \eg \cite[Theorem 4.10(b)]{vantiel.convex_analysis}---the decomposition in \eqref{face: e}  implies that $\vv{u} = \vv{v} + \vv{w}$ with $\vv{v} \in \lambda \conv(\col(A) \cap \O)$ and $\vv{w} \in \ri \cone(\rb(\O))$.  Any realization of $\vv{v}$ as $\lambda$ times a convex combination of the points in $\col(A) \cap \O$ then determines a special point.

 We have just shown that $\sp(A, \vv{u})$ is nonempty, and it is clear that this set is convex.
 Before proceeding to show that $\sp(A, \vv{u})$ contains the relative interior of the optimal set of $\LP = \LP(A,\vv{u})$,
  recall that the relative interior of a polytope consists of all convex combinations with \emph{positive coefficients} of the vertices of the polytope, while the polytope itself consists of all convex combinations of its vertices.

  Let $\vv{s} \in \ri \opt \LP$, and suppose $\vv{s}$ is not special for $(A,\vv{u})$.
  Then there exists some $i$ with $\canvec_i \in \rb(\O)$ for which $A\vv{s}$ agrees with $\vv{u}$ in the $i$-th coordinate.
  The characterization of relative interior above then implies that the same is true for every \emph{vertex} of $\opt \LP$, and consequently for every \emph{point} of $\opt \LP$.
  But this shows that $\sp(A,\vv{u})$ is empty---a contradiction.
\end{proof}

\begin{example}
   The containments asserted in \Cref{opt versus mc: P} may all be proper.
   Indeed, if $A$ and $\vv{u}$ are as in \Cref{ex: ft}, then $\opt \LP(A,\vv{u})$, highlighted in green in \Cref{fig: splitting polytope}, is the line segment connecting $(1/17, 0, 3/17)$ and $(2/17, 1/17, 1/17)$.
   Its relative interior excludes both endpoints, while the set of special points for $(A,\vv{u})$ excludes only the first endpoint.
\end{example}

% Consider the monomial matrix \[ A = \begin{bmatrix} a & 0 & c \\ 0 & b & c \\ 0 & 0 & d \end{bmatrix} \]
% where $a,b,c$ are positive integers with $1/a + 1/b = 1/c$ and $d$ is any integer with $d>c$.  The maximal face of the splitting polytope is the edge connecting the points \[ \left( \frac{d-c}{da}, \frac{d-c}{db}, \frac{1}{d} \right) \text{ and } \left( \frac{1}{a}, \frac{1}{b}, 0 \right).\]  On the other hand, it is easy to check that the special points for $(A, \vv{1})$ consist of the points on this edge except for the first of these two  points.

\begin{definition}  Suppose that $A$ is a monomial matrix. A \emph{special denominator} for $A$ is a positive integer $\denom = \denom(A)$ such that for every monomial pair $(A, \vv{u})$, there exists a point $\vv{s} \in \sp(A, \vv{u})$ so that $\denom \cdot \vv{s}$ has integer coordinates.
\end{definition}

\begin{theorem}
\label{special-denominators-exist:  T}
Special denominators exist.
\end{theorem}

\begin{proof}
   Let $A$ be a monomial matrix.
   Let $\ell_{\circ}$ be an integer satisfying the property described in \Cref{uniform denominators for vertices:  T} relative to $A$, and fix a monomial pair $(A, \vv{u})$.
   If $\O = \mf(A, \vv{u})$ is bounded, then $\sp(A, \vv{u}) = \opt \LP(A, \vv{u})$ by \Cref{opt versus mc: P}, and so every vertex in this set has denominator~$\ell_{\circ}$.

   Next, suppose $\O$ is unbounded, so that $A$ has $\numvars \geq 2$ many rows.
   Without loss of generality, suppose that $\rb(\O) = \{ \canvec_1, \ldots, \canvec_t \}$ for some $1 \leq t \leq \numvars-1$, and fix \emph{positive} integers $\numvars_1, \ldots, \numvars_t$ that sum to $\numvars-1$.
   As demonstrated in the  proof of \Cref{opt versus mc: P}, for every index  $1 \leq i \leq t$, there exists a vertex $\vv{s}_i$ of $\opt \LP(A, \vv{u})$ for which $A \vv{s}_i$ is less than $\vv{u}$ in the $i$-th coordinate.
   It then follows from the definition of special point that the point
   \[ \frac{ d_1 \cdot \vv{s}_1 + \cdots + d_t \cdot  \vv{s}_t}{\numvars-1}  \]
   lies in $\sp_{\QQ}(A, \vv{u})$ and has denominator $(\numvars-1)\ell_{\circ}$.
\end{proof}

\daniel[inline]{I cannot think of where to place this.  In fact, I am not certain that it is used anymore;  at some point, I think it came up when discussing the tree later, but I don't think it is relevant anymore.  Maybe we should omit it in that case?}

\subsection{Discreteness of $F$-thresholds}

The following is a consequence of the discreteness of the $F$-jumping exponents of an ideal in a regular ring \cite[Theorem~3.1]{blickle+mustata+smith.discr_rat_FPTs}.
However, to keep our discussion self-contained, we include an elementary proof in our specialized setting.

\begin{proposition}
\label{discreteness: L}
Given a monomial matrix $A$ and a real number $\beta > 0 $, there are only finitely many numbers of the form $\ft{A}{\vv{u}}$ less than $\beta$.
\end{proposition}

\begin{proof}
   It suffices to show that there are only finitely many numbers $\ft{A}{\vv{u}}$ less than $\beta$ with $\mf(A, \vv{u}) = \O$ being fixed.
   Consider such a pair, and suppose $\defpt$ defines $\O$ in the Newton polyhedron of $A$.
   Then $\ft{A}{\vv{u}} = \iprod{\defpt}{\vv{u}}$, and the nonnegativity of $\defpt$ implies that, as $\vv{u}$ ranges over positive integral points, this inner product takes on only finitely many values less than $\beta$.
   Indeed, when $c_i \ne 0$, this upper bound allows only finitely many choices for $u_i$, while when $c_i = 0$, the value of $u_i$ does not affect the inner product.
\end{proof}


\newpage
\section{Collapsing}

The following proposition is well known to experts, but we include the short proof to keep the article self-contained.

\begin{proposition}
   \label{face: P}
   If $\defpt \in \RR^\numvars$ defines a face $\O$ of a Newton polyhedron $\N$, then the $i$-th coordinate of $\defpt$ is zero if and only if $\vv{u} + \lambda \canvec_i \in \O$  for every $\vv{u} \in \O$ and $\lambda > 0$.
   In particular, the supporting indices of $\defpt$ depend only on $\O$, and $\O$ is bounded if and only if $\defpt$ is positive.
\end{proposition}

\begin{proof}
   Let $\vv{u} \in \O$, and suppose $\iprod{\defpt}{\vv{u}} = \alpha$.
   Then $\vv{u} + \lambda \canvec_i$ lies in $\O$ if and only if $\iprod{\defpt}{\vv{u} + \lambda \canvec_i} = \alpha + \lambda c_i$ equals $\alpha$.
   Clearly, this will happen for every (or equivalently, some) $\lambda > 0$ if and only if $c_i = 0$. 

Similar logic will show that if $\rb(\O) \coloneqq  \{ \canvec_i \in \RR^\numvars : \iprod{\defpt}{\canvec_i} = 0\}$, then
\begin{equation}
\label{face: e}
\O =  \conv( \col(A) \cap \O ) + \cone(\rb(\O))
\end{equation}
where we agree that the $\cone(\emptyset) = \{\vv{0}\}$.  We see from this that $\O$ is bounded if and only if $\rb(\O)$ is empty, which is equivalent to the third assertion.
\end{proof}

\begin{definition}
   If $\defpt \in \RR^\numvars$ defines $\O$, then the \emph{recession basis} of $\O$ is the set $\rb(\O)$ of all standard basis vectors $\canvec_i$ in $\RR^\numvars$ such that the $i$-th coordinate of $\defpt$ is zero, and the \emph{recession subspace} of $\O$ is the subspace $\rs(\O)$ of $\RR^\numvars$ spanned by $\rb(\O)$.
\end{definition}

As noted above, these definitions depend only on $\O$, but not on the choice of $\defpt$.
In view of the Minkowski--Weyl Theorem (see \Cref{ss: euclidean spaces and convexity}), equation \eqref{face: e} implies that the cone generated by $\rb(\O)$ is the recession cone of $\O$, motivating our choice of terminology.



This section concerns an operation used to get around some technical difficulties that arise when dealing with an unbounded minimal face.
Recall that if a face $\O$ of a Newton polyhedron in $\RR^\numvars$ is defined by a point~$\defpt$, then its recession basis is the set $\rb(\O)$ consisting of all standard basis vectors $\canvec_i$ in $\RR^\numvars$ such that $c_i = 0$, and its recession subspace is the subspace $\rs(\O)$ of $\RR^\numvars$ spanned by $\rb(\O)$.

\begin{definition}
\label{collapse: D}
 Suppose that $\O$ is a proper face of the Newton polyhedron $\N$ of a monomial matrix $A$ with $\numvars$ rows.

\begin{enumerate}
   \item The set $\rb(\O)^{\perp}$ is the complement of $\rb(\O)$ in $\{ \canvec_1, \ldots, \canvec_\numvars \}$, and $\rs(\O)^\perp$ is the subspace of $\RR^\numvars$ spanned by $\rb(\O)^\perp$, that is, the orthogonal complement of $\rs(\O)$ in $\RR^\numvars$.
   (The assumption that $\O \neq \N$ implies that $\rb(\O)^{\perp} \neq \emptyset$.)
\item The \emph{collapse} of a subset $X$ of $\RR^\numvars$ along $\O$ is the image of $X$ under the projection $\RR^\numvars \stackrel{B}{\longrightarrow} \RR^{\#\rb(\O)^\perp}$, where $B$ is the matrix whose rows are the vectors $\canvec_i$ in $\rb(\O)^\perp$.
In other words, the collapse of each point $\vv{v}$ along $\O$ is obtained by deleting each coordinate $v_i$ with $\canvec_i\in \rb(\O)$.
\item The \emph{collapse} of $A$ along $\O$ is the matrix obtained from $A$ by collapsing each of its columns along $\O$, or in other words, the matrix $BA$, obtained by deleting the $i$-th row of $A$ whenever $\canvec_i\in \rb(\O)$.
\end{enumerate}
\end{definition}

Below, we adopt the notation established in \Cref{collapse: D}, and we use an overbar to denote the collapse along $\O$.

\begin{remark}
   It is clear from the definition that $\collapse{A\vv{k}} = \collapse{A}\vv{k}$, for each $\vv{k}$ in the domain of $A$.
   Indeed, both are obtained by left-multiplying $\vv{k}$ by $BA$, where $B$ is as in \Cref{collapse: D}.
\end{remark}

\begin{remark}
   \label{collapse of a defining vector: R}
   If $\defpt \in \RR^\numvars$ defines $\O$ in $\N$, then the coordinates deleted in $\collapse{\defpt}$ correspond precisely to the zero coordinates of $\defpt$.
   Consequently, $\iprod{\defpt}{\vv{u}} = \iprod{\collapse{\defpt}}{\collapse{\vv{u}}}$ for every $\vv{u} \in \RR^\numvars$. \daniel[inline]{Restate this?}
\end{remark}



\begin{remark}
\label{collapse of monomial is monomial: R}
The collapsed matrix $\collapse{A}$ is monomial.  Indeed, each row of $\collapse{A}$ is a row of $A$, and hence is nonzero.   On the other hand, if $\defpt \in \RR^\numvars$ defines $\O$, then \Cref{collapse of a defining vector: R} implies that the inner product of $\collapse{\defpt}$ with every column of $\collapse{A}$ is at least one.  In particular, the columns of $\collapse{A}$ are nonzero.
\end{remark}

\begin{proposition}\label{collapse of Newton polyhedron: P}
   If $\M$ is the Newton polyhedron in the coordinate subspace $\rs(\O)^{\perp}$ of the monomial matrix $\collapse{A}$, then $\collapse{\O}$ is a bounded face of $\M = \collapse{\N}$.
   In addition, if $\defpt \in \RR^\numvars$ defines $\O$ in $\N$, then $ \collapse{\defpt}$ defines $\collapse{\O}$ in $\M$.
\end{proposition}

\begin{proof}
By definition, the Newton polyhedron $\M$ equals
%
\[  \conv( \col(\collapse{A}) ) + \cone(\rb(\O)^{\perp}) =  \ol{\conv( \col(A))} + \ol{\cone(\canvec_1, \ldots, \canvec_\numvars)} =  \collapse{\N}.\]

Given \Cref{collapse of a defining vector: R}, it is not difficult to verify that $\collapse{\defpt}$ defines $\collapse{\O}$ in $\M$ whenever $\defpt \in \RR^\numvars$ defines $\O$ in $\N$.  The positivity of $\collapse{\defpt}$ in $\rs(\O)^{\perp}$ then implies that $\collapse{\O}$ is bounded.  Alternatively, one may project \eqref{face: e} to $\rs(\O)^{\perp}$ to see that the collapsed face $\collapse{\O}$ is the polytope $\conv( \collapse{ \col(A) \cap \O}) = \conv( \col(\collapse{A}) \cap \collapse{\O})$.
\end{proof}

Below, we describe the relationship between collapses and the other notions introduced in \Cref{sec: LPs}.

\begin{proposition}
   \label{collapse of mf and mc: P}
   Consider a monomial pair $(A, \vv{u})$, and let an overbar denote collapse along $\O = \mf(A, \vv{u})$.
   Then the following hold.
\begin{enumerate}
\item $\mf(\collapse{A}, \collapse{\vv{u}}) = \collapse{\O}$ and $\ft{A}{\vv{u}} = \ft{\collapse{A}}{\collapse{\vv{u}}}$.
\item Each optimal point for $\LP(A, \vv{u})$ is also optimal for $\LP(\collapse{A}, \collapse{\vv{u}})$.
\item Each special point for $(A, \vv{u})$ is a special point for $(\collapse{A}, \collapse{\vv{u}})$.
\end{enumerate}
\end{proposition}

\begin{proof}
   Set $\lambda = \ft{A}{\vv{u}}$, so that $\lambda^{-1} \vv{u}$ lies in the relative interior of $\O$.
   It is clear that projection preserves relative interiors, and so $\lambda^{-1}\collapse{\vv{u}}$ must lie in the relative interior of $\collapse{\O}$, which is a bounded face of $\collapse{\N}$ by \Cref{collapse of Newton polyhedron: P}.
   This observation demonstrates both that $\collapse{\O}$ is the minimal face of $\collapse{\N}$ containing $\lambda^{-1} \collapse{\vv{u}}$, and that $\lambda = \ft{\collapse{A}}{\collapse{\vv{u}}}$.
   Consequently,
   \[ \val \LP(A, \vv{u}) = \ft{A}{\vv{u}} = \ft{\collapse{A}}{\collapse{\vv{u}}} = \val \LP(\collapse{A}, \collapse{\vv{u}}). \]
%
By construction, each nonzero row of $\collapse{A}$ is a row of $A$, and so the constraints of $\LP(\collapse{A}, \collapse{\vv{u}})$ are a subset of those of $\LP(A, \vv{u})$.  It follows that any optimal point for $\LP(A, \vv{u})$ must be optimal for $\LP(\collapse{A}, \collapse{\vv{u}})$.  The boundedness of $\collapse{\O}$, \Cref{opt versus mc: P}, and the preceding observation allows us to conclude that
\begin{equation*}
   \sp(A, \vv{u}) \subseteq \opt \LP(A, \vv{u}) \subseteq \opt \LP(\collapse{A}, \collapse{\vv{u}}) = \sp(\collapse{A}, \collapse{\vv{u}}).
   \qedhere
\end{equation*}
\end{proof}

\begin{example}
   \label{ex: collapse}
   Let $A$ and $\vv{u}$ be as in \Cref{ex: ft}.
   Then $\O \coloneqq \mf(A,\vv{u}) = \conv(\col A) + \cone(\canvec_2)$, so $\rb(\O) = \{\canvec_2\}$.
   \Cref{fig: newton polyhedron of collapse} shows the Newton polyhedron of the collapse $\collapse{A}$ of $A$ along $\O$ (compare with \Cref{fig: newton polyhedron}).
   \begin{figure}
   \centering
   \begin{subfigure}{.49\textwidth}
      \centering

      \ \\[.1mm] \

      \includegraphics[width=.9\textwidth]{Pictures/newton_polyhedron_of_collapse.pdf}\\[2mm]
      \caption{The Newton polyhedron of $\collapse{A}$}
      \label{fig: newton polyhedron of collapse}
   \end{subfigure}
   \begin{subfigure}{.49\textwidth}
      \centering
      \includegraphics[width=.8\textwidth]{Pictures/opt_for_collapse_may_change.pdf}
      \caption{The feasible region of $\LP(\collapse{A},\collapse{\vv{u}})$}
      \label{fig: splitting polytope of collapse}
   \end{subfigure}
   \caption{Illustration for \Cref{ex: collapse}}
   \label{fig: collapse}
   \end{figure}
   The point $(17/4)\cdot\collapse{\vv{u}}$, shown in white, lies in the relative interior of $\collapse{\O}$, shown in blue; thus, $\collapse{\O} = \mf(\collapse{A},\collapse{\vv{u}})$ and $\ft{\collapse{A}}{\collapse{\vv{u}}} = 4/17 = \ft{A}{\vv{u}}$.

   The feasible region of $\LP(A,\vv{u})$, shown in \Cref{fig: splitting polytope}, is properly contained in the feasible region for $\LP(\collapse{A},\collapse{\vv{u}})$, shown in \Cref{fig: splitting polytope of collapse}, which highlights the difference between those sets in yellow.
   \Cref{fig: splitting polytope of collapse} also shows that the optimal set of $\LP(A,\vv{u})$ is properly contained in the optimal set of $\LP(\collapse{A},\collapse{\vv{u}})$, highlighting the difference between these sets in red.
   Thus, the containment established in \Cref{collapse of mf and mc: P}(2) may be proper.
\end{example}

\newpage
\section{An auxiliary integer program}



\subsection{Canonical feasible points}

We highlight a simple construction that associates to any point in $\sp_{\QQ}(A, \vv{u})$ a feasible point for $\IP(A, \vv{u}, q)$.
As a part of this, we call upon some basic notions from modular arithmetic.

\begin{definition} If $m,n \in \ZZ$ are positive, then $\lpr{m}{n}$ is the \emph{least positive residue} of $m$ modulo $n$, \ie $m \equiv \lpr{m}{n} \bmod n$ and $1 \leq \lpr{m}{n} \leq n$.
\end{definition}

\begin{definition}
   \label{tail: D}
   Let $q$ be a positive integer.
   If $\lambda = a/b$ for some \emph{positive} integers $a$ and $b$, then we define
   \[ \tail{\lambda}_q = \frac{ \lpr{aq}{b}}{b}. \]
   Clearly, this expression depends only on $\lambda$, but not on the integers $a$ and $b$.
   Moreover, we set $[0]_q = 0$, and if $\vv{s} \in \QQ^n$ is nonnegative, then we define $\tail{\vv{s}}_q$ to be the point in $\QQ^n$ obtained by applying this operation to each coordinate of $\vv{s}$.
\end{definition}

\begin{remark}
\label{tail-basics: R}
If $q \in \ZZ$ is positive and $\lambda = a/b$ with $a$ and $b$ positive integers, then $\tail{\lambda}_q$ is positive and rational, at most $1$,  and depends on  $q$ modulo $b$, but not on $q$ itself.
Furthermore,
%
\[ \lambda q - \tail{\lambda}_q = \frac{aq-\lpr{aq}{b}}{b} \] is an integer, and in fact, is the \emph{greatest integer less than $\lambda q$}. \daniel[inline]{Elaborate, describe in terms of rounding up, compare with rounding down}
\end{remark}


\begin{lemma}
   \label{less than u: L}  Suppose that $\vv{s}$ is a special point for a monomial pair $(A, \vv{u})$.
   If $\vv{t}$ is a point in the domain of $A$ with $\vv{0} \leq \vv{t} \leq \vv{s}$, with the latter bound strict in every coordinate in which $\vv{s}$ is positive, then $A \vv{t} < \vv{u}$.
\end{lemma}

\begin{proof}  Set $\O = \mf(A, \vv{u})$.  The fact that $\vv{s} \in \sp(A, \vv{u})$  implies that \[ \vv{u} = A \vv{s} + \vv{w}\] for some point $\vv{w}$ that is positive in $\rs(\O)$.     The inequality $\vv{t} \leq \vv{s}$ induces the bound $A \vv{t} \leq A \vv{s} = \vv{u} - \vv{w}$, which shows that $A\vv{t}$ is less than $\vv{u}$ in $\rs(\O)$.  To conclude the proof, it suffices to show that the same is true in the complementary subspace $\rs(\O)^{\perp}$.

Towards this, let $(\collapse{A},\collapse{\vv{u}})$ be the collapse of $(A,\vv{u})$ along $\O$.  Our choice of $\vv{t}$ implies that $\collapse{A}( \vv{s} - \vv{t})$ and $\collapse{A} \vv{s} = \collapse{\vv{u}}$ are both linear combinations with positive coefficients of the same set of columns of $\collapse{A}$.  Therefore, since $\collapse{\vv{u}} = \collapse{A} \vv{s}$ is positive in $\rs(\O)^{\perp}$, then the same must be true for $\collapse{A}(\vv{s} - \vv{t})$.  In other words, $\collapse{A} \vv{t} < \collapse{A} \vv{s} = \collapse{\vv{u}}$, which shows that $A \vv{t}$ is less than  $\vv{u}$ in $\rs(\O)^{\perp}$.
\end{proof}

\begin{theorem}
   \label{canonical-feasible: T}
   If $\vv{s} \in \sp_{\QQ}(A, \vv{u})$ and $q \in \ZZ$ is positive, then
   \[ \vv{s}q - \tail{\vv{s}}_q \in \feas \IP(A, \vv{u}, q).\]
\end{theorem}

\begin{proof}
   \Cref{tail-basics: R} tells us that if $\vv{t} = \vv{s} - (1/q) {\tail{\vv{s}}_q}$, then $\vv{t}q = \vv{s}q -\tail{\vv{s}}_q$ has nonnegative integer coordinates, and $\vv{t}$ satisfies the hypotheses of \Cref{less  than u: L}, which shows that $A (\vv{s} q - \tail{\vv{s}}_q ) =  A\vv{t}q <  \vv{u}q$.
\end{proof}

\begin{remark}[Comparisons with canonical feasible points]
\label{comparison: R}
Adopt the context of \Cref{canonical-feasible: T}, and fix a point $\vv{k}$ that is feasible for $\IP = \IP(A, \vv{u}, q)$.
Our goal is to describe some natural constraints on the difference between $\vv{k}$ and the feasible point  described in \Cref{canonical-feasible: T}.  Toward this, set
%
\[ \vv{h} =  \vv{k} - \vv{s}q + \tail{\vv{s}}_q \]
%
and, let an overbar denote the collapse along the face $\O = \mf(A, \vv{u})$.

Notice that if $s_i = 0$, then $h_i  = k_i \geq 0$, where the last bound follows from the nonnegativity constraint of $\IP$.  The definition of $\vv{h}$ and constraints of $\IP$ also tell us that $A ( \vv{s}q-\tail{\vv{s}}_q + \vv{h}) = A \vv{k} < \vv{u}q = A \vv{s}q + \vv{w}q$, where $\vv{w} \in \rs(\O)$ is as in \Cref{mc: D}.  Collapsing this inequality, keeping in mind that $\collapse{\vv{w}} = \vv{0}$, and rearranging terms, shows that $\collapse{A \vv{h}} < \collapse{A \tail{\vv{s}}_q}$.
This motivates us to study another integer program, that will be formally introduced in the next subsection.
\end{remark}

\subsection{Another integer program}

\begin{definition}
A \emph{monomial list} $(A, \vv{u}, \vv{s}, q)$ consists of the following data.
\begin{enumerate}
\item A monomial pair $(A, \vv{u})$.
\item A rational special point $\vv{s} \in \sp_{\QQ}(A, \vv{u})$.
\item A positive integer $q$.
\end{enumerate}
We call a monomial list whose first term is the matrix $A$ an $A$-list.
\end{definition}

\begin{definition}
\label{aux program: D}
If $(A, \vv{u}, \vv{s}, q)$ is a monomial list, the integer program $\ip(A, \vv{u}, \vv{s}, q)$ in the domain lattice of $A$ consists of maximizing the function $\vv{h} \mapsto \norm{\vv{h}}$, subject to the constraints that the $i$-th coordinate of $\vv{h}$ is nonnegative whenever the $i$-th coordinate of $\vv{s}$ is zero, and $\collapse{A \vv{h}}  < \collapse{A \tail{\vv{s}}_q}$,
\daniel{We should remind the reader that the collapsed inequalities are a subset of the ones determined by $A$} where the overbar denotes collapse along the face $\O = \mf(A, \vv{u})$.
\end{definition}

\begin{definition}
The \emph{optimal image} of $\ip(A, \vv{u}, \vv{s}, q)$ is the set \[ \im \ip(A, \vv{u}, \vv{s}, q)  =  \collapse{A (\opt \ip(A, \vv{u}, \vv{s}, q))}\]
where the overbar denotes collapse along $\O = \mf(A, \vv{u})$.
\end{definition}

\daniel[inline]{

\begin{definition}
The \emph{shortfall} of $\ip(A, \vv{u}, \vv{s}, q)$ is the set \[ \short \ip(A, \vv{u}, \vv{s}, q) = \collapse{A} \tail{\vv{s}}_q - \collapse{A} (\opt \ip(A, \vv{u}, \vv{s}, q))\]
where the overbar denotes collapse along $\O = \mf(A, \vv{u})$.
\end{definition}

\!{\Cref{tail projection: L} and the constraints of $\ip$ imply that this is a positive set of lattice points in $\rs(\O)^{\perp}$.}

}

\Cref{comparison: R} gives us the following result.

\begin{proposition}
\label{comparison: P}
If $(A, \vv{u}, \vv{s}, q)$ is a monomial list, then $\feas \IP(A, \vv{u}, q)$ is contained in $\vv{s}q - \tail{\vv{s}}_q + \feas \ip (A, \vv{u}, \vv{s}, q)$.
\qed
\end{proposition}

\begin{lemma}
\label{tail projection: L}
If $(A, \vv{u}, \vv{s}, q)$ is a monomial list and $\O = \mf(A, \vv{u})$, then $\collapse{A}\tail{\vv{s}}_q$ is a positive lattice point in $\rs(\O)^{\perp}$, where the overbar denotes collapse along $\O$.
\end{lemma}

\begin{proof}  By construction, $\vv{s}q - \tail{\vv{s}}_q $ has nonnegative integer coordinates, and the identity
$\collapse{\vv{u}} q =\collapse{A} \vv{s} q = \collapse{A} ( \vv{s}q - \tail{\vv{s}}_q ) +\collapse{A} \tail{\vv{s}}_q$ then shows that $\collapse{A} \tail{\vv{s}}_q$ must also have integer coordinates.   To see that this vector is positive in $\rs(\O)^{\perp}$, note that $\collapse{\vv{u}} = \collapse{A} \vv{s}$ and $\collapse{A} \tail{\vv{s}}_q$ are both linear combinations with positive coefficients of the same set of columns of $\collapse{A}$.  Given this, it is easy to see that since $\collapse{\vv{u}} = \collapse{A} \vv{s}$ is positive in $\rs(\O)^{\perp}$, the same must be true for $\collapse{A} \tail{\vv{s}}_q$.
\end{proof}


\begin{remark}
   \label{collapsed aux program: R}
   Suppose $(A, \vv{u}, \vv{s}, q)$ is a monomial list and $(\collapse{A}, \collapse{\vv{u}})$ is the collapse of $(A ,\vv{u})$ along $\O = \mf(A, \vv{u})$.
   Then \Cref{collapse of mf and mc: P} implies that $(\collapse{A}, \collapse{\vv{u}}, \vv{s}, q)$ is also a monomial list, and it is then clear from \Cref{aux program: D} that $\ip(A, \vv{u}, \vv{s}, q) = \ip(\collapse{A}, \collapse{\vv{u}}, \vv{s}, q)$.
\end{remark}

The following may be regarded as a partial converse to \Cref{comparison: R}

\begin{proposition}
\label{uniform value: P}
Consider a monomial list $(A, \vv{u}, \vv{s}, q)$, denominator $\ell$ for the special point $\vv{s}$.  If $\vv{h} \in \opt \ip(A, \vv{u}, \vv{s}, q)$, and $q/\ell$ is greater than every coordinate of $\vv{1} - \vv{h}$, and every coordinate of $A \vv{h}$, then the point
$\vv{s}q - \tail{\vv{s}}_q + \vv{h}$ is optimal for $\IP(A, \vv{u}, q)$.
\end{proposition}

\begin{proof} We start by describing what it means $q$ to be large.  Fix a positive denominator  $\denom \in \ZZ$ for $\vv{s} \in \sp_{\QQ}(A, \vv{u})$, and choose $q \gg 0$ so that $q/\denom$ is greater than every coordinate of $\vv{1} - \vv{h}$, and every coordinate of $A \vv{h}$.

Let an overbar denote collapse along $\O = \mf(A, \vv{u})$, and write  \[ \vv{u} = A \vv{s} + \vv{w} \] for some $\vv{w}$ that is positive in $\rs(\O)$, as in \Cref{mc: D}.  As $\vv{u}$ has integer coordinates, it follows that $\denom$ is also a denominator for $\vv{w}$.

\Cref{tail-basics: R} tells us $\vv{k} \coloneqq \vv{s}q - \tail{\vv{s}}_q + \vv{h}$ has integer coordinates, and we claim that $\vv{k} \geq \vv{0}$, that is, $\vv{s}q \geq \tail{\vv{s}}_q - \vv{h}$, whenever $q \gg 0$.  Indeed, if the $i$-th coordinate of $\vv{s}$ is zero, then so is the $i$-th coordinate of $\tail{\vv{s}}_q$, while the feasibility of  $\vv{h}$ for $\ip = \ip(A, \vv{u}, \vv{s}, q)$ implies that the $i$-th coordinate of $\vv{h}$ is nonnegative.  On the other hand, if the $i$-th coordinates of $\vv{s}$ is positive, then it must be at least $q/\denom$, and so the $i$-th coordinate of $\vv{s}q$ is at least $q/\denom$, which is greater than $1 - h_i$ by our choice of $q \gg 0$.  However, \Cref{tail-basics: R} also tells us that the $i$-th coordinate of $\tail{\vv{s}}_q - \vv{h}$ is at most $1-h_i$.  In summary, we have just shown that $\vv{k}$ is a nonnegative lattice point whenever $q \gg 0$.

Thus, $\vv{k}$ is feasible for $\IP = \IP(A, \vv{u}, q)$ if and only if
\[ A\vv{k} = A (\vv{s}q - \tail{\vv{s}}_q + \vv{h})  < \vv{u}q = A {\vv{s}}q + \vv{w}q.\]
which we rewrite as
\begin{equation}
\label{equivalent ineq: e}
A \vv{h} < A \tail{\vv{s}}_q + \vv{w}q.
\end{equation}

After projecting to $\rs(\O)^{\perp}$, the bound \eqref{equivalent ineq: e} becomes $\collapse{A \vv{h}} < \collapse{A \tail{\vv{s}}}_q$, which holds by the feasibility of $\vv{h}$ for $\ip$.  If $\O$ is unbounded, then the projection of the right-hand side of \eqref{equivalent ineq: e} to $\rs(\O)$ is at least $\vv{w}q$.  However, as $\denom$ is a denominator for $\vv{w}$, every coordinate of $\vv{w}q$ is at least $q/\denom$,  and our choice of $q \gg 0$ then guarantees that \eqref{equivalent ineq: e} holds after projecting to $\rs(\O)$.  We conclude that \eqref{equivalent ineq: e} holds throughout $\RR^\numvars = \rs(\O) \oplus \rs(\O)^{\perp}$.

In summary, we have just shown that $\vv{k}$ is feasible for $\IP$, and so
\[ \val \IP \geq \norm{\vv{k}} = \ft{A}{\vv{u}} \cdot q - \norm{\tail{\vv{s}}_q} + \val \ip \]
where above we have used that $\vv{s} \in \sp_{\QQ}(A, \vv{u})$ and $\vv{h} \in \opt \ip$.  To establish the optimality of $\vv{k}$, it remains to show that the $\val \IP$ equals this lower bound.  However, this is a consequence of \Cref{comparison: P}.
\end{proof}




\subsection{Some finiteness properties}
We now explore some finiteness properties, and our results are of two types:  \Cref{bounded value: L} and \Cref{finite image: C} concern the integer program $\ip$ associated with some fixed monomial list,  while \Cref{finitely many secondary programs: L} and \Cref{finitely many coord sums: C} concern the nature of these programs as the list varies.

\begin{lemma}
\label{bounded value: L}
If $(A, \vv{u}, \vv{s}, q)$ is a monomial list, then $0 \leq  \val  \ip(A, \vv{u}, \vv{s}, q) < \norm{\tail{\vv{s}}_q}$.
\end{lemma}

\daniel[inline]{I propose the following definition. \!{Will require \Cref{bounded value: L} to be moved up.}

\begin{definition}
\label{deficit ip: D}
The \emph{deficit} of $\ip(A, \vv{u}, \vv{s}, q)$ is the positive rational number
\[ \deficit \ip(A, \vv{u}, \vv{s}, q) = \norm{\tail{\vv{s}}_q} - \val \ip(A, \vv{u}, \vv{s}, q). \] 

\end{definition}
}

\begin{proof}
   Fix a point $\defpt \in \RR^\numvars$ that defines $\O  = \mf(A, \vv{u})$, and let an overbar denote collapse along $\O$.
   Thus,
\[ \iprod{{\defpt}}{A \vv{t}} = \iprod{\collapse{\defpt}}{\collapse{A \vv{t}}} = \iprod{\collapse{\defpt}}{\collapse{A} \vv{t}} \] for every $\vv{t}$.  With this notation in hand, we begin the proof below.

The product $\defpt^{\mathrm{T}} A $ is a row vector whose $i$-th coordinate is the inner product of $\defpt$ with the $i$-th column of $A$ (and so is at least one).   In fact, if the $i$-th coordinate of a point $\vv{k}$ feasible for $\ip = \ip(A, \vv{u}, \vv{s}, q)$ were negative, then the $i$-th coordinate of $\vv{s}$ must be positive;  thus, the $i$-th column of $A$ must lie in $\O$, and so the $i$-th coordinate of $\defpt^{\mathrm{T}} A$ must equal one.  In particular,
%
\begin{equation}
\label{bound in inner product: e}
\norm{\vv{k}} \leq (\defpt^{\mathrm{T}} A) \vv{k} =  \defpt^{\mathrm{T}} (A \vv{k}) = \iprod{\defpt}{A \vv{k}} = \iprod{\collapse{\defpt}}{\collapse{A} \vv{k}}
\end{equation}
whenever $\vv{k}$ is feasible for $\ip$, and a similar argument will show that
\begin{equation}
\label{norm of tail: e}
\norm{\tail{\vv{s}}_q} =  \iprod{\defpt}{A \tail{\vv{s}}_q} = \iprod{\collapse{\defpt}}{\collapse{A} \tail{\vv{s}}_q}.
\end{equation}

Consequently, if $\vv{k}$ is feasible for $\ip$, then the constraint $\collapse{A}\vv{k} <\collapse{A} \tail{\vv{s}_q}$ and the above observations combine to tell us that \[ \norm{\vv{k}} \leq \iprod{\collapse{\defpt}}{\collapse{A} \vv{k}} < \iprod{\collapse{\defpt}}{\collapse{A} \tail{\vv{s}_q}} = \norm{\tail{\vv{s}_q}}\]
which demonstrates that $\val \ip < \norm{\tail{\vv{s}}_q}$.  Finally, the positivity of $\collapse{A}\tail{\vv{s}}_q$ described in \Cref{tail projection: L} implies that $\vv{0}$ is feasible for $\ip$.
\end{proof}

\emily[inline]{Let's try to construct an example in which the optimal set of $\ip$ is infinite.}

\begin{corollary}
\label{finite image: C}
If $(A, \vv{u}, \vv{s}, q)$ is a monomial list, then $\im \ip(A, \vv{u}, \vv{s}, q)$ is finite.
\end{corollary}

\begin{proof}  Adopt the notation from the proof of \Cref{bounded value: L}.
If $\vv{k}$ is optimal for $\ip$, then \eqref{bound in inner product: e} implies that \[ \val \ip = \norm{\vv{k}} \leq \iprod{\collapse{\defpt}}{\collapse{A} \vv{k}}\] and so $\collapse{A} \vv{k}$ is a lattice point in the polyhedron of all points $\vv{v}$  in $\rs(\O)^{\perp}$ with $\vv{v} < \collapse{A} \tail{\vv{s}_q}$  and $\iprod{\collapse{\defpt}}{\vv{v}} \geq \val \ip$.  The positivity of $\collapse{\defpt}$ in $\rs(\O)^{\perp}$ and \Cref{bounded polytope: P} then tell us  that this polyhedron is bounded.
\end{proof}

\emily[inline]{Maybe we should make this a Theorem, and explain that this is a very important finiteness property.
Potentially move it up before \Cref{bounded value: L}.}

\begin{lemma}
\label{finitely many secondary programs: L}
If $A$ is fixed, then there are only finitely many integer programs of the form $\ip(A, \vv{u}, \vv{s}, q)$ as we vary over all $A$-lists $(A, \vv{u}, \vv{s}, q)$.
\end{lemma}

\begin{proof}  Consider a monomial list $(A, \vv{u}, \vv{s}, q)$.  As $A$ is fixed, there are only finitely many possibilities for $\O = \mf(A, \vv{u})$, and only finitely many possibilities for the set of supporting indices of any point $\vv{s} \in \sp_{\QQ}(A ,\vv{u})$.

Next, let $\collapse{A}$ be the collapse of $A$ along the face $\O$.  If $\vv{s} \in \sp_{\QQ}(A, \vv{u})$, then $\vv{0} \leq \tail{\vv{s}}_q \leq \vv{1}$ for every integer $q > 0$, where $\vv{1}$ is the vector in the domain lattice of $A$ consisting of all ones.  Consequently, $\vv{0} \leq \collapse{A} \tail{\vv{s}}_q \leq \collapse{A}\, \vv{1}$, and as \Cref{tail projection: L} tells us that $\collapse{A} \tail{\vv{s}}_q$ has integer coordinates, it follows that there are only finitely many possibilities for this point.
\end{proof}

\begin{corollary}
\label{finitely many coord sums: C}
 If $A$ is fixed, then there are only finitely many rational numbers of the form $ \norm{\tail{\vv{s}}_q}$ as we vary over all $A$-lists $(A, \vv{u}, \vv{s}, q)$.
\end{corollary}

\begin{proof}  This follows from \eqref{norm of tail: e} and the proof of \Cref{finitely many secondary programs: L}.
\end{proof}


These finiteness properties above facilitate the following result.

\newcommand{\fsr}{\mathcal{R}}

\begin{theorem}[Existence of finite sets of representatives]
\label{fsr-exist: T}
Given a monomial matrix $A$, there exists a finite subset $\fsr = \fsr(A)$ of the domain lattice of $A$ with the following property\textup:  For every monomial list $(A, \vv{u}, \vv{s}, q)$, and for every $\vv{v} \in \im \ip(A, \vv{u}, \vv{s}, q)$, there exists $\vv{h} \in \fsr \cap \opt \ip(A, \vv{s}, \vv{u}, q)$ with $\collapse{ A \vv{h}} = \collapse{A} \vv{h} =  \vv{v}$, where the overbar denotes collapse along $\O = \mf(A, \vv{u})$.
\end{theorem}

\begin{proof}  \Cref{finite image: C} implies that for every monomial list $(A, \vv{u}, \vv{s}, q)$,  there exists a \emph{finite} subset $\fsr(A, \vv{u}, \vv{s}, q)$ of $\opt (A, \vv{u}, \vv{s}, q)$ such that
\[  \im \ip (A, \vv{u}, \vv{s}, q) = \ol{ A(\fsr(A, \vv{u}, \vv{s}, q)) } =  \collapse{A}(\fsr(A, \vv{u}, \vv{s}, q)) \]
and \Cref{finitely many secondary programs: L} then implies that these sets may be chosen in such a way so that $\fsr(A) = \cup \, \fsr(A, \vv{u}, \vv{s}, q)$ is finite, where the union is over all $A$-lists.
\end{proof}


\newpage
\section{Toward solving $\IP$}
\label{solving: S}

Suppose that $(A, \vv{u})$ is a monomial pair and that $q$ is positive integer.
The goal in this subsection is to demonstrate that the value and optimal image of $\IP(A, \vv{u}, q)$ vary with $q$ in a uniform way as $q \to \infty$.

\subsection{Relating the two integer programs}
\label{relating-programs: ss}

\begin{corollary}
\label{uniform value and image: C}
Given a monomial matrix $A$, there exists $\beta = \beta(A) \in \ZZ$ satisfying the following condition\textup:
If $(A, \vv{u})$ is a monomial pair with $\O = \mf(A, \vv{u})$, $\vv{s} \in \sp_{\QQ}(A, \vv{u})$ is a point with denominator $D$, and $q>\beta D$, then
%
\[ \val \IP(A, \vv{u}, q) = \ft{A}{\vv{u}} \cdot q - \norm{\tail{\vv{s}}_q} + \val \ip(A, \vv{u}, \vv{s}, q) \]
\daniel[inline]{This will become 
\[  \val \IP(A, \vv{u}, q) = \ft{A}{\vv{u}} \cdot q -  \deficit \ip(A, \vv{u}, \vv{s}, q) \] 
\!{Can we easily prove that $\val \IP(A, \vv{u}, q) < \ft{A}{\vv{u}} q$?  If so, would it make sense to describe the deficit of $\IP$  as the difference?  If we do this, then the above instead becomes $\deficit \IP = \deficit \ip$.  }

}
%
and
\[ \ol{\im \IP(A, \vv{u}, q)} = \collapse{\vv{u}} q - \collapse{A} \tail{\vv{s}}_q + \im \ip(A, \vv{u}, \vv{s}, q), \]
\daniel[inline]{This is more interesting, and will become 
\[  \collapse{ \short \IP(A, \vv{u}, q)} = \short \ip (A, \vv{u}, \vv{s}, q) \] 
}


where the overbar denotes collapse along $\O$.
\end{corollary}

\begin{proof}
Let $\beta$ be as in \Cref{uniform value: P}.  Fix a monomial pair $(A, \vv{u})$ with $\O = \mf(A, \vv{u})$, a point $\vv{s} \in \sp_{\QQ}(A, \vv{u})$ with denominator $D$, and an integer $q > \beta D$.  Let $\collapse{A}$ and $\collapse{X}$ be as above.

The asserted value of $\IP(A, \vv{u}, q)$ follows from \Cref{uniform value: P}.  Next, fix a point $\vv{k} \in \opt \IP(A, \vv{u}, q)$, and let $\vv{h}$ be the unique lattice point such that $\vv{k} = \vv{s}q - \tail{\vv{s}}_q + \vv{h}$.  \Cref{comparison: R} implies that $\vv{h}$ is feasible for $\ip = \ip(A, \vv{u}, \vv{s}, q)$, while the optimality of $\vv{k}$ tells us that $\norm{\vv{k}} = \val \IP(A, \vv{u}, q)$.  Keeping in mind our formula for $\val \IP(A, \vv{u}, q)$, this equality tells us $\norm{\vv{h}} = \val \ip$.    Therefore, $\vv{h}$ must be optimal for $\ip$,  and so $\collapse{A} \vv{h} \in \im \ip$.  Furthermore, as $\collapse{A} \vv{s} = \collapse{\vv{u}}$,
%
\[ \collapse{A} \vv{k} = \collapse{\vv{u}} q - \collapse{A} \tail{\vv{s}}_q + \collapse{A} \vv{h}\]
which shows that $\collapse{A} ( \opt \IP(A, \vv{u}, q))$ = $\ol{\im \IP(A, \vv{u}, q)}$ is contained in
\[ \collapse{\vv{u}} q - \collapse{A} \tail{\vv{s}}_q + \im \ip.\]

We now establish the opposite containment:  \Cref{uniform value: P} tells us that \[  \vv{s}q - \tail{\vv{s}}_q + \orep(A, \vv{u}, \vv{s}, q)\] is optimal for $\IP(A, \vv{u}, q)$,  while \[ \collapse{A}( \orep(A, \vv{u}, \vv{s}, q)) = \im \ip \] by \Cref{orep: D}.   It follows that $\collapse{A}(\opt \IP(A, \vv{u}, q)) = \ol{\im \IP(A, \vv{u}, q)}$ contains the set $\collapse{\vv{u}} q - \collapse{A} \tail{\vv{s}}_q + \im \ip$.
\end{proof}

\subsection{Some useful invariants}
\label{useful-invariants: ss}

In this subsection, we study the quantities appearing in \Cref{uniform value and image: C} above.  We begin with a fundamental observation.

\emily[inline]{Can we give a direct proof that $\delta$ does not depend on $\vv{s}$?}

\begin{corollary}
\label{independence: C} Fix a monomial pair $(A, \vv{u})$ and an integer $q>0$.  If $\collapse{A}$ is the collapse of $A$ along $\O = \mf(A, \vv{u})$, then the quantities
\[   \delta(A, \vv{u}, \vv{s}, q)  = \norm{\tail{\vv{s}}_q}  - \val \ip(A, \vv{u}, \vv{s}, q)\] and
\[ \Delta(A, \vv{u}, \vv{s}, q)  = \collapse{A} \tail{\vv{s}}_q - \im  \ip( A, \vv{u}, \vv{s}, q)  \]
\daniel[inline]{the deficit and shortfall of $\ip(A, \vv{u}, \vv{s}, q)$}
do not depend on  $\vv{s} \in \sp_{\QQ}(A, \vv{u})$.
\end{corollary}

\begin{proof}
Fix $\vv{s}$ and $\vv{s}'$ in $\sp_{\QQ}(A, \vv{u})$, as well as a common denominator $\denom$ for these points.  As these quantities clearly depend only on $q \bmod \denom$, it suffices to show that $\delta(A, \vv{u}, \vv{s}, q) = \delta(A, \vv{u}, \vv{s}', q)$  and $ \Delta(A, \vv{u}, \vv{s}, q) = \Delta(A, \vv{u}, \vv{s}', q)$ whenever $q \gg 0$.  However, this is follows from \Cref{uniform value and image: C}.
\end{proof}

\begin{definition}
\label{independence: D}

Given a monomial pair $(A, \vv{u})$ and positive integer $q$, we set
 \[ \delta(A, \vv{u}, q) = \norm{\tail{\vv{s}}_q}  - \val \ip(A, \vv{u}, \vv{s}, q)\] and
\[\Delta(A, \vv{u}, q) = \collapse{A} \tail{\vv{s}}_q - \im  \ip( A, \vv{u}, \vv{s}, q)  \]
where  $\vv{s} \in \sp_{\QQ}(A, \vv{u})$, and $\collapse{A}$ is the collapse of $A$ along $\O = \mf(A, \vv{u})$.
\daniel[inline]{Stick with $\delta$ and $\Delta$? Or maybe $\mathcal{S} (A, \vv{u}, q)$ and $\mathcal{D}(A, \vv{u}, q)$}
\end{definition}

%\begin{remark} Above, we referred to \Cref{uniform value and image: C} to deduce the independence of $\delta(A, \vv{u}, q)$ and $\Delta(A, \vv{u}, q)$ on the rational point $\vv{s} \in \sp_{\QQ}(A, \vv{u})$.  Though it seems likely that this can be established with a more direct argument, we have yet to identify one.
%\end{remark}

\begin{lemma}
\label{independence: L}
If $\collapse{A}$ is the collapse of $A$ along $\O = \mf(A, \vv{u})$ and $q>0$ is an integer, then the following hold.

\begin{enumerate}
\item $\delta(A, \vv{u}, q)$  is a positive rational number.
\item $\Delta(A, \vv{u}, q)$ is a finite set of positive lattice points in $\rs(\O)^{\perp}$.
\item No column of $\collapse{A}$ is less than any point in $\Delta(A, \vv{u}, q)$.
\end{enumerate}
\end{lemma}

\begin{proof}
Fix a point $\vv{s} \in \sp_{\QQ}(A, \vv{u})$ with which to compute $\delta = \delta(A, \vv{u}, q)$ and $\Delta = \Delta(A, \vv{u}, q)$.  \Cref{bounded value: L} implies that $\delta$ is positive, and \Cref{finite image: C} that $\Delta$ is a finite subset of $\ZZ \rb(\O)^{\perp}$.   The positivity of $\Delta$ in this lattice is a consequence of the constraints of $\ip = \ip(A, \vv{u}, \vv{s}, q)$.  These constraints also imply that no column of $\collapse{A}$ is less than any point in $\Delta$.  Indeed, if $\vv{k}$ is optimal for $\ip$, then optimality implies that  $\collapse{A}( \vv{k} + \canvec_i) \not < \collapse{A} \tail{\vv{s}}_q$ for each standard basis vector $\canvec_i$ in the domain of $\collapse{A}$, which we rewrite as  \[ \collapse{A} \canvec_i \not < \collapse{A} \tail{\vv{s}}_q - \collapse{A} \vv{k}.\]
We conclude that no column of $\collapse{A}$ is less than $\collapse{A}\tail{\vv{s}}_q - \collapse{A}\vv{k}$.
\end{proof}


We conclude with some finiteness properties.


\emily[inline]{Maybe we should restate \Cref{finitely many deltas for a fixed A: P} more precisely. }


\begin{proposition}
\label{finitely many deltas for a fixed A: P}
 Given a monomial matrix $A$, there are only finitely many objects of the form $\delta(A, \vv{u}, q)$ and $\Delta(A, \vv{u}, q)$.
\end{proposition}

\begin{proof}
This follows immediately from \Cref{finitely many secondary programs: L} and \Cref{finitely many coord sums: C}.
\end{proof}

\begin{remark}
\label{comparing deltas: R}
If $(\collapse{A}, \collapse{\vv{u}})$ is the collapse of $(A, \vv{u})$ along $\O = \mf(A, \vv{u})$, then
\[ \delta(A, \vv{u}, q) = \delta(\collapse{A}, \collapse{\vv{u}}, q)  \text{ and }  \Delta(A, \vv{u},q) = \Delta(\collapse{A}, \collapse{\vv{u}}, q)\] for all integers $q>0$ (e.g., this follows from \Cref{collapsed aux program: R}).   Consequently, one may replace the point in $\sp_{\QQ}(A, \vv{u})$ in \Cref{independence: D}   with one in $\sp_{\QQ}(\collapse{A}, \collapse{\vv{u}})$ without affecting the value of $\delta(A, \vv{u}, q)$ and $\Delta(A, \vv{u}, q)$.
\end{remark}

\begin{remark}
\label{pair periodicity: R}
If $(A, \vv{u})$ is fixed, then $\delta(A, \vv{u}, q)$ and $\Delta(A, \vv{u}, q)$ are periodic in $q$.  Indeed, if $\denom$ is the denominator of some point in $\sp_{\QQ}(A, \vv{u})$, then
\begin{equation}
\label{periodicity: e}
 \delta(A, \vv{u}, p) = \delta(A, \vv{u}, q)  \text{ and } \Delta(A, \vv{u}, p) = \Delta(A, \vv{u}, q)
\end{equation} whenever $p \equiv q \bmod \denom$.    In fact, \Cref{comparing deltas: R} tells us that the same is true if instead $\denom$ is the denominator of a point in $\sp_{\QQ}(\collapse{A}, \collapse{\vv{u}})$.
\end{remark}

\begin{remark}
\label{uniform periodicity: R}
 If only $A$ is specified, then there exists a uniform integer $\denom$ such that \eqref{periodicity: e} holds for every monomial pair $(A, \vv{u})$ whenever $p \equiv q \bmod \denom$.

 Indeed,  this follows from the observation that if $\denom$ is as in \Cref{uniform denominators for mc:  T}, then we may compute  $\delta(A, \vv{u}, q)$ and $\Delta(A, \vv{u}, q)$ for all monomial pairs $(A, \vv{u})$ and integers $q>0$ in terms of a point in $\sp_{\QQ}(A, \vv{u})$ with denominator $\denom$.
\end{remark}

We record another application of \Cref{uniform denominators for mc:  T} below.

\begin{theorem}
\label{uniform uniform value and image: T}
Given a monomial matrix $A$, there exists an integer $\beta = \beta(A)$ with the following property\textup:
If $q > \beta$ and $(A, \vv{u})$ is a monomial pair, then
\[ \val \IP(A, \vv{u}, q) = \ft{A}{\vv{u}} \cdot q - \delta(A, \vv{u}, q) \] 
\daniel[inline]{
\[  \deficit \IP(A, \vv{u}, q) = \delta (A, \vv{u}, \vv{s}, q) \]}

and
\[ \ol{ \im \IP(A, \vv{u}, q)} = \collapse{\vv{u}}q - \Delta(A, \vv{u},q) \] where the overbar denotes collapse along $\O = \mf(A, \vv{u})$.

\daniel[inline]{\[  \collapse{ \short \IP(A, \vv{u}, q)} = \Delta (A, \vv{u}, \vv{s}, q) \] 
}

\end{theorem}

\begin{proof}
\Cref{uniform denominators for mc:  T}  tells us that once $A$ has been fixed, there exists a positive integer $D$ such that for every monomial pair $(A, \vv{u})$, there exists a point in $\sp_{\QQ}(A, \vv{u})$ with denominator $D$.  Therefore, if $\beta_{\circ}$  is any integer satisfying the condition stated in \Cref{uniform value and image: C}, then we may take $\beta = D \beta_{\circ}$.
\end{proof}

\begin{remark}[An algebraic interpretation]
\label{leftover p large: R}

Let $(A, \vv{u})$ be a monomial pair associated to an ideal pair $(\ideala, \ideald)$. If $\beta = \beta(A)$ is as in \Cref{uniform uniform value and image: T}, then this same theorem tells us that if $q > \beta$, then\daniel{Update this, since we already motivated shortfalls.  If we like this description, I guess we'll have to define $F$-thresholds for ideals, which I am not sure we have done} \[ \nu(\ideala, \ideald, q) = \ft{\ideala}{\ideald} \cdot q - \delta(A, \vv{u},q). \] 

Next, we consider the algebraic implications of the description of the collapse of the optimal image of $\IP(A, \vv{u}, q)$ in \Cref{uniform uniform value and image: T}.  For simplicity, we first consider the case that $\O = \mf(A, \vv{u})$ is bounded.

In this case,  the collapse along $\O$ is the identity map, and so ${ \im \IP(A, \vv{u}, q)}$ equals ${\vv{u}}q - \Delta(A, \vv{u},q)$.  In light of this, it follows from \Cref{shortfall motivation: R} that 
\[ \ideala^{\nu(\ideala, \ideald, q)} \equiv \ideal{x^{\vv{u}q-\vv{w}}: \vv{w} \in \im \Delta(A,\vv{u},q)} \bmod \ideald^{[q]}.\]
In other words, the ``leftovers" of $\ideala^{\nu(\ideala, \ideald, q)}$ modulo $\ideald^{[q]}$, the diagonal ideal determined by $\vv{u}q$, differ from $\vv{u}q$ in a uniform way, at least when $q > \beta$.

When $\O$ is unbounded, the situation is slightly more complicated. \daniel{Return to this later;  you'll need to settle on how to describe collapses first}

\end{remark}



The following is a consequence of \Cref{comparing deltas: R} and \Cref{uniform uniform value and image: T}.

\begin{corollary}
Given a monomial matrix $A$, there exists an integer $\beta$ with the following property\textup:  If $q > \beta$ and $(A, \vv{u})$ is a monomial pair with $\O = \mf(A, \vv{u})$, then $\val \IP(A, \vv{u}, q) = \val \IP(\collapse{A}, \collapse{\vv{u}}, q)$ and $\ol{ \im \IP(A, \vv{u}, q)} = \im \IP(\collapse{A}, \collapse{\vv{u}}, q)$ where the overbar denotes collapse along $\O$.
\end{corollary}

\emily[inline]{The following could just replace the above statement?  Replace integer programming language with algebraic language? For instance, as follows?}

\begin{corollary}
\label{collapsed ft agree: C}
Given a monomial ideal $\ideala$, there exists an integer $\beta = \beta(\ideala)$ with the following property:  If $q> \beta$, and $(\ideala, \ideald)$ is an ideal pair, then $\nu(\ideala, \ideald, q) = \nu(\collapse{\ideala}, \collapse{\ideald}, q)$. \daniel{This will also require us to carefully define the collapse of an ideal pair}
\end{corollary}

% \newpage
% \section{Connections with arithmetic and fractal programs}
%
% \subsection{$\mu$ invariants as values of arithmetic integer programs}
%
% Let $\ideala$ be a monomial ideal in a polynomial ring in the variables $x=x_1,\ldots,x_\numvars$ over a field of positive characteristic $p$.
% Let $\vv{u}$ be a positive point in $\NN^\numvars$, and $\ideald = \diag(\vv{u}) = \ideal{x_1^{u_1},\ldots,x_\numvars^{u_\numvars}}$.
% Recall that for each $q$ a power of $p$ we define the number
% \[\mu(\ideala,\ideald,q) \coloneqq \max\big\{k\in \NN : \ideala^{[k]} \not\subseteq \ideald^{[q]}\big\},\]
% used in the computation of the critical exponent of $\ideala$ with respect to $\ideald$.
% If $A \in \NN^{d\times n}$ is an exponent matrix of $\ideala$, then $\ideala^{[k]}$ is generated by monomials $x^{A\vv{k}}$, with $\vv{k}\in \NN^n$ and $\binom{k}{\vv{k}} \not\equiv 0 \bmod{p}$.
% Since the condition $\ideala^{[k]} \not\subseteq \ideald^{[q]}$ is equivalent to the existence of $\vv{k}$ as above satisfying $A\vv{k} < \vv{u}q$, finding the maximum defining $\mu(\ideala,\ideald,q)$ is equivalent to maximizing $\norm{\vv{k}}$, with $\vv{k} \in \NN^n$ subject to the linear constraint $A\vv{k} < \vv{u}q$ and the arithmetic constraint $\binom{\norm{\vv{k}}}{\vv{k}} \not\equiv 0 \bmod{p}$.
%
% In what follows, $(A, \vv{u})$ is a $d \times n$ monomial pair and $p$ is a prime integer.
% Moreover, the letter $q$ from here on will always denote a power of $p$.
% Motivated by the above discussion, we introduce a variant of an integer program in which we impose an additional and highly nonlinear constraint.
% %As this new constraint is arithmetic in nature, we call such an optimization problem an \emph{arithmetic integer program}.
%
% \begin{definition}
% \label{aip: D}
% The \emph{arithmetic integer program} $\IP_p(A, \vv{u}, q)$ in the domain lattice of $A$ consists of maximizing $\vv{k} \mapsto \norm{\vv{k}}$ subject to the linear constraints $\vv{k} \geq \vv{0}$ and $A \vv{k} < \vv{u}q$, and the arithmetic constraint $\binom{\norm{\vv{k}}}{\vv{k}} \not \equiv 0 \bmod p$.
% \end{definition}
%
%
% \begin{remark} \label{dickson: R}
% Let $p$ be a prime integer, $k\in \NN$, and $\vv{u} \in \NN^n$.
% Write the terminating base $p$ expansions of $k$ and $\vv{u}$ as follows\textup:
% \begin{equation*}
% k = k_0+k_1p+k_2p^2+\cdots+k_rp^r\quad \text{and} \quad \vv{u}=\vv{u}_0+\vv{u}_1p+\vv{u}_2p^2+\cdots+\vv{u}_rp^r,
% \end{equation*}
% where $0\le k_i < p$ and $\vv{0}\le\vv{u}_i < p \cdot \vv{1}$ for each $i$
% (so that it is possible that $k_r = 0$ or $\vv{u}_r = \vv{0}$).
% Then by \cite{dickson.multinomial},
% \[
%     \binom{k}{\vv{u}}\equiv \binom{k_0}{\vv{u}_0}\binom{k_1}{\vv{u}_1}\cdots \binom{k_r}{\vv{u}_r} \mod{p}.
% \]
% In particular, $\binom{k}{\vv{u}}\not\equiv 0\bmod{p}$ if and only if $\norm{\vv{u}_i}=k_i$ for each $i$, that is, the components of $\vv{u}$ add up to $k$ without carrying \textup(base $p$\textup).
%
% Hence the condition that $\binom{\norm{\vv{k}}}{\vv{k}} \not \equiv 0 \bmod p$ in \Cref{aip: D} is equivalent to the condition that
%  if $\vv{k} = \vv{k}_0 + \cdots + \vv{k}_l \cdot  p^l$ is the
% %  \daniel{I removed \emph{unique terminating} since $\vv{k}$ is a lattice point.  Both $r$ and $l$ appeared as the terminating index, so I just made both $l$, since I prefer it.  We can change it back to $r$, though.}
%  base $p$ expansion of $\vv{k}$, then $\norm{\vv{k}_e} < p$ for all $0 \leq e \leq l$.
% \end{remark}
%
% % \begin{theorem}[\cite{dickson.multinomial}]
% %    \label{thm: dickson}
% %    Let $p$ be a prime integer, $k\in \NN$, and $\vv{u} \in \NN^n$.
% %    Write the terminating base $p$ expansions of $k$ and $\vv{u}$ as follows\textup:
% %    \begin{equation*}
% %       k = k_0+k_1p+k_2p^2+\cdots+k_rp^r\quad \text{and} \quad \vv{u}=\vv{u}_0+\vv{u}_1p+\vv{u}_2p^2+\cdots+\vv{u}_rp^r,
% %    \end{equation*}
% %    where $0\le k_i < p$ and $\vv{0}\le\vv{u}_i < p \cdot \vv{1}$ for each $i$.
% %    \textup{(}Note that it is possible that $k_r = 0$ or $\vv{u}_r = \vv{0}$.\textup{)}
% %    Then
% %    \[
% %       \binom{k}{\vv{u}}\equiv \binom{k_0}{\vv{u}_0}\binom{k_1}{\vv{u}_1}\cdots \binom{k_r}{\vv{u}_r} \mod{p}.
% %    \]
% %    In particular, $\binom{k}{\vv{u}}\not\equiv 0\bmod{p}$ if and only if $\norm{\vv{u}_i}=k_i$ for each $i$, that is, the components of $\vv{u}$ add up to $k$ without carrying \textup(base $p$\textup).
% % \qed
% % \end{theorem}
%
% % \begin{corollary}
% %    \label{cor: multinomial congruence}
% %    Let $k,l,e\in \NN$, with $l<p^e$, and $\vv{u},\vv{v}\in \NN^n$, with $\vv{v}<p^e\cdot \vv{1}$.
% %    Then
% %    \[
% %       \pushQED{\qed}
% %       \binom{kp^e+l}{\vv{u}p^e+\vv{v}}\equiv \binom{k}{\vv{u}}\binom{l}{\vv{v}} \mod{p}.\qedhere
% %       \popQED
% %    \]
% % \end{corollary}
% %
% %
% % By \Cref{thm: dickson}, the latter  is equivalent to the condition that  if \[ \vv{k} = \vv{k}_0 + \cdots + \vv{k}_l \cdot  p^l\] is the  \daniel{I removed \emph{unique terminating} since $\vv{k}$ is a lattice point.  Both $r$ and $l$ appeared as the terminating index, so I just made both $l$, since I prefer it.  We can change it back to $r$, though.} base $p$ expansion of $\vv{k}$, then $\norm{\vv{k}_e} < p$ for all $0 \leq e \leq l$.
%
%
% We define the terms \emph{feasible, optimal}, and \emph{value} relative to an arithmetic program in analogy with those for linear programs.
% %the analogous way.
% It is easy to see that the feasible set of $\IP_p(A,\vv{u},q)$ is finite, and consequently this program has a well-defined value.
% In the algebraic context given above, $\val \IP_p(A,\vv{u},q)$ equals $\mu(\ideala,\ideald,q)$, which we shall often denote $\mu(A,\vv{u},q)$.
%
% % By \Cref{thm: dickson}, the latter  is equivalent to the condition that  if \[ \vv{k} = \vv{k}_0 + \cdots + \vv{k}_l \cdot  p^l\] is the  \daniel{I removed \emph{unique terminating} since $\vv{k}$ is a lattice point.  Both $r$ and $l$ appeared as the terminating index, so I just made both $l$, since I prefer it.  We can change it back to $r$, though.} base $p$ expansion of $\vv{k}$, then $\norm{\vv{k}_e} < p$ for all $0 \leq e \leq l$.
% %
% % We define the terms \emph{feasible, optimal}, and \emph{value} relative to an arithmetic program in the analogous way.
% % It is easy to see that the feasible set of $\IP_p(A,\vv{u},q)$ is finite, and consequently this program has a well-defined value.
% % In the algebraic context given above, $\val \IP_p(A,\vv{u},q)$ equals $\mu(\ideala,\ideald,q)$, which we shall often denote $\mu(A,\vv{u},q)$.
%
% % \daniel[inline]{It is possible that we don't use the image of this program anywhere.  Maybe only the image of $\ip$.}
% % \pedro[inline]{
% %    This image appears in multiple places: \Cref{follow-leftovers: P}, \Cref{arithmetic uniform value and image: T}, \Cref{cor: upper bound for higher mus}.
% % }
% \begin{definition}
% The \emph{optimal image} of $\IP_p(A, \vv{u}, q)$ is the set $\im \IP_p(A, \vv{u}, q)$ of all points  $A \vv{k}$ with $\vv{k} \in \opt \IP_p(A, \vv{u}, q)$.
% \end{definition}
%
% If $\ideala$ and $\ideald$ are as above, and $\mu = \mu(\ideala,\ideald,q) = \val \IP_p(A,\vv{u},q)$, then the optimal image of $\IP_p(A,\vv{u},q)$ characterizes the ``leftovers'' of $\ideala^{[\mu]}$ modulo $\ideald^{[q]}$:
% \begin{align*}
%   \ideala^{[\mu]} &= \ideal{x^{A\vv{k}}: \textstyle\binom{\mu}{\vv{k}} \not\equiv 0 \bmod{p}}\\
%   &\equiv \ideal{x^{A\vv{k}}: \textstyle\binom{\mu}{\vv{k}} \not\equiv 0 \bmod{p}\text{ and } A\vv{k} <\vv{u}q} \bmod \ideald^{[q]}\\
%   &\equiv \ideal{x^{A\vv{k}}: \vv{k}\in \opt\IP_p(A,\vv{u},q)} \bmod \ideald^{[q]}\\
%   &\equiv \ideal{x^{\vv{v}}: \vv{v} \in \im \IP_p(A,\vv{u},q)} \bmod \ideald^{[q]}.
% \end{align*}
% The critical exponent of $\ideala$ with respect to $\ideald$ is the limit
% \begin{equation*}
% \crit(\ideala,\ideald) = \lim_{e\to\infty} \frac{\mu(\ideala,\ideald,p^e)}{p^e} = \lim_{e\to\infty} \frac{\val \IP_p(A,\vv{u},p^e)}{p^e}.
% \end{equation*}
% In the next subsection we show that this critical exponent, henceforth denoted $\crit(A,\vv{u})$, can be realized as the value of a \emph{fractal linear program} in the domain of $A$.
%
% \subsection{Critical exponents as values of fractal linear programs}
%
% \begin{definition}
%    The \emph{Sierpi\'nski $p$-gasket} in $\RRnn^n$ is the set $\sierp_{p,n}$ consisting of all points $\vv{t}\in \RRnn^n$ for which there exist a sequence of points $\{ \vv{t}_e \}_{e=1}^\infty$ in $\NN^n$ for which all $\norm{\vv{t}_e} < p$ and $q$, a power of $p$, such that
%  \[
% \vv{t} = q\cdot\Big(\frac{\vv{t}_1}{p} +\frac{\vv{t}_2}{p^2}+\cdots +\frac{\vv{t}_e}{p^e} + \cdots \Big).
%  \]
% \end{definition}
%
% From the definition, we immediately see that a point in $\RRnn^n$ is in $\sierp_{p,n}$ if the unique nonterminating base $p$ expansions of its coordinates \emph{add without carrying}.
% However, this is not a complete description of $\sierp_{p,n}$.
% For instance, when $p=2$, the sum of $\frac{1}{4} = \frac{1}{2^3} + \frac{1}{2^4} + \frac{1}{2^5} + \cdots$ with itself carries at infinitely many places, yet we see that $\left(\frac{1}{4}, \frac{1}{4}\right)$ is in the Sierpi\'nski $2$-gasket in $\RRnn^2$ after writing one summand simply as $\frac{1}{4} = \frac{1}{2^2}$, \ie in its \emph{terminating} expansion.
%
% This description of the Sierpi\'nski $p$-gasket in terms of expansions is not hard to translate geometrically into its geometric description as a fractal.
% For instance, the points of $\sierp_{2,2}$ in $[0,1]^2$ form the familiar Sierpi\'nski triangle:
% The points $\vv{t}$ in the unit square that have \emph{no} expansion $\vv{t} = \frac{\vv{t}_1}{2} +\frac{\vv{t}_2}{2^2}+\cdots +\frac{\vv{t}_e}{2^e} + \cdots$ for which all $\vv{t}_e \in \NN^2$, and $\norm{\vv{t}_1} < 2$ are precisely the points $(x,y) \in [0,1]^2$ lying above the line $x+y=1$, so we remove the triangle given by $x+y>1$.
% At the next stage, the points with no expansion satisfying $\norm{\vv{t}_1} < 2$ and  $\norm{\vv{t}_2} < 2$ are those in the triangle given by $x, y < \frac{1}{2}$ and $x+y > \frac{1}{2}$.
% The condition on expansions at the third place removes three triangles from the remaining set, and we can continue on analogously.
%
% \Cref{fig: sierpinski 3-gasket} illustrates the self-similarity of the Sierpi\'nski $3$-gasket in $\RRnn^2$.
% \begin{figure}
% \begin{subfigure}{.49\textwidth}
%   \centering
%   \includegraphics[width=.9\linewidth]{Pictures/sierpinski3_a.pdf}
%   \caption{Restriction to $[0,1]^2$}
% \end{subfigure}
% \begin{subfigure}{.49\textwidth}
%   \centering
%   \includegraphics[width=.9\linewidth]{Pictures/sierpinski3_b.pdf}
%   \caption{Restriction to $[0,9]^2$}
% \end{subfigure}
% \caption{The Sierpi\'nski 3-gasket in $\RRnn^2$}
% \label{fig: sierpinski 3-gasket}
% \end{figure}
% In general, notice that since each can be realized by removing a union of open sets from $\RR^n$, all $\sierp_{p,n}$ are closed sets.
%
% Remarkably, the critical exponent of a monomial pair $(A, \vv{u})$ can be computed in terms of the Sierpi\'nski $p$-gasket, providing a geometric realization of this value.
% Toward making this relation precise, consider the following optimization problem, which essentially adds the extra ``fractal'' constraint from the definition of $\sierp_{p,n}$ to
% the linear program $\LP(A, \vv{u})$, after replacing the condition $A \vv{t} \leq \vv{u}$ with $A \vv{t} < \vv{u}$.
%
% \begin{definition}
% Given a monomial pair $(A, \vv{u})$, the \emph{fractal linear program} $\fip_p(A,\vv{u})$ consists of maximizing $\vv{t}\mapsto \norm{\vv{t}}$, with $\vv{t}$ in the \emph{closure} of the set of all $\vv{t}\in \sierp_{p,n}$ satisfying $A\vv{t} < \vv{u}$.
% % The value of the problem, $\val \fip_p(A,\vv{u})$, is defined as the supremum of $\norm{\vv{t}}$ among all $\vv{t} \in \feas \fip_p(A, \vv{u})$.
% \end{definition}
%
% \begin{remark}
%    The set $\mathcal{S}$ consisting of the points $\vv{t}\in \sierp_{p,n}$ satisfying $A\vv{t} < \vv{u}$ is contained in the intersection of the feasible set of the linear program $\LP(A, \vv{u})$ with the Sierpi\'nski $p$-gasket.
%    Since this intersection is compact, so is the closure of $\mathcal{S}$, namely $\feas\fip_p(A,\vv{u})$.
%    Thus, $\fip_p(A,\vv{u})$ has a well-defined value.
% %Moreover, there is exists an optimal point
% %$\vv{t} \in \feas \fip_p(A, \vv{u})$ such that $\norm{\vv{t}} = \val \fip_p(A, \vv{u})$.
% \end{remark}
%
% \begin{example} \label{ex: feas fip}
%  Consider the fractal linear program $\fip_p(A, \vv{u})$, where
% \[ A = \begin{pmatrix}
%  3&11\\ 11&2 \\ 5&10 \\ 2&0
%  \end{pmatrix}
% \text{ and } \vv{u} = \begin{pmatrix} 1 \\ 1 \\ 1 \\ \end{pmatrix}.
% \]
% \Cref{fig: feas fip} shows the feasible set for $\fip_p = \fip_p(A,\vv{u})$ in blue, the feasible set for $\LP = \LP(A,\vv{u})$ in gray, and the line of points with coordinate sum $\val \fip_p$ in green, for small values of $p$.
% \begin{figure}
%   \centering
% \begin{subfigure}{.49\textwidth}
% \centering
%   \includegraphics[width=.9\textwidth]{Pictures/ex4_char2.pdf}\hskip .04\textwidth
%   \caption{
%      $
%      \begin{array}{l}
%        \opt \fip_2 = \conv\big(\big(\frac1{20},\frac3{40}\big),\big(\frac1{12},\frac1{24}\big)\big)\\[2mm]
%        \val \fip_2 = \frac18
%      \end{array}
%      $
%   }
% \end{subfigure}
% \begin{subfigure}{.49\textwidth}
% \centering
% \includegraphics[width=.9\textwidth]{Pictures/ex4_char3.pdf}
% \caption{
%      $
%      \begin{array}{l}
%        \opt \fip_3 = \conv\big(\big(\frac1{36},\frac1{12}\big),\big(\frac7{81},\frac2{81}\big)\big)\\[2mm]
%        \val \fip_3 = \frac19
%      \end{array}
%      $
% }
% \end{subfigure}
%
% \bigskip
%
% \begin{subfigure}{.49\textwidth}
% \centering
%   \includegraphics[width=.9\textwidth]{Pictures/ex4_char5.pdf}\hskip .04\textwidth
%   \caption{
%      $
%      \begin{array}{l}
%        \opt \fip_5 = \opt \LP = \big\{\big(\frac2{25}, \frac3{50}\big)\big\}\\[2mm]
%        \val \fip_5 = \frac7{50}
%      \end{array}
%      $
%   }
% \end{subfigure}
% \begin{subfigure}{.49\textwidth}
% \centering
%   \includegraphics[width=.9\textwidth]{Pictures/ex4_char7.pdf}
%   \caption{
%      $
%      \begin{array}{l}
%        \opt \fip_7 = \big\{\big(\frac{19}{245}, \frac3{49}\big)\big\}\\[2mm]
%        \val \fip_7 = \frac{34}{245}
%      \end{array}
%      $ }
% \end{subfigure}
% \caption{The feasible and optimal sets of $\fip_p = \fip_p(A, \vv{u})$ for $A$ and $\vv{u}$ described in \Cref{ex: feas fip}, for small values of $p$}
% \label{fig: feas fip}
% \end{figure}
% \end{example}
%
% \begin{proposition}
% Given a monomial pair $(A, \vv{u})$, we have that
% \[\crit(A,\vv{u}) = \val\fip_p(A,\vv{u}).\]
% \end{proposition}
%
% \begin{proof}
% For each $q=p^e$,  $\frac{1}{q}\cdot\feas\IP_p(A,\vv{u}, q)$ is contained in $\feas \fip_p(A,\vv{u})$, so
% \[
% \val\fip_p(A,\vv{u}) \ge \displaystyle \lim_{q\to \infty}\frac{\val\IP_p(A,\vv{u}, q)}{q} = \crit(A,\vv{u}).
%  \]
% On the other hand, fix $\vv{t} \in \feas \LP(A, \vv{u}) \cap \sierp_{p,n}$, fix $q \geq 1$ a power of $p$, and a sequence $\{ \vv{t}_e \}_{e=1}^\infty$ in $\NN^n$, for which all $\norm{\vv{t}_e} < p$ and
% $\vv{t} = q \cdot \big( \frac{\vv{t}_1}{p} +\frac{\vv{t}_2}{p^2}+\cdots \big)$.  For $e \geq 1$, let
%  $\vv{t}^\star_{p^e} = q \cdot \big( \frac{\vv{t}_1}{p} + \cdots + \frac{\vv{t}_e}{p^e}  \big)$, so that $p^e \cdot \vv{t}^\star_{p^e} \in \IP_p(A, \vv{u}, p^e)$ and $\val \IP_p(A, \vv{u}, p^e) \geq p^e \norm{\vv{t}^\star_{p^e}}$.
%  Dividing by $p^e$ and taking limits, we find that
%  \[
% \crit(A,\vv{u}) = \lim_{q\to \infty} \frac{\val \IP_p(A, \vv{u}, q)}{q} \geq \lim_{q \to \infty}   \norm{\vv{t}^\star_q} = \norm{\vv{t}}.
%  \]
% Finally, given $\vv{s} \in \feas \fip_p(A, \vv{u}, p)$, fix a sequence of points $\{ \vv{s}_e \}_{e=1}^\infty$ in $\feas \LP(A, \vv{u}) \cap \sierp_{p,n}$ that limit to $\vv{s}$. Since $\vv{t} \mapsto \norm{\vv{t}}$ defines a continuous function $\RR^n \to \RR$, we have that $\crit(A, \vv{u}) \geq \lim \limits_{e \to \infty} \norm{ \vv{s}_e } =  \norm{ \vv{s} }$.
% Now, since $\vv{s}$ is an arbitrary element of $\feas \fip_p(A, \vv{u}, p)$, we have that
% $\crit(A, \vv{u}) \geq \val \fip_p(A, \vv{u})$, and equality holds.
% \end{proof}


\newpage
\section{Arithmetic integer programs revisted}


The remainder of this paper is dedicated to the study of the arithmetic integer programs $\IP_p(A, \vv{u}, p^e)$ described in \Cref{aip: D}, which perhaps unsurprisingly, turn out to be far more complex in nature than their non-arithmetic counterparts $\IP(A, \vv{u}, p^e)$. \daniel{State the main result of this section here.  It should be like \Cref{general-mu-theorem: T}, except with $\idealb$ replaced with an arbitrary monomial ideal, and $\beta(\ideala, \idealb) = \beta(\ideala)$.   Then, remind the reader that we will use this main theorem to prove \Cref{general-mu-theorem: T} in the appendix.}   Throughout this section, $e$ will denote a natural number, and $q$ will denote a power of $p$.  

\subsection{Some general facts}

In this subsection, we gather a few simple facts about such arithmetic programs that are valid in an arbitrary positive characteristic.  More specialized arguments, most of which are only valid in sufficiently large characteristics, will appear in subsequent subsections.  

\begin{lemma}
 \label{optimal division: L}
 The quotient when dividing any optimal point of $\IP_p(A, \vv{u}, qp^e)$ by $p^e$ must be optimal for $\IP_p(A, \vv{u}, q)$.  Consequently,    
 \[ \val \IP_p(A, \vv{u}, q) \cdot p^e \leq \val \IP_p(A, \vv{u}, qp^e) < \val \IP_p(A, \vv{u}, q) \cdot p^e +p^e. \]
%book-keeping:  the bounds on values used to be a stand-alone {natural bounds: C}
\end{lemma}

\begin{proof}  Suppose $\vv{g} \in \opt \IP_p(A, \vv{u}, qp^e)$ and write
\[ \vv{g} = \vv{h} p^e + \vv{k} \]
with $\vv{h}$ and $\vv{k}$ in $\NN^\numvars$ such that every coordinate of $\vv{k}$ is less than $p^e$.  
By \Cref{dickson: R}, the arithmetic constraint satisfied by the optimal point $\vv{g}$ implies that $0 \not \equiv \binom{\norm{\vv{g}}}{\vv{g}} \equiv \binom{\norm{\vv{h}}}{\vv{h}} \binom{\norm{\vv{k}}}{\vv{k}} \bmod p$, so that both $\binom{\norm{\vv{h}}}{\vv{h}}$ and $\binom{\norm{\vv{k}}}{\vv{k}}$ must be nonzero modulo $p$.  In particular, $\vv{h}$ satisfies the arithmetic constraint of $\IP_p(A,\vv{u},q)$, and it is easy to see that it also satisfies its linear constraint.

Having established its feasibility, we now demonstrate the optimality of $\vv{h}$.  Observe that, by virtue of being a remainder upon division by $p^e$, the base $p$ expansion of $\vv{k}$ is of the form $\vv{k} = \sum_{0 \leq s < e} \vv{k}_s \, p^s$.  As noted \Cref{dickson: R}, the nonvanishing of $\binom{\norm{\vv{k}}}{\vv{k}}$ is equivalent to the condition that each $\norm{\vv{k}_s}$ is less than $p$, and so it follows that $\norm{\vv{k}} < p^e$.  If $\vv{h}$ were not optimal for $\IP_p(A, \vv{u}, q)$, then there would exist $\vv{m}$ feasible for $\IP_p(A, \vv{u}, q)$ with $\norm{\vv{m}} \geq \norm{\vv{h}} + 1$, which would lead to a point $\vv{m}p^e$ feasible for $\IP_p(A, \vv{u}, qp^e)$ whose norm is \[ \norm{\vv{m}} \, p^e \geq \norm{\vv{h}} \, p^e + p^e >  \norm{\vv{h}} \, p^e + \norm{\vv{k}} = \norm{\vv{g}}\] which contradicts the optimality of $\vv{g}$. We conclude that $\vv{h}$ is optimal.

Finally, given the optimality of $\vv{g}$ and $\vv{h}$, the asserted bounds on values follow from a direct computation of the norm of the optimal point $\vv{g}$.
\end{proof}

\begin{corollary}
   \label{cor: mu comparison}
   If $\val \IP_p(A, \vv{u}, q)$ is greater than $\val \IP_p(B, \vv{v}, q)$ for some $q$, then $\val \IP_p(A, \vv{u}, qp^e)$ is greater than $\val \IP_p(B, \vv{v}, qp^e)$ for all $e \geq 0$.
\end{corollary}

\begin{proof}
    \Cref{optimal division: L} tells us that $\val \IP_p(A, \vv{u}, qp^e)$ is at least $\val \IP_p(A, \vv{u}, q) \, p^e$, which our assumption implies is at least $(\val \IP_p(B, \vv{v}, q)+1)\, p^e$, which \Cref{optimal division: L} again tells us is greater than $\val \IP_p(B, \vv{u}, qp^e)$. 
%    
%    If $\val \IP_p(A, \vv{u}, q) \geq \val \IP_p(B, \vv{v}, q) + 1$, then tells us that
%   \begin{align*}
%     \val \IP_p(A, \vv{u}, qp^e)  &\geq \val \IP_p(A, \vv{u}, q) \cdot p^e \\
%                                  &\geq (\val \IP_p(B, \vv{v}, q)+1)\cdot p^e \\
%                                  & > \val \IP_p(B, \vv{u}, qp^e). \qedhere
%   \end{align*}
\end{proof}

\subsection{Small and very small pairs}

\ \pedro[inline]{Postpone introduction of medium small points until immediately before definition of $\widehat{\graph}$ graph? \daniel[inline]{I kind of makes sense to me to introduce them together, if only to be able to immediately provide the geometric description.}}

\begin{definition}
A monomial pair $(A, \vv{u})$ is \emph{small} if $\vv{u}$ is not greater than any column of $A$, and is \emph{very small} if $\ft{A}{\vv{u}}$ is at most one. 
\end{definition}

\begin{remark}[Geometric description of small and very small pairs]
\label{finitely many small but not very small: R}
Suppose that $A$ is a $\numvars \times n$ monomial matrix, and let $\N \subseteq \RR^{\numvars}$ be its Newton polyhedron.

Geometrically, a pair $(A, \vv{u})$ is small if and only if the positive lattice point $\vv{u} \in \ZZ^{\numvars}$ does not lie in the interior of the Minkowski sum of the set of columns of $A$ with the ambient nonnegative orthant, a subset of $\RR^{\numvars}$ sometimes called the \emph{upper staircase} determined by the columns of $A$. 

On the other hand, \Cref{FT: D} and \Cref{FT descriptions: P}\eqref{lambda} imply that a pair $(A, \vv{u})$ is very small if and only if $\vv{u}$ does not lie in the interior of $\N$.

It is clear from these interpretations that a very small pair must also be small, and that a small pair may fail to be very small.  However, once the monomial matrix $A$ is fixed, there are only finitely pairs $(A, \vv{u})$ that are small, but not very small, as the region lying outside of the interior of the staircase determined by $A$, but within the Newton polyhedron $\N$, is bounded.%\daniel{This doesn't seem to come up later, so I decided not to add a proof.  It is mostly here for intuition, I guess.} 
\end{remark}


\begin{proposition}
   \label{trivial value bound: P}
   If $(A, \vv{u})$ is small,  then $\val \IP_p(A, \vv{u}, p^e) \leq p^e-1$.
\end{proposition}

\begin{proof}
   Note that $(A, \vv{u})$ is small if and only if $\vv{0}$ is the only feasible point for (non-arithmetic) program $\IP(A, \vv{u}, 1)$.
   Consequently, the feasible set for the arithmetic program $\IP_p(A, \vv{u}, 1)$ must also consist of only $\vv{0}$.  Thus, $\val \IP_p(A, \vv{u}, 1) = 0$, and our assertion then follows from \Cref{optimal division: L}.
\end{proof}


Below, we identify an important case in which we are able to solve the arithmetic integer program $\IP_p(A, \vv{u}, p^e)$ in sufficiently large characteristics by appealing to our earlier work on the standard integer program $\IP(A, \vv{u}, p^e)$.  This result is fundamental, and using it, we will eventually be able to solve every such arithmetic program, at least when $p$ is large relative to $A$.

\begin{theorem}
\label{arithmetic uniform value and image: T}   Given a monomial matrix $A$, there exists an integer $\beta = \beta(A)$ with the following property\textup:
If $(A, \vv{u})$ is very small, and $p > \beta$, then the arithmetic integer program $\IP_p(A, \vv{u}, p)$ agrees with the standard integer program $\IP(A, \vv{u}, p)$.  In particular, if $p > \beta$, then \[ \val \IP_p(A, \vv{u}, p) = \val \IP(A, \vv{u}, p)  = \ft{A}{\vv{u}} \cdot p - \delta(A, \vv{u}, p) \] and 
\[ \ol{ \im \IP_p(A, \vv{u}, p)} = \ol{ \im \IP(A, \vv{u}, p)} = \collapse{\vv{u}}p - \Delta(A, \vv{u}, p) \] where the overbar denotes collapse along $\O = \mf(A, \vv{u})$.
\end{theorem}

\begin{proof}  If $\beta = \beta(A)$ is as in \Cref{uniform uniform value and image: T}, then \[ \val \IP(A, \vv{u}, p) = \ft{A}{\vv{u}} \cdot p - \delta(A, \vv{u}, p) \] for every monomial pair $(A, \vv{u})$ and $p > \beta$.  Given that $\ft{A}{\vv{u}} \leq 1$, the positivity of $\delta(A, \vv{u},p)$ will then imply that this quantity is less than $p$.  Consequently, every $\vv{k}$ feasible for $\IP(A, \vv{u}, p)$ satisfies $\norm{\vv{k}} \leq p-1$, and therefore automatically satisfies the arithmetic constraint of $\IP_p(A, \vv{u}, p)$.  We conclude that $\IP(A, \vv{u}, p) = \IP_p(A, \vv{u}, p)$ whenever $\ft{A}{\vv{u}}$ is at most $1$ and $p > \beta$, and our assertion follows from applying \Cref{uniform uniform value and image: T} once again.
\end{proof}






\emily[inline]{If $(A, \vv{u})$ is small but not very small, then $\mu_\ideala^{\vv{u}}(p) = p-1$, so $\mu_\ideala^{\vv{u}}(p) \neq \nu_\ideala^{\vv{u}}(p)$.
In this case, although our description of $\ideala^{\nu_\ideala^{\vv{u}}(p)}$ does not depend on $p$, we \emph{can} have that the generators of $\ideala^{\mu_\ideala^{\vv{u}}(p)}
= \ideala^{p-1}$ depend on $p$:  For instance, if $\ideala = \langle x, y, \rangle$, then $\nu_\ideala^{\vv{u}}(p) = 2p-2$  and $\mu_\ideala^{\vv{u}}(p) = p-1$.  Moreover, $x^{p-(p+1)/2}y^{p-(p+1)/2} \in \ideala^{p-1}$.
(Actually, all minimal generators of $\ideala^{p-1}$ depend on $p$.)
}

\daniel[inline]{I'm not sure what the purpose of these results in {\color{red} red} below was, but it doesn't look like they appear anywhere, and so I propose we omit them}

{\color{red}
\begin{lemma}
\label{refined-discreteness: L}
Given a monomial matrix $A$, there exists $\delta = \delta(A)$ such that $\ft{A}{\vv{u}} < \delta$ whenever $(A, \vv{u})$ is small.
\end{lemma}

\begin{proof}   Fix a small pair $(A, \vv{u})$ with $\O = \mf(A, \vv{u})$.  If $\epsilon$ is the number of columns of $A$ lying on $\O$, then it suffices to prove that $\ft{A}{\vv{u}} \leq \epsilon$.

By means of contradiction, suppose that $\ft{A}{\vv{u}} > \epsilon$.  If $\vv{s} \in \sp(A,\vv{u})$, then $\norm{\vv{s}} = \ft{A}{\vv{u}} > \epsilon$, and as $\vv{s}$ has at most $\epsilon$ nonzero entries, some entry of $\vv{s}$ must be greater than $1$.  Thus, $A \vv{s}$ is greater than some column of $A$.  However, our choice of $\vv{s}$ also implies that $A \vv{s} \leq \vv{u}$, which then implies that $\vv{u}$ is greater than some column of $A$, contradicting the smallness of $(A, \vv{u})$.
\end{proof}


\begin{proposition}
   \label{follow-leftovers: P}
   Suppose $(A, \vv{u})$ is a monomial pair.
   If
   \[ \im \IP(A, \vv{u}, 1) = \vv{u} - \Z\]
   then every monomial pair $(A, \vv{z})$ with $\vv{z} \in \Z$ is small, and if $p \gg 0$ \daniel{Specify what $p \gg 0$ means here?} and $e \geq 0$, then
   \[ \val \IP_p(A, \vv{u}, p^e) = \val \IP(A, \vv{u}, 1) \cdot p^e + \max \val \IP_p(A, \vv{z}, p^e) \]
   where the maximum is over all points $\vv{z} \in \Z$.
\end{proposition}

\daniel[inline]{This proof seems way too long.}

\begin{proof}
   The constraints of $\IP(A, \vv{u}, 1)$ imply that $\Z$ is a finite set of lattice points with positive coordinates.   These constraints and optimality also imply that if $\canvec$ is a standard basis vector in the domain of $A$, then no point in the Minkowski sum $\canvec + \opt \IP(A, \vv{u}, 1)$ can be feasible for $\IP(A, \vv{u}, 1)$.  Applying $A$ to this shows that no point in
\[ A \canvec + \im \IP(A, \vv{u}, 1) = A \canvec + \vv{u} - \Z \]
is less than $\vv{u}$.  Thus, $A \canvec$ is not less than any point in $\Z$, and as $\canvec$ was arbitrary, it follows that $(A, \vv{z})$ is small for every $\vv{z} \in \Z$.

The finiteness of $\Z$ allows us to choose $p$ large enough so that \[ \val \IP(A, \vv{v}, 1) \leq p -1 \] for every point $\vv{v} \in \Z \cup \{ \vv{u} \}$.  In this case, every feasible point for $\IP(A, \vv{v}, 1)$  automatically satisfies the arithmetic constraint of $\IP_p(A, \vv{v}, 1)$, which allows us to conclude that $\IP(A, \vv{v}, 1) = \IP_p(A, \vv{v}, 1)$.  In particular,
\[ \im \IP_p(A, \vv{u}, 1) =\vv{u} - \Z. \]

Next, fix $\vv{g}$ optimal for $\IP_p(A, \vv{u}, p^e)$.  If $\vv{h}$ is the quotient, and $\vv{k}$ the remainder, when dividing $\vv{g}$ by $p^e$, then \Cref{optimal division: L} tells us that $\vv{h}$ is optimal for $\IP_p(A, \vv{u}, 1)$, so that $A \vv{h} = \vv{u}-\vv{z}$ for some $\vv{z} \in \Z$.  The feasibility of $\vv{g}=\vv{h}p^e + \vv{k}$ for $\IP_p(A, \vv{u}, p^e)$ then implies the feasibility of $\vv{k}$ for $\IP_p(A, \vv{z}, p^e)$.  This establishes that $\norm{\vv{g}} = \val \IP_p(A, \vv{u}, p^e)$ is at most the asserted value.

To establish the opposite inequality, suppose $\vv{z}^{\ast}$ is a point in $\Z$ with $\val \IP_p(A, \vv{z}^{\ast}, p^e)$ maximal.  By virtue of being in $\Z$, we may write $\vv{z}^{\ast} = \vv{u} - A \vv{g}^{\ast}$ for some $\vv{g}^{\ast} \in \opt \IP(A, \vv{u}, 1)$.  If $\vv{k}^{\ast}$ is optimal for $\IP_p(A, \vv{z}^{\ast}, p^e)$, then a direct computation will show that
$\vv{h}^{\ast} = \vv{g}^{\ast} p^e + \vv{k}^{\ast}$ satisfies the linear constraint of  $\IP_p(A, \vv{u}, p^e)$.  Furthermore, the feasibility of $\vv{k}^{\ast}$ implies that $\binom{\norm{\vv{k}^{\ast}}}{\vv{k}^{\ast}} \not \equiv 0 \bmod p$, and the smallness of $(A, \vv{z}^{\ast})$ and \Cref{trivial value bound: P} tell us that $\norm{\vv{k}^{\ast}} \leq p^e-1$.  On the other hand, our choice of $p \gg 0$ tells us that $\norm{\vv{g}^{\ast}} = \val \IP_p(A, \vv{u}, 1) = \val \IP(A, \vv{u}, 1) \leq p-1$, and it follows that $\vv{h}^{\ast}$ also satisfies that the arithmetic constraint of $\IP_p(A, \vv{u}, p^e)$.
\end{proof}
}


\subsection{Sprouting}

\daniel[inline]{Perhaps we should motivate why we would want to look at $p$-sprouts.  The point is that they determine which $\mu$'s we should compute next, at least in terms of the collapse.}


\begin{definition}
\label{p-sprout: D}
We say that $(B, \vv{v})$ is a \emph{$p$-sprout} of a monomial pair $(A, \vv{u})$ whenever the following conditions are satisfied.
\begin{enumerate}
\item $B$ is the collapse of $A$ along the minimal face $\O = \mf(A, \vv{u})$.
\item $\vv{v}$ is any point in $\Delta(A, \vv{u}, p)$.
\end{enumerate}
\end{definition}



\begin{remark}
\label{p-sprout: R}
As noted in \Cref{collapse of monomial is monomial: R}, the collapse of a monomial matrix along a face of its Newton polyhedron is monomial, and so a $p$-sprout of a monomial pair is also a monomial pair.  Furthermore,   \Cref{independence: L} tells us that there are only finitely many $p$-sprouts of a fixed monomial pair, and that each such sprouted pair is small.
 \end{remark}


\begin{corollary}\label{cor: upper bound for higher mus}
Given a monomial matrix $A$, there exists an integer $\beta=\beta(A)$ with the following property\textup:  If $(A, \vv{u})$ is very small, then
%
\[ \val \IP_p(A, \vv{u}, p^{e+1})  \leq  \val \IP_p(A, \vv{u}, p) \cdot p^e +  \max \val \IP_p(B, \vv{v}, p^e) \]
%
for all $p > \beta$ and $e \geq 1$, where the maximum is over all $p$-sprouts $(B, \vv{v})$ of $(A, \vv{u})$.
\end{corollary}

\!{Update this proof.  What is B?}

\begin{proof}  Let $\beta$ be as in \Cref{arithmetic uniform value and image: T}, and fix a monomial pair $(A, \vv{u})$ that is very small.
Suppose $\vv{g}$ is optimal for $\IP_p(A, \vv{u}, p^{e+1})$, and let $\vv{h}$ and $\vv{k}$ be the quotient and remainder, respectively, when dividing $\vv{g}$ by $p^e$.

Let an overbar denote collapse along $\O = \mf(A, \vv{u})$.  \Cref{optimal division: L} tells us that $\vv{h}$ must be optimal for $\IP_p(A, \vv{u}, p)$, and \Cref{arithmetic uniform value and image: T} then implies that $B \vv{h} = \collapse{A \vv{h}} \in \ol{\im \IP_p(A, \vv{u}, p)} = \collapse{\vv{u}}p - \Delta(A, \vv{u}, p)$ for all $p > \beta$.
Therefore, for $p > \beta$, we may write \[ B \vv{h} = \collapse{\vv{u}}p - \vv{v}\] for some $\vv{v} \in \Delta(A, \vv{u}, p)$.  On the other hand, our choice of $\vv{g}$ guarantees that $A \vv{g} < \vv{u}p^{e+1}$, which leads to the inequality $B \vv{h} p^e + B \vv{k} = B \vv{g} <  \collapse{\vv{u}}p^{e+1}$  in $\rs(\O)^{\perp}$.  Comparing this with the above description of $B \vv{h}$ shows that \[ B \vv{k} < \vv{v} p^e \] which allows us to conclude that $\vv{k} \in \IP_p(B, \vv{v}, p^e)$.  %The corollary then follows from the fact that $\norm{\vv{g}} = \norm{\vv{h}} \cdot p^e + \norm{\vv{k}}$.
\end{proof}


\subsection{The Sprouting graph associated to a monomial matrix}


\begin{definition} \daniel{By the way, I think Emily and I have shown that the collapse of a collapse of $A$ is a collapse of $A$.  This will mean that the only matrices that can appear in $\S_p(A)$ are the collapses of $A$.  We don't gain any stronger theoretical finiteness properties, but this might simplify any implementations}
Given a monomial matrix $A$ and a prime $p$, define
\begin{enumerate}
   \item $\graph^0(A,p) = \{(A,\vv{u}) : (A,\vv{u})\text{ is a small monomial pair} \}$; \!{Is this semi-colon cool?}
   \item $\displaystyle\graph^{e+1}(A,p) = \bigcup_{(B,\vv{v})\in \graph^e(A,p)}\sprout(B,\vv{v},p)$ for $e \geq 0$. \!{Update this definition.  Make sure sprout notation is correct.  }
\end{enumerate}
\end{definition}


\emily[inline]{We think that $\{ \graph^e : e \geq 1 \}$ is finite, but only care that $\bigcup_{e=1}^\infty \graph^e(A)$ is.}

\emily[inline]{verify that $\graph_e(A, \vv{u})$ and $\graph_e(A)$ are eventually periodic}
\daniel[inline]{In the remark (or wherever) when we gather some basic finiteness properties, at least state that there are only finitely many matrices appearing in any vertex of $\graph_p(A)$ as $p$ varies}

%\subsection*{Finiteness properties}
%
%Once $A$ is fixed,
%\begin{itemize}
% \item $\bigcup_{e=1}^\infty \graph^e(A,p)$ is finite.
% \item There exist $D$ such that for all $e \geq 1$ and $(B, \vv{v}) \in \graph^e(A,p)$, there exists $\vv{s} \in \sp(B, \vv{v})$ with denominator $D$.
% \item $\mathbb{O}(A)$ is finite, and $\bigcup_{(B, \vv{v}) \in \graph^e(A), \text{ some } e} \mathbb{O}(B, \vv{v}, \vv{s}, p)$ is finite.
% \item Add the last point
%\end{itemize}

\begin{theorem}[Iterated lifting]
\label{ILL: T}
   For each monomial matrix $A$, there exists an integer $\beta = \beta(A)$ with the following property\textup:
   If $p>\beta$ and $(A_1, \vv{u}_1) \to (A_2, \vv{u}_2) \to \cdots \to (A_e, \vv{u}_e)$ is a path in $\graph_p(A)$, then for every $1 \leq i \leq e$, there exists a point $\vv{k}_i \in \opt \IP(A_i, \vv{u}_i,p)$  for which
 \[
  \vv{k}_1 p^{e-1} + \vv{k}_2 p^{e-2} + \cdots + \vv{k}_{e-1} p + \vv{k}_e \in \feas \IP(A_1, \vv{u}_1, p^e).
 \] \!{Define path notation}
\end{theorem}

\begin{proof}\daniel{The ``finiteness properties" part of this proof is slightly different than what we sketched in Lawrence, but the rest of the argument follows what we talked about then.  I think it is correct, but it would be nice if someone could verify this.}  We start by describing what it means $p$ to be large.  Toward this, let $M_1, \cdots, M_l$\daniel{Update these $M_i$ if we include a proof that the collapse of a collapse of $A$ is a collapse of $A$.} be the finitely many monomial matrices obtained from $A$ by omitting some of its columns.  \Cref{special-denominators-exist:  T} allows us to fix a positive integer $\denom$ that is a special denominator for each such matrix.  We may also fix a finite set of representatives $\fsr(M_i)$ for each such matrix, as described in \Cref{fsr-exist: T}.  Set $\fsr = \fsr(M_1) \cup \cdots \cup \fsr(M_l)$, and let $\Omega$ be the set consisting of all coordinates of all points in $A(\fsr)$.  We stress that $\fsr$ and $\Omega$ are finite sets determined by $A$, and do not depend in any way on $p$.

Consider the conditions \eqref{p-big-1} and \eqref{p-big-2} below.
%
\begin{align}
\tag{$\heartsuit$} \label{p-big-1}
\text{$p/\denom$ is greater than any coordinate of any point in $\vv{1} - \fsr$.} \\
 \label{p-big-2}
\tag{$\diamondsuit$}\text{$p^e / \denom > \sum_{i=1}^e \omega_i \cdot p^{e-i}$ for every $e \geq 1$ and $\omega_1, \cdots, \omega_e \in \Omega$,}
\end{align}

The finiteness of $\fsr$,  and \Cref{positive-polynomial: L} below, imply that there exists an integer $\beta = \beta(\denom, \fsr)$ for which \eqref{p-big-1} and \eqref{p-big-2} hold whenever $p > \beta$.  In what follows, we assume that $p$ is chosen so that these conditions are satisfied.

Now, consider a finite path \[ (A_1, \vv{u}_1) \to (A_2, \vv{u}_2) \to \cdots \to (A_e, \vv{u}_e) \] in $\graph_p(A)$.  For every $1 \leq i \leq e$, set $\O_i = \mf(A, \vv{u}_i)$, and fix a special point $\vv{s}_i \in \sp(A_i, \vv{u}_i)$ with denominator $\denom$.  If $1 \leq i < e$, then the sprouting $(A_i, \vv{u}_i) \to (A_{i+1}, \vv{u}_{i+1})$ tells us that $A_{i+1}$ is the collapse of $A_i$ along $\O_i$, and that $\vv{u}_{i+1} \in \Delta(A_i, \vv{u}_i, q) = A_{i+1} \tail{\vv{s}_i}_p - \im  \ip( A_i, \vv{u}_i, \vv{s}_i, p)$.  Theorem \Cref{fsr-exist: T}, and our choice of $\fsr$, then allow us to fix a point $\vv{h}_i$ in $\fsr \cap \opt \ip ( A_i, \vv{u}_i, \vv{s}_i, p)$ such that
$\vv{u}_{i+1} = A_{i+1} \tail{\vv{s}_i}_p - A_{i+1} \vv{h}_i$.  Finally, we take $\vv{h}_e$ to be an arbitrary point in the nonempty set $\fsr \cap \opt \ip ( A_e, \vv{u}_e, \vv{s}_e, p)$.


Next, for every $1 \leq i \leq e$,  we define
  \[
\vv{k}_i = \vv{s}_i \cdot p - [\vv{s}_i]_p + \vv{h}_i.
\]
Observe that \eqref{p-big-1} and \eqref{p-big-2} imply that for every $1 \leq i \leq e$, the quantity $p/\ell$ is greater than every coordinate of $\vv{1}-\vv{h}_i$, and every coordinate of $A_i \vv{h}_i$.  It then follows from \Cref{uniform value: P} (or rather, its proof) that
\begin{equation}
\label{optimality-for-each-component: e}
\vv{k}_i \in \opt \IP(A_i, \vv{u}_i,p)
\end{equation}
for every $1 \leq i \leq e$.

We will now induce on $e$ to prove that $\sum_{i=1}^e \vv{k}_i \cdot p^{e-i}$ is feasible for $\IP(A_1, \vv{u}_1, p^e)$.  When $e = 1$, this follows from \eqref{optimality-for-each-component: e}.  Next, suppose that $e \geq 2$.  Our induction hypothesis applied to the truncated path
\[ (A_2, \vv{u}_2) \to \cdots \to (A_e, \vv{u}_e) \]
%
tells us that $\vv{k}^{\ast} = \sum_{i=2}^e \vv{k}_i \cdot p^{e-i} \in \feas \IP(A_2, \vv{u}_2, p^{e-1})$.  To complete the induction step, we must show that $\vv{k}_1 p^{e-1} + \vv{k}^{\ast}$ is feasible for $\IP(A_1, \vv{u}_1, p^e)$.  However,  \eqref{optimality-for-each-component: e} implies that this point has nonnegative integer coordinates, and hence, we must only show that $A_1 ( \vv{k}_1 p^{e-1} + \vv{k}^{\ast} ) < \vv{u}_1 p^e$.

To do so,  recall that our choice of $\vv{s}_1 \in \sp(A_1, \vv{u}_1)$ allows us to express $\vv{u}_1$ as
$\vv{u}_1 = A_1 \vv{s}_1 + \vv{w}$, where $\vv{w}$ is some point in $\rs(\O_1)$ that is positive in this Euclidean space.  This expression implies that the special denominator $\denom$ is also a denominator for $\vv{w}$.  It then follows from the definition of $\vv{k}_1$ and this expression for $\vv{u}_1$ that the inequality $A_1 ( \vv{k}_1 p^{e-1} + \vv{k}^{\ast} ) < \vv{u}_1 p^e$ is equivalent to
%
\begin{equation}
\label{target-inequality: e}
  A_1( - \tail{\vv{s}_1}_p + \vv{h}_1 ) \cdot p^{e-1} + A_1\vv{k}^{\ast} < \vv{w} p^e.
\end{equation}

Given that the target of $A_1$ is $\rs(\O_1) \oplus \rs(\O_1)^{\perp}$, it suffices to verify that \eqref{target-inequality: e} holds after projection to each of these Euclidean spaces.  We first consider the projection to $\rb(\O_1)^{\perp}$.  As $A_2$ is the collapse of $A_1$ along $\O_1$, the projection of $A_1 \vv{m}$ to $\rb(\O_1)^{\perp}$ equals $A_2 \vv{m}$ for every $\vv{m}$ in the domain of $A_1$.  In particular, the projection of $A_1( - \tail{\vv{s}_1}_p + \vv{h}_1 )$ to this subspace is $A_2 ( - \tail{\vv{s}_1}_p + \vv{h}_1 )$, which equals $-\vv{u}_2$, by our choice of $\vv{h}_1$.  Thus, projecting \eqref{target-inequality: e} to $\rs(\O)^{\perp}$ yields $-\vv{u}_2 \cdot p^{e-1} + A_2 \vv{k}^{\ast} < \vv{0}$, which holds as $\vv{k}^{\ast} \in \feas \IP(A_2, \vv{u}_2, p^{e-1})$.

We now consider the projection of \eqref{target-inequality: e} to $\rs(\O_1)$, and given that $\tail{\vv{s}_1}_p \geq \vv{0}$, it suffices to verify that the projection of the stronger inequality \[ \sum_{i=1}^{e} A_1 \vv{h}_i \cdot p^{e-i} = A_1 \vv{h}_1 \cdot p^{e-1} + A_1 \vv{k}^{\ast} < \vv{w}p^e \] to $\rs(\O)^{\perp}$ holds.  However, keeping in mind that $\ell$ is a denominator of $\vv{w}$, which is positive in $\rs(\O_1)$, every coordinate of the projection of $\vv{w} p^e$ to $\rs(\O_1)$ is at least $p^e / \ell$, while every coordinate of $\sum_{i=1}^{e} A_1 \vv{h}_i \cdot p^{e-i}$ is of the form $\sum_{i=1}^{e} \omega_i \cdot p^{e-i}$ for some $\omega_1, \cdots, \omega_e \in \Omega$.  Thus, the condition \eqref{p-big-2} tells us that this stronger inequality holds after projecting to $\rs(\O_1)$.

We have just verified that \eqref{target-inequality: e} holds throughout $\rs(\O_1) \oplus \rs(\O_1)^{\perp}$, which allows us to conclude the induction step, and hence, our proof.
\end{proof}

\begin{lemma}
   \label{positive-polynomial: L}
   Given a real number $w > 0$, and a set $\Omega$ of real numbers that is bounded from above, there exists an integer $\beta = \beta(w, \Omega)$ satisfying the following condition\textup:
   If $p > \beta$, then for every integer $e \geq 1$, and for every $\omega_1, \ldots, \omega_e \in \Omega$, we have that $wp^{e} >  \omega_1 \cdot p^{e-1} + \cdots + \omega_{e-1} \cdot p + \omega_e$.
\end{lemma}

\begin{proof}
Let $\lambda$ be any positive upper bound for $\Omega$.  Suppose that $p > (w+\lambda)/w$, which after rearranging terms, is equivalent to the condition $w(p-1) - \lambda > 0$.  Multiplying this by $p^e$ and adding the positive number $\lambda$ then shows that
%
\[ wp^e ( p-1 ) - \lambda (p^e-1) > 0 \] for every integer $e \geq 1$.   If, in addition, we also suppose that $p -1 > 0$, then we may divide the above by this quantity to conclude that \[ w p^e - \lambda \cdot \frac{ p^e - 1}{p-1} = wp^e - \lambda p^{e-1} - \cdots - \lambda p - \lambda > 0 \] for every integer $e \geq 1$.   Moving every term but $wp^e$ to the right-hand side of this inequality, the resulting bound is enough to conclude our proof.
\end{proof}

%
\daniel[inline]{I ended up combining these.  We can split them up later if anyone (possibly, me) prefers this.  I also added the hypothesis that the pairs in the first path were small, which was missing.}
%
\begin{corollary}\label{cor: iterated lifting}
Given a monomial matrix $A$, there exists an integer $\beta = \beta(A)$ such that the following hold for every $p > \beta$ and path \daniel{We haven't defined the graph $\graph_p(A)$ yet (i.e., the arrows and ``levels" of the vertices hasn't been discussed.  Once we do this, we should define ``$\in''$ to mean ``path in", as this might save us some writing, and it is intuitive.  What does everyone think?}\[ (A_1, \vv{u}_1) \to \cdots \to (A_e, \vv{u}_e)  \in \graph_p(A).\]
\begin{enumerate}
\item If $(A_i, \vv{u}_i)$ is very small for every $1 \leq i \leq e$, then \[ \mu(A_1, \vv{u}_1, p^e) \geq \sum_{i=1}^e \mu(A_i, \vv{u}_i, p) \, p^{e-i}.\]
\item If $(A_i, \vv{u}_i)$ is very small for $1 \leq i < e$, but the last pair $(A_e, \vv{u}_e)$ is not very small, then for every integer $s \geq 0$,
 \[ \mu(A_1, \vv{u}_1, p^{e+s}) \geq \sum_{i=1}^{e-1} \mu(A_i, \vv{u}_i, p) \, p^{e+s-i} + p^{s+1}-1. \]
\end{enumerate}
\end{corollary}

\begin{proof} The proofs of each assertion are similar; we only prove the second, which is more involved.  Let $\beta = \beta(A)$ be as in \Cref{ILL: T}.  If $p > \beta$, then \Cref{ILL: T} tells us that there exists $\vv{k}_i \in \opt \IP(A_i, \vv{u}_i, p)$ for which \[ \vv{k}^{\ast} = \sum_{1 \leq i < e} \vv{k}_i \cdot p^{e-i} + \vv{k}_e \in \feas \IP(A_1, \vv{u}_1, p^e).\]

  The assumption on the points in the path implies that $\norm{\vv{k}_i} \leq p-1$ for all $1 \leq i < e$, while $\norm{\vv{k}_e} \geq p$.  Thus, there exists a point $\vv{g}$ in the domain lattice of $A$ such that $\norm{\vv{g}} = p-1$ and $\vv{0} \leq \vv{g} \leq \vv{k}_e$, with the last inequality strict in at least one coordinate, say, in the first coordinate.  Thus, $\vv{0} \leq \vv{g} + \canvec_1 \leq \vv{k}_e$.

Fix an integer $s \geq 0$, and set
%
\[ \vv{h} = \sum_{1 \leq i < e} \vv{k}_i \cdot p^{e+s-i} + (\vv{g} + \canvec_1) \cdot p^{s} - \canvec_1 \]
%
The bound $\vv{0} \leq \vv{g} + \canvec_1 \leq \vv{k}_e$ implies that $\vv{h} \leq \vv{k}^{\ast}  p^s$, and the feasibility of $\vv{k}^{\ast}$ for $\IP(A_1, \vv{u}_1, p^e)$ then implies that  $\vv{h}$ is feasible for $\IP(A_1, \vv{u}_1, p^{e+s})$.  To see that $\vv{h}$ is also feasible for the arithmetic version of this program, simply observe that the base $p$ expansion of $\vv{h}$ is given by
%
\[ \vv{h} = \sum_{1 \leq i < e} \vv{k}_i \cdot p^{e+s-i} + \vv{g} \cdot p^{s} + (p-1) \canvec_1 \cdot p^{s-1} + \cdots + (p-1) \canvec_1 \]
%
and recall that $\norm{\vv{k}_i} \leq p-1$ for every $1 \leq i < e$.  Therefore,
%
\[ \mu(A_1, \vv{u}_1, p^{e+s}) \geq \norm{\vv{h}} = \sum_{1 \leq i < e} \mu(A_i, \vv{u}_i, p) \cdot p^{e+s-i}+ p^{s+1}-1. \qedhere\]
%
\end{proof}

\begin{corollary}
   Given a monomial matrix $A$, there exists an integer $\beta = \beta(A)$ with the following property\textup: If $p > \beta$ and $(A, \vv{u})$ is small, but not very small, then $\mu(A,u,p^e) = p^e-1$ for every $e \geq 1$.
   \qed
\end{corollary}

\subsection{The Sprouting graph associated to a very small pair}
\begin{definition}
   Suppose that $(A, \vv{u})$ is very small.
   For $e \geq 0$, we define the set $\widehat{\graph}^e(A,\vv{u},p)$ inductively as follows:
\begin{enumerate}
\item $\widehat{\graph}^0(A, \vv{u}, p)$ consists of the single monomial pair $(A, \vv{u})$.
\item Suppose that $\widehat{\graph}^e(A, \vv{u}, p)$ has been defined for some integer $e \geq 0$, and let $S$ be the set of all $p$-sprouts of all monomial pairs in $\widehat{\graph}^e(A, \vv{u}, p)$.
If  $S$ is empty (that is, $\widehat{\graph}^e(A, \vv{u}, p)$ itself is empty) or contains a pair that is not very small, then \[ \widehat{\graph}^{e+1}(A, \vv{u}, p) = \emptyset.\]
\emily{or say $\emptyset$ whenever $\graph^e(A, \vv{u}, p)$ is empty, or contains a medium-small pair}
Otherwise, $\widehat{\graph}^{e+1}(A, \vv{u}, p)$ is the set of all sprouts $(B, \vv{v})$ in $S$ satisfying the following conditions:

\begin{enumerate}
\item Among all pairs in $S$, the value of  $\ft{B}{\vv{v}}$ is maximal.
\item Among all pairs in $S$ that achieve this maximum, the value of $\delta(B, \vv{v}, p)$ is minimal.
\end{enumerate}
Consequently, the value $\mu(B, \vv{v},p)$ is maximized among all elements in $S$ when $p \gg 0$.
\end{enumerate}
\end{definition}


\begin{proposition}
   Given a monomial matrix $A$, there exists an integer $D$ such that $\widehat{\graph}(A, \vv{u}, p) = \widehat{\graph}(A, \vv{u}, q)$ for every monomial pair $(A, \vv{u})$ whenever $p \equiv q \bmod D$.
\end{proposition}

\alert[inline]{Include the proof.  It has to do with some finiteness property of $\ip$.}


% \begin{corollary}
% If $(A, \vv{u})$ is a monomial pair and $p \gg 0$, then
% \[ \val \IP_p(B, \vv{v}, p) = \ft{B}{\vv{v}} \cdot p - \delta_p(B, \vv{v}) \] for any vertex $(B, \vv{v})$ of $\widehat{\graph}_p(A, \vv{u})$.    In addition, if $(B, \vv{v})$ and $(D, \vv{z})$ are any two such vertices, then $\val \IP_p(B, \vv{v}) < \val \IP_p(D, \vv{z})$ if and only if $\ft{B}{\vv{v}} < \ft{D}{\vv{z}}$, or these two quantities agree and $\delta_p(B, \vv{v}) > \delta_p(D, \vv{z})$.
% \end{corollary}


\begin{lemma}\label{lem: upper bound for higher mu}
   Given a monomial matrix $A$, there exists an integer $\beta= \beta(A)$ for which the following holds\textup:
   For each $p>\beta$ and $e\ge 1$, if $(A, \vv{u})$ is a very small monomial pair and $(A_1, \vv{u}_1) \to \cdots \to (A_e, \vv{u}_e)$ is a path in $\widehat{\graph}(A, \vv{u}, p)$,  then
   \[
      \mu(B, \vv{v}, p^e) \le \mu(A_1, \vv{u}_1, p)p^{e-1} + \mu(A_2, \vv{u}_2, p)p^{e-2} + \cdots + \mu(A_{e}, \vv{u}_{e}, p)
   \]
   for any vertex $(B, \vv{v})$ of $\widehat{\graph}(A, \vv{u}, p)$ on the same level as $(A_1, \vv{u}_1)$.
\end{lemma}

\begin{proof}
   Choose $\beta = \beta(A)$ so that the conclusions of \Cref{arithmetic uniform value and image: T,cor: upper bound for higher mus} \pedro{Maybe more?} hold for each one of the finitely many monomial pairs in $\widehat{\graph}(A, \vv{u},p)$ whenever $p > \beta$, and fix such $p$.
    By virtue of \Cref{arithmetic uniform value and image: T} and the construction of $\widehat{\graph}(A,\vv{u},p)$, the assumption that $(A_1, \vv{u}_1)$ and $(B, \vv{v})$ lie on the same level implies that $\mu(B,\vv{v},p) = \mu(A_1, \vv{u}_1, p)$, proving the result for $e = 1$.

    Suppose that $e \geq 2$ and the result holds for paths of length $e-2$.
    \Cref{cor: upper bound for higher mus} tells us that
    \begin{align*}
      \mu(B,\vv{v},p^e) &\le \mu(B,\vv{v},p) p^{e-1} + \max_{\sproutsfrom{(C,\vv{z})}{(B,\vv{v})}} \ \mu(C,\vv{z},p^{e-1}) \\
      &= \mu(A_1,\vv{u}_1,p) p^{e-1} + \max_{\sproutsfrom{(C,\vv{z})}{(B,\vv{v})}} \ \mu(C,\vv{z},p^{e-1}),
    \end{align*}
    and to complete our inductive step, it suffices to show that
    \begin{equation}\label{ineq}
        \mu(C,\vv{z},p^{e-1}) \le \mu(A_2,\vv{u}_1,p) p^{e-2} + \cdots + \mu(A_e,\vv{u}_e,p)
    \end{equation}
    for each $(C,\vv{z})$ sprouting from $(B,\vv{v})$.

    Towards this, first note that if $(C, \vv{z})$ does not lie in $\widehat{\graph}(A,\vv{u},p)$, then \Cref{cor: mu comparison} implies that $\mu(C, \vv{z},p^{e-1}) < \mu(A_2,\vv{u}_2, p^{e-1})$, and the induction hypothesis applied to $(A_2, \vv{u}_2) \to \cdots \to (A_e, \vv{u}_e)$ and $(A_2,\vv{u}_2)$ itself gives \eqref{ineq}.
    On the other hand, if $(C, \vv{z})$ lies in $\widehat{\graph}(A, \vv{u},p)$, then it lies on the same level as $(A_2, \vv{u}_2)$, and  our induction hypothesis applied to the path $(A_2, \vv{u}_2) \to \cdots \to (A_e, \vv{u}_e)$ and the point $(C, \vv{z})$ once again gives us \eqref{ineq}, completing the proof.
\end{proof}

\begin{theorem}\label{thm: formula for higher mu}
   Given a monomial matrix $A$, there exists an integer $\beta= \beta(A)$ for which the following holds\textup:
   For each $p>\beta$ and $e\ge 1$, if $(A, \vv{u})$ is a very small monomial pair and $(A_1, \vv{u}_1) \to \cdots \to (A_e, \vv{u}_e)$ is a path in $\widehat{\graph}(A, \vv{u}, p)$,  then
   \[
      \mu(A_1, \vv{u}_1, p^e) = \mu(A_1,\vv{u}_1,p)p^{e-1} + \cdots + \mu(A_{e-1},\vv{u}_{e-1},p)p + \mu(A_e,\vv{u}_e,p).
   \]
   If, in addition, that path is terminal \textup(that is, $(A_e,\vv{u}_e)$ is not very small\textup), then
      \[
 \mu(A_1, \vv{u}_1, p^{e+s}) = \mu(A_1, \vv{u}_1, p) p^{e+s-1} + \cdots + \mu(A_{e-1}, \vv{u}_{e-1}, p) p^{s+1} + p^{s+1} - 1
\]
for every nonnegative integer $s$.
\end{theorem}

\begin{proof}
   The first identity follows from \Cref{cor: iterated lifting}(1) and \Cref{lem: upper bound for higher mu}, while the second follows from \Cref{cor: iterated lifting}(2) and the first, together with the fact that $\mu(A_1, \vv{u}_1, p^{e+s}) \le \mu(A_1, \vv{u}_1, p^{e}) p^s+p^s-1$, which follows from the second inequality in \Cref{natural bounds: C}.
\end{proof}

\alert[inline]{
\begin{corollary}\label{cor: constant mus}
   If $(A,\vv{u})$ is a small monomial pair and  $\beta = \beta(A)$ is as in \Cref{lem: upper bound for higher mu}, then $\crit$ and $\mu$ are constant on each level of $\widehat{\graph}(A,\vv{u},p)$ for every $p\ge \beta$.
\end{corollary}

\pedro[inline]{
   I'm not a believer anymore.
   If $(A_1,\vv{u}_1)$ and $(B,\vv{v})$ are as in \Cref{lem: upper bound for higher mu}, we only get the inequality $\mu(B,\vv{v},p^e) \le \mu(A_1,\vv{u}_1,p^e)$, and to conclude that we have equality we'd need to know that there is a path of length $e-1$ starting at $(B,\vv{v})$.
   But we don't know if we have this kind of uniformity in the graph.
}

\daniel[inline]{I think this can be remedied, but we will need to define the graph in a different way, to allow for paths that terminate and infinite paths at the same time.  The point is that at some fixed level $e$, a vertex should sprout if and only if that vertex is very small.  This sounds more like what you were suggesting in Lawrence.}
}

\comment[inline]{The point of this Lemma is to show positivity of the coefficients in the polynomials that define the $\mu$'s. A consequence is that if $\ideala$ is $\idealm$-primary and homogeneous, then all coefficients of every intermediate power of $p$ in this polynomials vanishes.}

\begin{lemma}  If  $p>0$ is prime and $(B, \vv{v})$ is a $p$-sprout of  $(A, \vv{u})$, then \[ \delta(A, \vv{u}, p) \geq \ft{B}{\vv{v}}\]
with equality if $A$ is the monomial matrix associated with a monomial ideal that is homogeneous with respect to some positive $\ZZ$-grading, and primary to the ambient homogeneous maximal ideal.
\end{lemma}

\begin{proof}
By definition of $p$-sprout,  $B$ is the collapse of $A$ along the face $\O = \mf(A, \vv{u})$ of the Newton polyhedron $\N$ of $A$.  Suppose that $A$ has $\numvars$ rows, and let an overbar denote collapse along $\O$.

If $\defpt \in \RR^\numvars$ defines the face $\O$ in $\N$, then \Cref{collapse of Newton polyhedron: P}  states that $\collapse{\defpt}$ defines the standard face $\collapse{\O}$ of $\collapse{\N}$, the Newton polyhedron of $B$, and since $\vv{v} \in \ft{B}{\vv{v}} \collapse{\N}$, \Cref{alpha=1: convention} implies that
\[\ft{B}{\vv{v}} \leq \iprod{\collapse{\defpt}}{\vv{v}}.\]
Thus, it suffices to show that $\iprod{\collapse{\defpt}}{\vv{v}} \leq \delta(A, \vv{u}, p)$.  However, by definition of $p$-sprout, $\vv{v} \in \Delta(A, \vv{u}, p)$, and so fixing a point $\vv{s} \in \sp_{\QQ}(A, \vv{u})$, we may write $ \vv{v} = B \tail{\vv{s}}_p - B \vv{h}$ for some $\vv{h}$  optimal for $\ip = \ip(A, \vv{u}, \vv{s}, q)$.  Our choice of $\defpt$ guarantees that the inner product of $\collapse{\defpt}$ with each column of $B$ is at least one, with equality whenever that column lies on $\collapse{\O}$.  Arguing as in the proof of \Cref{bounded value: L}, one may show that $\iprod{\collapse{\defpt}}{B \tail{\vv{s}}_p} = \norm{\tail{\vv{s}}_p}$ and \[ \iprod{\collapse{\defpt}}{ B \vv{h}} \geq \norm{\vv{h}} = \val \ip \]
which allows us to conclude that \[ \iprod{\collapse{\defpt}}{\vv{v}} \leq \norm{\tail{\vv{s}}_p} - \val \ip= \delta(A, \vv{u}, p).\]

We now address the last assertion:  If $A$ satisfies these additional conditions, then homogeneity implies that the convex hull of the columns of $A$ is a proper face $\O$ of $\N$.  The $\mathfrak{m}$-primary assumption further implies that $\O$ is a facet, and that $\mf(A, \vv{z}) = \O$ for every $\vv{z} \in \ZZ_+^\numvars$.  Furthermore, the positivity of the grading implies that the point $\defpt \in \RR^\numvars$ defining this face must have positive coordinates, and so $\O$ must be bounded.  In this case, collapsing along this face is simply the identity map on $\RR^\numvars$, and so in particular, $B=A$.  Given this, one may retrace the steps above to see that every inequality involving the inner product of $\defpt = \collapse{\defpt}$ with another point must be, in fact, an equality.  The details are left to the reader.
\end{proof}


\todo[inline]{Point out that the levels of $\widehat{\graph}(A, \vv{u})$ are eventually periodic. This will give an independent proof the rationality of critical exponents.  The way we present a formula for critical exponents may need this observation.}


\emily[inline]{It seems like we prove that $\mu(A, \vv{u}, q) = \mu(\collapse{A}, \collapse{\vv{u}}, q)$.  Let's make sure to state that later\daniel[inline]{I guess this will be an analog of \Cref{leftover p large: R}}}

\newpage

\appendix

\section{Convex geometry}

A (convex) \emph{polyhedron} in $\RR^n$ is a subset of $\RR^n$ obtained by intersecting finitely many closed halfspaces or, equivalently, a set consisting of all points $\vv{x}\in \RR^n$ satisfying an inequality of the form $A\vv{x}\le \vv{b}$, where $A$ is a matrix with $n$ columns.
The (convex) \emph{cone generated by $\vv{u}_1,\ldots,\vv{u}_k \in \RR^n$}, denoted $\cone(\vv{u}_1,\ldots,\vv{u}_k)$, is the set consisting of all \emph{conical combinations} of $\vv{u}_1, \ldots, \vv{u}_k$, that is, points of the form $\sum_{i=1}^k \lambda_i \vv{u}_i$, where the $\lambda_i$ are nonnegative real numbers.
Likewise, the \emph{convex hull of $\vv{u}_1,\ldots,\vv{u}_k$}, denoted $\conv(\vv{u}_1,\ldots,\vv{u}_k)$, is the set of all \emph{convex combinations} of $\vv{u}_1, \ldots, \vv{u}_k$, that is, points of the form $\sum_{i=1}^k \lambda_i \vv{u}_i$, where the $\lambda_i$ are nonnegative and $\sum_{i=1}^k \lambda_i = 1$.
The convex hull of a finite set of points is called a \emph{polytope}.

If $\mathcal{U}$ and $\mathcal{V}$ are subsets of $\RR^n$, their \emph{Minkowski sum} is the set
\[\mathcal{U}+\mathcal{V} \coloneqq \{\vv{u}+\vv{v}: \vv{u}\in \mathcal{U}\text{ and }\vv{v}\in \mathcal{V}\}.\]
The \emph{Minkowski--Weyl Theorem} asserts that a subset $\mathcal{P}$ of $\RR^n$ is a polyhedron if and only if $\mathcal{P}$ is the Minkowski sum of a polytope and a finitely generated cone.
The cone in this decomposition is the set of all directions $\vv{d} \in \RR^n$ in which $\mathcal{P}$ recedes, that is, $\vv{c} + \lambda \vv{d} \in \mathcal{P}$ for every $\vv{c} \in \mathcal{P}$ and $\lambda > 0$; it is uniquely determined by $\mathcal{P}$, and called the \emph{recession cone of $\mathcal{P}$}.
The Minkowski--Weyl Theorem gives us a couple of useful characterizations of polytopes: a polyhedron $\mathcal{P}$ is a polytope if and only if it is a bounded polyhedron or, equivalently, a polyhedron with a trivial recession cone.

%\pedro[inline]{
%   Maybe we should gather what we need about faces and vertices of polyhedra right here.
%}
%\daniel[inline]{I'm not so sure about this.  At the moment, I feel like it is less distracting to just remind the reader of something (beyond the absolute basic definitions already covered) at the time they are used.  But, I could be convinced otherwise}
%\pedro[inline]{
%   Yes, maybe it's more efficient to introduce what we need ``on demand'', so the reader does not need to be coming back to this section all the time.
%   That said, I think \emph{some} definition (perhaps the most generic definition) of face and vertex should be given here, for completeness (but maybe not every fact or every characterization we need).
%}
The \emph{relative interior} of a subset $\mathcal{U}$ of $\RR^n$, denoted $\ri \mathcal{U}$, is its interior relative to the smallest affine subset of $\RR^n$ containing $\mathcal{U}$.
% \pedro{
%    Is there a more concrete characterization for polyhedra/polytopes?
%    E.g., points not in any proper face? Positive convex combinations of vertices?
% }
% \daniel{Yep!  Points not in any proper face.  If $S$ is a finite set with $\mathcal{P} = \conv(S)$, then $\ri \mathcal{P}$  consists of all points of the form $\sum_{\vv{s} \in S} \lambda_{\vv{s}} \vv{s}$ where the coefficients $\lambda_{\vv{s}}$ are positive, and sum to $1$.  So in particular, you could take $S$ to be the vertex set of $\mathcal{P}$.  Similarly, if $S$ is finite and $\mathcal{P} = \cone(S)$, then $\ri \mathcal{P}$ has a similar description, but we don't require that the coefficients sum to $1$. But, as I mentioned above, I'm not sure whether it is better to gather things here, or just mention them as we go along.}
When restricted to convex sets, the relative interior operator commutes with Minkowski sums: if $\mathcal{U}$ and $\mathcal{V}$ are convex subsets of $\RR^n$, then $\ri(\mathcal{U}+\mathcal{V})=\ri \mathcal{U}+\ri \mathcal{V}$.

\begin{proposition}
   \label{bounded polytope: P}
   Let $\vv{c}$ and $\vv{u}$ be points in $\RR^n$, and suppose that $\vv{c}$ has positive coordinates.
   If $\alpha$ is any real number, then the polyhedron consisting of all points $\vv{v} \in \RR^n$ such that  $\vv{v} \le \vv{u}$ and $\iprod{\vv{c}}{\vv{v}} \geq \alpha$ is bounded.
\end{proposition}

\begin{proof}
   It suffices to show that the given set is bounded from below.
   For each $\vv{v}$ in that set and each $i$ we have $\vv{v}\le \vv{u} + (v_i - u_i)\canvec_i$.
   As $\vv{c}$ has positive coordinates, $\alpha\le \iprod{\vv{c}}{\vv{v}}\le \iprod{\vv{c}}{\vv{u} + (v_i -u_i)\canvec_i} =
  \iprod{\vv{c}}{\vv{u}} + c_i(v_i - u_i)$, so $v_i \ge (\alpha + c_iu_i - \iprod{\vv{c}}{\vv{u}})/c_i$.
\end{proof}

%We conclude this subsection with a useful technical result.
Though variations of the following useful technical result are well known, we include a simple proof, for lack of an appropriate reference.

\begin{proposition}
\label{vertex: P}
Let $M$ be an $m \times n$ matrix and let $\vv{b} \in \RR^m$ be a point contained in the cone generated by the columns of $M$.  If $\Q$ is the polyhedron in $\RR^n$  consisting of all points $\vv{t}$ with $\vv{t} \geq \vv{0}$ and $M \vv{t} = \vv{b}$, then a point $\vv{t}^{\ast} \in \Q$ is a vertex of $\Q$ if and only if the columns of $M$ corresponding to the nonzero coordinates of $\vv{t}^{\ast}$ are linearly independent.  %In particular, $\Q$ contains a vertex.
\end{proposition}

\begin{proof}
   The fact that $\vv{b}$ lies in the cone generated by the columns of $M$ implies that $\Q$ is nonempty.
   Fix a point $\vv{t}^{\ast} \in \Q$.
   Before proceeding, recall that $\vv{t}^{\ast}$ is a vertex of $\Q$ if and only if an expression of $\vv{t}^{\ast}$ as a convex combination of points $\vv{r}$ and $\vv{s}$ in $\Q$ is only possible when $\vv{r}=\vv{s}=\vv{t}^{\ast}$.

   First, assume that the columns of $M$ corresponding to the nonzero coordinates of $\vv{t}^{\ast}$ are linearly independent, and suppose that $\vv{t}^{\ast} = \lambda \vv{r} + \mu \vv{s}$ is a convex combination of points $\vv{r}, \vv{s} \in \Q$.
   Since $\vv{r},\vv{s}\ge \vv{0}$, the $i$-th coordinate of $\vv{r}$ and of $\vv{s}$ are zero whenever the $i$-th coordinate of $\vv{t}^{\ast}$ is zero.
   On the other hand, the fact that $\vv{r}$ and $\vv{s}$ lie in $\Q$ also implies that
   \[ M \vv{t}^{\ast} = \vv{b} = M \vv{r} = M \vv{s}, \]
   and the assumption that the columns of $M$ corresponding to the nonzero coordinates of $\vv{t}^{\ast}$ are linearly independent then implies that $\vv{r}=\vv{s}=\vv{t}^{\ast}$.

Next, suppose that the columns of $M$ corresponding to the nonzero coordinates of $\vv{t}^{\ast}$ are linearly dependent.   In this case, we may fix a nonzero point $\vv{k} \in \RR^n$ with the property that $M \vv{k} = \vv{0}$, and such that the $i$-th coordinate of $\vv{k}$ is zero whenever the $i$-th coordinate of $\vv{t}^{\ast}$ is zero.  We claim that if $\varepsilon > 0$ is sufficiently small, then the points $\vv{t}^{\ast} \pm \varepsilon \vv{k}$ must lie in $\Q$.   As $\vv{t}^{\ast}$ is a convex combination of these points, it will then follow that $\vv{t}^{\ast}$ is not a vertex of $\Q$.  Towards the claim, note that $M(\vv{t}^{\ast} \pm \varepsilon \vv{k}) = M \vv{t}^{\ast} = \vv{b}$ for every $\varepsilon > 0$.  On the other hand, the condition relating the coordinates of $\vv{t}^{\ast}$ and $\vv{k}$ guarantees that $\vv{t}^{\ast} \pm \varepsilon \vv{k}$ is nonnegative for all $0 < \varepsilon \ll 1$.
\end{proof}

\pedro[inline]{
   An alternative to the previous proposition is the following result, which we could just mention and give a reference (it appears in several books):

   \begin{proposition}
      Let $\mathcal{P}$ be the polyhedron defined by a system of inequalities $A \vv{x} \le \vv{b}$, where $A\in \RR^{m\times n}$, and $\vv{v}$ a vertex of $\mathcal{P}$.
      Then there exists $I \subseteq \{1,\ldots,m\}$ such that $\vv{v}$ is the unique solution to the system $A_I \vv{x} = \vv{b}_I$, where $A_I$ and $\vv{b}_I$ are obtained by selecting the $i$-th rows of $A$ and $\vv{b}$, for each $i\in I$.
   \end{proposition}

   Then \Cref{uniform denominators for vertices:  T} can be approached as follows:
   By \Cref{opt set: P}, $\opt \LP(A,\vv{u})$ is defined by $A\vv{s} \le \vv{u}$ and $\vv{s}\ge \vv{0}$, with equality in some specific coordinates, and thus defined by a system of inequalities $B \vv{x} \le \vv{b}$, where $\vv{b}$ is an integral vector and $B$ is a submatrix of the matrix $M$ obtained by stacking $A$, $-A$, the identity matrix $I_n$, and $-I_n$.
   Let $\denom$ be the least common multiple of the nonzero minors of $M$; then by the above result, every vertex of $\opt \LP$ is rational, with denominator $\denom$.

   \bigskip

   Hope this makes sense; if so, then I think this argument is slightly simpler, avoiding the linear bijection business.
}


\section{Monomial ideals and reduction}
\label{monomial-reduction: A}

% \daniel[inline]{
% \begin{itemize}
% \item This appendix needs an introduction.
% \item We should acknowledge the Budur-Mustata-Saito paper.  I still think it is worth presenting some details, given that the original arguments are somewhat hard to read, and that we establish a more general result.
% \end{itemize}
% }

The purpose of this appendix is to describe the behavior of the monomials contained in some possibly non-monomial ideal under reduction to prime characteristic.  To make this precise, consider the following definition. 

\begin{definition}
If $\kk$ is a field and $I$ is an ideal of $\kk[x_1, \ldots, x_\numvars]$, then $\mon(I)$ is the ideal generated by the monomials in $I$. % That is, \[ \mon(I) = \langle x^{\vv{u}}: x^{\vv{u}} \in I \rangle \subseteq \kk[x_1, \ldots, x_\numvars].\]
\end{definition}

The following theorem, a generalization of \cite[Lemma 6.1]{budur+mustata+saito.roots_bs_polys_monomial}, is the main result of \Cref{monomial-reduction: A}.


\begin{theorem}
\label{mon-operation-modulo-p: T}
Given an ideal $I \subseteq \QQ[x_1, \cdots, x_\numvars]$, there exists an integer $\beta = \beta(I)$ with the following property:  If $p > \beta$, then $\mon(I)_p = \mon(I_p)$.
\end{theorem}

The subscript ``$p$'' indications reduction modulo $p$, which will be described in the first subsection. 
Our methods in proving \Cref{mon-operation-modulo-p: T} are adapted from those in \emph{loc.\,cit}.
Applications of the theorem related to critical exponents and $F$-thresholds 
appear in the final subsection of this appendix. 

\subsection{Reduction to prime characteristic}

\begin{remark}[Generic freeness] \label{generic-freeness} If $A$ is a finitely generated $\ZZ$-algebra, and $M$ is a finitely-generated $A$-module, then there exists a nonzero integer $\ell$ such that $M \otimes_{\ZZ} \ZZ[\ell^{-1}]$, the localization of $M$ at $\ell$, is free over $\ZZ[\ell^{-1}]$.
\end{remark}

\emily[inline]{Add description of reduction modulo $p$ here.}

\begin{lemma}
\label{noncontainment mod p: L}
Suppose that $I$ and $J$ are ideals of a finitely generated $\ZZ$-algebra $A$ with $I \subseteq J$.  If $IA_{\QQ} \neq JA_{\QQ}$, then $IA_p \neq JA_p$ for all $p \gg 0$.
\end{lemma}

\begin{proof}
Let $\ell$ be a nonzero integer, and $B = \ZZ[\ell^{-1}] \otimes_{\ZZ} A$ be the localization of $A$ at $\ell$.  The assumption that $IA_{\QQ} \neq JA_{\QQ}$ implies that
%
\begin{equation}
\label{localized-quotient: e}
\tag{$\heartsuit$}
0 \neq JB/IB \cong B \otimes_A (J/I) \cong \ZZ[\ell^{-1}] \otimes_{\ZZ} (J/I) .
\end{equation}

Thus, by Generic Freeness (\Cref{generic-freeness}), we can assume that each term $M$ in \eqref{localized-quotient: e} is nonzero and free over $\ZZ[\ell^{-1}]$.  However,  if $p \nmid \ell$, then $\FF_p$ is an algebra over $\ZZ[\ell^{-1}]$, and so $\FF_p \otimes_{\ZZ[\ell^{-1}]} M \cong J_p / I_p$ is nonzero and free over $\FF_p$.
\end{proof}

\begin{lemma}
\label{colon mod p: L}
 If $I$ and $J$ are ideals of a finitely generated $\ZZ$-algebra $A$, then $(J:_A I)A _p = (JA_p :_{A_p} IA_p)$ for all $p \gg 0$.
\end{lemma}

\begin{proof}
Consider the exact sequence of $A$-modules
\[ 0  \longrightarrow (I:_A f) \longrightarrow A \stackrel{f}{\longrightarrow} A/I \longrightarrow 0.  \]

Given that this is also an exact sequence of $\ZZ$-modules,  we may localize at a nonzero integer $\ell$ to obtain an exact sequence of $\ZZ[\ell^{-1}]$ modules
\[ 0  \longrightarrow (I:_A f)B \longrightarrow B \stackrel{f}{\longrightarrow} B/IB \longrightarrow 0  \] where $B = \ZZ[\ell^{-1}] \otimes_{\ZZ} A$ is the localization of $A$ at $\ell$.  By Generic Freeness (\Cref{generic-freeness}), we further suppose that each module in this sequence is free over  $\ZZ[\ell^{-1}]$.

If $p$ does not divide $\ell$, then $\ZZ/p\ZZ$ is an algebra over $\ZZ[\ell^{-1}]$, and we may take the tensor product of this exact sequence with $\ZZ/p\ZZ$ over $\ZZ[\ell^{-1}]$ to obtain the exact sequence of free $\ZZ/p\ZZ$-modules
\[ 0  \longrightarrow (I:_A f)A_p \longrightarrow A_p \stackrel{f_p}{\longrightarrow} A_p/IA_p \longrightarrow 0  \] where $A_p = A \otimes_{\ZZ} \ZZ/p\ZZ$.  Clearly, this is also an exact sequence of $A_p$-modules, from which it follows that $ (I:_A f) A_p = (IA_p :_{A_p} f_p)$.
\end{proof}

\subsection{Monomial ideals}


By definition, both $\mon(I)_p$ and $\mon(I_p)$ are monomial ideals, and so they are equal if and only if they contain the same monomials.  However, if $h$ is a monomial contained in $I$, then it is clear that $h_p$ is a monomial contained in $I_p$ for all $p$.  Thus, to prove \Cref{mon-operation-modulo-p: T} it suffices to prove the following statement:  If $p \gg 0$ and $h$ is a monomial with $h \notin I$, then $h_p \notin I_p$.

Given any $h$ not contained in $I$, then setting $J = I + \langle h \rangle$ in \Cref{noncontainment mod p: L} tells us that $h_p \notin I_p$ for all primes $p$ that are large relative to $h$.  This argument establishes \Cref{noncontainment mod p: L} in the case that $I$ contains all but finitely many monomials;  that is, in the case that $ \mon(I)$ is $\idealm$-primary.

If $\mon(\idealb)$ is not $\idealm$-primary, then this argument breaks down.  In this case, our strategy will rely on the fact that, even though there may be infinitely many monomials not in $I$, there are still only finitely many ideals of the form $(I:x^{\vv{u}})$ with $x^{\vv{u}} \notin I$.    This is established in \Cref{monomial-noetherian-decomposition: L} below.  We begin our arguments by recalling a basic and useful fact about colon ideals.




\begin{lemma}
\label{colon-product-stabilization: L}
  Suppose that $f,g_1, \cdots, g_s$ are elements of a ring $R$, and that $I$ is an ideal of $R$.
If $m$ is a nonnegative integer such that $(I: f g_i^m) = (I: f g_i^k)$ for every index $1 \leq i \leq s$ and every integer $k \geq m$, then \[ (I: f g_1^m \cdots g_s^m) = (I: f g_1^{k_1} \cdots g_s^{k_s})\] for all integers $k_1, \cdots, k_s \geq m$.
\end{lemma}

\begin{proof}
We induce on $k_1 + \cdots + k_s$.  The base case is when $k_1 = \cdots = k_s = m$, which is trivial, and the induction step, which is left to the reader, involves repeatedly applying the fact that $(I:fab) = ((I:fa):b)$ for all $a,b \in R$.
\end{proof}

\begin{lemma}
\label{monomial-noetherian-decomposition: L}
Given a proper ideal $I$ of $\kk[x_1, \ldots, x_\numvars]$, there exists a finite subset $\mathcal{V}$ of $\NN^\numvars$, and for every $\vv{v} \in \mathcal{V}$ a finite subset $\mathcal{W}(\vv{v})$ of the set of standard basis vectors of $\ZZ^\numvars$,  satisfying the following conditions.
\begin{enumerate}
\item $x^{\vv{u}} \notin I$ if and only if $\vv{u} \in \mathbf{v} + \NN \mathcal{W}(\vv{v})$ for some $\vv{v} \in \mathcal{V}$.
\item $(I:x^{\vv{v}}) = (I: x^{\vv{v}+\vv{w}})$ for every $\vv{v} \in \mathcal{V}$ and $\vv{w} \in  \NN  \mathcal{W}(\vv{v})$.
\end{enumerate}
\end{lemma}

\begin{proof}  By definition, a monomial lies in $I$ if and only if it lies in $\mon(I)$.  However, $\mon(I)$ is the intersection of ideals of the form $\langle x_i^{b_i} : i \in \Omega \rangle$, where $\Omega$ is some nonempty subset of $\{ 1, \ldots, \numvars \}$, and each exponent $b_i$ is positive.   It follows that a monomial is not in $I$ if and only if it is not in one of these components of $\mon(I)$.  Given this, it suffices to establish the following.

\vspace{.2cm}

\noindent \emph{Claim.} Given positive integers  $b_1, \cdots, b_{\ell}$ with $1 \leq \ell \leq \numvars$, there exists a finite subset $\mathcal{V}$ of $\NN^\numvars$, and for every $\vv{v} \in \mathcal{V}$ a finite subset $\mathcal{W}(\vv{v})$ of the set of standard basis vectors of $\ZZ^\numvars$,  satisfying the following conditions.
\begin{enumerate}
\setcounter{enumi}{2}
\item \label{stab-1: e} $x^{\vv{u}} \notin \langle x_1^{b_1}, \ldots, x_{\ell}^{b_{\ell}} \rangle $ if and only if $\vv{u} \in \vv{v} + \NN \mathcal{W}(\vv{v})$ for some $\vv{v} \in \mathcal{V}$.
\item \label{stab-2: e} $(I:x^{\vv{v}}) = (I: x^{\vv{v}+\vv{w}})$ for every $\vv{v} \in \mathcal{V}$ and $\vv{w} \in  \NN  \mathcal{W}(\vv{v})$.
\end{enumerate}

\vspace{.15cm}


We now prove the claim.  Let $\mathcal{A}$ be the finite set of all points $\vv{a} \in \NN^\numvars$ such that $0 \leq a_i < b_i$ for every $1 \leq i \leq \ell$ and $a_i = 0$ whenever $\ell < i \leq \numvars$.  If $\ell = \numvars$, then it is clear that we may take $\mathcal{V} = \mathcal{A}$ and $\mathcal{W}(\vv{v}) = \emptyset$ for every $\vv{v} \in \mathcal{V}$.

We now assume that $1 \leq \ell < \numvars$.
For every $\vv{a} \in \mathcal{A}$ and $\ell < i \leq \numvars$, the ideals $(I: x^{\vv{a}} x_i^k)$ are increasing in $k$.  Hence, there exists an integer $m \geq 0$ such that
\begin{equation}
\label{colon-simple: e}
(I: x^{\vv{a}}x_i^m) = (I: x^{\vv{a}}x_i^k)
\end{equation}
for every point $\vv{a} \in \mathcal{A}$, every index $\ell < i \leq \numvars$, and every integer $k \geq m$.


Let $\mathcal{V}$ be the set consisting of all points of the form $\vv{v} = \vv{a} + \sum_{\ell <i \leq d} m_i \vv{e}_i$ where $\vv{a} \in \mathcal{A}$, and each integer $m_i$ satisfies $0 \leq m_i \leq m$.  For every such $\vv{v}$, set $\mathcal{W}(\vv{v}) = \emptyset$ if some $m_i$ is not $m$, and set $\mathcal{W}(\vv{v}) = \{ \vv{e}_i : \ell < i \leq \numvars \}$ otherwise.

%\[ \mathcal{W}(\vv{v}) = \begin{cases}
%\emptyset & \text{ if $m_i \neq m$ for some $\ell < i \leq d$} \\
%\{ \vv{e}_i : \ell < i \leq d \} & \text{ if $m_i = m$ for every $\ell < i \leq d$}.
%\end{cases} \]

After unraveling these definitions, it is obvious that \eqref{stab-1: e} is satisfied.  Furthermore, the only case in which \eqref{stab-2: e} is nontrivial is when $\vv{v} = \vv{a} + \sum_{\ell <i \leq d} m \vv{e}_i$ for some $\vv{a} \in \mathcal{A}$,  in which case we must prove that
%
\[( I : x^{\vv{a}} x_{\ell+1}^m \cdots x_{d}^m ) =   ( I : x^{\vv{a}} x_{\ell+1}^{m+w_{\ell+1}} \cdots x_{d}^{m+w_\numvars}) \]
for all nonnegative integers $w_{\ell+1}, \ldots, w_{d}$.  However, this is an immediate consequence of \eqref{colon-simple: e} and \Cref{colon-product-stabilization: L}.
\end{proof}

With \Cref{colon-product-stabilization: L,monomial-noetherian-decomposition: L} in hand, we now prove \Cref{mon-operation-modulo-p: T}.

\begin{proof}[Proof of \Cref{mon-operation-modulo-p: T}] Let $\mathcal{V}$ and $\mathcal{W}(\vv{v})$ be as in \Cref{monomial-noetherian-decomposition: L}, so that \[ \langle 1 \rangle \neq (I : x^{\vv{v}}) = (I : x^{\vv{v}+\vv{e}_i}) \text{ for all } \vv{v} \in \mathcal{V} \text{ and } \vv{e}_i \in \NN \mathcal{W}(\vv{v}).\]
The finiteness of $\mathcal{V}$ and \Cref{noncontainment mod p: L,colon mod p: L} then imply that if $p \gg 0$, then
\[ \langle 1 \rangle \neq (I_p : x^{\vv{v}}) = (I_p : x^{\vv{v}+\vv{e}_i}) \text{ for all } \vv{v} \in \mathcal{V} \text{ and } \vv{e}_i \in \NN \mathcal{W}(\vv{v}).\]
Given this, \Cref{colon-product-stabilization: L} then implies that if $p \gg 0$, then
\begin{equation}
\label{canonical-set-reduces: e}
\langle 1 \rangle \neq (I_p : x^{\vv{v}}) = (I_p : x^{\vv{v}+\vv{w}}) \text{ for all } \vv{v} \in \mathcal{V} \text{ and } \vv{w} \in \NN \mathcal{W}(\vv{v}).
\end{equation}

For the remainder of the proof, suppose that $p \gg 0$ so that \eqref{canonical-set-reduces: e} holds.  Finally, consider a monomial $x^{\vv{u}} \notin I$.  As discussed immediately after the statement of \Cref{mon-operation-modulo-p: T}, it suffices to prove that $x^{\vv{u}} \notin I_p$.  However,  \Cref{monomial-noetherian-decomposition: L} the fact that $x^{\vv{u}} \notin I$ imply that $\vv{u} = \vv{v} + \vv{w}$ for some $\vv{v} \in \mathcal{V}$ and $\vv{w} \in \mathcal{W}(\vv{v})$, and \eqref{canonical-set-reduces: e} then tells us that $(I_p:x^{\vv{u}}) \neq \langle 1 \rangle$, so that $x^{\vv{u}} \notin I_p$.
\end{proof}


\subsection{Some applications}


\begin{lemma}
Suppose that $\operatorname{char} \kk = p > 0$.  If $\idealb$ is an ideal of $\kk[x_1, \ldots, x_\numvars]$, then $\mon(\idealb^{[p^e]}) = \mon(\idealb)^{[p^e]}$ for all nonnegative integers $e$.
\end{lemma}


% If $x^{\vv{u}} \in R$,  $\idealb$ is an ideal of $R$, and $e$ is a natural number, then $x^{\vv{u}} \in \idealb^{[p^e]}$ if and only if $x^{\vv{u}} \in \mon(\idealb)^{[p^e]}$.


\begin{proof}  Set $\idealc = \mon(\idealb)$ and $q=p^e$.  Given that both ideals in question are monomial, it suffices to prove that a monomial lies in $\mon(\idealb^{[q]})$ if and only if it lies in $\idealc^{[q]}$.  However, as $\idealc^{[q]} \subseteq \idealb^{[q]}$, it follows that
$\idealc^{[q]} = \mon(\idealc^{[q]}) \subseteq \mon(\idealb^{[q]})$.

Thus, to conclude the proof, it suffices to prove that any monomial in $\idealb^{[q]}$ must lie in $\idealc^{[q]}$.  Towards this, consider a monomial $x^{\vv{u}} \in \idealb^{[q]}$, and write $\vv{u} = \vv{v}q + \vv{w}$ where $\vv{v}$ and $\vv{w}$ are points in $\NN^\numvars$, and $\vv{0} \leq \vv{w} < \vv{1}q$.

As $x^{\vv{u}} = x^{\vv{v}q} x^{\vv{w}} \in \idealb^{[q]}$, the fact that $R$ is free over $R^q$ implies that $x^{\vv{w}} \in ( \idealb^{[q]}: x^{\vv{v}q}) = (\idealb:x^{\vv{v}})^{[q]}$.  Applying the $q$-root operation to this inclusion then illustrates that $\langle 1 \rangle = \langle x^{\vv{w}} \rangle^{[1/q]}  \subseteq (\idealb: x^{\vv{v}})$,  where the first equality here follows from the bounds on $\vv{w}$ noted above.   Therefore, $x^{\vv{v}} \in \idealb$, which by definition of the ideal $\idealc$ implies that $x^{\vv{v}} \in \idealc$, and so $x^{\vv{v}} \in \langle x^{\vv{w}q} \rangle \subseteq \idealc^{[q]}$.
\end{proof}


\begin{corollary}
\label{reduce-to-monomial-case: C}
Suppose that $\operatorname{char} \kk = p > 0$, that $\ideala$ is a monomial ideal of $\kk[x_1, \ldots, x_\numvars]$, and that $\idealb$ is an ideal of this same ring with $\ideala \subseteq \sqrt{\idealb}$.  If $e$ is an arbitrary nonnegative integer and $\idealc = \mon(\idealb)$, then the following hold.
\begin{enumerate}
\item $\nu(\ideala, \idealb, p^e) = \nu(\ideala, \idealc, p^e)$ and $\ft{\ideala}{\idealb} = \ft{\ideala}{\idealc}$.
\item $\mu(\ideala, \idealb, p^e) = \mu(\ideala, \idealc, p^e)$ and $\crit(\ideala,\idealb) = \crit(\ideala,\idealc)$. \qed
\end{enumerate}
\end{corollary}

\begin{corollary}
Suppose that $\ideala$ is a monomial ideal of $\QQ[x_1, \ldots, x_\numvars]$, and that $\idealb$ is an ideal of this same ring with $\ideala \subseteq \sqrt{\idealb}$.  If $\idealc = \mon(\idealb)$, then there exists an integer $\beta = \beta(\ideala, \idealb)$ such that $\nu(\ideala_p, \idealb_p, p^e) = \nu(\ideala_p, \idealc_p, p^e)$ and $\mu(\ideala_p, \idealb_p, p^e) = \mu(\ideala_p, \idealc_p, p^e)$ for all primes $p > \beta$ and integers $e \geq 1$. \qed
\end{corollary}

\daniel[inline]{We should add that if $\ideala$ is monomial, then we can always assume that we are computing thresholds and $\nu/\mu$ with respect to some point with positive coordinates. }


\subsection{Reduction to prime characteristic}

\begin{remark}[Generic freeness]  If $A$ is a finitely generated $\ZZ$-algebra, and $M$ is a finitely-generated $A$-module, then there exists a nonzero integer $\ell$ such that $M \otimes_{\ZZ} \ZZ[\ell^{-1}]$, the localization of $M$ at $\ell$, is free over $\ZZ[\ell^{-1}]$.
\end{remark}

\begin{lemma}
\label{noncontainment mod p: L}
Suppose that $I$ and $J$ are ideals of a finitely generated $\ZZ$-algebra $A$ with $I \subseteq J$.  If $IA_{\QQ} \neq JA_{\QQ}$, then $IA_p \neq JA_p$ for all $p \gg 0$.
\end{lemma}

\begin{proof}
Let $\ell$ be a nonzero integer, and $B = \ZZ[\ell^{-1}] \otimes_{\ZZ} A$ be the localization of $A$ at $\ell$.  The assumption that $IA_{\QQ} \neq JA_{\QQ}$ implies that
%
\begin{equation}
\label{localized-quotient: e}
\tag{$\heartsuit$}
0 \neq JB/IB \cong B \otimes_A (J/I) \cong \ZZ[\ell^{-1}] \otimes_{\ZZ} (J/I) .
\end{equation}

Thus, by Generic Freeness, we may assume that each term $M$ in \eqref{localized-quotient: e} is nonzero and free over $\ZZ[\ell^{-1}]$.  However,  if $p \nmid \ell$, then $\FF_p$ is an algebra over $\ZZ[\ell^{-1}]$, and so $\FF_p \otimes_{\ZZ[\ell^{-1}]} M \cong J_p / I_p$ is nonzero and free over $\FF_p$.
\end{proof}

\begin{lemma}
\label{colon mod p: L}
 If $I$ and $J$ are ideals of a finitely generated $\ZZ$-algebra $A$, then $(J:_A I)A _p = (JA_p :_{A_p} IA_p)$ for all $p \gg 0$.
\end{lemma}

\begin{proof}
Consider the exact sequence of $A$-modules
\[ 0  \longrightarrow (I:_A f) \longrightarrow A \stackrel{f}{\longrightarrow} A/I \longrightarrow 0.  \]

Given that this is also an exact sequence of $\ZZ$-modules,  we may localize at a nonzero integer $\ell$ to obtain an exact sequence of $\ZZ[\ell^{-1}]$ modules
\[ 0  \longrightarrow (I:_A f)B \longrightarrow B \stackrel{f}{\longrightarrow} B/IB \longrightarrow 0  \] where $B = \ZZ[\ell^{-1}] \otimes_{\ZZ} A$ is the localization of $A$ at $\ell$.  By Generic Freeness, we further suppose that each module in this sequence is free over  $\ZZ[\ell^{-1}]$.

If $p$ does not divide $\ell$, then $\ZZ/p\ZZ$ is an algebra over $\ZZ[\ell^{-1}]$, and we may take the tensor product of this exact sequence with $\ZZ/p\ZZ$ over $\ZZ[\ell^{-1}]$ to obtain the exact sequence of free $\ZZ/p\ZZ$-modules
\[ 0  \longrightarrow (I:_A f)A_p \longrightarrow A_p \stackrel{f_p}{\longrightarrow} A_p/IA_p \longrightarrow 0  \] where $A_p = A \otimes_{\ZZ} \ZZ/p\ZZ$.  Clearly, this is also an exact sequence of $A_p$ modules, from which it follows that $ (I:_A f) A_p = (IA_p :_{A_p} f_p)$.
\end{proof}


\newpage
\section{Frobenius powers of monomial ideals}


{\color{green}

In this section, we recall the definition and basic properties of (generalized) Frobenius powers and critical exponents, as introduced in \cite{hernandez+etal.frobenius_powers}.
Let $R$ be a regular domain of characteristic $p > 0$, and let $\ideala$ be an ideal of $R$.
If $q$ is a power of $p$, then $\ideala^{[q]}$ denotes the standard $q$-th Frobenius power of $\ideala$, that is, the ideal generated by the $q$-th powers of the elements of $\ideala$.
Given a nonnegative integer $k$, with base $p$ expansion $k = k_0 + k_1 p + \cdots + k_r p^r$, the $k$-th Frobenius power of $\ideala$ is the ideal
\[\ideala^{[k]} \coloneqq \ideala^{k_0}\big(\ideala^{k_1}\big)^{[p]}\cdots \big(\ideala^{k_r}\big)^{[p^r]}.\]


More relevant to this article, though, is the description of $\ideala^{[k]}$ in terms of generators of $\ideala$:
Given $\vv{u} \in \NN^n$, we use
$\binom{k}{\vv{u}}$ to denote the binomial coefficient $\binom{k}{u_1,\ldots,u_n}$, which equals zero if $\norm{\vv{u}} \neq k$.
If $\ideala = \ideal{f_1,\ldots,f_n}$, then $\ideala^{[k]}$ is the ideal generated by the products $f_1^{u_1}\cdots f_n^{u_n}$, ranging over all $\vv{u} \in \NN^n$ for which $\binom{k}{\vv{u}}\not\equiv 0\bmod{p}$ \cite[Proposition~3.5]{hernandez+etal.frobenius_powers}.


Frobenius powers are extended to allow nonnegative real exponents, through the use of the Frobenius roots introduced in \cite{blickle+mustata+smith.discr_rat_FPTs}.
Explicitly, for a nonnegative rational exponent of the form $k/p^e$, we define
\[\ideala^{[k/p^e]} \coloneqq \big(\ideala^{[k]}\big)^{[1/p^e]},\]
and for an arbitrary positive real number $t$, we define $\ideala^{[t]}$ by taking approximations of $t$ from above by such rational numbers, in a way analogous to the definition of test ideals in \loccit\
More explicitly, if $\left\{t_j\right\}_{j=1}^\infty$ is a sequence of real numbers limiting to $t$ from above, each of the form $k/p^e$ for some $k>0$ and $e \geq 0$, then $\ideala^{[t]}$ is defined the union of the ideals $\ideala^{[t_j]}$.

Like test ideals and multiplier ideals, as $t$ varies, the Frobenius powers $\ideala^{[t]}$ form a nonincreasing chain, and are right-constant for positive $t$, \ie $\ideala^{[t+\epsilon]} = \ideala^{[t]}$, for $0<\epsilon \ll 1$.
The positive exponents where $\ideala^{[t]}$ ``jumps'' (that is, $\ideala^{[t-\epsilon]}\ne \ideala^{[t]}$, for all $0<\epsilon \le t$) are called the \emph{critical exponents} of $\ideala$.
These are the analogues of the jumping numbers of multiplier ideals, and of the $F$-jumping exponents of test ideals, and like their counterparts, they form a discrete set of rational numbers \cite[Corollary~5.8]{hernandez+etal.frobenius_powers}.

If $\ideala$ and $\idealb$ are nonzero proper ideals of $R$, with $\ideala \subseteq \sqrt\idealb$, the \emph{critical exponent of $\ideala$ with respect to $\idealb$} is the number
\begin{equation}\label{eq: defn of crit(a,b)}
   \crit(\ideala,\idealb) \coloneqq \min\big\{t\in \RRnn: \ideala^{[t]} \subseteq \idealb\big\}
      = \sup\big\{t\in \RRnn: \ideala^{[t]} \not\subseteq \idealb\big\}.
\end{equation}
This is indeed a critical exponent of $\ideala$, and moreover, every critical exponent $\lambda$ of $\ideala$ is of this form, for some $\idealb$ (take, for instance, $\idealb = \ideala^{[\lambda]}$).

We now describe a more explicit realization of the critical exponents of an ideal, which is central to this paper.
With $\ideala$ and $\idealb$ as above, given a nonnegative integer $e$, we set
\[\mu(\ideala,\idealb,p^e) \coloneqq \max\big\{k\in \NN : \ideala^{[k]} \not\subseteq \idealb^{[p^e]}\big\}.\]
Then $\big(\mu(\ideala,\idealb,p^e)/p^e\big)_e$ is a nondecreasing bounded sequence, and
\begin{equation}\label{eq: crit as a limit of mus}
   \crit(\ideala,\idealb) = \lim_{e\to \infty} \frac{\mu(\ideala,\idealb,p^e)}{p^e} = \sup_{e\in \NN} \frac{\mu(\ideala,\idealb,p^e)}{p^e}.
\end{equation}
The $\mu(\ideala,\idealb,p^e)$ not only determine $\crit(\ideala,\idealb)$, but can also be recovered from $\crit(\ideala,\idealb)$, via truncations:
\begin{equation}\label{eq: recovering mus from crit}
   \mu(\ideala,\idealb,p^e) = \up{p^e\crit(\ideala,\idealb)} - 1.
\end{equation}

Before moving forward, we observe that the notions introduced in the last two paragraphs run parallel to the theory of $F$-thresholds.
With $\ideala$ and $\idealb$ as above, the \emph{$F$-threshold of $\ideala$ with respect to $\idealb$}, denoted $\ft{\ideala}{\idealb}$, is defined as in \eqref{eq: defn of crit(a,b)}, replacing the Frobenius power $\ideala^{[t]}$ with the test ideal $\tau(\ideala^t)$.
There is an explicit description for $\ft{\ideala}{\idealb}$ analogous to \eqref{eq: crit as a limit of mus}, where $\mu(\ideala,\idealb,p^e)$ is replaced with
\[\nu(\ideala,\idealb,p^e) \coloneqq \max\big\{k\in \NN : \ideala^{k} \not\subseteq \idealb^{[p^e]}\big\}.\]
However, there is no analogue to \eqref{eq: recovering mus from crit}, unless $\ideala$ is a principal ideal.


\subsection{Frobenius powers and critical exponents of monomial ideals}

We now introduce some notation and gather some basic results concerning Frobenius powers and critical exponents of monomial ideals.
We work in a polynomial ring over a field of characteristic $p>0$, in the variables $x=x_1,\ldots,x_\numvars$.

\begin{notation}
   % We adopt standard multi-index notation: if $\vv{u} = (u_1,\ldots,u_\numvars)\in \NN^\numvars$, then $x^\vv{u} = x_1^{u_1}\cdots x_\numvars^{u_\numvars}$.
   If $\vv{u}\in \NN^\numvars$, then $\diag(\vv{u})$ denotes the \emph{diagonal ideal}
   \[ \diag(\vv{u}) = \ideal{x_1^{u_1},\ldots,x_\numvars^{u_\numvars}} = \ideal{x^{\vv{v}} : \vv{v} \in \NN^\numvars \text{ and } \vv{v} \not < \vv{u}}.\]
   When dealing with notation involving diagonal ideals, we shall typically replace $\diag(\vv{u})$ in the notation with $\vv{u}$.
   For instance, $\crit(\ideala,\diag(\vv{u}))$ will be simply denoted $\crit(\ideala,\vv{u})$.
   Likewise, we shall often replace a monomial ideal with its exponent matrix in our notation.
\end{notation}

\begin{remark}\label{rmk: Frobenius powers of monomial ideals are monomial ideals}
   Since integral Frobenius powers and Frobenius roots of monomial ideals are themselves monomial ideals, the same is true for arbitrary Frobenius powers of monomial ideals.
\end{remark}

\begin{proposition}\label{prop: description of frobenius powers in terms of crits}
   If $\ideala$ is a monomial ideal, then
   \[ x^{\vv{v}} \in \ideala^{[t]} \iff \ideala^{[t]} \not \subseteq \diag(\vv{v}+\vv{1}) \iff \crit(\ideala, \vv{v}+\vv{1}) > t.\]
   Consequently, $\ideala^{[t]} = \ideal{x^{\vv{v}} : \crit(\ideala, \vv{v}+\vv{1}) > t}$.
\end{proposition}

\begin{proof}
   The second equivalence follows immediately from \eqref{eq: defn of crit(a,b)}.
   As for the first, the forward implication is trivial, since $x^\vv{v} \notin\diag(\vv{v}+\vv{1})$, and conversely, if $\ideala^{[t]} \not \subseteq \diag(\vv{v}+\vv{1})$, then there exists $x^\vv{u} \in \ideala^{[t]}$ with $\vv{u} \le \vv{v}$, so $x^\vv{v}\in \ideal{x^\vv{u}} \subseteq \ideala^{[t]}$.
   The final conclusion holds because $\ideala^{[t]}$ is a monomial ideal, as noted in \Cref{rmk: Frobenius powers of monomial ideals are monomial ideals}.
\end{proof}

\begin{corollary}
   \label{cor: every crit is crit wrt diagonal ideal}
   If $\ideala$ is a monomial ideal, then every critical exponent of~$\ideala$ is of the form $\crit(\ideala,\vv{u})$, for some $\vv{u} > \vv{0}$ in $\NN^\numvars$.
\end{corollary}

\begin{proof}
   Let $\lambda$ be a critical exponent of $\ideala$, so that $\ideala^{[\lambda]}$ is properly contained in $\ideala^{[t]}$ for every $0 \le t <\lambda$.
   Since the critical exponents of $\ideala$ form a discrete set \cite[Corollary~5.8]{hernandez+etal.frobenius_powers}, the intersection of all such $\ideala^{[t]}$ properly contains $\ideala^{[\lambda]}$.
   This intersection---a monomial ideal by \Cref{rmk: Frobenius powers of monomial ideals are monomial ideals}---thus contains a monomial $x^\vv{v}$ not in $\ideala^{[\lambda]}$.
   \cref{prop: description of frobenius powers in terms of crits} then shows that $\ideala^{[t]}\not\subseteq \diag(\vv{v}+\vv{1})$, whenever $0\le t < \lambda$, but $\ideala^{[\lambda]} \subseteq \diag(\vv{v}+\vv{1})$, hence $\lambda = \crit(\ideala,\vv{v}+\vv{1})$.
\end{proof}

\daniel[inline]{
   When we look for corollaries, our ability to compute $\crit(\ideala, \vv{u})$ with $\vv{u} > \vv{1}$ will show that the ideals $\ideala^{[\lambda]}$ also vary ``uniformly'' with respect to the class of $p$ modulo some denominator $\denom$, in a way that we can make precise.
}

\daniel[inline]{What about adding a connection between $\ideala^{[t]}$ and $\tau(\ideala^{t'})$ where $t'$ is the greatest $F$-threshold less than $t$?  Like in the diagonal section of our examples paper?  Maybe we could wait to do this until we compare Frobenius powers and test ideals.}
}








\newpage


\emily[inline]{

\textbf{Important Questions}.

\begin{enumerate}
 \item Does a medium-small pair always have a medium-small sprout?
 We think the answer is NO:
 Let $A = \begin{bmatrix} 3 & 0 \\ 0 & 3 \end{bmatrix}$ and $\vv{u} = (2,2)$, so that
$(A, \vv{u})$ is small but not very small.  Then the unique special point is $\vv{s} = (2/3,2/3)$, so that if $p=2 \bmod 3$, then $[\vv{s}]_p = (2[p\%3]/3, 2[p\%3]/3) = (1/3,1/3)$.

The value of $\Theta(A, \vv{u}, \vv{s}, p)$ is $0$
using the bounds in this paper, and from this, we can find that the only element of $\Delta(A, \vv{u}, \vv{s}, p)$ is $(1,1)$, which is very small.

\item Is it true that if some digit of a critical exponent of the monomial ideal $\ideala$ equals $p-1$, then \emph{all subsequent digits} must also be $p-1$.  This seems to be true if we run into a \emph{medium small} point.  Are there points $\vv{v}$ with $\mu(A,\vv{v}, p) = p-1$ where $(A,\vv{v})$ is not medium small?  Sure, look at $A = \begin{bmatrix} 2 & 0 \\ 0 & 2 \end{bmatrix}$ and $\vv{v} = (1,1)$.  Then our Frobenius examples paper should tell us that $(A, \vv{v})$ is very small but $\mu(A, \vv{u}, p)$ should equal $1$ often.
\item We saw earlier that a medium small pair need not sprout a medium small pair.  But does a pair $(A, \vv{u})$ that is small and satisfies $\mu(A, \vv{u}, p) = p-1$ then must it sprout a pair $(B, \vv{v})$ with $\mu(B, \vv{v}, p) = p-1$?

\item Pedro pointed out the better question is that if $(A, \vv{u})$ is small and $\mu(A, \vv{u}, p) = p-1$, then is the whole critical exponent $\crit(A, \vv{u}) = 1$?
\item The answer to the last question is FALSE:  In our Frobenius examples paper, there is a critical point $1-(1/p^2) = p-1 : p-2 : \overline{p-1}$.
\end{enumerate}
}




{\small
\bibliographystyle{amsalpha}
\bibliography{bibdatabase}
}


\end{document}
